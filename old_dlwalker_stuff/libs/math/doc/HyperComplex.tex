\documentclass{book}
\usepackage{amsmath}
\usepackage{amsfonts}
\usepackage{url}
\usepackage{xspace}
\usepackage{dsfont}
\usepackage[colorlinks=true,%
            linkcolor=blue,%
            anchorcolor=blue,%
            citecolor=blue,%
            pagecolor=blue,]{hyperref}

\title{%
  The Quaternionic Exponential\\
  (and beyond)%
}
\author{%
  Hubert HOLIN%
  \medskip\\%
  \small\url{mailto:Hubert.Holin@Bigfoot.com}\\
  \small\url{http://www.bigfoot.com/~Hubert.Holin}%
}
 
\date{08/12/1999}  % plus the errata and addenda from 23/03/2001

\newtheorem{theorem}{Theorem}[section]
\newtheorem{lemma}[theorem]{Lemma}
\newtheorem{scholie}[theorem]{Scholie}
\newtheorem{proposition}[theorem]{Proposition}
\newtheorem{corollary}[theorem]{Corollary}

\newcommand{\funcDef}[5]{
  \begin{array}{rccc}
    {#1}: & {#2} & \to     & {#4} \\
          & {#3} & \mapsto & {#5}
  \end{array}
}
\newcommand{\Funcify}[1]{\mathop{\mathrm{#1}}}
% Usually \xspace does a good job when deciding whether to add a space or not.
% In the rare case that you want to force \xspace to not add a space append a
% pair of braces after the command. E.g., \Fc{N}{} is typeset like
% \ensuremath{\mathfrak{N}}. 
\newcommand{\Bl}[1]{\ensuremath{\mathds{#1}}\xspace}
\newcommand{\Fc}[1]{\ensuremath{\mathcal{#1}}\xspace}
\newcommand{\Ck}[1]{\ensuremath{\mathfrak{#1}}\xspace}
\newcommand{\fcT}{\Fc{T}}
\newcommand{\fcN}{\Fc{N}}
\newcommand{\Trace}[1]{\ensuremath{{\fcT}_{\Bl{#1}}}}
\newcommand{\Norm}[1]{\ensuremath{{\fcN}_{\Bl{#1}}}}
\newcommand{\fnRe}{\Funcify{Re}}
\newcommand{\fnIm}{\Funcify{Im}}
\newcommand{\fnUr}{\Funcify{Ur}}
\newcommand{\fnVect}{\Funcify{Vect}}
\newcommand{\fnId}{\Funcify{Id}}
\newcommand{\fnSinc}{\Funcify{sinc}}
\newcommand{\fnSinhc}{\Funcify{sinhc}}
\newcommand{\fnSincPi}{{\fnSinc}_{\pi}}
\newcommand{\fnSinhcPi}{{\fnSinhc}_{\pi}}
\newcommand{\setN}{\Bl{N}}
\newcommand{\setZ}{\Bl{Z}}
\newcommand{\setQ}{\Bl{Q}}
\newcommand{\setR}{\Bl{R}}
\newcommand{\setC}{\Bl{C}}
\newcommand{\setH}{\Bl{H}}
\newcommand{\setO}{\Bl{O}}
\newcommand{\setX}{\Bl{X}}
\newcommand{\xsetA}{\Bl{A}}
\newcommand{\xsetE}{\Bl{E}}
\newcommand{\xsetF}{\Bl{F}}
\newcommand{\setS}{\Bl{S}}
\newcommand{\cBasis}{\Fc{C}}
\newcommand{\cMatrix}{\Fc{M}}

% a comma that allows linebreaks - used for overlong tuples
% (A perfectionist would probably imitate the url package's implementation and
% add potential line breaks after math punctuation characters. But I am not
% familiar enough with low level TeX programming and the following gets the
% job done.)
% The optional argument is passed on to \linebreak. Default is \lbc[1].
\newcommand{\lbc}[1][1]{,\linebreak[#1]}

\newcommand{\sst}{\scriptstyle}

% Special spacing for the "C++" logo
\newcommand{\Cpp}{C\nolinebreak %
  \hspace{-0.05em}\raisebox{0.4ex}{\tiny\bf +}\nolinebreak %
  \hspace{-0.1em}\raisebox{0.4ex}{\tiny\bf +}\xspace}

\begin{document}
\maketitle
\tableofcontents

\chapter*{Motivation}
\addcontentsline{toc}{chapter}{Motivation}
\label{part:N}

I felt the need to take a closer look at quaternions when, some time
back, I was looking for new applications to Harthong-Reeb circles (on
which I was working at the time), and came across~\cite{Pletincks:1989ds}.
That paper, on one hand, did indicate one potential application
for that method, but, on the other hand, alluded to some odd
constructions involving quaternions, the validity of which was
propitiously left in the shadows.  The present text is therefore a
compilation of many well-known but apparently scattered results about
quaternions (and related entities), as well as some new developments,
notably the explicit formula for the quaternionic exponential (and
friends).  Incidentally, these results enables one to solve the problem
found in~\cite{Pletincks:1989ds}, but without the unsalvageable
constructions.

\chapter{Quaternions redux}
\label{part:I}

\section{What to find here}
\label{part:I.1}

This chapter only contains a quick-and-dirty (but sufficient for most
uses) presentation of the quaternions, along with their most classical
properties, inspired very largely by~\cite{Leborgne:1982hf},
\cite{Lelong-Ferrand:1978gq} and~\cite{Berger:1990ce}.
This approach, however, obscures the deep relationship which
links the quaternions, the complex and real numbers and more exotic
things known as octonions; this relationship will be the thrust of the
next chapter.

It should be said that other important uses of quaternions exist
 (\cite{Gurlebeck:1989mb},~\ldots), but that they will not be touched
upon here.  As well, quaternionic analysis (\cite{Sudbery:1979ei})
and geometry (\cite{Salamon:1982sg}), though perhaps not as vibrant
as their complex counterparts, do keep evolving; though these
usually involve fairly sophisticated mathematical machinery,
very nice results can also be had with very elementary ones
 (\cite{Casteljau:1987bc},~\ldots).  All are beyond the scope of
this article, however.

\section{The nature of the Beast}
\label{part:I.2}

Let $\setH = \setR^{4}$ with the usual four-dimensional vector
space structure over \setR.  We define $e = (1,0,0,0)$, $i =
(0,1,0,0)$, $j = (0,0,1,0)$ and $k = (0,0,0,1)$.
 
The first important thing we need is a multiplication, denoted $*$,
which we \emph{define\/} to be a (non-commutative) \setR-bilinear
operation on \setH such that $i*i = j*j = k*k = -e$, $i*j =
-(j*i) = k$, $j*k = -(k*j) = i$, $k*i = -(i*k) = j$ and $e$ is its
neutral element.

The second important thing we need is the \emph{conjugation\/} on
\setH (and we will usually denote by $\bar{q}$ the conjugate of
$q$) which we define by  $(\alpha, \beta, \gamma, \delta) \mapsto
(\alpha, -\beta, -\gamma, -\delta)$.  Important properties are that
$\overline{q * q'} = \overline{q'} * \overline{q}$, that $\bar{e} = e$,
that $q * \bar{q} = \bar{q} * q \in \setR \cdot e$ and that $q +
\bar{q} \in \setR \cdot e$.  Actually $q * \bar{q} = 0$ if and only
if $q = 0$, as is easily seen.
 
A straightforward verification then shows that $(\setH, +, *,
\cdot)$ is an effectively non-commutative, but associative,
\setR-algebra with unit $e$, and that $[\setR \to
\setH, x \mapsto (x, 0, 0, 0)]$ and $[\setC \to \setH, z
\mapsto (\fnRe z, \fnIm z, 0, 0)]$  are algebra
homomorphisms, bijective from their sources onto their images.  The
image of the conjugate of a complex number is also seen to be the
conjugate (in \setH) of the image of that complex, by the above
function.  We will therefore assimilate \setH to a superset of
 (both) \setR and \setC, and identify $e$ with $1$ and $i$
with its complex counterpart.  We see at once that the operations we
have defined on \setH extend their counterparts on \setC
and \setR.  The multiplication can then be memorized thru the
well-known formula: 
\[
i*i = j*j = k*k = i*j*k = -1
\]

It is important to notice that given any quaternion $q$ and any
\emph{real\/} number $x$, we always have $q * x = x * q = x \cdot q$.

We will usually write a quaternion under the form $q = \alpha + \beta i
+ \gamma j + \delta k$ with $\alpha$, $\beta$, $\gamma$ and $\delta$
reals, omitting the ``$\cdot$'' when multiplying a quaternion by a real
number (as per the vector space structure).  We will also omit
the ``$*$'' when multiplying a quaternion by a real number, from the
left as well as from the right.  When no confusion may arise, we will do
away entirely with the ``$*$''.

With the above notations, the conjugate of $q = \alpha + \beta i +
\gamma j + \delta k$ will then simply be $\bar{q} = \alpha - \beta i -
\gamma j - \delta k$.

Looking at \setH as a 4-dimensional \setR-vector space, it
is easy to see the usual euclidian scalar product is equal to the
following:
\begin{eqnarray*}
(p | q) & = & (p + q)\overline{(p + q)} - p\bar{p} - q\bar{q}  \\
        & = & \frac{1}{2} (p\bar{q} + q\bar{p})                \\
        & = & \frac{1}{2} (\bar{p}q + \bar{q}p)
\end{eqnarray*}

All the same, the usual euclidian norm on $\setR^{4}$, coincides
with $[q \mapsto \|q\| = \sqrt{q * \bar{q}}]$, and of course $(q | q) =
\|q\|^{2} = \alpha^{2} + \beta^{2} + \gamma^{2} + \delta^{2}$.  Note
that, if $q \neq 0$ then $q^{-1} = {(q * \bar{q})}^{-1} * \bar{q} =
\bar{q} * {(q * \bar{q})}^{-1}$.  For the quaternions, we will also use a
notation compatible with real and complex numbers and define $|q|$ as
$\|q\|$ (of course, if $q$ is actually complex, $|q|$ has exactly the
value of the modulus of $q$).

It is important to remember that $(\setH, +, *, \cdot, ||)$ is a
Banach \setR-algebra.  The norm is better than what we might
expect, though, as we have $|p * q| = |p| |q|$ instead of just $|p * q|
\leq |p| |q|$.\footnote{We ask for a norm of a Banach algebra to verify
$|p * q| \leq |p| |q|$ and not just that there exists some positive $k$
such that $|p * q| \leq k |p| |q|$.  The superior case for the norm, $|p
* q| = |p| |q|$, also applies to real numbers, complex numbers and
octonions.}

We will call the real and unreal parts of quaternion, respectively,
$\fnRe q = \frac{1}{2} (q + \bar{q})$ and $\fnUr q =
\frac{1}{2} (q - \bar{q})$.  We will say that a quaternion is pure if
its real part is zero.  For a complex number, the quaternionic real part
is what is already known as the complex real part, and the unreal part
is just the imaginary part multiplied by $i$.

\section{Quaternions' kin}
\label{part:I.3}

As we have just seen, quaternions are related to both real numbers and
complex numbers.  As we shall see in some details in the next chapter,
quaternions are actually part of an infinite family of
sets\footnote{Actually, several families, but we will focus on just one
here; for others, see~\cite{Dixon:1994rh}.} which we will call the
Cayley ladder, some of which we will introduce here as we will have some
need of them for our purposes.

First relative in that family, beyond the quaternions, are the
octonions. We denote by \setO the set $\setR^{8}$, with its
usual vector space structure on \setR, we identify $1 =
(1,0,0,0,0,0,0,0)$, $i = (0,1,0,0,0,0,0,0)$, $j = (0,0,1,0,0,0,0,0)$ and
$k = (0,0,0,1,0,0,0,0)$ and we define $e' = (0,0,0,0,1,0,0,0)$, $i' =
(0,0,0,0,0,1,0,0)$, $j' = (0,0,0,0,0,0,1,0)$ and $k' =
(0,0,0,0,0,0,0,1)$.  We now consider \setO to be a super-set of
\setH. We can now define a multiplication on \setO by the
following table (the value at line $n$ and column $m$ is
the product of the element in the left column by the element in the top
row; for instance $i * i' = -e'$):

\begin{align*}
  \begin{array}{c|cccccccc}
    & 1  & i   & j   & k   & e' & i'  & j'  & k'  \\
    \hline
    1  & 1  & i   & j   & k   & e' & i'  & j'  & k'  \\
    i  & i  & -1  & k   & -j  & i' & -e' & -k' & j'  \\
    j  & j  & -k  & -1  & i   & j' & k'  & -e' & -i' \\
    k  & k  & j   & -i  & -1  & k' & -j' & i'  & -e' \\
    e' & e' & -i' & -j' & -k' & -1 & i   & j   & k   \\
    i' & i' & e'  & -k' & j'  & -i & -1  & -k  & j   \\
    j' & j' & k'  & e'  & -i' & -j & k   & -1  & -i  \\
    k' & k' & -j' & i'  & e'  & -k & -j  & i   & -1
  \end{array}
\end{align*}
Other presentations, perhaps more useful, exist (\cite{Dixon:ex}). 
This multiplication still has a unit ($1$), but is no longer associative
 (for instance $i' * (e' * j) = +k \neq -k = (i' * e') * j$).  Real
numbers still commute with every octonion.  We define a conjugation by
$\overline{\alpha + \beta i + \gamma j + \delta k + \epsilon e' + \zeta
i' + \eta j' + \theta k'} = \alpha - \beta i - \gamma j - \delta k -
\epsilon e' - \zeta i' - \eta j' - \theta k'$, a scalar product and a
norm which, as with the quaternions turn out to be exactly the euclidian
scalar product and euclidian norm on $\setR^{8}$.  Again, we have
just extended the quaternionic operations.  As with complex numbers and
quaternions, we have $|o * o'| = |o| |o'|$ for any two octonions $o$ and
$o'$, and an octonion $o$ is invertible if and only if it is non-zero,
and then $o^{-1} = \frac{1}{|o|^{2}} \bar{o}$.

Beyond even the octonions, we find $\setR^{16}$, which appears not
to have any agreed-upon name.  We shall here call them hexadecimalions,
and denote the set by \setX (after the C/\Cpp notation\ldots).  We
have the usual vector space structure on \setR, we identify
$1,\ldots, k'$ with
$(1\lbc 0\lbc 0\lbc 0\lbc 0\lbc 0\lbc 0\lbc 0\lbc 0\lbc 0\lbc 0\lbc 0\lbc
0\lbc 0\lbc 0\lbc 0), \ldots\lbc
(0\lbc 0\lbc 0\lbc 0\lbc 0\lbc 0\lbc 0\lbc 1\lbc 0\lbc 0\lbc 0\lbc 0\lbc 0\lbc
0\lbc 0\lbc 0)$ respectively, and define
$e''$, $i''$, $j''$, $k''$, $e'''$, $i'''$, $j'''$, $k'''$ as
$(0\lbc 0\lbc 0\lbc 0\lbc 0\lbc 0\lbc 0\lbc 0\lbc 1\lbc 0\lbc 0\lbc 0\lbc
0\lbc 0\lbc 0\lbc 0), \ldots\lbc (0\lbc 0\lbc 0\lbc 0\lbc 0\lbc 0\lbc 0\lbc
0\lbc 0\lbc 0\lbc 0\lbc 0\lbc 0\lbc 0\lbc 0\lbc 1)$ respectively.  We define a
multiplication on  \setX as explicited in the next chapter, for
which $1$ is still a unit and for which reals commute with every
hexadecimalion.  We define as well a conjugation, a scalar product and a
norm (for details, see next chapter), which once again coincide with the
euclidian scalar product and euclidian norm on $\setR^{16}$.  These
all extend the octonionic case.  However, the product has even fewer
properties than in the octonionic case (the algebra is no longer even
alternative\footnote{We will define this in the next chapter.}, as for
instance $(i + e''') * ((i + e''') * j) = -2j + 2k \neq -2j = ((i +
e''') * (i + e''')) * j$), and the norm is not even an algebra norm any
longer, as for instance ${\|(i + j'') * (e' + k''')\|}^{2} = 8 > 4 =
{\|i + j''\|}^{2} {\|e' + k'''\|}^{2}$.

\section{Quaternions and rotations}
\label{part:I.4}

It is pleasant to think that perhaps the relationship between
quaternions and rotations has been stumbled upon while running a
check-list of classical constructs on the then-newly discovered
quaternions.  At any rate, the easiest way to explain that link is thru
interior automorphisms.

More precisely, given a \emph{non-zero\/} quaternion $q = \alpha + \beta i
+ \gamma j + \delta k$, we can consider the interior automorphism:
\begin{align*}
  \funcDef{\lambda_{q}}{\setH}{p}{\setH}{(q)p(q^{-1})}
\end{align*}

These objects have several fundamental properties: $\lambda_{q q'} =
\lambda_{q} \circ \lambda_{q'}$ and $\lambda_{q}(q) = q$, $\lambda_{q}$
leaves \setR invariant (since reals commute with all
quaternions), and $\lambda_{q}$ respects the norm on \setH.

It is interesting to see $\lambda_{q}$ as an \setR-linear
function on \setH.  As it preserves the norm, it preserves the
scalar product, and hence $\lambda_{q} \in \mathrm{O}(4, \setH)$. 
Then, as it leaves \setR globally invariant, it must leave its
orthogonal (i.e.\ the unreals) globally invariant.

Consider now the matrix of $\lambda_{q}$; expressed in the canonical
basis $\cBasis = (1, i, j, k)$ that matrix is:
\begin{align*}
  \cMatrix(\lambda_{q}, \cBasis, \cBasis) =
  \tfrac{1}{{\|q\|}^{2}}
  \begin{bmatrix}
    \sst \alpha^2+\beta^2+\gamma^2+\delta^2 &
    \sst 0                                  &
    \sst 0                                  &
    \sst 0                                  \\
    \sst 0                                  &
    \sst \alpha^2+\beta^2-\gamma^2-\delta^2 &
    \sst -2\alpha\delta+2\beta\gamma        &
    \sst +2\alpha\gamma+2\beta\delta        \\
    \sst 0                                  &
    \sst +2\alpha\delta+2\beta\gamma        &
    \sst \alpha^2-\beta^2+\gamma^2-\delta^2 &
    \sst -2\alpha\beta+2\gamma\delta        \\
    \sst 0                                  &
    \sst -2\alpha\gamma+2\beta\delta        &
    \sst +2\alpha\beta+2\gamma\delta        &
    \sst \alpha^2-\beta^2-\gamma^2+\delta^2
  \end{bmatrix}
\end{align*}

It is quite obvious\footnote{We will note $\mathrm{M}(U, n, m)$ the set of
matrices, $n$ rows by $m$ columns, with elements in $U$.} that
$\Fc{Q}:[\setR^{4} - \{0\} \to \mathrm{M}(\setR, 4, 4); q
\mapsto \cMatrix(\lambda_{q}, \cBasis, \cBasis)]$ is
continuous, and a group homomorphism.  As we have seen,
$\Fc{Q}(\setR^{4} - \{0\}) \subset \mathrm{O}(4, \setR)$,
and as $\Fc{Q}(1) = \mathrm{I}_{4}$, the identity
matrix\footnote{More generally, we will denote by $\mathrm{I}_{n}$ the
identity matrix on $\setR^{n}$.} on $\setH = \setR^{4}$,
$\Fc{Q}(\setR^{4} - \{0\})$ must actually be included in the
connected component of $\mathrm{I}_{4}$ in $\mathrm{O}(4, \setR)$,
and that is $\mathrm{SO}(4, \setR)$, i.e., $\lambda_{q}$
is a rotation on $\setR^{4}$, and hence on \setR, where it
is the identity $\mathrm{I}_{1}$, and thus must also be a rotation on
$\{0\} \times \setR^{3}$, i.e.\ the unreals.  We can find a
far simpler (if somewhat tedious) proof of that by simply computing the
determinant of $\cMatrix(\lambda_{q}, \cBasis, \cBasis)$,
which of course turns out to be $1$ (also see next section)\ldots.

We can therefore extract a rotation matrix on $\setR^{3}$ from
$\cMatrix(\lambda_{q}, \cBasis, \cBasis)$:
\begin{align*}
  \rho_{q} = \tfrac{1}{\alpha^2+\beta^2+\gamma^2+\delta^2}
  \begin{bmatrix}
    \sst \alpha^2+\beta^2-\gamma^2-\delta^2 &
    \sst -2\alpha\delta+2\beta\gamma        &
    \sst +2\alpha\gamma+2\beta\delta        \\
    \sst +2\alpha\delta+2\beta\gamma        &
    \sst \alpha^2-\beta^2+\gamma^2-\delta^2 &
    \sst -2\alpha\beta+2\gamma\delta        \\
    \sst -2\alpha\gamma+2\beta\delta        &
    \sst +2\alpha\beta+2\gamma\delta        &
    \sst \alpha^2-\beta^2-\gamma^2+\delta^2
  \end{bmatrix}
\end{align*}

Let us introduce $\Fc{R}:[\setR^{4} - \{0\} \to
\mathrm{M}(\setR, 3, 3); q \mapsto \rho_{q}]$.  It is trivial to see
that $\Fc{Q}$ and $\Fc{R}$ are both $C^{\infty }$ (because
they are rational).  It is important to note that they are both
\setR-homogeneous of degree $0$, which means that given any
non-zero real number $x$, $\lambda_{q}$ and $\lambda_{xq}$ are
identical, and therefore yield identical rotations (i.e.\ 
$\rho_{q} = \rho_{xq}$).

A fundamental result is that $\Fc{R}$ is surjective.  There are at
least two well-known ways to prove this.

The easiest way also has the advantage of being completely constructive:
we just compute the elements of the rotation $\rho_{q}$.

This is possible because we always know one invariant vector.  Indeed
 (as an immediate consequence of $\lambda_{q}(q) = q$):
\[
  \rho_{q}
  \left[ \begin{array}{c} \beta \\ \gamma \\ \delta \end{array} \right]
  =
  \left[ \begin{array}{c} \beta \\ \gamma \\ \delta \end{array} \right]
\]

Furthermore, the angle, $\theta \in [0; \pi]$, is given by considering
the trace of $\rho_{q}$:
\[
  1 + 2 \cos \theta = \frac{3\alpha^2 - \beta^2 - \gamma^2 - \delta^2}
    {\alpha^2 + \beta^2 + \gamma^2 + \delta^2}
\]

We now exploit the homogeneity of $\Fc{R}$, which implies that
$\Fc{R}(\setH - \{0\}) = \Fc{R}(\setS^{3})$, and
therefore that we can restrict our search to \emph{unit\/} quaternions. 
For unit quaternions, the trace relation simplifies to $1 + \cos \theta
= 2 \alpha^{2}$.

Therefore, the identity rotation $\mathrm{I}_{3}$ is associated with $q
= \pm 1$ (which we already knew), and these unit quaternions only.

Let $(\vec{\imath}, \vec{\jmath}, \vec{k})$ be the canonical basis of
$\setR^{3}$.  Consider now a rotation $r \neq \mathrm{I}_{3}$
 (hence $\theta \in ]0; \pi]$), it possesses a unique rotation axis, and
a unique unit vector $\vec{u} = x \vec{\imath} + y \vec{\jmath} + z \vec{k}$
directing that axis such that $r(\vec{a}) = (\cos \theta)[\vec{a} -
(\vec{u} \cdot \vec{a}) \vec{u}] + (\sin \theta)(\vec{u} \wedge \vec{a})
+ (\vec{u} \cdot \vec{a}) \vec{u}$ for all $\vec{a} \in \setR^{3}$.
 It follows that $r$ is associated with the two unit quaternions
\[
  q = \pm \left[ \begin{array}{c}
      \cos \frac{\theta}{2} \\
    x \sin \frac{\theta}{2} \\
    y \sin \frac{\theta}{2} \\
    z \sin \frac{\theta}{2} 
  \end{array} \right]
\]
and these two unit quaternions only.

The second method is non-constructive, but has the advantage of
highlighting the regularity of the connection between rotations and
quaternions, which is harder to read using the first method.

We once again exploit the homogeneity of $\Fc{R}$ and use unit
quaternions.  Given that we know that in fact
$\Fc{R}(\setS^{3}) \subset \mathrm{SO}(3, \setR)$, we can
consider $\Fc{R}|^{\mathrm{SO}(3, \setR)}_{\setS^{3}}$
which is $C^{\infty }$ (because it is rational).  It is slightly
tedious, but possible, to prove that in fact
$\Fc{R}|^{\mathrm{SO}(3, \setR)}_{\setS^{3}}$ is a local
diffeomorphism at $1$.  It is also a group homomorphism (stemming from
the fact that $\lambda_{q q'} = \lambda_{q} \circ \lambda_{q'}$).  Since
in a connected topological group, every neighborhood of the neutral
element is a generator of the whole group (\cite{Pichon:1973oh}),
$\Fc{R}|^{\mathrm{SO}(3, \setR)}_{\setS^{3}}$ is
surjective upon the connected component of $\mathrm{I}_{3} =
\Fc{R}(1)$ in $\mathrm{SO}(3, \setR)$, i.e.\ upon
$\mathrm{SO}(3, \setR)$, and of course is everywhere a local
diffeomorphism (though it is of course not a global diffeomorphism).

Combining these two approaches, one finds a global $C^{\infty
}$-diffeomorphism between $\mathrm{SO}(3, \setR)$ and
$\setR\mathrm{P}^{3}$ (which is nothing more than $\setS^{3}$
where every couple of opposite points have been identified).

Another thing worth noting is that $\Fc{R}|^{\mathrm{SO}(3,
\setR)}_{\setS^{3}}$ is more than just a locally diffeomorphic
bijection.  If we call $\sigma_{\setS^{3}}$ the positive Borel
measure on $\setS^{3}$ induced by $\setH = \setR^{4}$ and
$\sigma_{\mathrm{SO}(3, \setR)}$ that induced on $\mathrm{SO}(3,
\setR)$ by $\mathrm{M}(\setR, 3, 3)$ (by assimilation of the
rotations with their matrix in the canonical basis of $\setR^{3}$),
seen as $\setR^{9}$, then we can compute\footnote{A fact that is
supposed to be found, but is not, in~\cite{Misner:1973es}.} that
$\Fc{R}* \sigma_{\mathrm{SO}(3, \setR)} = 16 \sqrt{2}
\sigma_{\setS^{3}}$.  Furthermore, $\Fc{R}|^{\mathrm{SO}(3,
\setR)}_{\setS^{3}}$ actually has no critical point.

\section{Miscellany}
\label{part:I.5}

As we have seen, the main power of the quaternions is their ability to
pleasantly parameter $\mathrm{SO}(3, \setR)$.  It should be said
that what is, perhaps their greatest strengths in this regard, with
respect to other parameterization of $\mathrm{SO}(3, \setR)$ such
as Euler angles, is that $\Fc{R}|^{\mathrm{SO}(3,
\setR)}_{\setS^{3}}$ has no critical points (no ``Gimbal
Lock''), and that the composition of rotations is extremely simple to
compute in terms of the parameter.  Also and they can be shown to allow
interpolations of orientations under constraints (such has having one
axis stay ``horizontal'').

Quaternions also allow a nice parameterization of $\mathrm{SO}(4,
\setR)$ (\cite{Berger:1990ce} the application $\setS^{3}
\times \setS^{3} \to \mathrm{SO}(4, \setR), (s, r) \mapsto [q
\mapsto sq\bar{r}]$ is a continuous group homomorphism, surjective, with
kernel $\{(1, 1), (-1, -1)\}$).

Quaternions have other uses, though.  For instance, they can be also be
used to parameter $\mathrm{SU}(2, \setC)$.  More precisely, an
isomorphism exists between $\{0\} \times \setS^{3}$ and
$\mathrm{SU}(2, \setC)$ (consider, the application
\begin{align*}
\funcDef{\Psi}
  {\setH}{q = \alpha + \beta i + \gamma j + \delta k}
  {\mathrm{M}(\setC, 2, 2)}{\left[ \begin{array}{cc}
                                      u & -\bar{v} \\
                                      v &  \bar{u} \\
                                    \end{array} \right]}
\end{align*}
with $u = \alpha + \delta i$ and $v = \gamma + \beta i$ is a ring
isomorphism from \setH on a sub-ring of $\mathrm{M}(\setC,
2, 2)$, which induces an isomorphism).  There are also applications of
quaternions to the Riemann sphere (\cite{Lelong-Ferrand:1978gq}).

It should be mentioned that research exists to find more efficient
algorithm for the product of quaternions (\cite{Howell:1975ft}),
but has so far not reached a conclusion, one way or the other.

Given the power of the quaternions, the question naturally arises as to
whether something similar can be done for rotations on spaces of higher
dimensions (the multiplication being commutative on the reals and
complex numbers, interiors automorphisms are just the identity).  The
answer to that question is partly positive, but it should be now stated
that the right tool, in general, for that problem turns out to be
Clifford algebras rather than Cayley algebras.

When we turn to the octonions, the multiplication is not only not
associative, it is no longer even associative.  Fortunately, the
sub-algebra engendered by any two elements (and the unity) is still
associative, and therefore interior automorphism do not depend on the
order in which the products are carried out.  The interesting fact is
that, as with the quaternions, the interior automorphisms leave
\setR invariant, and induce a rotation, on $\setR^{7}$ this
time.  The catch is that $\mathrm{SO}(7, \setR)$ is a
21-dimentional manifold, whereas the interior automorphisms we just
described only have 7 degrees of freedom.  In short, we do not get all
the rotations on $\setR^{7}$ by this method.  It is still useful,
though, for theoretical purposes.

Beyond the even the octonions, the hexadecimalions have two different
flavors of interior automorphism, $p \mapsto ((q)p)(q^{-1})$ and $p
\mapsto (q)(p(q^{-1}))$, neither of which is, in general, a rotation (on
either $\setR^{16}$ or $\setR^{15}$).  The average of the two
is not a rotation either, by the way\ldots.

Interior automorphisms having apparently reached the limits of their
usefulness, we turn now to something else, with the same objects.  It
turns out that we can find rotations with even simpler constructions!

Let $x = \alpha \in \setR$, then $\mathrm{M}_{x} = \cMatrix([y
\mapsto xy], 1, 1) = \cMatrix([y \mapsto yx], 1, 1) = [\alpha]$,
hence ${}^{\mathrm{t}}\mathrm{M}_{x}\mathrm{M}_{x} = |x|
\mathrm{I}_{1}$, and $\det \mathrm{M}_{x} = x$.  Therefore if $|x| = 1$,
we find that $\mathrm{M}_{x} \in \mathrm{O}(1, \setR)$, and we of
course get all two elements of $\mathrm{O}(1, \setR)$ that
way\ldots but $\mathrm{M}_{x} \in \mathrm{SO}(1, \setR)$ only if $x
= 1$!  Obviously, given $x \in \setR$ and $x' \in \setR$,
$\mathrm{M}_{x x'} = \mathrm{M}_{x} \mathrm{M}_{x'} = \mathrm{M}_{x'}
\mathrm{M}_{x}$.

Let $c = \alpha + \beta i \in \setC$, then $\mathrm{M}_{c} =
\cMatrix([z \mapsto cz], (1, i), (1, i)) = \cMatrix([z \mapsto
zc], (1, i), (1, i)) = \left[ \begin{array}{cc}
  \alpha & -\beta \\
  +\beta & \alpha
\end{array} \right]$, hence ${}^{\mathrm{t}}\mathrm{M}_{c}\mathrm{M}_{c}
= |c| \mathrm{I}_{2}$, and $\det \mathrm{M}_{c} = \alpha^{2} +
\beta^{2}$.  Therefore if $|c| = 1$, $\mathrm{M}_{c} \in \mathrm{SO}(2,
\setR)$, and we get all rotations on $\setR^{2}$ that way, as
is well-known.  And given $c \in \setC$ and $c' \in \setC$, we
still have $\mathrm{M}_{c c'} = \mathrm{M}_{c} \mathrm{M}_{c'} =
\mathrm{M}_{c'} \mathrm{M}_{c}$.

Let now $q = \alpha + \beta i + \gamma j + \delta k \in \setH$,
then:
\[
  \mathrm{M}^{G}_{q}
    = \cMatrix([p \mapsto qp], (1, i, j, k), (1, i, j, k))
    = \left[ \begin{array}{cccc}
      \alpha  & -\beta  & -\gamma & -\delta \\
      +\beta  & \alpha  & -\delta & +\gamma \\
      +\gamma & +\delta & \alpha  & -\beta  \\
      +\delta & -\gamma & +\beta  & \alpha
    \end{array} \right]
\]
and
\[
  \mathrm{M}^{D}_{q}
    = \cMatrix([p \mapsto pq], (1, i, j, k), (1, i, j, k))
    = \left[ \begin{array}{cccc}
      \alpha  & -\beta  & -\gamma & -\delta \\
      +\beta  & \alpha  & +\delta & -\gamma \\
      +\gamma & -\delta & \alpha  & +\beta  \\
      +\delta & +\gamma & -\beta  & \alpha
    \end{array} \right]
,\]
hence ${}^{\mathrm{t}} \mathrm{M}^{G}_{q} \mathrm{M}^{G}_{q} =
{}^{\mathrm{t}} \mathrm{M}^{D}_{q} \mathrm{M}^{D}_{q} = |q|
\mathrm{I}_{4}$, and $\det \mathrm{M}^{G}_{q} = \det \mathrm{M}^{D}_{q}
= {(\alpha^{2} + \beta^{2} + \gamma^{2} + \delta^{2})}^{2}$.  Therefore if
$|q| = 1$, $\mathrm{M}^{G}_{q} \in \mathrm{SO}(4, \setR)$ and
$\mathrm{M}^{D}_{q} \in \mathrm{SO}(4, \setR)$, but we only get a
tiny fraction of $\mathrm{SO}(4, \setR)$ that way.

This, of course can be used as an alternate proof that the interior
automorphisms on the quaternions actually induce rotations on
$\setR^{4}$.

It is interesting to note that given $q \in \setH$ and $q' \in
\setH$, we still have $\mathrm{M}^{G}_{q q'} = \mathrm{M}^{G}_{q}
\mathrm{M}^{G}_{q'}$ and $\mathrm{M}^{D}_{q q'} = \mathrm{M}^{D}_{q'}
\mathrm{M}^{D}_{q}$, though we now sometimes have $\mathrm{M}^{G}_{q}
\mathrm{M}^{G}_{q'} \neq \mathrm{M}^{G}_{q'} \mathrm{M}^{G}_{q}$ and
$\mathrm{M}^{D}_{q'} \mathrm{M}^{D}_{q} \neq \mathrm{M}^{D}_{q}
\mathrm{M}^{D}_{q'}$.

Turning to the octonions, let $o = \alpha  + \beta  i + \gamma  j +
\delta  k + \epsilon  e' + \zeta  i' + \eta  j' + \theta  k' \in
\setO$, then:
\begin{multline*}
  \mathrm{M}^{G}_{o} =
  \cMatrix([o' \mapsto o o'], (1, i, j, k, e', i', j', k'),
  (1, i, j, k, e', i', j', k'))\\
  =
  \begin{bmatrix}
    \alpha &    -\beta &   -\gamma &   -\delta
    & -\epsilon &    -\zeta &    -\eta  & -\theta   \\
    +\beta &    \alpha &   -\delta &   +\gamma
    &    -\zeta & +\epsilon &   +\theta & -\eta     \\
    +\gamma &   +\delta &    \alpha &    -\beta
    &     -\eta &   -\theta & +\epsilon & +\zeta    \\
    +\delta &   -\gamma &    +\beta &    \alpha
    &   -\theta &     +\eta &    -\zeta & +\epsilon \\
    +\epsilon &    +\zeta &     +\eta &   +\theta
    &    \alpha &    -\beta &   -\gamma & -\delta   \\
    +\zeta & -\epsilon &   +\theta &     -\eta
    &    +\beta &    \alpha &   +\delta & -\gamma   \\
    +\eta & -  \theta & -\epsilon &    +\zeta
    &   +\gamma &   -\delta &    \alpha & +\beta    \\
    +\theta &     +\eta &    -\zeta & -\epsilon
    &   +\delta &   +\gamma &    -\beta & \alpha
  \end{bmatrix}
\end{multline*}
and
\begin{multline*}
  \mathrm{M}^{D}_{o}
  = \cMatrix([o' \mapsto o' o], (1, i, j, k, e', i', j', k'),
  (1, i, j, k, e', i', j', k'))\\
  =
  \begin{bmatrix}
    \alpha &    -\beta &   -\gamma &   -\delta
    & -\epsilon &    -\zeta &    -\eta  & -\theta   \\
    +\beta &    \alpha &   +\delta &   -\gamma
    &    +\zeta & -\epsilon &   -\theta & +\eta     \\
    +\gamma &   -\delta &    \alpha &    +\beta
    &     +\eta &   +\theta & -\epsilon & -\zeta    \\
    +\delta &   +\gamma &    -\beta &    \alpha
    &   +\theta &     -\eta &    +\zeta & -\epsilon \\
    +\epsilon &    -\zeta &     -\eta &   -\theta
    &    \alpha &    +\beta &   +\gamma & +\delta   \\
    +\zeta & +\epsilon &   -\theta &     +\eta
    &    -\beta &    \alpha &   -\delta & +\gamma   \\
    +\eta & +  \theta & +\epsilon &    -\zeta
    &   -\gamma &   +\delta &    \alpha & -\beta    \\
    +\theta &     -\eta &    +\zeta & +\epsilon
    &   -\delta &   -\gamma &    +\beta & \alpha
  \end{bmatrix}\,,
\end{multline*}
hence ${}^{\mathrm{t}} \mathrm{M}^{G}_{o} \mathrm{M}^{G}_{o} =
{}^{\mathrm{t}} \mathrm{M}^{D}_{o} \mathrm{M}^{D}_{o} = |o|
\mathrm{I}_{8}$, and $\det \mathrm{M}^{G}_{o} = \det \mathrm{M}^{D}_{o}
= (\alpha^{2} + \beta^{2} + \gamma^{2} + \delta^{2} + \epsilon^{2} +
\zeta^{2} + \eta^{2} + \theta^{2})^{4}$.  Therefore if $|o| = 1$,
$\mathrm{M}^{G}_{o} \in \mathrm{SO}(8, \setR)$ and
$\mathrm{M}^{D}_{o} \in \mathrm{SO}(8, \setR)$.  Again, we only get
a very tiny fraction of $\mathrm{SO}(8, \setR)$ that way.

Also, and contrary to the case for the real numbers, the complex numbers
and the quaternions, in general $\mathrm{M}^{G}_{o o'} \neq
\mathrm{M}^{G}_{o} \mathrm{M}^{G}_{o'}$ and $\mathrm{M}^{D}_{o o'} \neq
\mathrm{M}^{D}_{o'} \mathrm{M}^{D}_{o}$, due to the non-associativity of
the product on \setO.  For instance, $i' e' = -i$, but
$\mathrm{M}^{G}_{i'} \mathrm{M}^{G}_{e'} \neq \mathrm{M}^{G}_{-i}$.

If we try to do the same thing with hexadecimalions, we find that
neither 
$\cMatrix([l \mapsto hl]\lbc%
(1\lbc i\lbc j\lbc k\lbc e'\lbc i'\lbc j'\lbc k'\lbc e''\lbc
i''\lbc j''\lbc k''\lbc e'''\lbc i'''\lbc j'''\lbc k''')\lbc%
(1\lbc i\lbc j\lbc k\lbc e'\lbc i'\lbc j'\lbc k'\lbc
e''\lbc i''\lbc j''\lbc k''\lbc e'''\lbc i'''\lbc j'''\lbc k'''))$
nor its right-hand version
are rotation in general, even if $\|l\| = 1$.  That trail ends here as
well!

\chapter{Building the Quaternions}
\label{part:II}

\section{What to find here}
\label{part:II.1}

This chapter, except for Section~\ref{part:II.5}, only consists of
well-known classical results (\cite{Bourbaki:sc},~\cite{Lang:1971vf},
\ldots).  Some have been slightly restated (usually with
simplifications) from their original sources, but hardly anything new is
presented here.  In case the sources disagree on definitions,~\cite{Bourbaki:sc}
will take precedence.

\section[Cayley \& alternate algebras]{Cayley algebra, alternative
algebra}
\label{part:II.2}

Some of the structures we will be considering will not even be
associative.  To save what may be, a weaker structure, which is
interesting in its own right is presented first.  An algebra
\xsetE is said to be \emph{alternative\/} if the following trilinear
application, known as the \emph{associator\/} of \xsetE, is
alternating (which means its value is zero if two of its arguments are
identical):
\begin{align*}
  \funcDef{\mathrm{a}}
  {\xsetE \times \xsetE \times \xsetE}{(x, y, z)}
  {\xsetE}{x * (y * z) - (x * y) * z}
\end{align*}

This notion is interesting as, though an alternative algebra is not as
wieldy as an associative algebra, it is such that every sub-algebra
engendered by any two elements \emph{is\/} associative.  It also implies that an
alternative algebra is a \emph{division\/} algebra (which means that for any $x
\in \xsetE, x \neq 0$, the applications $\xsetE \to \xsetE;
y \mapsto x * y$ and $\xsetE \to \xsetE; y \mapsto y * x$ are
bijective, or that elements are ``simplifiable'').  In particular the
inverse of a non-zero element (if it exists) is unique in such an
algebra.

The meat of this chapter is the following structure.

Let \xsetA be a commutative ring, and \xsetE an algebra over
\xsetA, not necessarily commutative or associative, but having a
unit element $e$ (remember that since \xsetE is an
\xsetA-algebra, then $(\forall \lambda \in \xsetA) (\forall x
\in \xsetE) \lambda \cdot x = (\lambda \cdot e) * x = x * (\lambda
\cdot e)$).

A \emph{conjugation\/} over \xsetE is any (there may be none)
bijective, \xsetA-linear, function $\sigma: \xsetE \to
\xsetE$ such that:
\begin{enumerate}
\item $\sigma(e) = e$.
\item $(\forall (x, y) \in \xsetE^{2}) \sigma(x * y) = \sigma(y) *
      \sigma(x)$ (\emph{beware\/} the inversion of $x$ and $y$!).
\item $(\forall x \in \xsetE) (x + \sigma(x)) \in \xsetA \cdot
      e$ and $(\forall x \in \xsetE) (x * \sigma(x)) \in \xsetA
      \cdot e$.
\end{enumerate}

These properties imply\footnote{$(x + \sigma(x)) \in \xsetA \cdot e
\Rightarrow x * \sigma(x) = x * (x + \sigma(x)) - x * x = (x +
\sigma(x)) * x - x * x = \sigma(x) * x$.} $(\forall x \in \xsetE) x
* \sigma(x) = \sigma(x) * x$, and\footnote{Given $x \in \xsetE$,
there exists $\alpha \in \xsetA$ such that $x + \sigma(x) = \alpha
\cdot e$; the \xsetA-linearity of $\sigma$ then implies $\sigma(x)
+ \sigma \circ \sigma(x) = \sigma(x + \sigma(x)) = \alpha \cdot
\sigma(e)$, and finally, $\sigma(e) = e$.} $(\forall x \in \xsetE)
\sigma \circ \sigma(x) = x$.

We will also write $\bar{x}$ for $\sigma(x)$.

If \xsetE is such an algebra, and if $\sigma$ is a conjugation
over \xsetE, the structure $(\xsetE, +, *, \cdot, \sigma)$ is
said to be a cayley algebra over \xsetA.  On such a structure, it
is convenient to consider the cayley trace and cayley norm (an
unfortunate misnomer as it is actually quadratic\ldots), defined
respectively by $\Trace{\xsetE}(x) = x + \sigma(x)$ and
$\Norm{\xsetE}(x) = x * \sigma(x)$.

Note that if $(\xsetE, +, *)$ has no zero divisors, for instance if
it is a field, then $\Norm{\xsetE}(x) = 0$ if and only if $x
= 0$.

We have the important relations:
\begin{itemize}
\item $\Trace{\xsetE}(\sigma(x))
  = \Trace{\xsetE}(x)$
\item $\Norm{\xsetE}(\sigma(x))
  = \Norm{\xsetE}(x)$
\item $\fcT(x * y) = \fcT(y * x)$
\item $\Trace{\xsetE}(x * \sigma(y))
  = \Trace{\xsetE}(y * \sigma(x))
  = \Trace{\xsetE}(x) * \Trace{\xsetE}(y)
    - \Trace{\xsetE}(x * y)
  = \Norm{\xsetE}(x + y) - \Norm{\xsetE}(x)
    - \Norm{\xsetE}(y)$
\end{itemize}

It is interesting to note that $\fcT(x * y) = \fcT(y * x)$
regardless of whether or not \xsetE is associative or commutative.
 For the cayley norm, no such broad result seem to
hold;\footnote{Indeed, we have seen that such an equality does not hold
for hexadecimalions!} however if \xsetE is alternative, then we
also have $\Norm{\xsetE}(x * y) = \Norm{\xsetE}(x)
\Norm{\xsetE}(y)$.

Finally, the following lemma will be useful for our purposes:
\begin{lemma}[Complexo\"{\i}d]\label{lemma:complexoid}
  Given $x \in \xsetE$, $\fnVect_{\xsetA}(e, x)$,
    the \xsetA-module spanned by $x$ and $e$, is stable for $*$;
    it is a sub-cayley algebra of \xsetE which is associative and
    commutative.  If $x \not\in \xsetA \cdot e$, let
    $y = \alpha \cdot e + \beta \cdot x$,
    $\mathrm{M}^{G}_{y} = \cMatrix([u \mapsto y * u], (e, x), (e, x))$
    and $\mathrm{M}^{D}_{y} = \cMatrix([u \mapsto u * y], (e, x), (e, x))$;
    then (with $\mathrm{T} \cdot e = \Trace{\xsetE}(x)$ and
    $\mathrm{N} \cdot e = \Norm{\xsetE}(x)$)
  \[
    \mathrm{M}_{y}
      = \mathrm{M}^{G}_{y}
      = \mathrm{M}^{D}_{y}
      = \left[ \begin{array}{cc}
          \alpha & -\beta \mathrm{N}          \\
          \beta  & \beta \mathrm{T} + \alpha
        \end{array} \right]
  .\]
  Given $z \in \fnVect(e, x)$, we have
    $\mathrm{M}_{y * z}
    = \mathrm{M}_{y} \mathrm{M}_{z}
    = \mathrm{M}_{z} \mathrm{M}_{y}
    = \mathrm{M}_{z * y}$.

  This is a simple consequence of the fact that
    $x * x = \mathrm{T} \cdot x - \mathrm{N} \cdot e$, with
    $\mathrm{T} \cdot e = \Trace{\xsetE}(x)$ and
    $\mathrm{N} \cdot e = \Norm{\xsetE}(x)$!
\end{lemma}

This lemma allows us, in particular, to define unambiguously the $n$th
power, with $n \in \setN$, of any $x \in \xsetE$ by the usual
recursion rules, we will write the result, as usual $x^{n}$.  It also
trivially induces the following scholie: 
\begin{scholie}[Powers]\label{scholie:powers}
  Given $x \in \xsetE$,
    and $n \in \setN$,
  $x^{n} \in \fnVect_{\xsetA}(e, x)$,
    and $\Norm{\xsetE}(x^{n}) = {(\Norm{\xsetE}(x))}^{n}$.
\end{scholie}

\section{The Cayley doubling procedure}
\label{part:II.3}

It should be noted that this is simply the plain vanilla version of the
doubling process;\footnote{The general procedure involves abitrary
coefficients which parameterize the operations.} it will suffice here,
however.

Let \xsetA be a commutative ring, and $(\xsetE, +, *, \cdot,
\sigma)$ a cayley algebra over \xsetA, not necessarily commutative
or associative, with unit element $e$.  Let $\xsetF = \xsetE
\times \xsetE$ and $e_{\xsetF} = (e, 0) \in \xsetF$;
furthermore, let:
\[
\begin{array}{c}
  \funcDef{+_{\xsetF}}
    {\xsetF \times \xsetF}{((x, y), (x', y'))}
    {\xsetF}{(x + x', y + y')} \\
  \funcDef{*_{\xsetF}}
    {\xsetF \times \xsetF}{((x, y), (x', y'))}
    {\xsetF}{(x * x' - \overline{y'} * y, y * \overline{x'} + y' * x)} \\
  \funcDef{\cdot_{\xsetF}}
    {\xsetA \times \xsetF}{(\lambda, (x, y))}
    {\xsetF}{(\lambda \cdot x, \lambda \cdot y)} \\
  \funcDef{\sigma_{\xsetF}}
    {\xsetF}{(x, y)}
    {\xsetF}{(\sigma(x), -y)}
\end{array}
\]

\begin{proposition}[Structure]\label{proposition:structure}
  $(\xsetF, +_{\xsetF}, *_{\xsetF}, \cdot_{\xsetF})$ is
  an \xsetA-algebra, with unit $e_{\xsetF}$, and
  $\sigma_{\xsetF}$ is a conjugation over \xsetF; \xsetF
  is associative if and only if \xsetE is both associative and
  commutative; \xsetF is alternative if and only if \xsetE
  is associative.  Furthermore, $\Trace{\xsetF}((x, y))
  = \Trace{\xsetE}(x)$ and $\Norm{\xsetF}((x, y))
  = \Norm{\xsetE}(x) + \Norm{\xsetE}(y)$.
\end{proposition}

Keep in mind that since \xsetF is also an \xsetA-algebra
then $(\forall \lambda \in \xsetA) (\forall (x, y) \in \xsetF)
\lambda \cdot_{\xsetF} (x, y) = (\lambda \cdot_{\xsetF}
e_{\xsetF}) *_{\xsetF} (x, y) = (x, y) *_{\xsetF} (\lambda
\cdot_{\xsetF} e_{\xsetF})$.  It is interesting to note that, if
\xsetE is associative, we still have $\Norm{\xsetF}((x,
y) *_{\xsetF} (x', y')) = \Norm{\xsetF}((x, y))
\Norm{\xsetF}((x', y'))$, even if \xsetF is not
associative.

Given the proposition, we can (and will) identify \xsetE with
$\xsetE \times \{0_{\xsetE}\}$.  Alternatively, we can identify
\xsetF with a superset of \xsetE.  It is also possible to
identify \xsetA with a subset of \xsetE (and hence of
\xsetF as well), in that case we have noted that all elements of
\xsetA commute with all elements of \xsetE, for the
multiplication in \xsetE, as well as with all elements of
\xsetF, for the multiplication in \xsetF, even though
\xsetE or \xsetF might not be commutative.  With this
identification, $\fcT$ and $\fcN$ have value in
\xsetA.

\section{%
  \texorpdfstring{%
    \setR, \setC, \setH, \setO, \setX, \dots%
  }{%
    R, C, H, O, X, \dots%
  }%
}
\label{part:II.4}

We now consider $\xsetA = \setR$ and $\xsetE = \setR$, with $\sigma(x)
= x$ and $e = 1$, then $\Norm{\setR}(x) = x^{2}$ is always
positive (and zero if and only $x = 0$, as \setR is a field).  When we
build \xsetF as above, we get exactly \setC, and $\sigma_{\xsetF}$ is
the usual conjugation on \setC.  We define $i = (0; 1)$, and as stated
earlier, we identify \setR with $\setR \times \{0\}$.  As is well
known, \setC is a commutative field, in particular, real numbers
commute with complex numbers.  Due to our identifications,
$\Trace{\setC}$ and $\Norm{\setC}$ have values in \setR,
and actually, if $z = x + i y$ then $\Norm{\setC}(z) =
\Norm{\setR}(x) + \Norm{\setR}(y) = x^{2} + y^{2} =
|z|^{2} \geq 0$, and $\Norm{\setC}(z) = 0$ if and only if $z =
0$.  We lose some of the original properties of \setR as we build
\setC, for instance we lose the existence of an order compatible with
the multiplication; we do get new and interesting properties at the same
time, of course.

Let�s do the doubling again, this time with $\xsetA = \setR$ and
$\xsetE = \setC$, with the usual conjugation, and this time we get
exactly \setH, the conjugation being the same as defined earlier, given
the definition of $j = (0; 1)$ and $k = (0; i)$, and the identification
of \setC with $\setC \times \{0\}$.  Once again, we note that, as
predicted, for quaternion multiplication, real numbers commute with
quaternions, though some quaternions do not commute (for instance $i * j
\neq j * i$).  As already stated \setH is a (non-commutative) field. 
Once again, due to our new identifications, \Trace{H} and \Norm{H} have
values in \setR, and actually, \Norm{H} is always positive and
$\Norm{H}(q) = 0$ if and only if $q = 0$.  We keep losing original
properties, most notably the commutativity, when we go from \setC to
\setH, but the new properties we gain, notably the link with rotations
in $\setR^{3}$, which we saw earlier, still makes it worthwhile.  We
also see that $\Trace{H}(q) = 2 \fnRe q$ and $\Norm{H}(q) =
\|q\|^{2} = |q|^{2}$, as defined earlier.

There being not such thing as too much of a good thing, let�s do the
doubling once again, this time with $\xsetA = \setR$ and $\xsetE =
\setH$, and the conjugation just built on \setH.  What the process
yields this time is known as the set of (Cayley) octonions, whose symbol
is \setO.  We, as is now usual, identify \setH with $\setH \times
\{0\}$.  Yet again, we note that, for octonion multiplication, real
numbers commute with octonions, though some octonions do not commute (as
some quaternions already do not commute).  Yet again, due to our new
identifications, \Trace{O} and \Norm{O} have values in \setR, and
actually, \Norm{O} is always positive and $\Norm{O}(o) = 0$ if and only
if $o = 0$.  The situation keeps deteriorating, though, as this time the
algebra is not associative anymore (but it is still associative). 
Octonions do have uses, apart from being an example of a non-associative
algebra.  They can be used to find a basis of non-vanishing vector field
on $\setS^{7}$ (the euclidian unit sphere in $\setR^{8}$), in the same
way quaternions can be used to find one on $\setS^{3}$, and complexes
are used to find one on $\setS^{1}$.  They also see use in theoretical
physics (\cite{Dixon:1994rh}).  Octonions still are a division
algebra, and non-zero octonions $O$ have ${[\Norm{O}(O)]}^{-1}
\sigma_{\setO}(O)$ for inverse.  Despite the non-associativity of the
multiplication, we still have $\Norm{O}(o * o') = \Norm{O}(o)
\Norm{O}(o')$, since the multiplication \emph{is\/} associative on \setH.

We can keep doubling \emph{ad nauseam}, but things really get unwieldy. 
At the stage after octonions, the hexadecimalions, \setX, the algebra
is not even alternating.  This author does not know of any use the
ulterior echelons may have been put to, if any.

\section[The full Cayley ladder]{The full Cayley ladder all at once}
\label{part:II.5}

One might wonder if the whole doubling procedure might be ``carried out
to infinity''.  As it turns out, it can, after a fashion.  We will
present here a special version\footnote{That is, the object built by our
``plain vanilla'' version of the Cayley doubling procedure.} of the
global object, for simplicity.

Let \xsetA be a commutative ring, whose unit element will be called $e$.

Let us call $\Fc{A}_{0} = \xsetA$ and $\sigma_{0}$ the identity over
\xsetA.  It is quite obvious that $(\xsetA, +, \times, \times, \sigma)$
is a cayley algebra over \xsetA.  Using the doubling procedure, we build
$\Fc{A}_{1} = \xsetA \times \xsetA$ and $\sigma_{1}$, and by induction
we build $\Fc{A}_{n}$ and $\sigma_{n}$ for all $n \in \setN$.

Consider $\xsetA[X]$ the set of polynomials (in one indeterminate $X$)
with coefficients in \xsetA.  We already have an \xsetA-algebra
structure, which we will denote by $(\xsetA[X], +, \cdot, \times)$, and
is the usual commutative algebra.  We readily identify $\Fc{A}_{0} =
\xsetA$ with constant polynomials, through an homomorphism of
\xsetA-modules $\Fc{J}_{0}$.  It is trivial to see that $\Fc{A}_{n}$
identifies with polynomials of degree less or equal to $2^{n} - 1$, thru
the trivial \xsetA-modules isomorphism $\Fc{J}_{n}$.  Let us call
$I_{n}: \Fc{A}_{n} \to \Fc{A}_{n + 1}, x \mapsto (x, 0)$ the canonical
identification.  Then $(\forall n \in \setN) \Fc{J}_{n + 1} \circ I_{n}
= \Fc{J}_{n}$, which means our identifications are all coherent.

So every element of every rung of the Cayley ``ladder,'' build by
successively doubling the preceding rung and begun by \xsetA, \emph{a
finite number of times}, can be identified uniquely with some polynomial
with coefficients in \xsetA, and conversely every element of $\xsetA[X]$
can be seen a some unique element of the Cayley ladder.  As the
multiplication we will define differs, in general, from the polynomial
multiplication, we will choose a new symbol for our construction.

Let $\Ck{C}(\xsetA)$ be some set equipotent to $\xsetA[X]$, the set of
polynomials in one indeterminate $X$ over \xsetA, thru a bijection
$\Ck{I}: \Ck{C}(\xsetA) \mapsto \xsetA[X]$.  This bijection induces an
\xsetA-module on $\Ck{C}(\xsetA)$, from $(\xsetA[X], +, \cdot)$, which
we will denote by $(\Ck{C}(\xsetA), +, \cdot)$, and we identify
$\Fc{A}_{n}$ with $\Ck{I}^{-1}(\Fc{J}_{n}(\Fc{A}_{n}))$.

We will now define a multiplication on $\Ck{C}(\xsetA)$, which we will
denote by ``$*$''.  Let $\Ck{p} \in \Ck{C}(\xsetA)$ and $\Ck{q} \in
\Ck{C}(\xsetA)$; let $P = \Ck{I}(\Ck{p})$ and $Q = \Ck{I}(\Ck{q})$, then
there exists (at least) one $n \in \setN$ such that $P \in
\Fc{J}_{n}(\Fc{A}_{n})$ and $Q \in \Fc{J}_{n}(\Fc{A}_{n})$.  We chose
the smallest such $n$.  We now find the only $p_{n} \in \Fc{A}_{n}$ such
that $P = \Fc{J}_{n}(p_{n})$ and the only $q_{n} \in \Fc{A}_{n}$ such
that $Q = \Fc{J}_{n}(q_{n})$.  Finally $\Ck{I}(\Ck{p} * \Ck{q}) =
\Fc{J}_{n}(p_{n} *_{\Fc{A}_{n}} q_{n})$.  We note that for all $n' > n$,
we do have $P \in \Fc{J}_{n'}(\Fc{A}_{n'})$ and $Q \in
\Fc{J}_{n'}(\Fc{A}_{n'})$ and there are unique $p_{n'} \in \Fc{A}_{n'}$
such that $P = \Fc{J}_{n'}(p_{n'})$ and $q_{n'} \in \Fc{A}_{n'}$ such
that $Q = \Fc{J}_{n'}(q_{n'})$, but thanks to the coherence of the
identifications we also have $\Fc{J}_{n}(p_{n} *_{\Fc{A}_{n}} q_{n}) =
\Fc{J}_{n'}(p_{n'} *_{\Fc{A}_{n'}} q_{n'})$.

It is easy to verify that $(\Ck{C}(\xsetA), +, \cdot, *)$ is an
\xsetA-algebra.  However, in general $\Ck{I}(\Ck{p} * \Ck{q}) \neq
\Ck{I}(\Ck{p}) \times \Ck{I}(\Ck{q})$.  For instance if $\xsetA =
\setR$ then $X^{1} = \Ck{I}(i)$, $X^{2} = \Ck{I}(j)$, and $X^{3} =
\Ck{I}(k)$, and thus $\Ck{I}(i) = X^{1} \neq X^{5} = X^{2} \times X^{3}
= \Ck{I}(j) \times \Ck{I}(k)$.  So \Ck{I} is \emph{not}, in general, an
algebra isomorphism between $(\Ck{C}(\xsetA), +, \cdot, *)$ and
$(\xsetA[X], +, \cdot, \times)$, as stated earlier.

We likewise define the conjugation $\sigma$, and the cayley trace and
``norm,'' over $\Ck{C}(\xsetA)$ through the identifications
$\Fc{J}_{n}$, with values in $\Fc{A}_{0}$.  It is now easy to check that
$(\Ck{C}(\xsetA), +, \cdot, *, \sigma)$ is a cayley algebra over \xsetA
 (usually not commutative or associative), which, through the
identifications, contains all the rungs of the cayley doubling procedure
starting with \xsetA.  Elements of \xsetA commute with all elements of
$\Ck{C}(\xsetA)$, for $*$.

We will shortly use the fact that if $\Ck{I}(\Ck{p}) = \alpha_{0} +
\alpha_{1} X + \cdots + \alpha_{2^{n} - 1} X^{2^{n} - 1} =
\Fc{J}_{n}(p_{n})$, then $\Ck{I}(\Ck{p} * \Ck{p}) = \Fc{J}_{n}(p_{n}
*_{\Fc{A}_{n}} p_{n}) = (\alpha^{2}_{0} - (\alpha^{2}_{1} + \cdots +
\alpha^{2}_{2^{n} - 1})) + 2 \alpha_{1} \alpha_{0} X + \cdots + 2
\alpha_{2^{n} - 1} \alpha_{0} X^{2^{n} - 1}$; this is simply proved by
recurrence.

As a first example of Cayley ladders, let us consider
$(\Ck{C}(\frac{\setZ}{2 \setZ}), +, \times, *, \fnId)$.  It is a
commutative and associative cayley algebra, the conjugation being the
identity on $\Ck{C}(\frac{\setZ}{2 \setZ})$; however it has zero
divisors, as for instance $\Ck{I}^{-1}(1 + X) * \Ck{I}^{-1}(1 + X) = 0$,
but if $\Ck{p} \in \Ck{C}(\frac{\setZ}{2 \setZ})$ and $\Ck{I}(\Ck{p})$
has an odd number of 1 then $\Ck{I}(\Ck{p} * \Ck{p}) = 1$.

The second, perhaps more interesting example, is $\Ck{C}(\setR)$, which
we have actually used already.  In that case we can see that if $\Ck{a}
\in \Ck{C}(\setR)$, then $\Norm{\Ck{C}(\setR)}(\Ck{a})$ is always
positive, and $\Norm{\Ck{C}(\setR)}(\Ck{a}) = 0$ if and only if $\Ck{a}
= 0$; furthermore, $\Ck{a} \neq 0 \Rightarrow
[{(\Norm{\Ck{C}(\setR)}(\Ck{a}))}^{-1} \sigma(\Ck{a})] * \Ck{a} = \Ck{a}
* [{(\Norm{\Ck{C}(\setR)}(\Ck{a}))}^{-1} \sigma(\Ck{a})] = 1$.  It is
also possible in this case to compute square roots!  Indeed, let $x \in
\Fc{A}_{n}$, with $\Fc{J}_{n}(x) = A_{0} + A_{1} X + \cdots + A_{2^{n} -
1} X^{2^{n} - 1}$; we seek $y \in \Fc{A}_{n}$ with $\Fc{J}_{n}(y) =
\alpha_{0} + \alpha_{1} X + \cdots + \alpha_{2^{n} - 1} X^{2^{n} - 1}$
such that $y * y = x$.  This amounts to solving, in $\setR^{2^{n}}$ the
system:
\[
\left\{
\begin{array}{rcl}
  \alpha^{2}_{0} - (\alpha^{2}_{1} + \cdots
        + \alpha^{2}_{2^{n} - 1}) &    =   & A_{0}         \\
          2 \alpha_{1} \alpha_{0} &    =   & A_{1}         \\
                                  & \vdots &               \\
  2 \alpha_{2^{n} - 1} \alpha_{0} &    =   & A_{2^{n} - 1}
\end{array}
\right.
\]
This system is easily solved by considering first the case $x = 0$, for
which there is a unique solution $y = 0$, second the subcase $x \in
\setR^{*}_{+}$ for which there are exactly two solutions given by $y =
\pm \sqrt{x}$, third the subcase $x \not\in \setR$ (if $n \geq 1$, of
course) for which there are also exactly two solution given by
$\Fc{J}_{n}(y) = \alpha_{0} + \frac{A_{1}}{2 \alpha_{0}} X + \cdots +
\frac{A_{2^{n} - 1}}{2 \alpha_{0}} X^{2^{n} - 1}$ with $\alpha_{0} = \pm
\sqrt{\frac{\fnRe x + |x|}{2}}$, and finally the case $x \in
\setR^{*}_{-}$, for which the solutions are all the $y \in \Fc{A}_{n}$
such that $\Fc{R}(y) = 0$ and $|y| = |x|$.

This means that the solutions to $y^{2} = x$, where $x \not\in
\setR_{-}$ are the same in every rung of the real Cayley ladder (that
is, there are exactly two, opposite solutions, belonging to the same
rung), and the solution to $y^{2} = 0$ is always $y = 0$, in whatever
rung of the Cayley ladder.  However, solutions to $y^{2} = x$ for $x \in
\setR^{*}_{-}$ differ depending upon the precise rung: in \setR there
is no solution, in \setC there are exactly two, opposite, solutions, in
\setH and above there is an innumerable number of solutions (full
spheres)!

Note that in any case a $y$ such that $y^{2} = x$ commutes with $x$, but
that two such solutions need not commute with each other!

At least two topologies are interesting to consider on $\Ck{C}(\setR)$:
the norm topology induced by the square root of \fcN, the Cayley
``norm'' on $\Ck{C}(\setR)$ (we will write $\|c\| = \sqrt{\fcN(c)}$),
which we will call \fcT, and the strict inductive limit topology
 (\cite{Khoan:1972ld}) defined by the rungs $\Ck{C}_{n} =
\Fc{J}_{n}(\Fc{A}_{n})$ of $\Ck{C}(\setR)$ on which we consider the
norms $q_{n} = 2^{\sup(0, n - 2)} \|\cdot\|$, which we will call
$\Trace{\infty}$.

The problem with $\|\cdot\|$ is that it is not an algebra norm, as
evidenced by the hexadecimalions.  Furthermore, $(\Ck{C}(\setR),
\fcT)$ is not complete, its completion being $\ell^{2}(\setR)$ with
its usual topology.

On the other hand, $(\Ck{C}(\setR), \Trace{\infty})$ is complete, and
the product is (trivially) separately continuous (\cite{Bourbaki:do}),
but it is not known if it is continuous.

For both topologies, any finite-dimensional vector space is closed and
the restriction to that vector space is just the usual (euclidian)
topology.

At any rate, given $x \in \Ck{C}(\setR)$, the Powers Scholie proves
that $(\fnVect_{\setR}(1, x), \| \|)$ is a commutative
\setR-Banach algebra (of dimension 1 if and only if $x \in \setR$).

As a final thought, since $\Ck{C}(\xsetA)$ is an \xsetA-Cayley algebra,
we can perform the Cayley doubling procedure on it!  And again, and so
on and so forth\ldots.  We can actually perform an infinity of doubling
as above, and embed all these doublings in what, essentially, is
$\xsetA[X, Y]$.  And then we can start all over again\ldots.  As we can
readily see, there is no ``ultimate'' step\ldots.  What seems to be
going on is that we can build an object for any \emph{finite\/} ordinal
 (\cite{Exbrayat:1971pk}), and we have built an object,
which we have called $\Ck{C}(\xsetA)$, for the first infinite ordinal
$\omega$.  We have then seen that the doubling of $\Ck{C}(\xsetA)$
yields the object corresponding to $\omega^{*}$ (the successor of
$\omega$).  The next infinite ordinal with no predecessor ($2 \omega$)
corresponds to $\xsetA[X, Y]$.  Further on (corresponding to
$\omega^{2}$), we find the set of polynomials in an indeterminate number
of indeterminates (i.e.\footnote{Recall that if $X$ is a
mono\"{\i}d (~\cite{Bourbaki:sc}) and $Y$ is a set, $X^{[Y]}$ is
the set of functions from $Y$ to $X$ which take values different from
the neutral element of $X$ only for a finite numbers of elements of
$Y$.}\ $\xsetA^{[\setN^{[\setN]}]}$).  It is not clear, however, in
which way we can extend the construction to any set of ordinals
 (i.e.\ there is no clear transfinite ``recurrence formula'').

\chapter{The Exponential}
\label{part:III}

\section{What to find here}
\label{part:III.1}

This chapter is mostly designed to prove the explicit formula for the
exponential in $\Ck{C}(\setR)$, and give several related results.  As
far as I known, these results are new.

There are many notions of the exponential, and many ways to see several
of them.  These, of course, agree when various different definitions can
be put forward for the same object to be exponentiated.  We will be
concerned here mainly with the analyst�s point of view, and define the
exponential of quaternions thru the use of the usual power series
 (\cite{Beardon:1979ek},~\ldots).  It is known that the approach
detailed in~\cite{Pham:1996gf} could also be used, at least for
quaternions, though I believe it would then be necessary to derive the
power series representation (or the intermediary differential
representation we will also use) to achieve our present goal.  It
remains to be seen if it can also be carried over to the whole of
$\Ck{C}(\setR)$.

\section{Definition}
\label{part:III.2}

Given $x \in \Ck{C}(\setR)$, we will call \emph{exponential\/} of $x$,
and we will write $\exp x$ the element of $\Ck{C}(\setR)$ given by
$\sum^{+\infty}_{n = 0} \frac{x^{n}}{n!}$.  The unambiguity and
existence of $\exp x$ is given by the fact that
$(\fnVect_{\setR}(1, x), \| \|)$ is a commutative \setR-Banach
algebra, as we have said earlier.  This, of course agrees with the
definition on \setR and \setC. We must bear in mind that $\exp x \in
\fnVect_{\setR}(1, x)$.

We see at once that $(\forall x \in \Ck{C}(\setR)) \exp \overline{x} =
\overline{\exp x}$.  The exponential is continuous when restricted to
each rung of $\Ck{C}(\setR)$, and has its values into the same rung (we
will give a more precise result later on).

\section{Links with differentiation}
\label{part:III.3}

Differentiating a function of one or several quaternions (or higher in
the Cayley ladder) is quite problematic.  Of course, since
$\fnVect_{\setR}(1, x)$ is commutative, there is no ambiguity in
defining $\frac{f(y) - f(x)}{y - x}$ if $y \in \fnVect_{\setR}(1,
x)$, and we can therefore differentiate
$\exp|_{\fnVect_{\setR}(1, x)}$ with respect to some $y \in
\fnVect_{\setR}(1, x)$ and find that it is once again
$\exp|_{\fnVect_{\setR}(1, x)}$.

It is more fruitful, however, to differentiate a function of a
\emph{real\/} variable, with values in some topological \setR-vector
space.

Let us therefore consider, for some $x \in \Ck{C}(\setR)$, the function
$e_{x}: [\setR \to \Ck{C}(\setR), t \mapsto \exp tx]$.  It is clear
that $e_{x}$ takes its values in $\fnVect_{\setR}(1, x)$, is
differentiable and $e'_{x}(t) = x e_{x}(t) = e_{x}(t) x$, and of course
$e_{x}(0) = 1$.  This, of course proves that $e_{x}$ is the unique
solution to $f' = x f, f(0) = 1$ in $\cBasis^{1}(\setR,
\fnVect_{\setR}(1, x))$, the set of one-time continuously
differentiable functions from \setR to $\fnVect_{\setR}(1, x)$. 
Given any rung \Fc{E} of $\Ck{C}(\setR)$ such that $x \in \Fc{E}$,
$e_{x}$ is still the unique solution to $f' = x f, f(0) = 1$ in
$\cBasis^{1}(\setR, \Fc{E})$.

The perhaps surprising phenomenon is when we consider the equation $f' =
x f, f(0) = \gamma$ in $\cBasis^{1}(\setR, \Fc{E})$ for some rung \Fc{E}
of $\Ck{C}(\setR)$ such that $x \in \Fc{E}$, and $\gamma \in \Fc{E}$. 
If $\Fc{E} = \setR$ or $\Fc{E} = \setC$, then of course the solution
is $e_{x}(t) \gamma$, and it turns out this is still true if $\Fc{E} =
\setH$, because of the associativity of the quaternionic product (this,
actually, is how one can navigate the unit sphere of the quaternions,
which is useful for interpolating between orientations, and was the
problem under examination in~\cite{Pletincks:1989ds}).  It is
interesting to note that this is \emph{still\/} true if $\Fc{E} = \setO$
because of the alternative nature of that algebra.  This stops to be
true with hexadecimalions, however.  Indeed, consider $x = i + e'''$ and
$\gamma = j$, and let $\mathrm{g}(t) = e_{x}(t) \gamma$.  We will
shortly see that $e_{i + e'''}(\frac{\pi \sqrt{2}}{4}) =
\frac{\sqrt{2}}{2} (i + e''')$, from which we can deduce
$\mathrm{g}(\frac{\pi \sqrt{2}}{4}) = \frac{\sqrt{2}}{2} (i + e''') j$
and $\mathrm{g}'(\frac{\pi \sqrt{2}}{4}) = ((i + e''')
\frac{\sqrt{2}}{2} (i + e''')) j = -\sqrt{2} j$ whereas $(i + e''')
\mathrm{g}(\frac{\pi}{2}) = (i + e''') (\frac{\sqrt{2}}{2} (i + e''') j)
= \sqrt{2}(-j + k''')$, and therefore $\mathrm{g}'(\frac{\pi}{2}) \neq
(i + e''') \mathrm{g}(\frac{\pi}{2})$.  Numerical integration procedures
will yield the solution to the differential equation, and therefore not
the exponential function, unless care has been taken to chose the
starting point correctly.

\section[The exponential formula]{The closed formula for the exponential
in $\Ck{C}(\setR)$}
\label{part:III.4}

We now give the main result of this work.  Note that it is closed only
in that we assume the exponential and classical trigonometric functions
on \setR to be givens.\footnote{A family of special functions will be
of interest here, that of the ``Sinus Cardinal'' functions, defined for
some parameter $a \in \setR^{*}_{+}$ by $\fnSinc_{a}: [\setR \to
\setR, x \mapsto \frac{\sin \pi x / a}{\pi x / a}]$.  We will, by
similitude, define the ``Hyperbolic Sinus Cardinal'' family of functions
defined for some parameter $a \in \setR^{*}_{+}$ by
$\fnSinhc_{a}: [\setR \to \setR, x \mapsto \frac{\sinh \pi x /
a}{\pi x / a}]$.  These functions are entire functions on all of
\setR.}

\begin{theorem}[Exponential]\label{theorem:exponential}
If $x \in \Ck{C}(\setR)$ then $\exp x = \mathrm{e}^{\fnRe x}
[\cos \|\fnUr x\| + (\fnSincPi \|\fnUr x\|)
 (\fnUr x)]$.

Let $y \in \Ck{C}(\setR)$ such that $\fnRe y = 0, \|y\| = 1$;
then $y^{2} = [\fcT(y) - \bar{y}] y = -\fcN(y) = -1$.  Therefore, in
$\fnVect_{\setR}(1, y)$ computations are carried out exactly as
in \setC, with $y$ taking the place of $i$.  More precisely, $[\setC
\to \fnVect_{\setR}(1, y), a + ib \mapsto a + by]$ is a Banach
isomorphism.

Let now $x \in \Ck{C}(\setR)$.  If $x \in \setR$, we see
the result is trivially true.  Assume, then that $x \not\in \setR$, and
let $\hat{x} = \frac{\fnUr x}{\|\fnUr x\|}$.  Then
$\fnRe \hat{x} = 0, \|\hat{x}\| = 1, x = \fnRe x +
\|\fnUr x\| \hat{x}$, and of course $\fnVect_{\setR}(1, x)
= \fnVect_{\setR}(1, \hat{x})$.  The previous identification then
allows us to find $\exp x = \mathrm{e}^{\fnRe x} [\cos
\|\fnUr x\| + (\sin \|\fnUr x\|) \hat{x}]$.
\end{theorem}

As an example, we have, as announced earlier, $\exp \frac{\pi
\sqrt{2}}{4} (i + e''') = \frac{\sqrt{2}}{2} (i + e''')$.

\section[Exponential properties]{Some properties of the exponential and
further consequences}
\label{part:III.5}

We compute at once $\|\exp x\| = \mathrm{e}^{| \fnRe x|}$.

As should be expected when we lose the benefit of commutativity, the
exponential of a sum is in general different from the product of the
exponentials; for instance we have $(\exp i) (\exp j) = {(\cos 1)}^{2} + i
(\sin 1) (\cos 1) + j (\sin 1) (\cos 1) + k {(\sin 1)}^{2}$ whereas $\exp
(i + j) = (\cos \sqrt{2}) + \frac{\sqrt{2}}{2} (i + j)$.  We also see
immediately that the exponential is not injective on any rung \Fc{E} of
$\Ck{C}(\setR)$ containing \setC, as it is already not injective on
\setC!  We, however also lose the periodicity when \Fc{E} contains
\setH, as the periods would make an additive subgroup of \Fc{E} but the
solutions of $\exp x = 1$ on \setH are exactly the set $2 \pi . \setN
. \{0\} \times \setS^{2}$ (with $\setS^{2}$ the unit sphere of
$\setR^{3}$); the rest is number theory (and trying to fit square pegs
into round holes).  We have more details on the surjectivity of the
exponential:
\begin{corollary}[surjectivity]\label{corollary:surjectivity}
If \Fc{E} is a rung of $\Ck{C}(\setR)$ containing \setC, then the
exponential is a surjection from \Fc{E} onto $ \Fc{E} -\{0\}$.

We first note that given any $x \in \Ck{C}(\setR)$, $\|\exp x\| =
\mathrm{e}^{| \fnRe x|}$ proves that the exponential never take
the value 0 on $\Ck{C}(\setR)$.

Let now $y \in \Fc{E}, y \neq 0$.  If $y \in \setR$ we know we can
solve our problem (in \setR if $y > 0$, in \setC if $y < 0$).  Assume
therefore that $y \not\in \setR$.  We can find $\rho \in \setR$ such
that $\mathrm{e}^{\rho} = \|y\|$.  Let $\tilde{y} = \frac{y}{\|y\|}$;
$\|\tilde{y}\| = 1$ and $\tilde{y} \not\in \setR$, so let $\hat{y} =
\frac{\fnUr \tilde{y}}{\|\fnUr \tilde{y}\|}$, so that
$\tilde{y} = \fnRe \tilde{y} + \|\fnUr \tilde{y}\|
\hat{y}$, $\fnRe \tilde{y} \neq 0$ and
${(\fnRe \tilde{y})}^{2} + {\|\fnUr \tilde{y}\|}^{2} = 1$. 
Let $\theta \in ]0; \pi[$; the unique number such that $\cos \theta =
\fnRe \tilde{y}$ and $\sin \theta = \|\fnUr \tilde{y}\|$. 
We see that $\exp (\rho + \theta \hat{y}) = y$.
\end{corollary}

We can likewise find closed formul\ae\ for other interesting entire
functions (defining $\cos x = \sum^{+\infty}_{n = 0} \frac{{(-1)}^{n} x^{2
n}}{(2 n)!}$, $\sin x = \sum^{+\infty}_{n = 0} \frac{{(-1)}^{n} x^{2 n +
1}}{(2 n + 1)!}$, $\cosh x = \sum^{+\infty}_{n = 0} \frac{x^{2 n}}{(2
n)!}$, $\sinh x = \sum^{+\infty}_{n = 0} \frac{x^{2 n + 1}}{(2 n +
1)!}$), to wit:
\[
  \begin{array}{rcl}
     \cos x & = &  (\cos \fnRe x) (\cosh \|\fnUr x\|)
       -  (\sin \fnRe x) (\fnSinhcPi \|\fnUr x\|)
       (\fnUr x) \\
     \sin x & = &  (\sin \fnRe x) (\cosh \|\fnUr x\|)
       +  (\cos \fnRe x) (\fnSinhcPi \|\fnUr x)\|)
       (\fnUr x) \\
    \cosh x & = & (\cosh \fnRe x)  (\cos \|\fnUr x\|)
       + (\sinh \fnRe x)  (\fnSincPi \|\fnUr x\|)
       (\fnUr x) \\
    \sinh x & = & (\sinh \fnRe x)  (\cos \|\fnUr x\|)
       + (\cosh \fnRe x)  (\fnSincPi \|\fnUr x\|)
       (\fnUr x)
  \end{array}
\]
and of course many other such.

\section{Conclusion}
\label{part:III.6}

We have found a closed formula for the exponential, for quaternions,
octonions, and beyond.

An interesting application of this formula is navigation on the unit
sphere of the quaternions, leading to an algorithm for the interpolation
of orientations, but which, in general, does not preserve the
horizontal.  This can also be achieved, however, and has been
implemented by the author and a colleague (\cite{Holin:lq}).

\appendix
\chapter{Addenda}
\label{part:A}

\section{More structure}
\label{part:A.1}

It is interesting to note that \setC, \setH and \setO are (left)
vector spaces over \setC.

The basis of \setH as a left \setC-vector space is $(1, j)$, and the
basis of \setO as a left \setC-vector space is $(1, j, e', j')$. 
However, if $q = \alpha  + \beta  i + \gamma  j + \delta  k \in \setH$,
then $q = (\alpha  + \beta  i) + (\gamma + \delta  i) j$, but if $o =
\alpha + \beta i + \gamma j + \delta k + \epsilon e' + \zeta i' + \eta
j' + \theta k' \in \setO$ then $o = (\alpha + \beta i) + (\gamma +
\delta i) j + (\epsilon + \zeta i) e' + (\eta - \theta i) j'$ (note the
minus sign in the last factor).

If we write $q = \Gamma + \Delta j$, with $\Delta \in \setC$ and
$\Gamma \in \setC$, then $\overline{q} = \overline{\Gamma} - \Delta j$,
and if we also have $p = \mathrm{A} + \mathrm{B} j$, with $\mathrm{A}
\in \setC$ and $\mathrm{B} \in \setC$, then $p q = (\mathrm{A} \Gamma
- \mathrm{B} \overline{\Delta}) + (\mathrm{A} \Delta + \mathrm{B}
\overline{\Gamma}) j$.  In particular, if $z \in \setC$ then $j z =
\overline{z} j$.

Things break down when we want to consider \setO as a structure over
\setH, however.  Indeed, there is no widely-accepted generalization of
vector field where the role of the scalars is taken by a non-commutative
structure, as is the case with \setH as most interesting properties of
vector spaces fail to remain true in that case, in general (though by
requiring the scalars to be merely a commutative ring instead of a full
blown field, quite a few properties remain true; this structure is known
as a module).

However, if $o = \alpha + \beta i + \gamma j + \delta k + \epsilon e' +
\zeta i' + \eta j' + \theta k' \in \setO$, then it is also true that $o
= (\alpha + \beta i + \gamma j + \delta k) + (\epsilon + \zeta i + \eta
j + \theta k) e'$.

\section{More Geometry}
\label{part:A.2}

Another interesting way to see \setH is as $\setR \times \setR^{3}$. 
In this case, if $q_{1} = (t_{1}, V_{1}) \in \setR \times \setR^{3}$
and $q_{2} = (t_{2}, V_{2}) \in \setR \times \setR^{3}$, then the
quaternionic product can be expressed as $q_{1} q_{2} = (t_{1} t_{2} -
V_{1} \cdot V_{2}, t_{1} V_{2} + t_{2} V_{1} + V_{1} \wedge V_{2})$,
with ``$\cdot$'' the scalar product on $\setR^{3}$ and ``$\wedge$'' the
vector product in $\setR^{3}$.

\section[Quaternions for rotations]{Finding the quaternions for a given
rotation of $\setR^{3}$}
\label{part:A.3}

If we are given the rotation in term of vector and angle in $[0; +\pi]$,
then this has been solved in the main text (if the angle is in $[-\pi;
0]$, we just take the opposite of both the angle and vector; the
identity and its opposite are trivial to solve).  The opposite
quaternion is also a solution, of course.

If we are simply given a rotation matrix, then, essentially, we first
find its elements (vector and angle), and use the procedure above.  To
find an invariant vector, we simply solve the linear system which
defines them.  For the angle, we first find its cosine using the trace. 
Then we build a vector orthogonal to the invariant vector we found
 (always possible starting from one of the canonical basis vector and
using some classical orthonormalization procedure) to check the sign of
the angle.

If we are given a succession of rotation, it may be advantageous in
applications to chose among the successions of pairs of opposite
solutions that for which the distance between successive quaternions is
the smallest.

\section{More rotations}
\label{part:A.4}

We have considered $[\setH \to \setH, p \mapsto p q]$ and $d_q =
[\setH \to \setH, p \mapsto q p]$ in Chapter~\ref{part:I} and seen,
through some amount of computation, that they gave rise to rotations on
$\setR^{4}$, when $|q| = 1$.

We present here another take on the same subject, aimed at giving
effective methods of parameterizing $\mathrm{SO}(4, \setR)$.

It is also interesting to consider $g_q = [\setH \to \setH, p \mapsto
p \overline{q}]$, as both $g_q$ and $d_q$ are \setC-linear operators on
\setH, which trivially verify $g_{q q'} = g_q \circ g_{q'}$ and $d_{q
q'} = d_q \circ d_{q'}$.  We can also very simply verify that $(g_{q}(p)
| g_{q}(p')) = \|q\|^{2} (p | p') = (d_{q}(p) | d_{q}(p'))$.  Obviously
$g_{1}$ and $d_{1}$ are both the identity on \setH.  Considering them
now as \setR-linear operators on \setH, we see that their determinant
in the canonical basis must stay of the same sign on $\setS^{3}$, hence
must stay positive (since $1 \in \setS^{3}$), therefore must be always
equal to 1 on $\setS^{3}$.

Hence $[\setS^{3} \to \mathrm{SO}(4, \setR), q \mapsto g_{q}]$ and
$[\setS^{3} \to \mathrm{SO}(4, \setR), q \mapsto d_{q}]$ are both
group homeomorphism, sending 1 to the identity.

Using the same kind of topological argument as in the case of
$\mathrm{SO}(3, \setR)$, we get the parameterization of $\mathrm{SO}(4,
\setR)$ that we announced (\cite{Berger:1990ce},~\cite{Godbillon:1971th})
simply by considering $(p, q) \mapsto d_{p} \circ g_{q}$,
save for the determination of the kernel.  We will, however, aim
here for a more constructive approach to surjectivity.

Given $r$ a rotation of $\setR^{4}$, we seek $p \in \setS^{3}$ and $q
\in \setS^{3}$ such that for any quaternion $s$ we have $p s
\overline{q} = r(s)$.  Hence, by applying that to $s = 1$, we find that
necessarily $p = r(1) q$ (since $\overline{q} = q^{-1}$ as $|q| = 1$). 
Therefore we are led to solve $q s \overline{q} = \rho(s)$ for all $s$,
with $\rho(s) = \overline{r(1)} r(s)$ (as $|r(t)| = |t|$ for all
quaternion $t$, since $r$ is a rotation, and hence $|r(1)| = 1$).  But
then $\rho(\mu) = \overline{r(1)} r(\mu) = \mu \overline{r(1)} r(1) =
\mu$ for all $\mu \in \setR$, which means \setR is invariant, and we
know that $\rho$ is a rotation $\setR^{4}$, as the composition of $r$,
which is one by hypothesis, and the multiplication on the left by a unit
quaternion, which is one also as we have seen in the main text.  This
means we are simply back to solving on $\setS^{3}$ the equation
$\rho_{q} = \cMatrix(\rho, \cBasis, \cBasis)$ (with $q \mapsto \rho_{q}$ as
presented in Chapter~\ref{part:I}).

Finally, given two unit quaternions $p = \alpha + \beta i + \gamma j +
\delta k$ and $q = \epsilon + \zeta i + \eta j + \theta k$, the rotation
matrix on $\setR^{4}$ is given explicitly by:
\begin{align*}
  \begin{bmatrix}
    \sst \alpha \epsilon + \beta \zeta + \gamma \eta + \delta \theta  &
    \sst +\alpha \zeta - \beta \epsilon - \gamma \theta + \delta \eta &
    \sst +\alpha \eta + \beta \theta - \gamma \epsilon - \delta \zeta &
    \sst +\alpha \theta - \beta \eta + \gamma \zeta - \delta \epsilon \\
    \sst -\alpha \zeta + \beta \epsilon - \gamma \theta + \delta \eta &
    \sst \alpha \epsilon + \beta \zeta - \gamma \eta - \delta \theta  &
    \sst -\alpha \theta + \beta \eta + \gamma \zeta - \delta \epsilon &
    \sst +\alpha \eta + \beta \theta + \gamma \epsilon + \delta \zeta \\
    \sst -\alpha \eta + \beta \theta + \gamma \epsilon - \delta \zeta &
    \sst +\alpha \theta + \beta \eta + \gamma \zeta + \delta \epsilon &
    \sst \alpha \epsilon - \beta \zeta + \gamma \eta - \delta \theta  &
    \sst -\alpha \zeta - \beta \epsilon + \gamma \theta + \delta \eta \\
    \sst -\alpha \theta - \beta \eta + \gamma \zeta + \delta \epsilon &
    \sst -\alpha \eta + \beta \theta - \gamma \epsilon + \delta \zeta &
    \sst +\alpha \zeta + \beta \epsilon + \gamma \theta + \delta \eta &
    \sst \alpha \epsilon - \beta \zeta - \gamma \eta + \delta \theta
   \end{bmatrix}
\end{align*}

\bibliographystyle{alpha}
\bibliography{HyperComplex}
\end{document}
