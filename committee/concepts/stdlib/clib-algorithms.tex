\documentclass[american,twoside]{book}
\usepackage{refbib}
\usepackage{hyperref}
\usepackage{pdfsync}
% Definitions and redefinitions of special commands

\usepackage{babel}      % needed for iso dates
\usepackage{savesym}		% suppress duplicate macro definitions
\usepackage{fancyhdr}		% more flexible headers and footers
\usepackage{listings}		% code listings
\usepackage{longtable}	% auto-breaking tables
\usepackage{remreset}		% remove counters from reset list
\usepackage{booktabs}		% fancy tables
\usepackage{relsize}		% provide relative font size changes
\usepackage[htt]{hyphenat}	% hyphenate hyphenated words: conflicts with underscore
\savesymbol{BreakableUnderscore}	% suppress BreakableUnderscore defined in hyphenat
									                % (conflicts with underscore)
\usepackage{underscore}	% remove special status of '_' in ordinary text
\usepackage{verbatim}		% improved verbatim environment
\usepackage{parskip}		% handle non-indented paragraphs "properly"
\usepackage{array}			% new column definitions for tables
\usepackage[iso]{isodate} % use iso format for dates
\usepackage{soul}       % strikeouts and underlines for difference markups
\usepackage{color}      % define colors for strikeouts and underlines
\usepackage{amsmath}    % additional math symbols
\usepackage{mathrsfs}
\usepackage{multicol}

\usepackage[T1]{fontenc}
\usepackage{ae}
\usepackage{mathptmx}
\usepackage[scaled=.90]{helvet}

%% Difference markups
\definecolor{addclr}{rgb}{0,.4,.4}
\definecolor{remclr}{rgb}{1,0,0}
\newcommand{\added}[1]{\textcolor{addclr}{\ul{#1}}}
\newcommand{\removed}[1]{\textcolor{remclr}{\st{#1}}}
\newcommand{\changed}[2]{\removed{#1}\added{#2}}
\newcommand{\remfn}{\footnote{\removed{removed footnote}}}
\newcommand{\addfn}[1]{\footnote{\added{#1}}}
\newcommand{\remitem}[1]{\item\removed{#1}}
\newcommand{\additem}[1]{\item\added{#1}}

%% Added by JJ
\long\gdef\metacomment#1{[{\sc Editorial note:} \begingroup\sf\aftergroup] #1\endgroup}

%% October, 2005 changes
\newcommand{\addedA}[1]{#1}
\newcommand{\removedA}[1]{}
\newcommand{\changedA}[2]{#2}

%% April, 2006 changes
\newcommand{\addedB}[1]{#1}
\newcommand{\removedB}[1]{}
\newcommand{\changedB}[2]{#2}
\newcommand{\remfootnoteB}[1]{}
\newcommand{\marktr}{}
\newcommand\ptr{}

%% October, 2006 changes
%\newcommand{\addedC}[1]{\added{#1}}
%\newcommand{\removedC}[1]{\removed{#1}}
%\newcommand{\changedC}[2]{\changed{#1}{#2}}
%\newcommand{\remfootnoteC}[1]{\remfn}
%\newcommand{\addfootnoteC}[1]{\addfn{#1}}
%\newcommand{\remitemC}[1]{\remitem{#1}}
%\newcommand{\additemC}[1]{\additem{#1}}
%\newcommand{\remblockC}{}

%% November registration ballot
\newcommand{\addedC}[1]{#1}
\newcommand{\removedC}[1]{}
\newcommand{\changedC}[2]{#2}
\newcommand{\remfootnoteC}[1]{}
\newcommand{\addfootnoteC}[1]{\footnote{#1}}
\newcommand{\remitemC}[1]{}
\newcommand{\additemC}[1]{\item{#1}}
\newcommand{\remblockC}{\remov_this_block}

\newcommand{\addedD}[1]{#1}
\newcommand{\removedD}[1]{}
\newcommand{\changedD}[2]{#2}
\newcommand{\remfootnoteD}[1]{}
\newcommand{\addfootnoteD}[1]{\footnote{#1}}
\newcommand{\remitemD}[1]{}
\newcommand{\additemD}[1]{\item{#1}}
\newcommand{\remblockD}{\remov_this_block}

%% Variadic Templates changes
\newcommand{\addedVT}[1]{\textcolor{addclr}{\ul{#1}}}
\newcommand{\removedVT}[1]{\textcolor{remclr}{\st{#1}}}
\newcommand{\changedVT}[2]{\removed{#1}\added{#2}}

%% Concepts changes
\newcommand{\addedConcepts}[1]{\added{#1}}
\newcommand{\removedConcepts}[1]{\removed{#1}}
\newcommand{\changedConcepts}[2]{\changed{#1}{#2}}
\newcommand{\addedConceptsC}[1]{\textcolor{addclr}{\tcode{\ul{#1}}}}
\newcommand{\remitemConcepts}[1]{\remitem{#1}}
\newcommand{\additemConcepts}[1]{\additem{#1}}

%% Concepts changes since the last revision
\definecolor{ccadd}{rgb}{0,.6,0}
\newcommand{\addedCC}[1]{\textcolor{ccadd}{\ul{#1}}}
\newcommand{\removedCC}[1]{\textcolor{remclr}{\st{#1}}}
\newcommand{\changedCC}[2]{\removedCC{#1}\addedCC{#2}}
\newcommand{\remitemCC}[1]{\remitem{#1}}
\newcommand{\additemCC}[1]{\item\addedCC{#1}}
\newcommand{\changedCCC}[2]{\textcolor{ccadd}{\st{#1}}\addedCC{#2}}
\newcommand{\removedCCC}[1]{\textcolor{ccadd}{\st{#1}}}
\newcommand{\remitemCCC}[1]{\item\removedCCC{#1}}

%% Concepts changes for the next revision
\definecolor{zadd}{rgb}{0.8,0,0.8}
\newcommand{\addedZ}[1]{\textcolor{zadd}{\ul{#1}}}
\newcommand{\removedZ}[1]{\textcolor{remclr}{\st{#1}}}
\newcommand{\changedCZ}[2]{\textcolor{addclr}{\st{#1}}\addedZ{#2}}


%%--------------------------------------------------
%% Sectioning macros.  
% Each section has a depth, an automatically generated section 
% number, a name, and a short tag.  The depth is an integer in 
% the range [0,5].  (If it proves necessary, it wouldn't take much
% programming to raise the limit from 5 to something larger.)


% The basic sectioning command.  Example:
%    \Sec1[intro.scope]{Scope}
% defines a first-level section whose name is "Scope" and whose short
% tag is intro.scope.  The square brackets are mandatory.
\def\Sec#1[#2]#3{{%
\ifcase#1\let\s=\chapter
      \or\let\s=\section
      \or\let\s=\subsection
      \or\let\s=\subsubsection
      \or\let\s=\paragraph
      \or\let\s=\subparagraph
      \fi%
\s[#3]{#3\hfill[#2]}\relax\label{#2}}}

% A convenience feature (mostly for the convenience of the Project
% Editor, to make it easy to move around large blocks of text):
% the \rSec macro is just like the \Sec macro, except that depths 
% relative to a global variable, SectionDepthBase.  So, for example,
% if SectionDepthBase is 1,
%   \rSec1[temp.arg.type]{Template type arguments}
% is equivalent to
%   \Sec2[temp.arg.type]{Template type arguments}

\newcounter{SectionDepthBase}
\newcounter{scratch}

\def\rSec#1[#2]#3{{%
\setcounter{scratch}{#1}
\addtocounter{scratch}{\value{SectionDepthBase}}
\Sec{\arabic{scratch}}[#2]{#3}}}

% Change the way section headings are formatted.
\renewcommand{\chaptername}{}
\renewcommand{\appendixname}{Annex}

\makeatletter
\def\@makechapterhead#1{%
  \hrule\vspace*{1.5\p@}\hrule
  \vspace*{16\p@}%
  {\parindent \z@ \raggedright \normalfont
    \ifnum \c@secnumdepth >\m@ne
        \huge\bfseries \@chapapp\space \thechapter\space\space\space\space
    \fi
    \interlinepenalty\@M
    \huge \bfseries #1\par\nobreak
  \vspace*{16\p@}%
  \hrule\vspace*{1.5\p@}\hrule
  \vspace*{48\p@}
  }}

\renewcommand\section{\@startsection{section}{1}{0pt}%
                                   {-3.5ex plus -1ex minus -.2ex}%
                                   {.3ex plus .2ex}%
                                   {\normalfont\normalsize\bfseries}}
\renewcommand\section{\@startsection{section}{1}{0pt}%
                                   {2.5ex}% plus 1ex minus .2ex}%
                                   {.3ex}% plus .1ex minus .2 ex}%
                                   {\normalfont\normalsize\bfseries}}

\renewcommand\subsection{\@startsection{subsection}{2}{0pt}%
                                     {-3.25ex plus -1ex minus -.2ex}%
                                     {.3ex plus .2ex}%
                                     {\normalfont\normalsize\bfseries}}

\renewcommand\subsubsection{\@startsection{subsubsection}{3}{0pt}%
                                     {-3.25ex plus -1ex minus -.2ex}%
                                     {.3ex plus .2ex}%
                                     {\normalfont\normalsize\bfseries}}

\renewcommand\paragraph{\@startsection{paragraph}{4}{0pt}%
                                     {-3.25ex plus -1ex minus -.2ex}%
                                     {.3ex \@plus .2ex}%
                                     {\normalfont\normalsize\bfseries}}

\renewcommand\subparagraph{\@startsection{subparagraph}{5}{0pt}%
                                     {-3.25ex plus -1ex minus -.2ex}%
                                     {.3ex plus .2ex}%
                                     {\normalfont\normalsize\bfseries}}
\@removefromreset{footnote}{chapter}
\@removefromreset{table}{chapter}
\@removefromreset{figure}{chapter}
\makeatother

%%--------------------------------------------------
% Heading style for Annexes
\newcommand{\Annex}[3]{\chapter[#2]{\\(#3)\\#2\hfill[#1]}\relax\label{#1}}
\newcommand{\infannex}[2]{\Annex{#1}{#2}{informative}}
\newcommand{\normannex}[2]{\Annex{#1}{#2}{normative}}

\newcommand{\synopsis}[1]{\textbf{#1}}

%%--------------------------------------------------
% General code style
\newcommand{\CodeStyle}{\ttfamily}
\newcommand{\CodeStylex}[1]{\texttt{#1}}

% Code and definitions embedded in text.
\newcommand{\tcode}[1]{\CodeStylex{#1}}
\newcommand{\techterm}[1]{\textit{#1}}

%%--------------------------------------------------
%% allow line break if needed for justification
\newcommand{\brk}{\discretionary{}{}{}}
%  especially for scope qualifier
\newcommand{\colcol}{\brk::\brk}

%%--------------------------------------------------
%% Macros for funky text
%%!\newcommand{\Rplus}{\protect\nolinebreak\hspace{-.07em}\protect\raisebox{.25ex}{\small\textbf{+}}}
\newcommand{\Rplus}{+}
\newcommand{\Cpp}{C\Rplus\Rplus}
\newcommand{\opt}{$_\mathit{opt}$}
\newcommand{\shl}{<{<}}
\newcommand{\shr}{>{>}}
\newcommand{\dcr}{-{-}}
\newcommand{\bigohm}[1]{\mathscr{O}(#1)}
\newcommand{\bigoh}[1]{$\bigohm{#1}$}
\renewcommand{\tilde}{{\smaller$\sim$}}		% extra level of braces is necessary

%%--------------------------------------------------
%% States and operators

\newcommand{\state}[2]{\tcode{#1}\ensuremath{_{#2}}}
\newcommand{\bitand}{\ensuremath{\, \mathsf{bitand} \,}}
\newcommand{\bitor}{\ensuremath{\, \mathsf{bitor} \,}}
\newcommand{\xor}{\ensuremath{\, \mathsf{xor} \,}}
\newcommand{\rightshift}{\ensuremath{\, \mathsf{rshift} \,}}
\newcommand{\leftshift}{\ensuremath{\, \mathsf{lshift} \,}}

%% Notes and examples
\newcommand{\EnterBlock}[1]{[\,\textit{#1:}}
\newcommand{\ExitBlock}[1]{\textit{\ ---\,end #1}\,]}
\newcommand{\enternote}{\EnterBlock{Note}}
\newcommand{\exitnote}{\ExitBlock{note}}
\newcommand{\enterexample}{\EnterBlock{Example}}
\newcommand{\exitexample}{\ExitBlock{example}}

%% Library function descriptions
\newcommand{\Fundescx}[1]{\textit{#1}}
\newcommand{\Fundesc}[1]{\Fundescx{#1:}}
\newcommand{\required}{\Fundesc{Required behavior}}
\newcommand{\requires}{\Fundesc{Requires}}
\newcommand{\effects}{\Fundesc{Effects}}
\newcommand{\postconditions}{\Fundesc{Postconditions}}
\newcommand{\postcondition}{\Fundesc{Postcondition}}
\newcommand{\preconditions}{\Fundesc{Preconditions}}
\newcommand{\precondition}{\Fundesc{Precondition}}
\newcommand{\returns}{\Fundesc{Returns}}
\newcommand{\throws}{\Fundesc{Throws}}
\newcommand{\default}{\Fundesc{Default behavior}}
\newcommand{\complexity}{\Fundesc{Complexity}}
\newcommand{\note}{\Fundesc{Remark}}
\newcommand{\notes}{\Fundesc{Remarks}}
\newcommand{\implimits}{\Fundesc{Implementation limits}}
\newcommand{\replaceable}{\Fundesc{Replaceable}}
\newcommand{\exceptionsafety}{\Fundesc{Exception safety}}
\newcommand{\returntype}{\Fundesc{Return type}}

%% Cross reference
\newcommand{\xref}{\textsc{See also:}}

%% NTBS, etc.
\newcommand{\NTS}[1]{\textsc{#1}}
\newcommand{\ntbs}{\NTS{ntbs}}
\newcommand{\ntmbs}{\NTS{ntmbs}}
\newcommand{\ntwcs}{\NTS{ntwcs}}

%% Function argument
\newcommand{\farg}[1]{\texttt{\textit{#1}}}

%% Code annotations
\newcommand{\EXPO}[1]{\textbf{#1}}
\newcommand{\expos}{\EXPO{exposition only}}
\newcommand{\exposr}{\hfill\expos}
\newcommand{\exposrc}{\hfill// \expos}
\newcommand{\impdef}{\EXPO{implementation-defined}}
\newcommand{\notdef}{\EXPO{not defined}}

%% Double underscore
\newcommand{\unun}{\_\,\_}
\newcommand{\xname}[1]{\unun\,#1}
\newcommand{\mname}[1]{\tcode{\unun\,#1\,\unun}}

%% Ranges
\newcommand{\Range}[4]{\tcode{#1\brk{}#3,\brk{}#4\brk{}#2}}
\newcommand{\crange}[2]{\Range{[}{]}{#1}{#2}}
\newcommand{\orange}[2]{\Range{(}{)}{#1}{#2}}
\newcommand{\range}[2]{\Range{[}{)}{#1}{#2}}

%% Change descriptions
\newcommand{\diffdef}[1]{\hfill\break\textbf{#1:}}
\newcommand{\change}{\diffdef{Change}}
\newcommand{\rationale}{\diffdef{Rationale}}
\newcommand{\effect}{\diffdef{Effect on original feature}}
\newcommand{\difficulty}{\diffdef{Difficulty of converting}}
\newcommand{\howwide}{\diffdef{How widely used}}

%% Miscellaneous
\newcommand{\uniquens}{\textrm{\textit{\textbf{unique}}}}
\newcommand{\stage}[1]{\item{\textbf{Stage #1:}}}

%%--------------------------------------------------
%% Adjust markers
\renewcommand{\thetable}{\arabic{table}}
\renewcommand{\thefigure}{\arabic{figure}}
\renewcommand{\thefootnote}{\arabic{footnote})}

%% Change list item markers from box to dash
\renewcommand{\labelitemi}{---}
\renewcommand{\labelitemii}{---}
\renewcommand{\labelitemiii}{---}
\renewcommand{\labelitemiv}{---}

%%--------------------------------------------------
%% Environments for code listings.

% We use the 'listings' package, with some small customizations.  The
% most interesting customization: all TeX commands are available
% within comments.  Comments are set in italics, keywords and strings
% don't get special treatment.

\lstset{language=C++,
        basicstyle=\CodeStyle\small,
        keywordstyle=,
        stringstyle=,
        xleftmargin=1em,
        showstringspaces=false,
        commentstyle=\itshape\rmfamily,
        columns=flexible,
        keepspaces=true,
        texcl=true}

% Our usual abbreviation for 'listings'.  Comments are in 
% italics.  Arbitrary TeX commands can be used if they're 
% surrounded by @ signs.
\lstnewenvironment{codeblock}
{
 \lstset{escapechar=@}
 \renewcommand{\tcode}[1]{\textup{\CodeStyle##1}}
 \renewcommand{\techterm}[1]{\textit{##1}}
}
{
}

% Permit use of '@' inside codeblock blocks (don't ask)
\makeatletter
\newcommand{\atsign}{@}
\makeatother

%%--------------------------------------------------
%% Paragraph numbering
\newcounter{Paras}
\makeatletter
\@addtoreset{Paras}{chapter}
\@addtoreset{Paras}{section}
\@addtoreset{Paras}{subsection}
\@addtoreset{Paras}{subsubsection}
\@addtoreset{Paras}{paragraph}
\@addtoreset{Paras}{subparagraph}
\def\pnum{\addtocounter{Paras}{1}\noindent\llap{{\footnotesize\arabic{Paras}}\hspace{\@totalleftmargin}\quad}}
\makeatother

% For compatibility only.  We no longer need this environment.
\newenvironment{paras}{}{}

%%--------------------------------------------------
%% Indented text
\newenvironment{indented}
{\list{}{}\item\relax}
{\endlist}

%%--------------------------------------------------
%% Library item descriptions
\lstnewenvironment{itemdecl}
{
 \lstset{escapechar=@,
 xleftmargin=0em,
 aboveskip=2ex,
 belowskip=0ex	% leave this alone: it keeps these things out of the
				% footnote area
 }
}
{
}

\newenvironment{itemdescr}
{
 \begin{indented}}
{
 \end{indented}
}


%%--------------------------------------------------
%% Bnf environments
\newlength{\BnfIndent}
\setlength{\BnfIndent}{\leftmargini}
\newlength{\BnfInc}
\setlength{\BnfInc}{\BnfIndent}
\newlength{\BnfRest}
\setlength{\BnfRest}{2\BnfIndent}
\newcommand{\BnfNontermshape}{\rmfamily\itshape\small}
\newcommand{\BnfTermshape}{\ttfamily\upshape\small}
\newcommand{\nonterminal}[1]{{\BnfNontermshape #1}}

\newenvironment{bnfbase}
 {
 \newcommand{\terminal}[1]{{\BnfTermshape ##1}}
 \newcommand{\descr}[1]{\normalfont{##1}}
 \newcommand{\bnfindentfirst}{\BnfIndent}
 \newcommand{\bnfindentinc}{\BnfInc}
 \newcommand{\bnfindentrest}{\BnfRest}
 \begin{minipage}{.9\hsize}
 \newcommand{\br}{\hfill\\}
 }
 {
 \end{minipage}
 }

\newenvironment{BnfTabBase}[1]
{
 \begin{bnfbase}
 #1
 \begin{indented}
 \begin{tabbing}
 \hspace*{\bnfindentfirst}\=\hspace{\bnfindentinc}\=\hspace{.6in}\=\hspace{.6in}\=\hspace{.6in}\=\hspace{.6in}\=\hspace{.6in}\=\hspace{.6in}\=\hspace{.6in}\=\hspace{.6in}\=\hspace{.6in}\=\hspace{.6in}\=\kill%
}
{
 \end{tabbing}
 \end{indented}
 \end{bnfbase}
}

\newenvironment{bnfkeywordtab}
{
 \begin{BnfTabBase}{\BnfTermshape}
}
{
 \end{BnfTabBase}
}

\newenvironment{bnftab}
{
 \begin{BnfTabBase}{\BnfNontermshape}
}
{
 \end{BnfTabBase}
}

\newenvironment{simplebnf}
{
 \begin{bnfbase}
 \BnfNontermshape
 \begin{indented}
}
{
 \end{indented}
 \end{bnfbase}
}

\newenvironment{bnf}
{
 \begin{bnfbase}
 \list{}
	{
	\setlength{\leftmargin}{\bnfindentrest}
	\setlength{\listparindent}{-\bnfindentinc}
	\setlength{\itemindent}{\listparindent}
	}
 \BnfNontermshape
 \item\relax
}
{
 \endlist
 \end{bnfbase}
}

% non-copied versions of bnf environments
\newenvironment{ncbnftab}
{
 \begin{bnftab}
}
{
 \end{bnftab}
}

\newenvironment{ncsimplebnf}
{
 \begin{simplebnf}
}
{
 \end{simplebnf}
}

\newenvironment{ncbnf}
{
 \begin{bnf}
}
{
 \end{bnf}
}

%%--------------------------------------------------
%% Drawing environment
%
% usage: \begin{drawing}{UNITLENGTH}{WIDTH}{HEIGHT}{CAPTION}
\newenvironment{drawing}[4]
{
\begin{figure}[h]
\setlength{\unitlength}{#1}
\begin{center}
\begin{picture}(#2,#3)\thicklines
}
{
\end{picture}
\end{center}
%\caption{Directed acyclic graph}
\end{figure}
}

%%--------------------------------------------------
%% Table environments

% Base definitions for tables
\newenvironment{TableBase}
{
 \renewcommand{\tcode}[1]{{\CodeStyle##1}}
 \newcommand{\topline}{\hline}
 \newcommand{\capsep}{\hline\hline}
 \newcommand{\rowsep}{\hline}
 \newcommand{\bottomline}{\hline}

%% vertical alignment
 \newcommand{\rb}[1]{\raisebox{1.5ex}[0pt]{##1}}	% move argument up half a row

%% header helpers
 \newcommand{\hdstyle}[1]{\textbf{##1}}				% set header style
 \newcommand{\Head}[3]{\multicolumn{##1}{##2}{\hdstyle{##3}}}	% add title spanning multiple columns
 \newcommand{\lhdrx}[2]{\Head{##1}{|c}{##2}}		% set header for left column spanning #1 columns
 \newcommand{\chdrx}[2]{\Head{##1}{c}{##2}}			% set header for center column spanning #1 columns
 \newcommand{\rhdrx}[2]{\Head{##1}{c|}{##2}}		% set header for right column spanning #1 columns
 \newcommand{\ohdrx}[2]{\Head{##1}{|c|}{##2}}		% set header for only column spanning #1 columns
 \newcommand{\lhdr}[1]{\lhdrx{1}{##1}}				% set header for single left column
 \newcommand{\chdr}[1]{\chdrx{1}{##1}}				% set header for single center column
 \newcommand{\rhdr}[1]{\rhdrx{1}{##1}}				% set header for single right column
 \newcommand{\ohdr}[1]{\ohdrx{1}{##1}}
 \newcommand{\br}{\hfill\break}						% force newline within table entry

%% column styles
 \newcolumntype{x}[1]{>{\raggedright\let\\=\tabularnewline}p{##1}}	% word-wrapped ragged-right
 																	% column, width specified by #1
 \newcolumntype{m}[1]{>{\CodeStyle}l{##1}}							% variable width column, all entries in CodeStyle
}
{
}

% General Usage: TITLE is the title of the table, XREF is the
% cross-reference for the table. LAYOUT is a sequence of column
% type specifiers (e.g. cp{1.0}c), without '|' for the left edge
% or right edge.

% usage: \begin{floattablebase}{TITLE}{XREF}{COLUMNS}{PLACEMENT}
% produces floating table, location determined within limits
% by LaTeX.
\newenvironment{floattablebase}[4]
{
 \begin{TableBase}
 \begin{table}[#4]
 \caption{\label{#2}#1}
 \begin{center}
 \begin{tabular}{|#3|}
}
{
 \bottomline
 \end{tabular}
 \end{center}
 \end{table}
 \end{TableBase}
}

% usage: \begin{floattable}{TITLE}{XREF}{COLUMNS}
% produces floating table, location determined within limits
% by LaTeX.
\newenvironment{floattable}[3]
{
 \begin{floattablebase}{#1}{#2}{#3}{htbp}
}
{
 \end{floattablebase}
}

% usage: \begin{tokentable}{TITLE}{XREF}{HDR1}{HDR2}
% produces six-column table used for lists of replacement tokens;
% the columns are in pairs -- left-hand column has header HDR1,
% right hand column has header HDR2; pairs of columns are separated
% by vertical lines. Used in "trigraph sequences" table in standard.
\newenvironment{tokentable}[4]
{
 \begin{floattablebase}{#1}{#2}{cc|cc|cc}{htbp}
 \topline
 \textit{#3}   &   \textit{#4}    &
 \textit{#3}   &   \textit{#4}    &
 \textit{#3}   &   \textit{#4}    \\ \capsep
}
{
 \end{floattablebase}
}

% usage: \begin{libsumtabase}{TITLE}{XREF}{HDR1}{HDR2}
% produces two-column table with column headers HDR1 and HDR2.
% Used in "Library Categories" table in standard, and used as
% base for other library summary tables.
\newenvironment{libsumtabbase}[4]
{
 \begin{floattable}{#1}{#2}{ll}
 \topline
 \lhdr{#3}	&	\hdstyle{#4}	\\ \capsep
}
{
 \end{floattable}
}

% usage: \begin{libsumtab}{TITLE}{XREF}
% produces two-column table with column headers "Subclause" and "Header(s)".
% Used in "C++ Headers for Freestanding Implementations" table in standard.
\newenvironment{libsumtab}[2]
{
 \begin{libsumtabbase}{#1}{#2}{Subclause}{Header(s)}
}
{
 \end{libsumtabbase}
}

% usage: \begin{LibSynTab}{CAPTION}{TITLE}{XREF}{COUNT}{LAYOUT}
% produces table with COUNT columns. Used as base for
% C library description tables
\newcounter{LibSynTabCols}
\newcounter{LibSynTabWd}
\newenvironment{LibSynTabBase}[5]
{
 \setcounter{LibSynTabCols}{#4}
 \setcounter{LibSynTabWd}{#4}
 \addtocounter{LibSynTabWd}{-1}
 \newcommand{\centry}[1]{\textbf{##1}:}
 \newcommand{\macro}{\centry{Macro}}
 \newcommand{\macros}{\centry{Macros}}
 \newcommand{\function}{\centry{Function}}
 \newcommand{\functions}{\centry{Functions}}
 \newcommand{\templates}{\centry{Templates}}
 \newcommand{\type}{\centry{Type}}
 \newcommand{\types}{\centry{Types}}
 \newcommand{\values}{\centry{Values}}
 \newcommand{\struct}{\centry{Struct}}
 \newcommand{\cspan}[1]{\multicolumn{\value{LibSynTabCols}}{|l|}{##1}}
 \begin{floattable}{#1 \tcode{<#2>}\ synopsis}{#3}
 {#5}
 \topline
 \lhdr{Type}	&	\rhdrx{\value{LibSynTabWd}}{Name(s)}	\\ \capsep
}
{
 \end{floattable}
}

% usage: \begin{LibSynTab}{TITLE}{XREF}{COUNT}{LAYOUT}
% produces table with COUNT columns. Used as base for description tables
% for C library
\newenvironment{LibSynTab}[4]
{
 \begin{LibSynTabBase}{Header}{#1}{#2}{#3}{#4}
}
{
 \end{LibSynTabBase}
}

% usage: \begin{LibSynTabAdd}{TITLE}{XREF}{COUNT}{LAYOUT}
% produces table with COUNT columns. Used as base for description tables
% for additions to C library
\newenvironment{LibSynTabAdd}[4]
{
 \begin{LibSynTabBase}{Additions to header}{#1}{#2}{#3}{#4}
}
{
 \end{LibSynTabBase}
}

% usage: \begin{libsyntabN}{TITLE}{XREF}
%        \begin{libsyntabaddN}{TITLE}{XREF}
% produces a table with N columns for C library description tables
\newenvironment{libsyntab2}[2]
{
 \begin{LibSynTab}{#1}{#2}{2}{ll}
}
{
 \end{LibSynTab}
}

\newenvironment{libsyntab3}[2]
{
 \begin{LibSynTab}{#1}{#2}{3}{lll}
}
{
 \end{LibSynTab}
}

\newenvironment{libsyntab4}[2]
{
 \begin{LibSynTab}{#1}{#2}{4}{llll}
}
{
 \end{LibSynTab}
}

\newenvironment{libsyntab5}[2]
{
 \begin{LibSynTab}{#1}{#2}{5}{lllll}
}
{
 \end{LibSynTab}
}

\newenvironment{libsyntab6}[2]
{
 \begin{LibSynTab}{#1}{#2}{6}{llllll}
}
{
 \end{LibSynTab}
}

\newenvironment{libsyntabadd2}[2]
{
 \begin{LibSynTabAdd}{#1}{#2}{2}{ll}
}
{
 \end{LibSynTabAdd}
}

\newenvironment{libsyntabadd3}[2]
{
 \begin{LibSynTabAdd}{#1}{#2}{3}{lll}
}
{
 \end{LibSynTabAdd}
}

\newenvironment{libsyntabadd4}[2]
{
 \begin{LibSynTabAdd}{#1}{#2}{4}{llll}
}
{
 \end{LibSynTabAdd}
}

\newenvironment{libsyntabadd5}[2]
{
 \begin{LibSynTabAdd}{#1}{#2}{5}{lllll}
}
{
 \end{LibSynTabAdd}
}

\newenvironment{libsyntabadd6}[2]
{
 \begin{LibSynTabAdd}{#1}{#2}{6}{llllll}
}
{
 \end{LibSynTabAdd}
}

% usage: \begin{LongTable}{TITLE}{XREF}{LAYOUT}
% produces table that handles page breaks sensibly.
\newenvironment{LongTable}[3]
{
 \begin{TableBase}
 \begin{longtable}
 {|#3|}\caption{#1}\label{#2}
}
{
 \bottomline
 \end{longtable}
 \end{TableBase}
}

% usage: \begin{twocol}{TITLE}{XREF}
% produces a two-column breakable table. Used in
% "simple-type-specifiers and the types they specify" in the standard.
\newenvironment{twocol}[2]
{
 \begin{LongTable}
 {#1}{#2}
 {ll}
}
{
 \end{LongTable}
}

% usage: \begin{libreqtabN}{TITLE}{XREF}
% produces an N-column brekable table. Used in
% most of the library clauses for requirements tables.
% Example at "Position type requirements" in the standard.

\newenvironment{libreqtab1}[2]
{
 \begin{LongTable}
 {#1}{#2}
 {x{.55\hsize}}
}
{
 \end{LongTable}
}

\newenvironment{libreqtab2}[2]
{
 \begin{LongTable}
 {#1}{#2}
 {lx{.55\hsize}}
}
{
 \end{LongTable}
}

\newenvironment{libreqtab2a}[2]
{
 \begin{LongTable}
 {#1}{#2}
 {x{.30\hsize}x{.68\hsize}}
}
{
 \end{LongTable}
}

\newenvironment{libreqtab3}[2]
{
 \begin{LongTable}
 {#1}{#2}
 {x{.28\hsize}x{.18\hsize}x{.43\hsize}}
}
{
 \end{LongTable}
}

\newenvironment{libreqtab3a}[2]
{
 \begin{LongTable}
 {#1}{#2}
 {x{.28\hsize}x{.33\hsize}x{.29\hsize}}
}
{
 \end{LongTable}
}

\newenvironment{libreqtab3b}[2]
{
 \begin{LongTable}
 {#1}{#2}
 {x{.40\hsize}x{.25\hsize}x{.25\hsize}}
}
{
 \end{LongTable}
}

\newenvironment{libreqtab3c}[2]
{
 \begin{LongTable}
 {#1}{#2}
 {x{.30\hsize}x{.25\hsize}x{.35\hsize}}
}
{
 \end{LongTable}
}

\newenvironment{libreqtab3d}[2]
{
 \begin{LongTable}
 {#1}{#2}
 {x{.32\hsize}x{.27\hsize}x{.36\hsize}}
}
{
 \end{LongTable}
}

\newenvironment{libreqtab3e}[2]
{
 \begin{LongTable}
 {#1}{#2}
 {x{.38\hsize}x{.27\hsize}x{.25\hsize}}
}
{
 \end{LongTable}
}

\newenvironment{libreqtab3f}[2]
{
 \begin{LongTable}
 {#1}{#2}
 {x{.40\hsize}x{.22\hsize}x{.31\hsize}}
}
{
 \end{LongTable}
}

\newenvironment{libreqtab4}[2]
{
 \begin{LongTable}
 {#1}{#2}
}
{
 \end{LongTable}
}

\newenvironment{libreqtab4a}[2]
{
 \begin{LongTable}
 {#1}{#2}
 {x{.14\hsize}x{.30\hsize}x{.30\hsize}x{.14\hsize}}
}
{
 \end{LongTable}
}

\newenvironment{libreqtab4b}[2]
{
 \begin{LongTable}
 {#1}{#2}
 {x{.13\hsize}x{.15\hsize}x{.29\hsize}x{.27\hsize}}
}
{
 \end{LongTable}
}

\newenvironment{libreqtab4c}[2]
{
 \begin{LongTable}
 {#1}{#2}
 {x{.16\hsize}x{.21\hsize}x{.21\hsize}x{.30\hsize}}
}
{
 \end{LongTable}
}

\newenvironment{libreqtab4d}[2]
{
 \begin{LongTable}
 {#1}{#2}
 {x{.22\hsize}x{.22\hsize}x{.30\hsize}x{.15\hsize}}
}
{
 \end{LongTable}
}

\newenvironment{libreqtab5}[2]
{
 \begin{LongTable}
 {#1}{#2}
 {x{.14\hsize}x{.14\hsize}x{.20\hsize}x{.20\hsize}x{.14\hsize}}
}
{
 \end{LongTable}
}

% usage: \begin{libtab2}{TITLE}{XREF}{LAYOUT}{HDR1}{HDR2}
% produces two-column table with column headers HDR1 and HDR2.
% Used in "seekoff positioning" in the standard.
\newenvironment{libtab2}[5]
{
 \begin{floattable}
 {#1}{#2}{#3}
 \topline
 \lhdr{#4}	&	\rhdr{#5}	\\ \capsep
}
{
 \end{floattable}
}

% usage: \begin{longlibtab2}{TITLE}{XREF}{LAYOUT}{HDR1}{HDR2}
% produces two-column table with column headers HDR1 and HDR2.
\newenvironment{longlibtab2}[5]
{
 \begin{LongTable}{#1}{#2}{#3}
 \\ \topline
 \lhdr{#4}	&	\rhdr{#5}	\\ \capsep
}
{
  \end{LongTable}
}

% usage: \begin{LibEffTab}{TITLE}{XREF}{HDR2}{WD2}
% produces a two-column table with left column header "Element"
% and right column header HDR2, right column word-wrapped with
% width specified by WD2.
\newenvironment{LibEffTab}[4]
{
 \begin{libtab2}{#1}{#2}{lp{#4}}{Element}{#3}
}
{
 \end{libtab2}
}

% Same as LibEffTab except that it uses a long table.
\newenvironment{longLibEffTab}[4]
{
 \begin{longlibtab2}{#1}{#2}{lp{#4}}{Element}{#3}
}
{
 \end{longlibtab2}
}

% usage: \begin{libefftab}{TITLE}{XREF}
% produces a two-column effects table with right column
% header "Effect(s) if set", width 4.5 in. Used in "fmtflags effects"
% table in standard.
\newenvironment{libefftab}[2]
{
 \begin{LibEffTab}{#1}{#2}{Effect(s) if set}{4.5in}
}
{
 \end{LibEffTab}
}

% Same as libefftab except that it uses a long table.
\newenvironment{longlibefftab}[2]
{
 \begin{longLibEffTab}{#1}{#2}{Effect(s) if set}{4.5in}
}
{
 \end{longLibEffTab}
}

% usage: \begin{libefftabmean}{TITLE}{XREF}
% produces a two-column effects table with right column
% header "Meaning", width 4.5 in. Used in "seekdir effects"
% table in standard.
\newenvironment{libefftabmean}[2]
{
 \begin{LibEffTab}{#1}{#2}{Meaning}{4.5in}
}
{
 \end{LibEffTab}
}

% Same as libefftabmean except that it uses a long table.
\newenvironment{longlibefftabmean}[2]
{
 \begin{longLibEffTab}{#1}{#2}{Meaning}{4.5in}
}
{
 \end{longLibEffTab}
}

% usage: \begin{libefftabvalue}{TITLE}{XREF}
% produces a two-column effects table with right column
% header "Value", width 3 in. Used in "basic_ios::init() effects"
% table in standard.
\newenvironment{libefftabvalue}[2]
{
 \begin{LibEffTab}{#1}{#2}{Value}{3in}
}
{
 \end{LibEffTab}
}

% Same as libefftabvalue except that it uses a long table and a
% slightly wider column.
\newenvironment{longlibefftabvalue}[2]
{
 \begin{longLibEffTab}{#1}{#2}{Value}{3.5in}
}
{
 \end{longLibEffTab}
}

% usage: \begin{liberrtab}{TITLE}{XREF} produces a two-column table
% with left column header ``Value'' and right header "Error
% condition", width 4.5 in. Used in regex clause in the TR.

\newenvironment{liberrtab}[2]
{
 \begin{libtab2}{#1}{#2}{lp{4.5in}}{Value}{Error condition}
}
{
 \end{libtab2}
}

% Like liberrtab except that it uses a long table.
\newenvironment{longliberrtab}[2]
{
 \begin{longlibtab2}{#1}{#2}{lp{4.5in}}{Value}{Error condition}
}
{
 \end{longlibtab2}
}

% enumerate with lowercase letters
\newenvironment{enumeratea}
{
 \renewcommand{\labelenumi}{\alph{enumi})}
 \begin{enumerate}
}
{
 \end{enumerate}
}

% enumerate with arabic numbers
\newenvironment{enumeraten}
{
 \renewcommand{\labelenumi}{\arabic{enumi})}
 \begin{enumerate}
}
{
 \end{enumerate}
}

%%--------------------------------------------------
%% Definitions section
% usage: \definition{name}{xref}
%\newcommand{\definition}[2]{\rSec2[#2]{#1}}
% for ISO format, use:
\newcommand{\definition}[2]
 {\hfill\vspace{.25ex plus .5ex minus .2ex}\\
 \addtocounter{subsection}{1}%
 \textbf{\thesubsection\hfill\relax[#2]}\\
 \textbf{#1}\label{#2}\\
 }


%%--------------------------------------------------
%% Set section numbering limit, toc limit
\setcounter{secnumdepth}{5}
\setcounter{tocdepth}{1}

%%--------------------------------------------------
%% Parameters that govern document appearance
\setlength{\oddsidemargin}{0pt}
\setlength{\evensidemargin}{0pt}
\setlength{\textwidth}{6.6in}

%%--------------------------------------------------
%% Handle special hyphenation rules
\hyphenation{tem-plate ex-am-ple in-put-it-er-a-tor}

% Do not put blank pages after chapters that end on odd-numbered pages.
\def\cleardoublepage{\clearpage\if@twoside%
  \ifodd\c@page\else\hbox{}\thispagestyle{empty}\newpage%
  \if@twocolumn\hbox{}\newpage\fi\fi\fi}

\begin{document}
\raggedbottom

\begin{titlepage}
\begin{center}
\huge
Concepts for the C++0x Standard Library: Algorithms \\
(Revision 2)

\vspace{0.5in}

\normalsize
Douglas Gregor and Andrew Lumsdaine \\
Open Systems Laboratory, Indiana University \\
\href{mailto:dgregor@osl.iu.edu}{dgregor@osl.iu.edu}, \href{mailto:lums@osl.iu.edu}{lums@osl.iu.edu}
\end{center}

\vspace{1in}
\par\noindent Document number: D2573=08-0083\vspace{-6pt}
\par\noindent Revises document number: N2084=06-0154\vspace{-6pt}
\par\noindent Date: \today\vspace{-6pt}
\par\noindent Project: Programming Language \Cpp{}, Library Working Group\vspace{-6pt}
\par\noindent Reply-to: Douglas Gregor $<$\href{mailto:dgregor@osl.iu.edu}{dgregor@osl.iu.edu}$>$\vspace{-6pt}

\section*{Introduction}
\libintrotext{Chapter 25}
\section*{Changes from N2084}
\begin{itemize}
\item Updated algorithm requirements to new concept syntax.
\item Updated algorithm requirements to use the new iterator and
  utility concepts, including the relaxed iterator requirements in
  N2323 and the move concepts of N2322.
\item Added new algorithms \tcode{move}, \tcode{move_backward},
  \tcode{is_sorted}, \tcode{is_sorted_until}, \tcode{is_heap},
  \tcode{is_heap_until}, \tcode{minmax} and \tcode{minmax_element},
  and the variadic \tcode{min}, \tcode{max}, and \tcode{minmax} overloads
  from N2551.
\end{itemize}

\end{titlepage}

%%--------------------------------------------------
%% Headers and footers
\pagestyle{fancy}
\fancyhead[LE,RO]{\textbf{\rightmark}}
\fancyhead[RE]{\textbf{\leftmark\hspace{1em}\thepage}}
\fancyhead[LO]{\textbf{\thepage\hspace{1em}\leftmark}}
\fancyfoot[C]{Draft}

\fancypagestyle{plain}{
\renewcommand{\headrulewidth}{0in}
\fancyhead[LE,RO]{}
\fancyhead[RE,LO]{}
\fancyfoot{}
}

\renewcommand{\sectionmark}[1]{\markright{\thesection\hspace{1em}#1}}
\renewcommand{\chaptermark}[1]{\markboth{#1}{}}

\color{black}

\setcounter{chapter}{24}
\rSec0[algorithms]{Algorithms library}

\begin{paras}

\pnum
This clause describes components that \Cpp\ programs may use to perform
algorithmic operations on containers (clause \ref{lib.containers}) and other sequences.

\pnum
The following subclauses describe components for
non-modifying sequence operation,
modifying sequence operations,
sorting and related operations,
and algorithms from the ISO C library,
as summarized in Table~\ref{tab:algorithms.lib.summary}.

\begin{libsumtab}{Algorithms library summary}{tab:algorithms.lib.summary}
\ref{alg.nonmodifying} Non-modifying sequence operations    &                                               \\
\ref{alg.modifying.operations} Mutating sequence operations &       \tcode{<algorithm>}     \\
\ref{alg.sorting} Sorting and related operations                    &                                               \\ \hline
\ref{alg.c.library} C library algorithms                                    &       \tcode{<cstdlib>}       \\ \hline
\end{libsumtab}

\synopsis{Header \tcode{<algorithm>}\ synopsis}
\index{algorithm@\tcode{<algorithm>}}%

\color{addclr}
\begin{codeblock}
namespace std {
  @\textcolor{black}{// \ref{alg.nonmodifying}, non-modifying sequence operations:}@
  template<InputIterator Iter, Callable<auto, Iter::reference> Function>
    Function for_each(Iter @\farg{first}@, Iter @\farg{last}@, Function @\farg{f}@);
  template<InputIterator Iter, class T>
    requires EqualityComparable<Iter::reference, T>
    Iter find(Iter @\farg{first}@, Iter @\farg{last}@, const T& @\farg{value}@);
  template<InputIterator Iter, Predicate<auto, Iter::reference> Pred>
    Iter find_if(Iter @\farg{first}@, Iter @\farg{last}@, Pred @\farg{pred}@);
  template<ForwardIterator Iter1, ForwardIterator Iter2>
    requires EqualityComparable<Iter1::reference, Iter2::reference>
    Iter1 find_end(Iter1 @\farg{first1}@, Iter1 @\farg{last1}@,
                   Iter2 @\farg{first2}@, Iter2 @\farg{last2}@);
  template<ForwardIterator Iter1, ForwardIterator Iter2, 
           Predicate<auto, Iter1::reference, Iter2::reference> Pred>
    Iter1 find_end(Iter1 @\farg{first1}@, Iter1 @\farg{last1}@,
                   Iter2 @\farg{first2}@, Iter2 @\farg{last2}@,
                   Pred @\farg{pred}@);

  template<InputIterator Iter1, ForwardIterator Iter2>
    requires EqualityComparable<Iter1::reference, Iter2::reference>
    Iter1 find_first_of(Iter1 @\farg{first1}@, Iter1 @\farg{last1}@,
                        Iter2 @\farg{first2}@, Iter2 @\farg{last2}@);
  template<InputIterator Iter1, ForwardIterator Iter2,
           Predicate<auto, Iter1::reference, Iter2::reference> Pred>
    Iter1 find_first_of(Iter1 @\farg{first1}@, Iter1 @\farg{last1}@,
                        Iter2 @\farg{first2}@, Iter2 @\farg{last2}@,
                        Pred @\farg{pred}@);

  template<ForwardIterator Iter>
    requires EqualityComparable<Iter::reference>
    Iter adjacent_find(Iter @\farg{first}@, Iter @\farg{last}@);
  template<ForwardIterator Iter, Predicate<auto, Iter::reference, Iter::reference> Pred>
    Iter adjacent_find(Iter @\farg{first}@, Iter @\farg{last}@, Pred @\farg{pred}@);

  template<InputIterator Iter, class T>
    requires EqualityComparable<Iter::reference, T>
    @\color{addclr}@Iter::difference_type count(Iter @\farg{first}@, Iter @\farg{last}@, const T& @\farg{value}@);
  template<InputIterator Iter, Predicate<auto, Iter::reference> Pred>
    Iter::difference_type count_if(Iter @\farg{first}@, Iter @\farg{last}@, Pred @\farg{pred}@);

  template<InputIterator Iter1, InputIterator Iter2>
    requires EqualityComparable<Iter1::reference, Iter2::reference>
    pair<Iter1, Iter2> mismatch(Iter1 @\farg{first1}@, Iter1 @\farg{last1}@,
                                Iter @\farg{first2}@);
  template<InputIterator Iter1, InputIterator Iter2,
           Predicate<auto, Iter1::reference, Iter2::reference> Pred>
    pair<Iter1, Iter2> mismatch(Iter1 @\farg{first1}@, Iter2 @\farg{last1}@,
                                Iter2 @\farg{first2}@, Pred @\farg{pred}@);

  template<InputIterator Iter1, InputIterator Iter2>
    requires EqualityComparable<Iter1::reference, Iter2::reference>
    bool equal(Iter1 @\farg{first1}@, Iter1 @\farg{last1}@,
               Iter2 @\farg{first2}@);
  template<InputIterator Iter1, InputIterator Iter2,
           Predicate<auto, Iter1::reference, Iter2::reference> Pred>
    bool equal(Iter1 @\farg{first1}@, Iter1 @\farg{last1}@,
               Iter2 @\farg{first2}@, Pred @\farg{pred}@);

  template<ForwardIterator Iter1, ForwardIterator Iter2>
    requires EqualityComparable<Iter1::reference, Iter2::reference>
    Iter1 search(Iter1 @\farg{first1}@, Iter1 @\farg{last1}@,
                 Iter2 @\farg{first2}@, Iter2 @\farg{last2}@);
  template<ForwardIterator Iter1, ForwardIterator Iter2,
           Predicate<auto, Iter1::reference, Iter2::reference> Pred>
    Iter1 search(Iter1 @\farg{first1}@, Iter1 @\farg{last1}@,
                 Iter2 @\farg{first2}@, Iter2 @\farg{last2}@,
                 Pred @\farg{pred}@);
  template<ForwardIterator Iter, class T>
    requires EqualityComparable<Iter::reference, T>
    Iter search_n(Iter @\farg{first}@, Iter @\farg{last}@, Iter::difference_type @\farg{count}@,
                  const T& @\farg{value}@);
  template<ForwardIterator Iter, class T,
           Predicate<auto, Iter::reference, T> Pred>
    Iter search_n(Iter @\farg{first}@, Iter @\farg{last}@, Iter::difference_type @\farg{count}@,
                  const T& @\farg{value}@, Pred @\farg{pred}@);

  @\textcolor{black}{// \ref{alg.modifying.operations}, modifying sequence operations:}@
  @\textcolor{black}{// \ref{alg.copy}, copy:}@
  template<InputIterator InIter, OutputIterator<auto, InIter::reference> OutIter>
    OutIter copy(InIter @\farg{first}@, InIter @\farg{last}@,
                 OutIter @\farg{result}@);
  template<BidirectionalIterator InIter, MutableBidirectionalIterator OutIter>
    requires CopyAssignable<OutIter::reference, InIter::reference>
    OutIter copy_backward(InIter @\farg{first}@, InIter @\farg{last}@,
                          OutIter @\farg{result}@);

  @\textcolor{black}{// \ref{alg.move}, move:}@
  template<InputIterator InIter, OutputIterator<auto, InIter::value_type&&> OutIter>
    OutIter move(InIter @\farg{first}@, InIter @\farg{last}@,
                 OutIter @\farg{result}@);
  template<BidirectionalIterator InIter, BidirectionalIterator OutIter>
    requires MoveAssignable<OutIter::reference, InIter::value_type>
    OutIter move_backward(InIter @\farg{first}@, InIter @\farg{last}@,
                          OutIter @\farg{result}@);

  @\textcolor{black}{// \ref{alg.swap}, swap:}@
  template<class T> 
    requires MoveAssignable<T> && MoveConstructible<T>
    void swap(T& @\farg{a}@, T& @\farg{b}@);
  template<MutableForwardIterator Iter1, MutableForwardIterator Iter2>
    requires SameType<Iter1::value_type, Iter2::value_type> && 
             SameType<Iter1::value_type&, Iter1::reference> && 
             SameType<Iter2::value_type&, Iter2::reference> && 
             Swappable<Iter1::value_type>
    Iter2 swap_ranges(Iter1 @\farg{first1}@, Iter1 @\farg{last1}@,
                      Iter2 @\farg{first2}@);
  template<MutableForwardIterator Iter1, MutableForwardIterator Iter2>
    requires SameType<Iter1::value_type, Iter2::value_type> && 
             SameType<Iter1::value_type&, Iter1::reference> && 
             SameType<Iter2::value_type&, Iter2::reference> && 
             Swappable<Iter1::value_type>
    void iter_swap(Iter1 @\farg{a}@, Iter2 @\farg{b}@);

  template<InputIterator InIter, class OutIter,
           Callable<auto, InIter::reference> Op>
    requires OutputIterator<OutIter, Op::result_type>
    OutIter transform(InIter @\farg{first}@, InIter @\farg{last}@,
                      OutIter @\farg{result}@, Op @\farg{op}@);
  template<InputIterator InIter1, InputIterator InIter2,
           class OutIter, Callable<auto, InIter1::reference, InIter2::reference> BinaryOp>
    requires OutputIterator<OutIter, BinaryOp::result_type>
    OutIter transform(InIter1 @\farg{first1}@, InIter1 @\farg{last1}@,
                      InIter2 @\farg{first2}@, OutIter @\farg{result}@,
                      BinaryOp @\farg{binary_op}@);

  template<MutableForwardIterator Iter, class T>
    requires EqualityComparable<Iter::reference, T> && CopyAssignable<Iter::reference, T>
    void replace(Iter @\farg{first}@, Iter @\farg{last}@,
                 const T& @\farg{old_value}@, const T& @\farg{new_value}@);
  @\textcolor{addclr}{template}@<MutableForwardIterator Iter, Predicate<auto, Iter::reference> Pred, class T>
    requires CopyAssignable<Iter::reference, T>
    void replace_if(Iter @\farg{first}@, Iter @\farg{last}@,
                    Pred @\farg{pred}@, const T& @\farg{new_value}@);
  template<InputIterator InIter, OutputIterator<auto, InIter::value_type> OutIter, class T>
    requires CopyAssignable<OutIter::reference, T> && EqualityComparable<InIter::reference, T>
    OutIter replace_copy(InIter @\farg{first}@, InIter @\farg{last}@,
                         OutIter @\farg{result}@,
                         const T& @\farg{old_value}@, const T& @\farg{new_value}@);
  template<InputIterator InIter, OutputIterator<auto, InIter::value_type> OutIter, 
           Predicate<auto, InIter::reference> Pred, class T>
    requires CopyAssignable<OutIter::reference, T>
    OutIter replace_copy_if(InIter @\farg{first}@, InIter @\farg{last}@,
                            OutIter @\farg{result}@,
                            Pred @\farg{pred}@, const T& @\farg{new_value}@);

  template<MutableForwardIterator Iter, class T>
    requires CopyAssignable<Iter::reference, T>
    void fill(Iter @\farg{first}@, Iter @\farg{last}@, const T& @\farg{value}@);
  template<class Iter, Integral Size, class T>
    requires OutputIterator<Iter, T>
    void fill_n(Iter @\farg{first}@, Size @\farg{n}@, const T& @\farg{value}@);

  template<MutableForwardIterator Iter, Callable Generator>
    requires CopyAssignable<Iter::reference, Generator::result_type>
    void generate(Iter @\farg{first}@, Iter @\farg{last}@,
                  Generator @\farg{gen}@);
  template<class Iter, Integral Size, Callable Generator>
    requires OutputIterator<Iter, Generator::result_type>
    void generate_n(Iter @\farg{first}@, Size @\farg{n}@, Generator @\farg{gen}@);

  template<ForwardIterator Iter, class T>
    requires MoveAssignable<Iter::reference> && EqualityComparable<Iter::reference, T>
    Iter remove(Iter @\farg{first}@, Iter @\farg{last}@,
                const T& @\farg{value}@);
  template<ForwardIterator Iter, Predicate<auto, Iter::reference> Pred>
    requires MoveAssignable<Iter::reference>
    Iter remove_if(Iter @\farg{first}@, Iter @\farg{last}@,
                   Pred @\farg{pred}@);
  template<InputIterator InIter, OutputIterator<auto, InIter::value_type> OutIter, class T>
    requires EqualityComparable<InIter::reference, T>
    OutIter remove_copy(InIter @\farg{first}@, InIter @\farg{last}@,
                        OutIter @\farg{result}@, const T& @\farg{value}@);
  template<InputIterator InIter, OutputIterator<auto, InIter::value_type> OutIter, 
           Predicate<auto, InIter::reference> Pred>
    OutIter remove_copy_if(InIter @\farg{first}@, InIter @\farg{last}@,
                           OutIter @\farg{result}@, Pred @\farg{pred}@);

  template<ForwardIterator Iter>
    requires MoveAssignable<Iter::reference> && EqualityComparable<Iter::reference>
    Iter unique(Iter @\farg{first}@, Iter @\farg{last}@);
  template<ForwardIterator Iter, Predicate<auto, Iter::reference, Iter::reference> Pred>
    requires MoveAssignable<Iter::reference>
    Iter unique(Iter @\farg{first}@, Iter @\farg{last}@,
                Pred @\farg{pred}@);
  template<InputIterator InIter, OutputIterator<auto, InIter::value_type> OutIter>
    requires EqualityComparable<InIter::value_type> && CopyAssignable<InIter::value_type> && 
             @\textcolor{addclr}{CopyConstructible}@<InIter::value_type> && !ForwardIterator<InIter> && 
             !MutableForwardIterator<OutIter>
    OutIter
      unique_copy(InIter @\farg{first}@, InIter @\farg{last}@,
                  OutIter @\farg{result}@);
  template<ForwardIterator InIter, OutputIterator<auto, InIter::value_type> OutIter>
    requires EqualityComparable<InIter::reference>
    OutIter unique_copy(InIter @\farg{first}@, InIter @\farg{last}@,
                        OutIter @\farg{result}@);
  template<InputIterator InIter, MutableForwardIterator OutIter>
    requires EqualityComparable<OutIter::reference, InIter::value_type> &&
             CopyAssignable<OutIter::reference, InIter::reference> &&
             !ForwardIterator<InIter>
    OutIter unique_copy(InIter @\farg{first}@, InIter @\farg{last}@,
                        OutIter @\farg{result}@);
  template<InputIterator InIter, OutputIterator<auto, InIter::value_type> OutIter,
           Predicate<auto, InIter::value_type, InIter::value_type> Pred>
    requires CopyAssignable<InIter::value_type> && CopyConstructible<InIter::value_type> &&
             !ForwardIterator<InIter> && !MutableForwardIterator<OutIter>
    OutIter unique_copy(InIter @\farg{first}@, InIter @\farg{last}@,
                        OutIter @\farg{result}@, Pred @\farg{pred}@);
  template<ForwardIterator InIter, OutputIterator<auto, InIter::value_type> OutIter,
           Predicate<auto, InIter::reference, InIter::reference> Pred>
    OutIter unique_copy(InIter @\farg{first}@, InIter @\farg{last}@,
                        OutIter @\farg{result}@);
  template<InputIterator InIter, MutableForwardIterator OutIter,
           Predicate<auto, OutIter::reference, InIter::reference> Pred>
    requires CopyAssignable<OutIter::reference, InIter::reference> &&
             !ForwardIterator<InIter>
    OutIter unique_copy(InIter @\farg{first}@, InIter @\farg{last}@,
                        OutIter @\farg{result}@, Pred @\farg{pred}@);

  template<MutableBidirectionalIterator Iter>
    requires SameType<Iter::value_type&, Iter::reference> && Swappable<Iter::value_type>
    void reverse(Iter @\farg{first}@, Iter @\farg{last}@);
  template<BidirectionalIterator InIter, OutputIterator<auto, InIter::value_type> OutIter>
    OutIter reverse_copy(InIter @\farg{first}@,
                         InIter @\farg{last}@, OutIter @\farg{result}@);

  template<ForwardIterator Iter>
    requires SameType<Iter::value_type&, Iter::reference> && Swappable<Iter::value_type>
    void rotate(Iter @\farg{first}@, Iter @\farg{middle}@,
                Iter @\farg{last}@);
  template<ForwardIterator InIter, OutputIterator<auto, InIter::value_type> OutIter>
    OutIter rotate_copy(InIter @\farg{first}@, InIter @\farg{middle}@,
                        InIter @\farg{last}@, OutIter @\farg{result}@);

  template<MutableRandomAccessIterator Iter>
    requires SameType<Iter::value_type&, Iter::reference> && Swappable<Iter::value_type>
    void random_shuffle(Iter @\farg{first}@,
                        Iter @\farg{last}@);
  template<MutableRandomAccessIterator Iter, Callable<auto, Iter::difference_type> Rand>
    requires SameType<Iter::value_type&, Iter::reference> && Swappable<Iter::value_type> && 
             Convertible<Rand::result_type, Iter::difference_type>
    void random_shuffle(Iter @\farg{first}@,
                        Iter @\farg{last}@,
                        Rand& @\farg{rand}@);
  template<class RandomAccessIterator, class UniformRandomNumberGenerator>
    void random_shuffle(RandomAccessIterator @\farg{first}@,
                        RandomAccessIterator @\farg{last}@, 
                        UniformRandomNumberGenerator& @\farg{rand}@);

  @\textcolor{black}{// \ref{alg.partitions}, partitions:}@
  template<BidirectionalIterator Iter, Predicate<auto, Iter::reference> Pred>
    @\color{addclr}@requires SameType<Iter::value_type&, Iter::reference> && Swappable<Iter::value_type>
    Iter partition(Iter @\farg{first}@, Iter @\farg{last}@, Pred @\farg{pred}@);
  template<MutableBidirectionalIterator Iter, Predicate<auto Iter::reference> Pred>
    requires SameType<Iter::value_type&, Iter::reference> && Swappable<Iter::value_type> && 
             MoveConstructible<Iter::value_type> && MoveAssignable<Iter::value_type>
    Iter stable_partition(Iter @\farg{first}@, Iter @\farg{last}@, Pred @\farg{pred}@);

  @\textcolor{black}{// \ref{alg.sorting}, sorting and related operations:}@
  @\textcolor{black}{// \ref{alg.sort}, sorting:}@
  template<MutableRandomAccessIterator Iter>
    requires SameType<Iter::value_type&, Iter::reference> && Swappable<Iter::value_type> &&
             MoveConstructible<Iter::value_type> && MoveAssignable<Iter::value_type> &&
             LessThanComparable<Iter::value_type> 
    void sort(Iter @\farg{first}@, Iter @\farg{last}@);
  template<MutableRandomAccessIterator Iter, 
           Predicate<auto, Iter::value_type, Iter::value_type> Compare>
    requires SameType<Iter::value_type&, Iter::reference> && Swappable<Iter::value_type> && 
             MoveConstructible<Iter::value_type> && MoveAssignable<Iter::value_type>
    void sort(Iter @\farg{first}@, Iter @\farg{last}@,
              Compare @\farg{comp}@);

  template<MutableRandomAccessIterator Iter>
    requires SameType<Iter::value_type&, Iter::reference> && Swappable<Iter::value_type> &&
             MoveConstructible<Iter::value_type> && MoveAssignable<Iter::value_type> &&
             LessThanComparable<Iter::value_type> 
    void stable_sort(Iter @\farg{first}@, Iter @\farg{last}@);
  template<MutableRandomAccessIterator Iter, 
           Predicate<auto, Iter::value_type, Iter::value_type> Compare>
    requires SameType<Iter::value_type&, Iter::reference> && Swappable<Iter::value_type> && 
             MoveConstructible<Iter::value_type> && MoveAssignable<Iter::value_type>
    void stable_sort(Iter @\farg{first}@, Iter @\farg{last}@,
                     Compare @\farg{comp}@);

  template<MutableRandomAccessIterator Iter>
    requires SameType<Iter::value_type&, Iter::reference> && Swappable<Iter::value_type> && 
             MoveConstructible<Iter::value_type> && MoveAssignable<Iter::value_type> @\addedCC{\&\&}@
             @\addedCC{LessThanComparable<Iter::value_type>}@
    void partial_sort(Iter @\farg{first}@,
                      Iter @\farg{middle}@,
                      Iter @\farg{last}@);
  template<MutableRandomAccessIterator Iter, 
           Predicate<auto, Iter::value_type, Iter::value_type> Compare>
    requires SameType<Iter::value_type&, Iter::reference> && Swappable<Iter::value_type> && 
             MoveConstructible<Iter::value_type> && MoveAssignable<Iter::value_type>
    void partial_sort(Iter @\farg{first}@,
                      Iter @\farg{middle}@,
                      Iter @\farg{last}@,
                      Compare @\farg{comp}@);
  template<InputIterator InIter, MutableRandomAccessIterator RAIter>
    requires CopyAssignable<RAIter::reference, InIter::value_type> &&
             SameType<RAIter::value_type&, RAIter::reference> && Swappable<RAIter::value_type> && 
             MoveConstructible<RAIter::value_type> && MoveAssignable<RAIter::value_type> &&
             LessThanComparable<InIter::value_type, RAIter::value_type> &&
             LessThanComparable<RAIter::value_type, InIter::value_type> &&
             LessThanComparable<RAIter::value_type>
    RAIter partial_sort_copy(InIter @\farg{first}@, InIter @\farg{last}@,
                             RAIter @\farg{result_first}@, RAIter @\farg{result_last}@);
  template<InputIterator InIter, MutableRandomAccessIterator RAIter, class Compare>
    requires CopyAssignable<RAIter::reference, InIter::value_type> &&
             SameType<RAIter::value_type&, RAIter::reference> && Swappable<RAIter::value_type> && 
             MoveConstructible<RAIter::value_type> && MoveAssignable<RAIter::value_type> &&
             Predicate<Compare, InIter::value_type, RAIter::value_type> &&
             Predicate<Compare, RAIter::value_type, InIter::value_type> &&
             Predicate<Compare, RAIter::value_type>
    RAIter partial_sort_copy(InIter @\farg{first}@, InIter @\farg{last}@,
                             RAIter @\farg{result_first}@, RAIter @\farg{result_last}@,
                             Compare @\farg{comp}@);

  template<ForwardIterator Iter>
    requires LessThanComparable<Iter::value_type>
    bool is_sorted(Iter @\farg{first}@, Iter @\farg{last}@);
  template<ForwardIterator Iter, Predicate<auto, Iter::value_type, Iter::value_type> Compare>
    bool is_sorted(Iter @\farg{first}@, Iter @\farg{last}@,
                   Compare @\farg{comp}@);
  template<ForwardIterator Iter>
    requires LessThanComparable<Iter::value_type>
    Iter is_sorted_until(Iter @\farg{first}@, Iter @\farg{last}@);
  template<ForwardIterator Iter, Predicate<auto, Iter::value_type, Iter::value_type> Compare>
    Iter is_sorted_until(Iter @\farg{first}@, Iter @\farg{last}@,
                         Compare @\farg{comp}@);

  template<MutableRandomAccessIterator Iter>
    requires SameType<Iter::value_type&, Iter::reference> && Swappable<Iter::value_type> && 
             MoveConstructible<Iter::value_type> && MoveAssignable<Iter::value_type> &&
             LessThanComparable<Iter::value_type>
    void nth_element(Iter @\farg{first}@, Iter @\farg{nth}@,
                     Iter @\farg{last}@);
  template<MutableRandomAccessIterator Iter, 
           Predicate<auto, Iter::value_type, Iter::value_type> Compare>
    requires SameType<Iter::value_type&, Iter::reference> && Swappable<Iter::value_type> && 
             MoveConstructible<Iter::value_type> && MoveAssignable<Iter::value_type>
    void nth_element(Iter @\farg{first}@, Iter @\farg{nth}@,
                     Iter @\farg{last}@,  Compare @\farg{comp}@);

  @\textcolor{black}{// \ref{alg.binary.search}, binary search:}@
  template<ForwardIterator Iter, class T>
    requires LessThanComparable<Iter::reference, T>
    Iter lower_bound(Iter @\farg{first}@, Iter @\farg{last}@,
                     const T& @\farg{value}@);
  @\textcolor{addclr}{template}@<ForwardIterator Iter, class T, Predicate<auto, Iter::reference, T> Compare>
    Iter lower_bound(Iter @\farg{first}@, Iter @\farg{last}@,
                     const T& @\farg{value}@, Compare @\farg{comp}@);

  template<ForwardIterator Iter, class T>
    requires LessThanComparable<T, Iter::reference>
    Iter upper_bound(Iter @\farg{first}@, Iter @\farg{last}@,
                     const T& @\farg{value}@);
  template<ForwardIterator Iter, class T, Predicate<auto, T, Iter::reference> Compare>
    Iter upper_bound(Iter @\farg{first}@, Iter @\farg{last}@,
                     const T& @\farg{value}@, Compare @\farg{comp}@);

  template<ForwardIterator Iter, class T>
    requires LessThanComparable<T, Iter::reference> && 
             LessThanComparable<Iter::reference, T>
    pair<Iter, Iter>
      equal_range(Iter @\farg{first}@,
                  Iter @\farg{last}@, const T& @\farg{value}@);
  template<ForwardIterator Iter, class T, class Compare>
    requires Predicate<Compare, T, Iter::reference> &&
             Predicate<Compare, Iter::reference, T>
    pair<Iter, Iter>
      equal_range(Iter @\farg{first}@,
                  Iter @\farg{last}@, const T& @\farg{value}@,
                  Compare @\farg{comp}@);

  template<ForwardIterator Iter, class T>
    requires LessThanComparable<T, Iter::reference> &&
             LessThanComparable<Iter::reference, T>
    bool binary_search(ForwardIterator @\farg{first}@, ForwardIterator @\farg{last}@,
                       const T& @\farg{value}@);
  template<ForwardIterator Iter, class T, class Compare>
    requires Predicate<Compare, T, Iter::reference> &&
             Predicate<Compare, Iter::reference, T>
    bool binary_search(Iter @\farg{first}@, Iter @\farg{last}@,
                       const T& @\farg{value}@, Compare @\farg{comp}@);

  @\textcolor{black}{// \ref{alg.merge}, merge:}@
  template<InputIterator InIter1, InputIterator InIter2,
           OutputIterator<auto, InIter1::value_type> OutIter>
    requires SameType<InIter1::value_type, InIter2::value_type> &&
             LessThanComparable<InIter1::value_type>
    OutIter merge(InIter1 @\farg{first1}@, InIter1 @\farg{last1}@,
                  InIter2 @\farg{first2}@, InIter2 @\farg{last2}@,
                  OutIter @\farg{result}@);
  @\textcolor{addclr}{template}@<InputIterator InIter1, InputIterator InIter2,
           OutputIterator<auto, InIter1::value_type> OutIter, 
           Predicate<auto, InIter1::value_type, InIter2::value_type> Compare>
    requires SameType<InIter1::value_type, InIter2::value_type>
    OutIter merge(InIter1 @\farg{first1}@, InIter1 @\farg{last1}@,
                  InIter2 @\farg{first2}@, InIter2 @\farg{last2}@,
                  OutIter @\farg{result}@, Compare @\farg{comp}@);

  template<MutableBidirectionalIterator Iter>
    requires SameType<Iter::value_type&, Iter::reference> && Swappable<Iter::value_type> && 
             MoveConstructible<Iter::value_type> && MoveAssignable<Iter::value_type> &&
             LessThanComparable<Iter::value_type>
    void inplace_merge(Iter @\farg{first}@,
                       Iter @\farg{middle}@,
                       Iter @\farg{last}@);
  template<MutableBidirectionalIterator Iter, 
           Predicate<auto, Iter::value_type, Iter::value_type> Compare>
    requires SameType<Iter::value_type&, Iter::reference> && Swappable<Iter::value_type> && 
             MoveConstructible<Iter::value_type> && MoveAssignable<Iter::value_type>
    void inplace_merge(Iter @\farg{first}@,
                       Iter @\farg{middle}@,
                       Iter @\farg{last}@, Compare @\farg{comp}@);

  @\textcolor{black}{// \ref{alg.set.operations}, set operations:}@
  template<InputIterator Iter1, InputIterator Iter2>
    requires SameType<Iter1::value_type, Iter2::value_type> && LessThanComparable<Iter1::value_type>
    bool includes(Iter1 @\farg{first1}@, Iter1 @\farg{last1}@,
                  Iter2 @\farg{first2}@, Iter2 @\farg{last2}@);
  template<InputIterator Iter1, InputIterator Iter2, 
           Predicate<auto, Iter1::value_type, Iter2::value_type> Compare>
    requires SameType<Iter1::value_type, Iter2::value_type>
    bool includes(Iter1 @\farg{first1}@, Iter1 @\farg{last1}@,
                  Iter2 @\farg{first2}@, Iter2 @\farg{last2}@,
                  Compare @\farg{comp}@);

  template<InputIterator InIter1, InputIterator InIter2,
           OutputIterator<auto, InIter1::value_type> OutIter>
    requires SameType<InIter1::value_type, InIter2::value_type> &&
             LessThanComparable<InIter1::value_type>
    OutIter set_union(InIter1 @\farg{first1}@, InIter1 @\farg{last1}@,
                      InIter2 @\farg{first2}@, InIter2 @\farg{last2}@,
                      OutIter @\farg{result}@);
  template<InputIterator InIter1, InputIterator InIter2,
           OutputIterator<auto, InIter1::value_type> OutIter, 
           Predicate<auto, InIter1::value_type, InIter2::value_type> Compare>
    requires SameType<InIter1::value_type, InIter2::value_type>
    OutIter set_union(InIter1 @\farg{first1}@, InIter1 @\farg{last1}@,
                      InIter2 @\farg{first2}@, InIter2 @\farg{last2}@,
                      OutIter @\farg{result}@, Compare @\farg{comp}@);

  template<InputIterator InIter1, InputIterator InIter2,
           OutputIterator<auto, InIter1::value_type> OutIter>
    requires SameType<InIter1::value_type, InIter2::value_type> &&
             LessThanComparable<InIter1::value_type>
    OutIter set_intersection(InIter1 @\farg{first1}@, InIter1 @\farg{last1}@,
                             InIter2 @\farg{first2}@, InIter2 @\farg{last2}@,
                             OutIter @\farg{result}@);
  template<InputIterator InIter1, InputIterator InIter2,
           OutputIterator<auto, InIter1::value_type> OutIter, 
           Predicate<auto, InIter1::value_type, InIter2::value_type> Compare>
    requires SameType<InIter1::value_type, InIter2::value_type>
    OutIter set_intersection(InIter1 @\farg{first1}@, InIter1 @\farg{last1}@,
                             InIter2 @\farg{first2}@, InIter2 @\farg{last2}@,
                             OutIter @\farg{result}@, Compare @\farg{comp}@);

  template<InputIterator InIter1, InputIterator InIter2,
           OutputIterator<auto, InIter1::value_type> OutIter>
    requires SameType<InIter1::value_type, InIter2::value_type> &&
             LessThanComparable<InIter1::value_type>
    OutIter set_difference(InIter1 @\farg{first1}@, InIter1 @\farg{last1}@,
                           InIter2 @\farg{first2}@, InIter2 @\farg{last2}@,
                           @\textcolor{addclr}{OutIter}@ @\farg{result}@);
  template<InputIterator InIter1, InputIterator InIter2,
           OutputIterator<auto, InIter1::value_type> OutIter, 
           Predicate<auto, InIter1::value_type, InIter2::value_type> Compare>
    requires SameType<InIter1::value_type, InIter2::value_type>
    OutIter set_difference(InIter1 @\farg{first1}@, InIter1 @\farg{last1}@,
                           InIter2 @\farg{first2}@, InIter2 @\farg{last2}@,
                           OutIter @\farg{result}@, Compare @\farg{comp}@);

  template<InputIterator InIter1, InputIterator InIter2,
           OutputIterator<auto, InIter1::value_type> OutIter>
    requires SameType<InIter1::value_type, InIter2::value_type> &&
             LessThanComparable<InIter1::value_type>
    OutIter set_symmetric_difference(InIter1 @\farg{first1}@, InIter1 @\farg{last1}@,
                                     InIter2 @\farg{first2}@, InIter2 @\farg{last2}@,
                                     OutIter @\farg{result}@);
  template<InputIterator InIter1, InputIterator InIter2,
           OutputIterator<auto, InIter1::value_type> OutIter, 
           Predicate<auto, InIter1::value_type, InIter2::value_type> Compare>
    requires SameType<InIter1::value_type, InIter2::value_type>
    OutIter set_symmetric_difference(InIter1 @\farg{first1}@, InIter1 @\farg{last1}@,
                                     InIter2 @\farg{first2}@, InIter2 @\farg{last2}@,
                                     OutIter @\farg{result}@, Compare @\farg{comp}@);

  @\textcolor{black}{// \ref{alg.heap.operations}, heap operations:}@
  template<MutableRandomAccessIterator Iter>
    requires MoveConstructible<Iter::value_type> && MoveAssignable<Iter::value_type> &&
             LessThanComparable<Iter::value_type>
    void push_heap(Iter @\farg{first}@, Iter @\farg{last}@);
  template<MutableRandomAccessIterator Iter, 
           Predicate<auto, Iter::value_type, Iter::value_type> Compare>
    requires MoveConstructible<Iter::value_type> && MoveAssignable<Iter::value_type>
    void push_heap(Iter @\farg{first}@, Iter @\farg{last}@,
                   Compare @\farg{comp}@);

  template<MutableRandomAccessIterator Iter>
    requires MoveConstructible<Iter::value_type> && MoveAssignable<Iter::value_type> &&
             Swappable<Iter::value_type> && LessThanComparable<Iter::value_type>
    void pop_heap(Iter @\farg{first}@, Iter @\farg{last}@);
  template<MutableRandomAccessIterator Iter, 
           Predicate<auto, Iter::value_type, Iter::value_type> Compare>
    requires MoveConstructible<Iter::value_type> && MoveAssignable<Iter::value_type> &&
             Swappable<Iter::value_type>
    void pop_heap(Iter @\farg{first}@, Iter @\farg{last}@,
                  Compare @\farg{comp}@);

  template<MutableRandomAccessIterator Iter>
    requires MoveConstructible<Iter::value_type> && MoveAssignable<Iter::value_type> &&
             LessThanComparable<Iter::value_type>
    void make_heap(Iter @\farg{first}@, Iter @\farg{last}@);
  template<MutableRandomAccessIterator Iter, 
           Predicate<auto, Iter::value_type, Iter::value_type> Compare>
    requires MoveConstructible<Iter::value_type> && MoveAssignable<Iter::value_type>
    void make_heap(Iter @\farg{first}@, Iter @\farg{last}@,
                   Compare @\farg{comp}@);

  template<MutableRandomAccessIterator Iter>
    requires MoveConstructible<Iter::value_type> && MoveAssignable<Iter::value_type> &&
             Swappable<Iter::value_type> && LessThanComparable<Iter::value_type>
    void sort_heap(Iter @\farg{first}@, Iter @\farg{last}@);
  template<MutableRandomAccessIterator Iter, 
           Predicate<auto, Iter::value_type, Iter::value_type> Compare>
    requires MoveConstructible<Iter::value_type> && MoveAssignable<Iter::value_type> &&
             Swappable<Iter::value_type>
    void sort_heap(Iter @\farg{first}@, Iter @\farg{last}@,
                   Compare @\farg{comp}@);

  template<RandomAccessIterator Iter>
    requires LessThanComparable<Iter::value_type>
    bool is_heap(Iter @\farg{first}@, Iter @\farg{last}@);
  template<RandomAccessIterator Iter, Predicate<auto, Iter::value_type, Iter::value_type> Compare>
    bool is_heap(Iter @\farg{first}@, Iter @\farg{last}@, Compare @\farg{comp}@);
  template<RandomAccessIterator Iter>
    Iter is_heap_until(Iter @\farg{first}@, Iter @\farg{last}@);
  template<RandomAccessIterator Iter, Predicate<auto, Iter::value_type, Iter::value_type> Compare>
    Iter is_heap_until(Iter @\farg{first}@, Iter @\farg{last}@,
                       Compare @\farg{comp}@);

  @\textcolor{black}{// \ref{alg.min.max}, minimum and maximum:}@
  template<LessThanComparable T> const T& min(const T& @\farg{a}@, const T& @\farg{b}@);
  template<class T, Predicate<auto, T, T> Compare>
    @\addedCC{requires !SameType<T, Compare>}@
    const T& min(const T& @\farg{a}@, const T& @\farg{b}@, Compare @\farg{comp}@);
  @\removedCC{template<LessThanComparable T> const T\& min(const T\& \mbox{\farg{a}}, const T\& \mbox{\farg{b}}, const T\& \mbox{\farg{c}});}@
  @\addedCC{template<LessThanComparable T, LessThanComparable... Args>}@
    @\addedCC{requires SameType<T, Args>...}@
    @\addedCC{const T\& min(const T\& \mbox{\farg{a}}, const Args\&... \mbox{\farg{args}});}@
  @\textcolor{black}{template<class T, class U, class... Args>}@
    @\textcolor{black}{const T\& min(const T\& \mbox{\farg{a}}, const U\& \mbox{\farg{b}}, const Args\&... \mbox{\farg{args}});}@

  template<LessThanComparable T> const T& max(const T& @\farg{a}@, const T& @\farg{b}@);
  template<class T, Predicate<auto, T, T> Compare>
    @\addedCC{requires !SameType<T, Compare>}@
    const T& max(const T& @\farg{a}@, const T& @\farg{b}@, Compare @\farg{comp}@);
  @\removedCC{template<LessThanComparable T> const T\& max(const T\& \mbox{\farg{a}}, const T\& \mbox{\farg{b}}, const T\& \mbox{\farg{c}});}@
  @\addedCC{template<LessThanComparable T, LessThanComparable... Args>}@
    @\addedCC{requires SameType<T, Args>...}@
    @\addedCC{const T\& max(const T\& \mbox{\farg{a}}, const Args\&... \mbox{\farg{args}});}@
  @\textcolor{black}{template<class T, class U, class... Args>}@
    @\textcolor{black}{const T\& max(const T\& \mbox{\farg{a}}, const U\& \mbox{\farg{b}}, const Args\&... \mbox{\farg{args}});}@

  template<LessThanComparable T> pair<const T&, const T&> minmax(const T& @\farg{a}@, const T& @\farg{b}@);
  template<class T, Predicate<auto, T, T> Compare>
    @\addedCC{requires !SameType<T, Compare>}@
    pair<const T&, const T&> minmax(const T& @\farg{a}@, const T& @\farg{b}@, Compare @\farg{comp}@);
  @\removedCC{template<LessThanComparable T>}@
    @\removedCC{pair<const T\&, const T\&> minmax(const T\& \mbox{\farg{a}}, const T\& \mbox{\farg{b}}, const T\& \mbox{\farg{c}});}@
  @\addedCC{template<LessThanComparable T, LessThanComparable... Args>}@
    @\addedCC{requires SameType<T, Args>...}@
    @\addedCC{pair<const T\&, const T\&> minmax(const T\& \mbox{\farg{a}}, const Args\&... \mbox{\farg{args}});}@
  @\textcolor{black}{template<class T, class U, class... Args>}@
    @\textcolor{black}{pair<const T\&, const T\&> minmax(const T\& \mbox{\farg{a}}, const U\& \mbox{\farg{b}}, const Args\&... \mbox{\farg{args}});}@

  template<ForwardIterator Iter>
    requires LessThanComparable<Iter::reference>
    Iter min_element(Iter @\farg{first}@, Iter @\farg{last}@);
  template<ForwardIterator Iter, Predicate<auto, Iter::reference, Iter::reference> Compare>
    Iter min_element(Iter @\farg{first}@, Iter @\farg{last}@,
                     Compare @\farg{comp}@);

  template<ForwardIterator Iter>
    requires LessThanComparable<Iter::reference>
    Iter max_element(Iter @\farg{first}@, Iter @\farg{last}@);
  template<ForwardIterator Iter, Predicate<auto, Iter::reference, Iter::reference> Compare>
    Iter max_element(Iter @\farg{first}@, Iter @\farg{last}@,
                     Compare @\farg{comp}@);

  template<ForwardIterator Iter>
    requires LessThanComparable<Iter::value_type>
    pair<Iter, Iter>
      @\textcolor{addclr}{}@minmax_element(Iter @\farg{first}@, Iter @\farg{last}@);
  template<ForwardIterator Iter, Predicate<auto, Iter::value_type, Iter::value_type> Compare>
    pair<Iter, Iter>
      minmax_element(Iter @\farg{first}@, Iter @\farg{last}@, Compare @\farg{comp}@);

  template<InputIterator Iter1, InputIterator Iter2>
    @\textcolor{addclr}{requires}@ LessThanComparable<Iter1::reference, Iter2::reference> &&
             LessThanComparable<Iter2::reference, Iter1::reference>
    bool lexicographical_compare(Iter1 @\farg{first1}@, Iter1 @\farg{last1}@,
                                 Iter2 @\farg{first2}@, Iter2 @\farg{last2}@);

  @\textcolor{addclr}{template}@<InputIterator Iter1, InputIterator Iter2, class Compare>
    requires Predicate<Compare, Iter1::reference, Iter2::reference> &&
             Predicate<Compare, Iter2::reference, Iter1::reference>
    bool lexicographical_compare(Iter1 @\farg{first1}@, Iter1 @\farg{last1}@,
                                 Iter2 @\farg{first2}@, Iter2 @\farg{last2}@,
                                 Compare @\farg{comp}@);

  @\textcolor{black}{// \ref{alg.permutation.generators}, permutations:}@
  template<MutableBidirectionalIterator Iter>
    requires SameType<Iter::value_type&, Iter::reference> && Swappable<Iter::value_type> &&
             @\textcolor{addclr}{LessThanComparable}@<Iter::reference>
    bool next_permutation(Iter @\farg{first}@, Iter @\farg{last}@);
  template<MutableBidirectionalIterator Iter, 
           Predicate<auto, Iter::reference, Iter::reference> Compare>
    requires SameType<Iter::value_type&, Iter::reference> && Swappable<Iter::value_type>
    bool next_permutation(Iter @\farg{first}@, Iter @\farg{last}@, Compare @\farg{comp}@);
  template<MutableBidirectionalIterator Iter>
    requires SameType<Iter::value_type&, Iter::reference> && Swappable<Iter::value_type> &&
             @\textcolor{addclr}{}@LessThanComparable<Iter::reference>
    bool prev_permutation(Iter @\farg{first}@, Iter @\farg{last}@);
  template<MutableBidirectionalIterator Iter, 
           Predicate<auto, Iter::reference, Iter::reference> Compare>
    requires SameType<Iter::value_type&, Iter::reference> && Swappable<Iter::value_type>
    bool prev_permutation(Iter @\farg{first}@, Iter @\farg{last}@, Compare @\farg{comp}@);
}
\end{codeblock}
\color{black}

\pnum
All of the algorithms are separated from the particular implementations of data structures and are
parameterized by iterator types.
Because of this, they can work with program-defined data structures, as long
as these data structures have iterator types satisfying the assumptions on the algorithms.

\pnum
\removedConcepts{Throughout this clause, the names of template parameters
are used to express type requirements.
If an algorithm's template parameter is
InputIterator,
InputIterator1,
or
InputIterator2,
the actual template argument shall satisfy the
requirements of an input iterator (24.1.1).
If an algorithm's template parameter is
OutputIterator,
OutputIterator1,
or
OutputIterator2,
the actual template argument shall satisfy the requirements
of an output iterator (24.1.2).
If an algorithm's template parameter is
ForwardIterator,
ForwardIterator1,
or
ForwardIterator2,
the actual template argument shall satisfy the requirements
of a forward iterator (24.1.3).
If an algorithm's template parameter is
BidirectionalIterator,
BidirectionalIterator1,
or
BidirectionalIterator2,
the actual template argument shall satisfy the requirements
of a bidirectional iterator (24.1.4).
If an algorithm's template parameter is
RandomAccessIterator,
RandomAccessIterator1,
or
RandomAccessIterator2,
the actual template argument shall satisfy the requirements
of a random-access iterator (24.1.5).}

\pnum
\removedConcepts{If an algorithm's
Effects
section says that a value pointed to by any iterator passed
as an argument is modified, then that algorithm has an additional
type requirement:
The type of that argument shall satisfy the requirements
of a mutable iterator (24.1).
[ Note:
this requirement does not affect arguments that are declared as
OutputIterator,
OutputIterator1,
or
OutputIterator2,
because output iterators must always be mutable.
-- end note]}

\pnum\
\color{black}
Both in-place and copying versions are provided for certain algorithms.%
\footnote{
The decision whether to include a copying version was usually based on complexity considerations.
When the cost of doing the operation dominates the cost of copy, the copying version is not included.
For example,
\tcode{sort_copy}
is not included
because the cost of sorting is much more significant, and users might as well do
\tcode{copy}
followed by
\tcode{sort}.
}
When such a version is provided for
\textit{algorithm}\ 
it is called
\textit{algorithm\farg{_copy}}.
Algorithms that take predicates end with the suffix
\tcode{_if}
(which follows the suffix
\tcode{_copy}).

\pnum
\removedConcepts{The
Predicate
parameter is used whenever an algorithm expects a function object that when applied to the result
of dereferencing the corresponding iterator returns a value testable as
true.
In other words, if an algorithm
takes
Predicate pred
as its argument and first
as its iterator argument, it should work correctly in the
construct
if (pred(*first))\{...\}.
The function object
pred
shall not apply any non-constant
function through the dereferenced iterator.
This function object may be a pointer to function,
or an object of a type with an appropriate function call operator.}

\pnum
\removedConcepts{The
Predicate
parameter is used whenever an algorithm expects a function object that when applied to
the result of dereferencing two corresponding iterators or to dereferencing an
iterator and type
T
when
T
is part of the signature returns a value testable as
true.
In other words, if an algorithm takes
Predicate binary_pred
as its argument and first1 and first2 as
its iterator arguments, it should work correctly in
the construct
if (binary_pred(*first1, *first2))\{...\}.
Predicate
always takes the first iterator
type as its first argument, that is, in those cases when
T value
is part of the signature, it should work
correctly in the context of
if (binary_pred(*first1, value))\{...\}.
binary_pred shall not
apply any non-constant function through the dereferenced iterators.}

\pnum
\enternote\
Unless otherwise specified, algorithms that take function objects as arguments
are permitted to copy those function objects freely. Programmers for whom object
identity is important should consider using a wrapper class that
points \textcolor{black}{}to a
noncopied implementation object, or some equivalent solution.
\exitnote\

\pnum\
\removedConcepts{
When the description of an algorithm gives an expression such as
*first == value
for a condition, the expression shall evaluate to
either true or false in boolean contexts.}

\pnum
In the description of the algorithms operators
\tcode{+}\
and
\tcode{-}\
are used for some of the iterator categories for which
they do not have to be defined.
In these cases the semantics of
\tcode{a+n}\
is the same as that of

\begin{codeblock}
  { X tmp = a;
    advance(tmp, n);
    return tmp;
  }
\end{codeblock}

and that of
\tcode{b-a}\
is the same as of

\begin{codeblock}
  return distance(a, b);
\end{codeblock}

\rSec1[alg.nonmodifying]{Non-modifying sequence operations}

\rSec2[alg.foreach]{For each}

\editorial{The standard does not state whether the function object
  takes a value of the iterator's value type or reference type. The
  SGI STL documentation says it's the value type, but that conflicts
  with user expectations that they are operating on references. Also,
  ``the result of dereferencing every iterator...'' is the reference
  type, not the value type. Therefore, we have chosen to use the
  reference type.}

\index{for_each@\tcode{for_each}}%

\color{addclr}
\begin{itemdecl}
template<InputIterator Iter, Callable<auto, Iter::reference> Function>
  Function for_each(Iter @\farg{first}@, Iter @\farg{last}@, Function @\farg{f}@);
\end{itemdecl}
\color{black}

\begin{itemdescr}
\pnum
\effects\ 
Applies
\farg{f}\ to the result of dereferencing every iterator in the range
\range{\farg{first}}{\farg{last}},
starting from
\tcode{\farg{first}}\
and proceeding to
\tcode{\farg{last} - 1}.

\pnum
\returns\ 
\farg{f}.

\pnum
\complexity\ 
Applies \farg{f}\
exactly
\tcode{\farg{last}\ - \farg{first}}\
times.

\pnum
\removedConcepts{Notes: If f returns a result, the result is ignored.}
\end{itemdescr}

\rSec2[alg.find]{Find}

\index{find@\tcode{find}}%
\index{find_if@\tcode{find_if}}%
\color{addclr}\begin{itemdecl}
@\textcolor{addclr}{template}@<InputIterator Iter, class T>
  requires EqualityComparable<Iter::reference, T>
  Iter find(Iter @\farg{first}@, Iter @\farg{last}@, const T& @\farg{value}@);

template<InputIterator Iter, Predicate<auto, Iter::reference> Pred>
  Iter find_if(Iter @\farg{first}@, Iter @\farg{last}@, Pred @\farg{pred}@);
\end{itemdecl}\color{black}

\begin{itemdescr}
\pnum
\returns\ 
The first iterator
\tcode{i}\
in the range
\range{\farg{first}}{\farg{last}}\
for which the following corresponding
conditions hold:
\tcode{*i == \farg{value}, \farg{pred}(*i) != false}.
Returns \farg{last}\ if no such iterator is found.

\pnum
\complexity\ 
At most
\tcode{\farg{last}\ - \farg{first}}\
applications of the corresponding predicate.
\end{itemdescr}

\rSec2[alg.find.end]{Find End}

\index{find_end@\tcode{find_end}}%
\color{addclr}\begin{itemdecl}
template<ForwardIterator Iter1, ForwardIterator Iter2>
  requires EqualityComparable<Iter1::reference, Iter2::reference>
  Iter1 find_end(Iter1 @\farg{first1}@, Iter1 @\farg{last1}@,
                 Iter2 @\farg{first2}@, Iter2 @\farg{last2}@);

template<ForwardIterator Iter1, ForwardIterator Iter2, 
         Predicate<auto, Iter1::reference, Iter2::reference> Pred>
  Iter1 find_end(Iter1 @\farg{first1}@, Iter1 @\farg{last1}@,
                 Iter2 @\farg{first2}@, Iter2 @\farg{last2}@,
                 Pred @\farg{pred}@);
\end{itemdecl}\color{black}

\begin{itemdescr}
\pnum
\effects\ 
Finds a subsequence of equal values in a sequence.

\pnum
\textcolor{black}{}\returns\ 
The last iterator
\tcode{i}\
in the range \range{\farg{first1}}{\farg{last1} - (\farg{last2} - \farg{first2})}\
such that for any non-negative integer
\tcode{n < (\farg{last2} - \farg{first2})},
the following corresponding conditions hold:
\tcode{*(i + n) == *(\brk{}\farg{first2}\ + n), \farg{pred}(*(i + n), *(\farg{first2}\ + n)) != false}.
Returns \farg{last1}\
if no such iterator is found.

\pnum
\complexity\ 
At most
\tcode{(\farg{last2} - \farg{first2}) * (\farg{last1} - \farg{first1} - (\farg{last2} - \farg{first2}) + 1)}
applications of the corresponding predicate.
\end{itemdescr}

\rSec2[alg.find.first.of]{Find First}

\editorial{This text assumes that the proposed resolution to DR 576 is
  accepted, weakening the requirements on the first type parameter
  (\tcode{Iter1}) to \tcode{Input Iterator}.}

\index{find_first_of@\tcode{find_first_of}}%
\color{addclr}\begin{itemdecl}
template<InputIterator Iter1, ForwardIterator Iter2>
  requires EqualityComparable<Iter1::reference, Iter2::reference>
  Iter1 find_first_of(Iter1 @\farg{first1}@, Iter1 @\farg{last1}@,
                      Iter2 @\farg{first2}@, Iter2 @\farg{last2}@);

template<InputIterator Iter1, ForwardIterator Iter2,
         Predicate<auto, Iter1::reference, Iter2::reference> Pred>
  Iter1 find_first_of(Iter1 @\farg{first1}@, Iter1 @\farg{last1}@,
                      Iter2 @\farg{first2}@, Iter2 @\farg{last2}@,
                      Pred @\farg{pred}@);
\end{itemdecl}\color{black}

\begin{itemdescr}
\pnum
\effects\ 
Finds an element that matches one of a set of values.

\pnum
\returns\ 
The first iterator
\tcode{i}\
in the range \range{\farg{first1}}{\farg{last1}}\
such that for some
iterator
\tcode{j}\
in the range \range{\farg{first2}}{\farg{last2}}\
the following conditions hold:
\tcode{*i == *j, \farg{pred}(*i,*j) != false}.
Returns \farg{last1}\
if no such iterator is found.

\pnum
\complexity\ 
At most
\tcode{(\farg{last1}-\farg{first1}) * (\farg{last2}-\farg{first2})}
applications of the corresponding predicate.
\end{itemdescr}

\rSec2[alg.adjacent.find]{Adjacent find}

\index{adjacent_find@\tcode{adjacent_find}}%
\color{addclr}\begin{itemdecl}
template<ForwardIterator Iter>
  requires EqualityComparable<Iter::reference>
  Iter adjacent_find(Iter @\farg{first}@, Iter @\farg{last}@);

template<ForwardIterator Iter, Predicate<auto, Iter::reference, Iter::reference> Pred>
  Iter adjacent_find(Iter @\farg{first}@, Iter @\farg{last}@, Pred @\farg{pred}@);
\end{itemdecl}\color{black}

\begin{itemdescr}
\pnum
\returns\ 
The first iterator
\tcode{i}\
such that both
\tcode{i}\
and
\tcode{i + 1}\
are in
the range
\range{\farg{first}}{\farg{last}}\
for which
the following corresponding conditions hold:
\tcode{*i == *(i + 1), \farg{pred}(*i, *(i + 1)) != false}.
Returns \farg{last}\
if no such iterator is found.

\pnum
\complexity\ 
For a nonempty range, exactly
\tcode{min((i - \farg{first}) + 1, (\farg{last}\ - \farg{first}) - 1)}\
applications of the corresponding predicate, where \tcode{i}\ is
\tcode{adjacent_find}'s
return value.
\end{itemdescr}

\rSec2[alg.count]{Count}

\index{count@\tcode{count}}%
\index{count_if@\tcode{count_if}}%
\color{addclr}\begin{itemdecl}
template<InputIterator Iter, class T>
  requires EqualityComparable<Iter::reference, T>
  @\color{addclr}@Iter::difference_type count(Iter @\farg{first}@, Iter @\farg{last}@, const T& @\farg{value}@);

template<InputIterator Iter, Predicate<auto, Iter::reference> Pred>
  Iter::difference_type count_if(Iter @\farg{first}@, Iter @\farg{last}@, Pred @\farg{pred}@);
\end{itemdecl}\color{black}

\begin{itemdescr}
\pnum
\effects\ 
Returns the number of iterators
\tcode{i}\
in the range \range{\farg{first}}{\farg{last}}\
for which the following corresponding
conditions hold:
\tcode{*i == \farg{value}, \farg{pred}(*i) != false}.

\pnum
\complexity\ 
Exactly
\tcode{\farg{last} - \farg{first}}\
applications of the corresponding predicate.
\end{itemdescr}

\rSec2[lib.mismatch]{\textcolor{black}{Mismatch}}

\index{mismatch@\tcode{mismatch}}%
\color{addclr}\begin{itemdecl}
template<InputIterator Iter1, InputIterator Iter2>
  requires EqualityComparable<Iter1::reference, Iter2::reference>
  pair<Iter1, Iter2> mismatch(Iter1 @\farg{first1}@, Iter1 @\farg{last1}@,
                              Iter @\farg{first2}@);

template<InputIterator Iter1, InputIterator Iter2,
         Predicate<auto, Iter1::reference, Iter2::reference> Pred>
  pair<Iter1, Iter2> mismatch(Iter1 @\farg{first1}@, Iter2 @\farg{last1}@,
                              Iter2 @\farg{first2}@, Pred @\farg{pred}@);
\end{itemdecl}\color{black}

\begin{itemdescr}
\pnum
\returns\ 
A pair of iterators
\tcode{i}\
and
\tcode{j}\
such that
\tcode{j == \farg{first2} + (i - \farg{first1})}\
and
\tcode{i}\
is the first iterator
in the range \range{\farg{first1}}{\farg{last1}}\
for which the following corresponding conditions hold:

\begin{codeblock}
  !(*i == *(@\farg{first2}@ + (i - @\farg{first1}@)))
  @\farg{pred}@(*i, *(@\farg{first2}@ + (i - @\farg{first1}@))) == false
\end{codeblock}

Returns the pair \farg{last1}\ and
\tcode{\farg{first2}\ + (\farg{last1}\ - \farg{first1})}
if such an iterator
\tcode{i}\
is not found.

\pnum
\complexity\ 
At most
\tcode{\farg{last1}\ - \farg{first1}}\
applications of the corresponding predicate.
\end{itemdescr}

\rSec2[alg.equal]{Equal}

\index{equal@\tcode{equal}}%
\color{addclr}\begin{itemdecl}
template<InputIterator Iter1, InputIterator Iter2>
  requires EqualityComparable<Iter1::reference, Iter2::reference>
  bool equal(Iter1 @\farg{first1}@, Iter1 @\farg{last1}@,
             Iter2 @\farg{first2}@);

template<InputIterator Iter1, InputIterator Iter2,
         Predicate<auto, Iter1::reference, Iter2::reference> Pred>
  bool equal(Iter1 @\farg{first1}@, Iter1 @\farg{last1}@,
             Iter2 @\farg{first2}@, Pred @\farg{pred}@);
\end{itemdecl}\color{black}

\begin{itemdescr}
\pnum
\returns\ 
\tcode{true}\
if for every iterator
\tcode{i}\
in the range \range{\farg{first1}}{\farg{last1}}\
the following corresponding conditions hold:
\tcode{*i == *(\farg{first2}\ + (i - \farg{first1})), \farg{pred}(*i, *(\farg{first2}\ + (i - \farg{first1}))) != false}.
Otherwise, returns
\tcode{false}.

\pnum
\complexity\ 
At most
\tcode{\farg{last1} - \farg{first1}}\
applications of the corresponding predicate.
\end{itemdescr}

\rSec2[alg.search]{Search}

\index{search@\tcode{search}}%
\color{addclr}\begin{itemdecl}
template<ForwardIterator Iter1, ForwardIterator Iter2>
  requires EqualityComparable<Iter1::reference, Iter2::reference>
  Iter1 search(Iter1 @\farg{first1}@, Iter1 @\farg{last1}@,
               Iter2 @\farg{first2}@, Iter2 @\farg{last2}@);

template<ForwardIterator Iter1, ForwardIterator Iter2,
         Predicate<auto, Iter1::reference, Iter2::reference> Pred>
  @\textcolor{addclr}{Iter1}@ search(Iter1 @\farg{first1}@, Iter1 @\farg{last1}@,
               Iter2 @\farg{first2}@, Iter2 @\farg{last2}@,
               Pred @\farg{pred}@);
\end{itemdecl}\color{black}

\begin{itemdescr}
\pnum
\effects\ 
Finds a subsequence of equal values in a sequence.

\pnum
\returns\ 
The first iterator
\tcode{i}\
in the range \range{\farg{first1}}{\farg{last1}\ - (\farg{last2}-\farg{first2})}\
such that for any non-negative integer
\tcode{n}\
less than
\tcode{\farg{last2} - \farg{first2}}\
the following corresponding conditions hold:
\tcode{*(i + n) == *(\farg{first2}\ + n), \farg{pred}(*(i + n), *(\farg{first2}\ + n)) != false}.
Returns \farg{last1}\
if no such iterator is found.

\pnum
\complexity\ 
At most
\tcode{(\farg{last1}\ - \farg{first1}) * (\farg{last2}\ - \farg{first2})}
applications of the corresponding predicate.
\end{itemdescr}

\color{addclr}\begin{itemdecl}
@\textcolor{addclr}{template}@<ForwardIterator Iter, class T>
  requires EqualityComparable<Iter::reference, T>
  Iter search_n(Iter @\farg{first}@, Iter @\farg{last}@, Iter::difference_type @\farg{count}@,
                const T& @\farg{value}@);

template<ForwardIterator Iter, class T,
         Predicate<auto, Iter::reference, T> Pred>
  Iter search_n(Iter @\farg{first}@, Iter @\farg{last}@, Iter::difference_type @\farg{count}@,
                const T& @\farg{value}@, Pred @\farg{pred}@);
\end{itemdecl}\color{black}

\begin{itemdescr}
\pnum
\removedConcepts{Requires: 
The type
Size
is convertible to integral type~(4.7, 12.3).} 

\editorial{We have
removed the \tcode{Size} parameter and instead chosen to use the
\tcode{difference_type} of the iterator. This change can break existing
code in two ways. First, if the \tcode{Size} parameter was originally
bound to a type larger than \tcode{difference_type} and the
\tcode{count} parameter contains a value outside of the range of
\tcode{difference_type} (in which case, \tcode{search_n} always
returns \tcode{last}). Second, if the user explicitly provides an
argument for the \tcode{Size} parameter. Note: This change has not yet
been reflected in libstdc++.} 

\pnum\ \color{black}
\effects\ 
Finds a subsequence of equal values in a sequence.

\pnum
\returns\ 
The first iterator
\tcode{i}\
in the range \range{\farg{first}}{\farg{last}-\farg{count}}\
such that for any non-negative integer
\tcode{n}\
less than
\tcode{count}\
the following corresponding conditions hold:
\tcode{*(i + n) == \farg{value}, \farg{pred}(*(i + n),\farg{value}) != false}.
Returns \farg{last}\
if no such iterator is found.

\pnum
\textcolor{black}{}\complexity\ 
At most
\tcode{(\farg{last}\ - \farg{first}) * \farg{count}}
applications of the corresponding predicate if \farg{count} is positive, or
0 otherwise.
\end{itemdescr}

\rSec1[alg.modifying.operations]{Mutating sequence operations}

\rSec2[alg.copy]{Copy}

\index{copy@\tcode{copy}}%
\color{addclr}\begin{itemdecl}
template<InputIterator InIter, OutputIterator<auto, InIter::reference> OutIter>
  OutIter copy(InIter @\farg{first}@, InIter @\farg{last}@,
               OutIter @\farg{result}@);
\end{itemdecl}\color{black}

\begin{itemdescr}
\pnum
\effects\ 
Copies elements in the range \range{\farg{first}}{\farg{last}}\
into the range \range{\farg{result}}{\farg{result}\ + (\farg{last}\ - \farg{first})}\
starting from \farg{first} and proceeding to \farg{last}.
For each non-negative integer
\tcode{n < (\farg{last}-\farg{first})},
performs
\tcode{\textcolor{black}{*}(\farg{result}\ + n) = *(\farg{first}\ + n)}.

\pnum
\returns\ 
\tcode{\farg{result}\ + (\farg{last}\ - \farg{first})}.

\pnum
\requires\ 
\farg{result}
shall not be in the range
\range{\farg{first}}{\farg{last}}.

\pnum
\complexity\ 
Exactly
\tcode{\farg{last}\ - \farg{first}}
assignments.
\end{itemdescr}

\index{copy_backward@\tcode{copy_backward}}%
\color{addclr}\begin{itemdecl}
template<BidirectionalIterator InIter, MutableBidirectionalIterator OutIter>
  requires CopyAssignable<OutIter::reference, InIter::reference>
  OutIter copy_backward(InIter @\farg{first}@, InIter @\farg{last}@,
                        OutIter @\farg{result}@);
\end{itemdecl}\color{black}

\begin{itemdescr}
\pnum
\effects\ 
Copies elements in the range \range{\farg{first}}{\farg{last}}\
into the
range \range{\farg{result}\ - (\farg{last}-\farg{first})}{\farg{result}}\
\textcolor{black}{starting} from
\tcode{\farg{last} - 1}\
and proceeding to \farg{first}.%
\footnote{
\tcode{copy_backward}\
should be used instead of copy when \farg{last}\
is in
the range
\range{\farg{result}\ - (\farg{last}\ - \farg{first})}{\farg{result}}.
}
For each positive integer
\tcode{n <= (\farg{last}\ - \farg{first})},
performs
\tcode{*(\farg{result}\ - n) = *(\farg{last}\ - n)}.

\pnum
\requires\ 
\tcode{\farg{result}}
shall not be in the range \range{\farg{first}}{\farg{last}}.

\pnum
\returns\ 
\tcode{\farg{result}\ - (\farg{last}\ - \farg{first})}.

\pnum
\complexity\ 
Exactly
\tcode{\farg{last}\ - \farg{first}}
assignments.
\end{itemdescr}

\rSec2[alg.move]{Move}

\index{move\tcode{move}}%
\color{addclr}
\begin{itemdecl}
@\textcolor{addclr}{}@template<InputIterator InIter, OutputIterator<auto, InIter::value_type&&> OutIter>
  OutIter move(InIter first, InIter last,
               OutIter result);
\end{itemdecl}
\color{black}

\begin{itemdescr}
\pnum
\addedD{\mbox{\effects}
Moves elements in the range \mbox{\range{\farg{first}}{\farg{last}}}
into the range \mbox{\range{\farg{result}}{\farg{result}\ + (\farg{last}\ - \farg{first})}}
starting from \mbox{\farg{first}} and proceeding to \mbox{\farg{last}}.
For each non-negative integer
\mbox{\tcode{n < (\farg{last}-\farg{first})}},
performs
\mbox{\tcode{*(\farg{result}\ + n)}} \mbox{\tcode{= std::move(*(\farg{first}\ + n))}}.}

\pnum
\addedD{\mbox{\returns}
\mbox{\tcode{\farg{result}\ + (\farg{last}\ - \farg{first})}}.}

\pnum
\addedD{\mbox{\requires}
\mbox{\farg{result}}
shall not be in the range
\mbox{\range{\farg{first}}{\farg{last}}}.}

\pnum
\addedD{\mbox{\complexity}
Exactly
\mbox{\tcode{\farg{last}\ - \farg{first}}}
move assignments.}
\end{itemdescr}

\index{move_backward@\tcode{move_backward}}%
\color{addclr}
\begin{itemdecl}
template<BidirectionalIterator InIter, BidirectionalIterator OutIter>
  requires MoveAssignable<OutIter::reference, InIter::value_type>
  OutIter move_backward(InIter first, InIter last,
                        OutIter result);
\end{itemdecl}
\color{black}

\begin{itemdescr}
\pnum
\addedD{\mbox{\effects}
Moves elements in the range \mbox{\range{\farg{first}}{\farg{last}}}
into the
range \mbox{\range{\farg{result}\ - (\farg{last}-\farg{first})}{\farg{result}}}
starting from
\mbox{\tcode{\farg{last} - 1}}
and proceeding to \mbox{\farg{first}}.}%
\footnote{
\mbox{\tcode{move_backward}}
should be used instead of move when \mbox{\farg{last}}
is in
the range
\mbox{\range{\farg{result}\ - (\farg{last}\ - \farg{first})}{\farg{result}}}.
}
\addedD{For each positive integer
\mbox{\tcode{n <= (\farg{last}\ - \farg{first})}},
performs
\mbox{\tcode{*(\farg{result}\ - n) = std::move(*(\farg{last}\ - n))}}.}

\pnum
\addedD{\mbox{\requires}
\mbox{\tcode{\farg{result}}}
shall not be in the range \mbox{\range{\farg{first}}{\farg{last}}}.}

\pnum
\addedD{\mbox{\returns}
\mbox{\tcode{\farg{result}\ - (\farg{last}\ - \farg{first})}}.}

\pnum
\addedD{\mbox{\complexity}
Exactly
\mbox{\tcode{\farg{last}\ - \farg{first}}}
assignments.}
\end{itemdescr}

\rSec2[alg.swap]{Swap}

\index{swap@\tcode{swap}}%
\color{addclr}
\begin{itemdecl}
template<class T> 
  requires MoveAssignable<T> && MoveConstructible<T>
  void swap(T& @\farg{a}@, T& @\farg{b}@);
\end{itemdecl}
\color{black}

\begin{itemdescr}
\pnum
\removedConcepts{
Requires:
Type
T
is
MoveConstructible~(33) and
MoveAssignable~(35).}

\pnum
\effects\ 
Exchanges values stored in two locations.
\end{itemdescr}

\index{swap_ranges@\tcode{swap_ranges}}%
\color{addclr}
\begin{itemdecl}
template<MutableForwardIterator Iter1, MutableForwardIterator Iter2>
  requires SameType<Iter1::value_type, Iter2::value_type> && 
           SameType<Iter1::value_type&, Iter1::reference> && 
           SameType<Iter2::value_type&, Iter2::reference> && 
           Swappable<Iter1::value_type>
  Iter2 swap_ranges(Iter1 @\farg{first1}@, Iter1 @\farg{last1}@,
                    Iter2 @\farg{first2}@);
\end{itemdecl}
\color{black}

\begin{itemdescr}
\pnum
\effects\ 
For each non-negative integer
\tcode{n < (\farg{last1}\ - \farg{first1})}
performs:
\tcode{swap(*(\farg{first1}\ + n), *(\farg{first2}\ + n))}.

\pnum
\textcolor{black}{}\requires\ 
The two ranges \range{\farg{first1}}{\farg{last1}}\
and
\range{\farg{first2}}{\farg{first2}\ + (\farg{last1}\ - \farg{first1})}\
shall not overlap. 
\removedConcepts{The type of *first1 shall be the same
as the type of *first2 and that type shall satisfy the
Swappable requirements (20.1.4).}

\pnum
\returns\ 
\tcode{\farg{first2}\ + (\farg{last1}\ - \farg{first1})}.

\pnum
\complexity\ 
Exactly
\tcode{\farg{last1}\ - \farg{first1}}\
swaps.
\end{itemdescr}

\index{iter_swap@\tcode{iter_swap}}%
\color{addclr}
\begin{itemdecl}
template<MutableForwardIterator Iter1, MutableForwardIterator Iter2>
  requires SameType<Iter1::reference, Iter2::reference> && 
           SameType<Iter1::value_type&, Iter1::reference> && 
           SameType<Iter2::value_type&, Iter2::reference> && 
           Swappable<Iter1::value_type>
  @\addedConcepts{void}@ iter_swap(Iter1 @\farg{a}@, Iter2 @\farg{b}@);
\end{itemdecl}
\color{black}

\begin{itemdescr}
\pnum
\effects\ 
\tcode{swap(*\farg{a}, *\farg{b})}.

\pnum
\removedConcepts{
Requires:
The type of *a shall be the same
as the type of *b and that type shall satisfy the
Swappable requirements (20.1.4).}
\end{itemdescr}

\rSec2[alg.transform]{Transform}

\index{transform@\tcode{transform}}%
\color{addclr}\begin{itemdecl}
template<InputIterator InIter, class OutIter,
         Callable<auto, InIter::reference> Op>
  requires OutputIterator<OutIter, Op::result_type>
  OutIter transform(InIter @\farg{first}@, InIter @\farg{last}@,
                    OutIter @\farg{result}@, Op @\farg{op}@);

template<InputIterator InIter1, InputIterator InIter2,
         class OutIter, Callable<auto, InIter1::reference, InIter2::reference> BinaryOp>
  requires OutputIterator<OutIter, BinaryOp::result_type>
  OutIter transform(InIter1 @\farg{first1}@, InIter1 @\farg{last1}@,
                    InIter2 @\farg{first2}@, OutIter @\farg{result}@,
                    BinaryOp @\farg{binary_op}@);
\end{itemdecl}\color{black}

\begin{itemdescr}
\pnum
\effects\ 
Assigns through every iterator
\tcode{i}\
in the range
\range{\farg{result}}{\farg{result}\ + (\farg{last1}\ - \farg{first1})}
a new
corresponding value equal to
\tcode{\farg{op}(*(\farg{first1}\ + (i - \farg{result}))}
or
\tcode{\farg{binary_op}(*(\farg{first1}\ + (i - \farg{result}), *(\farg{first2}\ + (i - \farg{result})))}.

\pnum
\requires\ 
\farg{op}\ and \farg{binary_op}\
shall not invalidate iterators or subranges, or modify elements in the ranges
\crange{\farg{first1}}{\farg{last1}},
\crange{\farg{first2}}{\farg{first2}\ + (\farg{last1}\ - \farg{first1})},
and
\crange{\farg{result}}{\farg{result}\ + (\farg{last1}\ - \farg{first1})}.%
\footnote{The use of fully closed ranges is intentional.
}

\pnum
\returns\ 
\tcode{\farg{result}\ + (\farg{last1}\ - \farg{first1})}.

\pnum
\complexity\ 
Exactly
\tcode{\farg{last1}\ - \farg{first1}}\
applications of
\farg{op}\ or \farg{binary_op}.

\pnum
\notes\ 
\farg{result}\ may be equal to \farg{first}\
in case of unary transform,
or to \farg{first1}\ or \farg{first2}\
in case of binary transform.
\end{itemdescr}

\rSec2[alg.replace]{Replace}

\index{replace@\tcode{replace}}%
\index{replace_if@\tcode{replace_if}}%
\color{addclr}\begin{itemdecl}
template<MutableForwardIterator Iter, class T>
  requires EqualityComparable<Iter::reference, T> && CopyAssignable<Iter::reference, T>
  void replace(Iter @\farg{first}@, Iter @\farg{last}@,
               const T& @\farg{old_value}@, const T& @\farg{new_value}@);

@\textcolor{addclr}{template}@<MutableForwardIterator Iter, Predicate<auto, Iter::reference> Pred, class T>
  requires CopyAssignable<Iter::reference, T>
  void replace_if(Iter @\farg{first}@, Iter @\farg{last}@,
                  Pred @\farg{pred}@, const T& @\farg{new_value}@);
\end{itemdecl}\color{black}

\begin{itemdescr}
\pnum
\requires\ 
The expression
\tcode{*\farg{first}\ = \farg{new_value}}\
must be valid.

\pnum
\effects\ 
Substitutes elements referred by the iterator
\tcode{i}\
in the range \range{\farg{first}}{\farg{last}}\
with \farg{new_value},
when the following corresponding conditions hold:
\tcode{*i == \farg{old_value}}, \tcode{\farg{pred}(*i) != false}.

\pnum
\complexity\ 
Exactly
\tcode{\farg{last}\ - \farg{first}}\
applications of the corresponding predicate.
\end{itemdescr}

\index{replace_copy@\tcode{replace_copy}}%
\index{replace_copy_if@\tcode{replace_copy_if}}%
\color{addclr}\begin{itemdecl}
template<InputIterator InIter, OutputIterator<auto, InIter::value_type> OutIter, class T>
  requires CopyAssignable<OutIter::reference, T> && EqualityComparable<InIter::reference, T>
  OutIter replace_copy(InIter @\farg{first}@, InIter @\farg{last}@,
                       OutIter @\farg{result}@,
                       const T& @\farg{old_value}@, const T& @\farg{new_value}@);

template<InputIterator InIter, OutputIterator<auto, InIter::value_type> OutIter, 
         Predicate<auto, InIter::reference> Pred, class T>
  requires CopyAssignable<OutIter::reference, T>
  OutIter replace_copy_if(InIter @\farg{first}@, InIter @\farg{last}@,
                          OutIter @\farg{result}@,
                          Pred @\farg{pred}@, const T& @\farg{new_value}@);
\end{itemdecl}\color{black}

\begin{itemdescr}
\pnum
\requires\ 
The results of the expressions
\tcode{*\farg{first}}\
and
\farg{new_value}\
shall be writable to the
\farg{result}\
output iterator.
The ranges
\range{\farg{first}}{\farg{last}}\
and
\range{\farg{result}}{\farg{result}\ + (\farg{last}\ - \farg{first})}\
shall not overlap.

\pnum
\effects\ 
Assigns to every iterator
\tcode{i}\
in the
range
\range{\farg{result}}{\farg{result} + (\farg{last} - \farg{first})}\
either
\tcode{\farg{new_value}}\
or
\tcode{*\brk(\farg{first} + (i - \farg{result}))}\
depending on whether the following corresponding conditions hold:

\begin{codeblock}
*(@\farg{first}@ + (i - @\farg{result}@)) == @\farg{old_value}@
@\farg{pred}@(*(@\farg{first}@ + (i - @\farg{result}@))) != false
\end{codeblock}

\pnum
\returns\ 
\tcode{\farg{result}\ + (\farg{last}\ - \farg{first})}.

\pnum
\complexity\ 
Exactly
\tcode{\farg{last}\ - \farg{first}}\
applications of the corresponding predicate.
\end{itemdescr}

\rSec2[alg.fill]{Fill}

\index{fill@\tcode{fill}}%
\index{fill_n@\tcode{fill_n}}%
\color{addclr}\begin{itemdecl}
template<MutableForwardIterator Iter, class T>
  requires CopyAssignable<Iter::reference, T>
  void fill(Iter @\farg{first}@, Iter @\farg{last}@, const T& @\farg{value}@);

template<class Iter, Integral Size, class T>
  requires OutputIterator<Iter, T>
  void fill_n(Iter @\farg{first}@, Size @\farg{n}@, const T& @\farg{value}@);
\end{itemdecl}\color{black}

\begin{itemdescr}
\pnum
\removedConcepts{Requires: The expression value shall be writable to
  the output iterator. The type Size is convertible to integral type~(4.7,
  12.3).}\efootnote{The ``Convertible to integral type'' requirements
  are very odd. We can model them with concepts if we need, but I've
  chosen the simpler route of requiring the \tcode{Size} parameter to model
  \tcode{Integral}. This could break existing code, although it
  appears that such existing code would fail to compile with at least
  libstdc++.} 

\pnum
\effects\ 
The first algorithm assigns \farg{value}\ through all the iterators in the range
\range{\farg{first}}{\farg{last}}. The second algorithm assigns \farg{value}\
through all the iterators in the range \range{\farg{first}}{\farg{first}\ + \farg{n}}
if \farg{n}\ is positive, otherwise it does nothing.

\textcolor{black}{\pnum}
\complexity\ 
Exactly
\tcode{\farg{last}\ - \farg{first}},
\farg{n}, or 0 assignments, respectively.
\end{itemdescr}

\rSec2[alg.generate]{Generate}

\index{generate@\tcode{generate}}%
\index{generate_n@\tcode{generate_n}}%
\color{addclr}\begin{itemdecl}
template<MutableForwardIterator Iter, Callable Generator>
  requires CopyAssignable<Iter::reference, Generator::result_type>
  void generate(Iter @\farg{first}@, Iter @\farg{last}@,
                Generator @\farg{gen}@);

template<class Iter, Integral Size, Callable Generator>
  requires OutputIterator<Iter, Generator::result_type>
  void generate_n(Iter @\farg{first}@, Size @\farg{n}@, Generator @\farg{gen}@);
\end{itemdecl}\color{black}

\begin{itemdescr}
\pnum
\effects\ 
The first algorithm invokes the function object \farg{gen}\ and assigns the return
value of \farg{gen}\ through all the iterators in the range
\range{\farg{first}}{\farg{last}}. The second algorithm invokes the function object
\farg{gen}\ and assigns the return value of \farg{gen}\ through all the iterators in
the range \range{\farg{first}}{\farg{first} + \farg{n}} if \farg{n}\ is positive,
otherwise it does nothing.

\pnum
\removedConcepts{Requires: gen takes no arguments, Size is convertible
  to integral type~(4.7, 
  12.3).}\efootnote{The ``Convertible to integral type'' requirements
  are very odd. We can model them with concepts if we need, but I've
  chosen the simpler route of requiring the \tcode{Size} parameter to model
  \tcode{Integral}. This could break existing code, although it
  appears that such existing code would fail to compile with at least
  libstdc++.} 

\pnum
\complexity\ 
Exactly
\tcode{\farg{last}\ - \farg{first}},
\farg{n}, or 0
invocations of \farg{gen}\ and assignments, respectively.
\end{itemdescr}

\rSec2[alg.remove]{Remove}

\index{remove@\tcode{remove}}%
\index{remove_if@\tcode{remove_if}}%
\color{addclr}\begin{itemdecl}
template<ForwardIterator Iter, class T>
  requires MoveAssignable<Iter::reference> && EqualityComparable<Iter::reference, T>
  Iter remove(Iter @\farg{first}@, Iter @\farg{last}@,
              const T& @\farg{value}@);

template<ForwardIterator Iter, Predicate<auto, Iter::reference> Pred>
  requires MoveAssignable<Iter::reference>
  Iter remove_if(Iter @\farg{first}@, Iter @\farg{last}@,
                 @\textcolor{addclr}{}@Pred @\farg{pred}@);
\end{itemdecl}\color{black}

\begin{itemdescr}
\pnum
\removedConcepts{Requires: The type of *first shall satisfy the
  MoveAssignable requirements (Table 35).}

\textcolor{black}{\pnum}
\effects\ 
Eliminates all the elements referred to by iterator
\tcode{i}\
in the range \range{\farg{first}}{\farg{last}}\
for which the following corresponding conditions hold:
\tcode{*i == \farg{value}, \farg{pred}(*i) != false}.

\pnum
\returns\ 
The end of the resulting range.

\textcolor{black}{\pnum}
\notes\ 
Stable.

\pnum
\textcolor{black}{}\complexity\ 
Exactly
\tcode{\farg{last} - \farg{first}}\
applications of the corresponding predicate.
\end{itemdescr}

\index{remove_copy@\tcode{remove_copy}}%
\index{remove_copy_if@\tcode{remove_copy_if}}%
\color{addclr}\begin{itemdecl}
template<InputIterator InIter, OutputIterator<auto, InIter::value_type> OutIter, class T>
  requires EqualityComparable<InIter::reference, T>
  OutIter remove_copy(InIter @\farg{first}@, InIter @\farg{last}@,
                      OutIter @\farg{result}@, const T& @\farg{value}@);

template<InputIterator InIter, OutputIterator<auto, InIter::value_type> OutIter, 
         Predicate<auto, InIter::reference> Pred>
  OutIter remove_copy_if(InIter @\farg{first}@, InIter @\farg{last}@,
                         OutIter @\farg{result}@, Pred @\farg{pred}@);
\end{itemdecl}\color{black}

\begin{itemdescr}
\pnum
\requires\ 
\removedConcepts{Type T is EqualityComparable~(20.1.1).}
The ranges
\range{\farg{first}}{\farg{last}}
and
\range{\farg{result}}{\farg{result} + (\farg{last} - \farg{first})}\
shall not overlap.

\pnum
\effects\ 
Copies all the elements referred to by the iterator
\tcode{i}\
in the range
\range{\farg{first}}{\farg{last}}\
for which the following corresponding conditions do not hold:
\tcode{*i == \farg{value}, \farg{pred}(*i) != false}.

\textcolor{black}{\pnum}
\returns\ 
The end of the resulting range.

\pnum
\complexity\ 
Exactly
\tcode{\farg{last}\ - \farg{first}}\
applications of the corresponding predicate.

\pnum
\notes\ 
Stable.
\end{itemdescr}

\rSec2[alg.unique]{Unique}

\index{unique@\tcode{unique}}%
\color{addclr}\begin{itemdecl}
template<ForwardIterator Iter>
  requires MoveAssignable<Iter::reference> && EqualityComparable<Iter::reference>
  Iter unique(Iter @\farg{first}@, Iter @\farg{last}@);

template<ForwardIterator Iter, Predicate<auto, Iter::reference, Iter::reference> Pred>
  requires MoveAssignable<Iter::reference>
  Iter unique(Iter @\farg{first}@, Iter @\farg{last}@,
              Pred @\farg{pred}@);
\end{itemdecl}\color{black}

\begin{itemdescr}
\pnum
\effects\ 
For a nonempty range, eliminates all but the first element from every
consecutive group of equivalent elements referred to by the iterator
\tcode{i}\
in the range
\range{\farg{first} + 1}{\farg{last}}\
for which the following conditions hold:
\tcode{*(i - 1) == *i}\
or
\tcode{\farg{pred}(*(i - 1), *i) != false}.

\pnum
\requires\
The comparison function shall be an equivalence relation.

\pnum
\returns\ 
The end of the resulting range.

\pnum
\complexity\ 
For nonempty ranges, exactly
\tcode{(\farg{last} - \farg{first}) - 1}\
applications of the corresponding predicate.
\end{itemdescr}

\index{unique_copy@\tcode{unique_copy}}%
\color{addclr}
\begin{itemdecl}
template<InputIterator InIter, OutputIterator<auto, InIter::value_type> OutIter>
  requires EqualityComparable<InIter::value_type> && CopyAssignable<InIter::value_type> && 
           CopyConstructible<InIter::value_type> && !ForwardIterator<InIter> && 
           !MutableForwardIterator<OutIter>
  @\textcolor{addclr}{OutIter}@
    unique_copy(InIter @\farg{first}@, InIter @\farg{last}@,
                OutIter @\farg{result}@);

template<ForwardIterator InIter, OutputIterator<auto, InIter::value_type> OutIter>
  requires EqualityComparable<InIter::reference>
  OutIter unique_copy(InIter @\farg{first}@, InIter @\farg{last}@,
                      OutIter @\farg{result}@);

@\textcolor{addclr}{template}@<InputIterator InIter, MutableForwardIterator OutIter>
  requires EqualityComparable<OutIter::reference, InIter::value_type> &&
           CopyAssignable<OutIter::reference, InIter::reference> &&
           !ForwardIterator<InIter>
  OutIter unique_copy(InIter @\farg{first}@, InIter @\farg{last}@,
                      OutIter @\farg{result}@);

template<InputIterator InIter, OutputIterator<auto, InIter::value_type> OutIter,
         Predicate<auto, InIter::value_type, InIter::value_type> Pred>
  requires CopyAssignable<InIter::value_type> && CopyConstructible<InIter::value_type> &&
           !ForwardIterator<InIter> && !MutableForwardIterator<OutIter>
  OutIter unique_copy(InIter @\farg{first}@, InIter @\farg{last}@,
                      @\textcolor{addclr}{OutIter}@ @\farg{result}@, Pred @\farg{pred}@);

template<ForwardIterator InIter, OutputIterator<auto, InIter::value_type> OutIter,
         Predicate<auto, InIter::reference, InIter::reference> Pred>
  OutIter unique_copy(InIter @\farg{first}@, InIter @\farg{last}@,
                      OutIter @\farg{result}@);

template<InputIterator InIter, MutableForwardIterator OutIter,
         Predicate<auto, OutIter::reference, InIter::reference> Pred>
  requires CopyAssignable<OutIter::reference, InIter::reference> &&
           !ForwardIterator<InIter>
  OutIter unique_copy(InIter @\farg{first}@, InIter @\farg{last}@,
                      OutIter @\farg{result}@, Pred @\farg{pred}@);
\end{itemdecl}
\color{black}

\editorial{We assume (and require!) the proposed resolution to DR 538,
  which adds the \tcode{CopyAssignable} requirement.}

\editorial{Note that we have split the two signatures of
  \tcode{unique_copy} into six signatures, to cover the actual
  variants required in the implementation.}

\begin{itemdescr}
\pnum
\requires\ 
The ranges
\range{\farg{first}}{\farg{last}}\
and
\range{\farg{result}}{\farg{result}+(\farg{last}-\farg{first})}\
shall not overlap. \removedConcepts{The expression
*result = *first
shall be valid. If neither
InputIterator
nor
OutputIterator
meets the requirements of forward iterator then the value type of
InputIterator
shall be CopyConstructible (20.1.3).
Otherwise CopyConstructible is not required.}

\pnum
\textcolor{black}{}\effects\ 
Copies only the first element from every consecutive group of equal elements referred to by
the iterator \color{black}
\tcode{i}\
in the range
\range{\farg{first}}{\farg{last}}\
for which the following corresponding conditions hold:
\tcode{*i == *(i - 1)}\
or
\tcode{\farg{pred}(*i, *(i - 1)) != false}.

\pnum
\returns\ 
The end of the resulting range.

\pnum
\complexity\ 
For nonempty ranges, exactly
\tcode{\farg{last}\ - \farg{first} - 1}\
applications of the corresponding predicate.
\end{itemdescr}

\rSec2[alg.reverse]{Reverse}

\index{reverse@\tcode{reverse}}%
\color{addclr}\begin{itemdecl}
template<MutableBidirectionalIterator Iter>
  requires SameType<Iter::value_type&, Iter::reference> && Swappable<Iter::value_type>
  void reverse(Iter @\farg{first}@, Iter @\farg{last}@);
\end{itemdecl}\color{black}

\begin{itemdescr}
\pnum
\effects\ 
For each non-negative integer
\tcode{i <= (\farg{last}\ - \farg{first})/2},
applies
\tcode{iter_swap}\
to all pairs of iterators
\tcode{\farg{first}\ + i, (\farg{last}\ - i) - 1}.

\pnum
\removedConcepts{Requires: 
The type of *first shall satisfy the
Swappable requirements (20.1.4).}

\pnum
\complexity\ 
Exactly
\tcode{(\farg{last}\ - \farg{first})/2}\
swaps.
\end{itemdescr}

\index{reverse_copy@\tcode{reverse_copy}}%
\color{addclr}\begin{itemdecl}
template<BidirectionalIterator InIter, OutputIterator<auto, InIter::value_type> OutIter>
  OutIter reverse_copy(InIter @\farg{first}@,
                       InIter @\farg{last}@, OutIter @\farg{result}@);
\end{itemdecl}\color{black}

\begin{itemdescr}
\pnum
\effects\ 
Copies the range
\range{\farg{first}}{\farg{last}}\
to the range
\range{\farg{result}}{\farg{result}+(\farg{last}-\farg{first})}\
such that
for any non-negative integer
\tcode{i < (\farg{last}\ - \farg{first})}\
the following assignment takes place:
\tcode{*(\farg{result}\ + (\farg{last}\ - \farg{first}) - i) = *(\farg{first}\ + i)}.

\pnum
\requires\ 
The ranges
\range{\farg{first}}{\farg{last}}\
and
\range{\farg{result}}{\farg{result}+(\farg{last}-\farg{first})}\
shall not overlap.

\pnum
\returns\ 
\tcode{\farg{result}\ + (\farg{last}\ - \farg{first})}.

\pnum
\complexity\ 
Exactly
\tcode{\farg{last}\ - \farg{first}}\
assignments.
\end{itemdescr}

\rSec2[alg.rotate]{Rotate}

\index{rotate@\tcode{rotate}}%
\color{addclr}\begin{itemdecl}
template<ForwardIterator Iter>
  requires SameType<Iter::value_type&, Iter::reference> && Swappable<Iter::value_type>
  void rotate(Iter @\farg{first}@, Iter @\farg{middle}@,
              Iter @\farg{last}@);
\end{itemdecl}\color{black}

\begin{itemdescr}
\pnum
\effects\ 
For each non-negative integer
\tcode{i < (\farg{last}\ - \farg{first})},
places the element from the position
\tcode{\farg{first}\ + i}\
into position
\tcode{\farg{first}\ + (i + (\farg{last}\ - \farg{middle})) \% (\farg{last}\ - \farg{first})}.

\pnum
\notes\ 
This is a left rotate.

\pnum
\requires\ 
\range{\farg{first}}{\farg{middle}}\
and
\range{\farg{middle}}{\farg{last}}\
are valid ranges.
\removedConcepts{The type of *first shall satisfy the
Swappable requirements (20.1.4), the \mbox{\tcode{MoveConstructible}} requirements (Table~33), and the \mbox{\tcode{MoveAssignable}} requirements (Table~35).}

\pnum
\complexity\ 
At most
\tcode{\farg{last}\ - \farg{first}}\
swaps.
\end{itemdescr}

\index{rotate_copy@\tcode{rotate_copy}}%
\color{addclr}\begin{itemdecl}
template<ForwardIterator InIter, OutputIterator<auto, InIter::value_type> OutIter>
  OutIter rotate_copy(InIter @\farg{first}@, InIter @\farg{middle}@,
                      InIter @\farg{last}@, OutIter @\farg{result}@);
\end{itemdecl}\color{black}

\begin{itemdescr}
\pnum
\effects\ 
Copies the range
\range{\farg{first}}{\farg{last}}\
to the range
\range{\farg{result}}{\farg{result}\ + (\farg{last}\ - \farg{first})}\
such that for each non-negative integer
\tcode{i < (\farg{last}\ - \farg{first})}\
the following assignment takes place:
\tcode{*(\farg{result}\ + i) =  *(\farg{first} +
(i + (\farg{middle}\ - \farg{first})) \% (\farg{last}\ - \farg{first}))}.

\pnum
\returns\ 
\tcode{\farg{result}\ + (\farg{last}\ - \farg{first})}.

\pnum
\requires\ 
The ranges
\range{\farg{first}}{\farg{last}}\
and
\range{\farg{result}}{\farg{result}\ + (\farg{last}\ - \farg{first})}\
shall not overlap.

\pnum
\complexity\ 
Exactly
\tcode{\farg{last}\ - \farg{first}}\
assignments.
\end{itemdescr}

\rSec2[alg.random.shuffle]{Random shuffle}

\index{random_shuffle@\tcode{random_shuffle}}%
\color{addclr}\begin{itemdecl}
template<MutableRandomAccessIterator Iter>
  requires SameType<Iter::value_type&, Iter::reference> && Swappable<Iter::value_type>
  void random_shuffle(Iter @\farg{first}@,
                      Iter @\farg{last}@);

template<MutableRandomAccessIterator Iter, Callable<auto, Iter::difference_type> Rand>
  requires SameType<Iter::value_type&, Iter::reference> && Swappable<Iter::value_type> && 
           Convertible<Rand::result_type, Iter::difference_type>
  void random_shuffle(Iter @\farg{first}@,
                      Iter @\farg{last}@,
                      Rand&& @\farg{rand}@);

template<class RandomAccessIterator, class UniformRandomNumberGenerator>
  void random_shuffle(RandomAccessIterator @\farg{first}@,
                      RandomAccessIterator @\farg{last}@, 
                      UniformRandomNumberGenerator& @\farg{rand}@);
\end{itemdecl}\color{black}

\editorial{TODO: We do not yet have the
\tcode{UniformRandomNumberGenerator} concept, so we leave the third
\tcode{random_shuffle} without concept constraints for now.}

\begin{itemdescr}
\pnum
\effects\ 
Shuffles the elements in the range
\range{\farg{first}}{\farg{last}}\
with uniform distribution.

\pnum
\removedConcepts{Requires: 
The type of *first shall satisfy the
Swappable requirements (20.1.4).}

\pnum
\complexity\ 
Exactly
\tcode{(\farg{last}\ - \farg{first}) - 1}\
swaps.

\pnum
\notes\ 
The underlying source of random numbers for the first form of the function
is implementation-defined. An implementation may use the
\tcode{rand}\
function from the standard C library.
The second form of the function
takes a random number generating function object
\farg{rand}
\removedConcepts{such that
if
n 
is an argument for rand, with a positive value, that has
type
iterator_traits<RandomAccessIterator>::difference_type,
then
rand(n)
returns a randomly chosen value,
which lies in the interval (0,n],
and which has a type that is convertible to
iterator_traits<RandomAccessIterator>:: difference_type}.
\end{itemdescr}

\rSec2[alg.partitions]{Partitions}

\index{partition@\tcode{partition}}%
\color{addclr}\begin{itemdecl}
@\textcolor{addclr}{}@template<BidirectionalIterator Iter, Predicate<auto, Iter::reference> Pred>
  @\color{addclr}@requires SameType<Iter::value_type&, Iter::reference> && Swappable<Iter::value_type>
  Iter partition(Iter @\farg{first}@, Iter @\farg{last}@, Pred @\farg{pred}@);
\end{itemdecl}\color{black}

\begin{itemdescr}
\pnum
\effects\ 
Places all the elements in the range
\range{\farg{first}}{\farg{last}}\
that satisfy \farg{pred}\
before all the elements that do not satisfy it.

\pnum
\returns\ 
An iterator
\tcode{i}\
such that for any iterator
\tcode{j}\
in the range
\range{\farg{first}}{i}\
\tcode{\farg{pred}(*j) != false},
and for any iterator
\tcode{k}\
in the range
\range{i}{\farg{last}},
\tcode{\farg{pred}(*k) == false}.

\pnum
\removedConcepts{Requires:
The type of *first shall satisfy the
Swappable requirements (20.1.4),
the \mbox{\tcode{MoveConstructible}} requirements
(Table~\mbox{\ref{moveconstructible}}), and the
the \mbox{\tcode{MoveAssignable}} requirements
(Table~\mbox{\ref{moveassignable}}).}

\pnum
\complexity\ 
At most
\tcode{(\farg{last}\ - \farg{first})/2}\
swaps.
Exactly
\tcode{\farg{last}\ - \farg{first}}\
applications of the predicate are done.
\end{itemdescr}

\index{stable_partition@\tcode{stable_partition}}%
\color{addclr}\begin{itemdecl}
template<MutableBidirectionalIterator Iter, Predicate<auto, Iter::reference> Pred>
  requires SameType<Iter::value_type&, Iter::reference> && Swappable<Iter::value_type> &&
           MoveConstructible<Iter::value_type> && MoveAssignable<Iter::value_type>
  Iter stable_partition(Iter @\farg{first}@, Iter @\farg{last}@, Pred @\farg{pred}@);
\end{itemdecl}\color{black}

\begin{itemdescr}
\pnum
\effects\ 
Places all the elements in the range
\range{\farg{first}}{\farg{last}}\
that satisfy \farg{pred}\ before all the
elements that do not satisfy it.

\pnum
\returns\ 
An iterator
\tcode{i}\
such that for any iterator
\tcode{j}\
in the range
\range{\farg{first}}{i},
\tcode{\farg{pred}(*j) != false},
and for any iterator
\tcode{k}\
in the range
\range{i}{\farg{last}},
\tcode{\farg{pred}(*k) == false}.
The relative order of the elements in both groups is \textcolor{black}{preserved}.

\pnum
\removedConcepts{Requires:
The type of *first shall satisfy the
Swappable requirements (20.1.4),
the \mbox{\tcode{MoveConstructible}} requirements
(Table~\mbox{\ref{moveconstructible}}), and the
the \mbox{\tcode{MoveAssignable}} requirements
(Table~\mbox{\ref{moveassignable}}).}

\pnum
\complexity\ 
At most
\tcode{(\farg{last}\ - \farg{first}) * log(\farg{last}\ - \farg{first})}\
swaps, but only linear number of swaps if there is enough extra memory.
Exactly
\tcode{\farg{last}\ - \farg{first}}\
applications of the predicate.
\end{itemdescr}

\rSec1[alg.sorting]{Sorting and related operations}

\rSec2[alg.sort]{Sorting}

\rSec3[lib.sort]{\tcode{sort}}

\index{sort@\tcode{sort}}%
\color{addclr}\begin{itemdecl}
template<MutableRandomAccessIterator Iter>
  requires SameType<Iter::value_type&, Iter::reference> && Swappable<Iter::value_type> && 
           MoveConstructible<Iter::value_type> && MoveAssignable<Iter::value_type> &&
           LessThanComparable<Iter::value_type>
  void sort(Iter @\farg{first}@, Iter @\farg{last}@);

template<MutableRandomAccessIterator Iter, Predicate<auto, Iter::value_type, Iter::value_type> Compare>
  requires SameType<Iter::value_type&, Iter::reference> && Swappable<Iter::value_type> && 
           MoveConstructible<Iter::value_type> && MoveAssignable<Iter::value_type>
  void sort(Iter @\farg{first}@, Iter @\farg{last}@,
            Compare @\farg{comp}@);
\end{itemdecl}\color{black}

\begin{itemdescr}
\pnum
\effects\ 
Sorts the elements in the range
\range{\farg{first}}{\farg{last}}.

\pnum
\removedConcepts{Requires:
The type of *first shall satisfy the
Swappable requirements (37),
the \mbox{\tcode{MoveConstructible}} requirements
(Table~33), and the
the \mbox{\tcode{MoveAssignable}} requirements
(Table~35).}

\pnum
\complexity\ 
Approximately $N \log(N)$
(where
\tcode{$N$ == \farg{last} - \farg{first}})
comparisons on the average.%
\footnote{
If the worst case behavior is important
\tcode{stable_sort()}~(\ref{lib.stable.sort}) or
\tcode{partial_sort()}~(\ref{lib.partial.sort}) should be used.
}
\end{itemdescr}

\rSec3[lib.stable.sort]{\tcode{stable_sort}}

\index{stable_sort@\tcode{stable_sort}}%
\color{addclr}\begin{itemdecl}
template<MutableRandomAccessIterator Iter>
  requires SameType<Iter::value_type&, Iter::reference> && Swappable<Iter::value_type> && 
           MoveConstructible<Iter::value_type> && MoveAssignable<Iter::value_type> &&
           LessThanComparable<Iter::value_type>
  void stable_sort(Iter @\farg{first}@, Iter @\farg{last}@);

template<MutableRandomAccessIterator Iter, Predicate<auto, Iter::value_type, Iter::value_type> Compare>
  requires SameType<Iter::value_type&, Iter::reference> && Swappable<Iter::value_type> && 
           MoveConstructible<Iter::value_type> && MoveAssignable<Iter::value_type>
  void stable_sort(Iter @\farg{first}@, Iter @\farg{last}@,
                   Compare @\farg{comp}@);
\end{itemdecl}\color{black}

\begin{itemdescr}
\pnum
\effects\ 
Sorts the elements in the range \range{\farg{first}}{\farg{last}}.

\pnum
\removedConcepts{Requires:
The type of *first shall satisfy the
Swappable requirements (20.1.4),
the \mbox{\tcode{MoveConstructible}} requirements
(Table~33), and the
the \mbox{\tcode{MoveAssignable}} requirements
(Table~35).}

\pnum
\complexity\ 
It does at most $N \log^2(N)$
(where
\tcode{$N$ == \farg{last} - \farg{first}})
comparisons; if enough extra memory is available, it is
$N \log(N)$.

\pnum
\notes\ 
Stable.
\end{itemdescr}

\rSec3[lib.partial.sort]{\tcode{partial_sort}}

\index{partial_sort@\tcode{partial_sort}}%
\color{addclr}\begin{itemdecl}
template<MutableRandomAccessIterator Iter>
  requires SameType<Iter::value_type&, Iter::reference> && Swappable<Iter::value_type> && 
           MoveConstructible<Iter::value_type> && MoveAssignable<Iter::value_type> &&
           LessThanComparable<Iter::value_type>
  void partial_sort(Iter @\farg{first}@,
                    Iter @\farg{middle}@,
                    Iter @\farg{last}@);

template<MutableRandomAccessIterator Iter, Predicate<auto, Iter::value_type, Iter::value_type> Compare>
  requires SameType<Iter::value_type&, Iter::reference> && Swappable<Iter::value_type> && 
           MoveConstructible<Iter::value_type> && MoveAssignable<Iter::value_type>
  void partial_sort(Iter @\farg{first}@,
                    Iter @\farg{middle}@,
                    Iter @\farg{last}@,
                    Compare @\farg{comp}@);
\end{itemdecl}\color{black}

\begin{itemdescr}
\pnum
\effects\ 
Places the first
\tcode{\farg{middle}\ - \farg{first}}\
sorted elements from the range
\range{\farg{first}}{\farg{last}}
into the range
\range{\farg{first}}{\farg{middle}}.
The rest of the elements in the range
\range{\farg{middle}}{\farg{last}}\
are placed in an unspecified order.
\index{unspecified}%

\pnum
\removedConcepts{Requires:
The type of *first shall satisfy the
Swappable requirements (20.1.4),
the \mbox{\tcode{MoveConstructible}} requirements
(Table~33), and the
the \mbox{\tcode{MoveAssignable}} requirements
(Table~35).}

\pnum
\complexity\ 
It takes approximately
\tcode{(\farg{last}\ - \farg{first}) * log(\farg{middle}\ - \farg{first})}\
comparisons.
\end{itemdescr}

\rSec3[lib.partial.sort.copy]{\tcode{partial_sort_copy}}

\index{partial_sort_copy@\tcode{partial_sort_copy}}%
\color{addclr}\begin{itemdecl}
template<InputIterator InIter, MutableRandomAccessIterator RAIter>
  requires CopyAssignable<RAIter::reference, InIter::value_type> &&
           SameType<RAIter::value_type&, RAIter::reference> && Swappable<RAIter::value_type> && 
           MoveConstructible<RAIter::value_type> && MoveAssignable<RAIter::value_type> &&
           LessThanComparable<InIter::value_type, RAIter::value_type> &&
           LessThanComparable<RAIter::value_type, InIter::value_type> &&
           LessThanComparable<RAIter::value_type>
  RAIter partial_sort_copy(InIter @\farg{first}@, InIter @\farg{last}@,
                           RAIter @\farg{result_first}@, RAIter @\farg{result_last}@);

template<InputIterator InIter, MutableRandomAccessIterator RAIter, class Compare>
  requires CopyAssignable<RAIter::reference, InIter::value_type> &&
           SameType<RAIter::value_type&, RAIter::reference> && Swappable<RAIter::value_type> && 
           MoveConstructible<RAIter::value_type> && MoveAssignable<RAIter::value_type> &&
           Predicate<Compare, InIter::value_type, RAIter::value_type> &&
           Predicate<Compare, RAIter::value_type, InIter::value_type> &&
           Predicate<Compare, RAIter::value_type>
  RAIter partial_sort_copy(InIter @\farg{first}@, InIter @\farg{last}@,
                           RAIter @\farg{result_first}@, RAIter @\farg{result_last}@,
                           Compare @\farg{comp}@);
\end{itemdecl}\color{black}

\begin{itemdescr}
\pnum
\effects\ 
Places the first
\tcode{min(\farg{last}\ - \farg{first}, \farg{result_last}\ - \farg{result_first})}\
sorted elements into the range
\range{\farg{result_first}}{\farg{result_first}\ + min(\farg{last}\ - \farg{first}, \farg{result_last}\ - \farg{result_first})}.

\pnum
\returns\ 
The smaller of:
\farg{result_last}\ or
\tcode{\farg{result_first}\ + (\farg{last}\ - \farg{first})}.

\pnum
\removedConcepts{\mbox{\requires}
The type of \mbox{\tcode{*result_first}} shall satisfy the
Swappable requirements (37),
the \mbox{\tcode{MoveConstructible}} requirements
(Table~33), and the
the \mbox{\tcode{MoveAssignable}} requirements
(Table~35).}

\pnum
\complexity\ 
Approximately
\tcode{(\farg{last}\ - \farg{first}) * log(min(\farg{last}\ - \farg{first}, \farg{result_last}\ - \farg{result_first}))}\
comparisons.
\end{itemdescr}

\rSec3[is.sorted]{\tcode{is_sorted}}

\index{is_sorted@\tcode{is_sorted}}%
\color{addclr}
\begin{itemdecl}
template<ForwardIterator Iter>
  requires LessThanComparable<Iter::value_type>
  bool is_sorted(Iter @\farg{first}@, Iter @\farg{last}@);
\end{itemdecl}
\color{black}

\begin{itemdescr}
\pnum
\addedD{\mbox{\returns} \mbox{\tcode{is_sorted_until(first, last) == last}}}
\end{itemdescr}

\index{is_sorted@\tcode{is_sorted}}%
\color{addclr}
\begin{itemdecl}
template<ForwardIterator Iter, Predicate<auto, Iter::value_type, Iter::value_type> Compare>
  bool is_sorted(Iter @\farg{first}@, Iter @\farg{last}@,
                 Compare @\farg{comp}@);
\end{itemdecl}
\color{black}

\begin{itemdescr}
\pnum
\addedD{\mbox{\returns} \mbox{\tcode{is_sorted_until(first, last, comp) == last}}}
\end{itemdescr}

\index{is_sorted_until@\tcode{is_sorted_until}}%
\color{addclr}
\begin{itemdecl}
template<ForwardIterator Iter>
  requires LessThanComparable<Iter::value_type>
  Iter is_sorted_until(Iter @\farg{first}@, Iter @\farg{last}@);
template<ForwardIterator Iter, Predicate<auto, Iter::value_type, Iter::value_type> Compare>
  Iter is_sorted_until(Iter @\farg{first}@, Iter @\farg{last}@,
                       Compare @\farg{comp}@);
\end{itemdecl}
\color{black}

\begin{itemdescr}
\pnum
\addedD{\mbox{\returns} If \mbox{\tcode{distance(first, last) < 2}}, returns
\mbox{\tcode{last}}. Otherwise, returns
the last iterator \mbox{\tcode{i}} in \mbox{\crange{first}{last}} for which the
range \mbox{\range{first}{i}} is sorted.}

\pnum
\addedD{\mbox{\complexity} Linear.}
\end{itemdescr}

\rSec2[alg.nth.element]{Nth element}

\index{nth_element@\tcode{nth_element}}%
\color{addclr}\begin{itemdecl}
template<MutableRandomAccessIterator Iter>
  requires SameType<Iter::value_type&, Iter::reference> && Swappable<Iter::value_type> && 
           MoveConstructible<Iter::value_type> && MoveAssignable<Iter::value_type> &&
           LessThanComparable<Iter::value_type>
  void nth_element(Iter @\farg{first}@, Iter @\farg{nth}@,
                   Iter @\farg{last}@);

template<MutableRandomAccessIterator Iter, Predicate<auto, Iter::value_type, Iter::value_type> Compare>
  requires SameType<Iter::value_type&, Iter::reference> && Swappable<Iter::value_type> && 
           MoveConstructible<Iter::value_type> && MoveAssignable<Iter::value_type>
  void nth_element(Iter @\farg{first}@, Iter @\farg{nth}@,
                   Iter @\farg{last}@,  Compare @\farg{comp}@);
\end{itemdecl}\color{black}

\begin{itemdescr}
\pnum
After
\tcode{nth_element}\
the element in the position pointed to by \farg{nth}\
is the element that would be
in that position if the whole range were sorted.
Also for any iterator
\tcode{i}\
in the range
\range{\farg{first}}{\farg{nth}}
and any iterator
\tcode{j}\
in the range
\range{\farg{nth}}{\farg{last}}\
it holds that:
\tcode{!(*i > *j)}\
or
\tcode{\farg{comp}(*j, *i) == false}.

\pnum
\removedConcepts{Requires:
The type of *first shall satisfy the
Swappable requirements (20.1.4),
the \mbox{\tcode{MoveConstructible}} requirements
(Table~33), and the
the \mbox{\tcode{MoveAssignable}} requirements
(Table~35).}

\pnum
\complexity\ 
Linear on average.
\end{itemdescr}

\rSec2[alg.binary.search]{Binary search}

\pnum
All of the algorithms in this section are versions of binary search
and assume that the sequence being searched is partitioned with respect to
an expression formed by binding the search key to an argument of the
implied or explicit comparison function.
They work on non-random access iterators minimizing the number of comparisons,
which will be logarithmic for all types of iterators.
They are especially appropriate for random access iterators,
because these \textcolor{black}{}algorithms do a logarithmic number of steps
through the data structure.
For non-random access iterators they execute a linear number of steps.

\rSec3[lib.lower.bound]{\tcode{lower_bound}}

\index{lower_bound@\tcode{lower_bound}}%
\color{addclr}\begin{itemdecl}
template<ForwardIterator Iter, class T>
  requires LessThanComparable<Iter::reference, T>
  Iter lower_bound(Iter @\farg{first}@, Iter @\farg{last}@,
                   const T& @\farg{value}@);

@\textcolor{addclr}{template}@<ForwardIterator Iter, class T, Predicate<auto, Iter::reference, T> Compare>
  Iter lower_bound(Iter @\farg{first}@, Iter @\farg{last}@,
                   const T& @\farg{value}@, Compare @\farg{comp}@);
\end{itemdecl}\color{black}

\begin{itemdescr}
\pnum
\requires\ 
The elements
\tcode{e}\
of
\range{\farg{first}}{\farg{last}}\
are partitioned with respect to the expression
\tcode{e < value}\
or
\tcode{\farg{comp}(e, value)}.

\pnum
\returns\ 
The furthermost iterator
\tcode{i}\
in the range
\crange{\farg{first}}{\farg{last}}
such that for any iterator
\tcode{j}\
in the range
\range{\farg{first}}{i}\
the following corresponding conditions hold:
\tcode{*j < \farg{value}}\
or
\tcode{\farg{comp}(*j, \farg{value}) != false}.

\pnum
\complexity\ 
At most
\tcode{log(\farg{last}\ - \farg{first}) + 1}\
comparisons.
\end{itemdescr}

\rSec3[lib.upper.bound]{\tcode{upper_bound}}

\index{upper_bound@\tcode{upper_bound}}%
\color{addclr}\begin{itemdecl}
template<ForwardIterator Iter, class T>
  requires LessThanComparable<T, Iter::reference>
  Iter upper_bound(Iter @\farg{first}@, Iter @\farg{last}@,
                   const T& @\farg{value}@);

template<ForwardIterator Iter, class T, Predicate<auto, T, Iter::reference> Compare>
  Iter upper_bound(Iter @\farg{first}@, Iter @\farg{last}@,
                   const T& @\farg{value}@, Compare @\farg{comp}@);
\end{itemdecl}\color{black}

\begin{itemdescr}
\pnum
\requires\ 
The elements
\tcode{e}\
of
\range{\farg{first}}{\farg{last}}\
are partitioned with respect to the expression
\tcode{!(value < e)}\
or
\tcode{!\farg{comp}(\brk{}value, e)}.

\pnum
\returns\ 
The furthermost iterator
\tcode{i}\
in the range
\range{\farg{first}}{\farg{last}}\
such that for any iterator
\tcode{j}\
in the range
\range{\farg{first}}{i}\
the following corresponding conditions hold:
\tcode{!(value < *j)}\
or
\tcode{\farg{comp}(\farg{value}, *j) == false}.

\pnum
\complexity\ 
At most
\tcode{log(\farg{last}\ - \farg{first}) + 1}\
comparisons.
\end{itemdescr}

\rSec3[lib.equal.range]{\tcode{equal_range}}

\index{equal_range@\tcode{equal_range}}%
\color{addclr}\begin{itemdecl}
template<ForwardIterator Iter, class T>
  requires LessThanComparable<T, Iter::reference> && 
           LessThanComparable<Iter::reference, T>
  pair<Iter, Iter>
    equal_range(Iter @\farg{first}@,
                Iter @\farg{last}@, const T& @\farg{value}@);

template<ForwardIterator Iter, class T, class Compare>
  requires Predicate<Compare, T, Iter::reference> &&
           Predicate<Compare, Iter::reference, T>
  pair<Iter, Iter>
    equal_range(Iter @\farg{first}@,
                Iter @\farg{last}@, const T& @\farg{value}@,
                Compare @\farg{comp}@);
\end{itemdecl}\color{black}

\begin{itemdescr}
\pnum
\requires\ 
The elements
\tcode{e}\
of
\range{\farg{first}}{\farg{last}}\
are partitioned with respect to the expressions
\tcode{e < value}\
and
\tcode{!(value < e)}\
or
\tcode{\farg{comp}(e, value)}\
and
\tcode{!\farg{comp}(value, e)}.
Also, for all elements
\tcode{e}\
of
\tcode{[\farg{first}, \farg{last})},
\tcode{e < value}\
\textcolor{black}{}implies
\tcode{!(value < e)}\
or
\tcode{\farg{comp}(e, value)}\
implies
\tcode{!\farg{comp}(value, e)}.

\pnum
\returns\
\begin{codeblock}
    make_pair(lower_bound(@\farg{first}@, @\farg{last}@, @\farg{value}@),
              upper_bound(@\farg{first}@, @\farg{last}@, @\farg{value}@))
\end{codeblock}
or
\begin{codeblock}
    make_pair(lower_bound(@\farg{first}@, @\farg{last}@, @\farg{value}@, @\farg{comp}@),
              upper_bound(@\farg{first}@, @\farg{last}@, @\farg{value}@, @\farg{comp}@))
\end{codeblock}

\pnum
\complexity\ 
At most
\tcode{2 * log(\farg{last}\ - \farg{first}) + 1}\
comparisons.
\end{itemdescr}

\rSec3[lib.binary.search]{\tcode{binary_search}}

\index{binary_search@\tcode{binary_search}}%
\color{addclr}\begin{itemdecl}
template<ForwardIterator Iter, class T>
  requires LessThanComparable<T, Iter::reference> &&
           LessThanComparable<Iter::reference, T>
  bool binary_search(ForwardIterator @\farg{first}@, ForwardIterator @\farg{last}@,
                     const T& @\farg{value}@);

template<ForwardIterator Iter, class T, class Compare>
  requires Predicate<Compare, T, Iter::reference> &&
           Predicate<Compare, Iter::reference, T>
  bool binary_search(Iter @\farg{first}@, Iter @\farg{last}@,
                     const T& @\farg{value}@, Compare @\farg{comp}@);
\end{itemdecl}\color{black}

\begin{itemdescr}
\pnum
\requires\ 
The elements
\tcode{e}\
of
\range{\farg{first}}{\farg{last}}\
are partitioned with respect to the expressions
\tcode{e < value}\
and
\tcode{!(value < e)}\
or
\tcode{\farg{comp}(e, value)}\
and
\tcode{!\farg{comp}(value, e)}.
Also, for all elements
\tcode{e}\
of
\tcode{[\farg{first}, \farg{last})},
\tcode{e < value}\
implies
\tcode{!(value < e)}\
or
\tcode{\farg{comp}(e, value)}\
implies
\tcode{!\farg{comp}(value, e)}.

\pnum
\returns\ 
\tcode{true}\
if there is an iterator
\tcode{i}\
in the range
\range{\farg{first}}{\farg{last}}\
that satisfies the corresponding conditions:
\tcode{!(*i < \farg{value}) \&\& !(\farg{value} < *i)}
or
\tcode{\farg{comp}(*i, \farg{value}) == false \&\& \farg{comp}(\farg{value}, *i) == false}.

\pnum
\complexity\ 
At most
\tcode{log(\farg{last}\ - \farg{first}) + 2}\
comparisons.
\end{itemdescr}

\rSec2[alg.merge]{Merge}

\index{merge@\tcode{merge}}%
\color{addclr}\begin{itemdecl}
@\textcolor{addclr}{}@template<InputIterator InIter1, InputIterator InIter2,
         OutputIterator<auto, InIter1::value_type> OutIter>
  @\textcolor{addclr}{requires}@ SameType<InIter1::value_type, InIter2::value_type> &&
           LessThanComparable<InIter1::value_type>
  OutIter merge(InIter1 @\farg{first1}@, InIter1 @\farg{last1}@,
                InIter2 @\farg{first2}@, InIter2 @\farg{last2}@,
                OutIter @\farg{result}@);

@\textcolor{addclr}{template}@<InputIterator InIter1, InputIterator InIter2,
         OutputIterator<auto, InIter1::value_type> OutIter, 
         @\textcolor{addclr}{}@Predicate<auto, InIter1::value_type, InIter2::value_type> Compare>
  requires SameType<InIter1::value_type, InIter2::value_type>
  OutIter merge(InIter1 @\farg{first1}@, InIter1 @\farg{last1}@,
                InIter2 @\farg{first2}@, InIter2 @\farg{last2}@,
                OutIter @\farg{result}@, Compare @\farg{comp}@);
\end{itemdecl}\color{black}

\begin{itemdescr}
\pnum
\effects\ 
Merges two sorted ranges
\range{\farg{first1}}{\farg{last1}}\
and
\range{\farg{first2}}{\farg{last2}}\
into the range
\range{\farg{result}}{\farg{result}\ + (\farg{last1}\ - \farg{first1}) + (\farg{last2}\ - \farg{first2})}.

\pnum
The resulting range shall not overlap with either of the original ranges.
The list will be sorted in non-decreasing order according to the
ordering defined by
\tcode{\farg{comp}};
that is, for every iterator
\tcode{i}\
in
\range{\farg{first}}{\farg{last}}\
other than
\tcode{\farg{first}},
the condition
\tcode{*i < *(i - 1)}\
or
\tcode{\farg{comp}(*i, *(i - 1))}\
will be false.

\pnum
\returns\ 
\tcode{\farg{result}\ + (\farg{last1}\ - \farg{first1}) + (\farg{last2}\ - \farg{first2})}.

\pnum
\complexity\ 
At most
\tcode{(\farg{last1}\ - \farg{first1}) + (\farg{last2}\ - \farg{first2}) - 1}\
comparisons.

\pnum
\notes\ 
Stable.
\end{itemdescr}

\index{inplace_merge@\tcode{inplace_merge}}%
\color{addclr}\begin{itemdecl}
@\textcolor{addclr}{template}@<MutableBidirectionalIterator Iter>
  requires SameType<Iter::value_type&, Iter::reference> && Swappable<Iter::value_type> && 
           MoveConstructible<Iter::value_type> && MoveAssignable<Iter::value_type> &&
           LessThanComparable<Iter::value_type>
  void inplace_merge(Iter @\farg{first}@,
                     Iter @\farg{middle}@,
                     Iter @\farg{last}@);

template<MutableBidirectionalIterator Iter, 
         Predicate<auto, Iter::value_type, Iter::value_type> Compare>
  requires SameType<Iter::value_type&, Iter::reference> && Swappable<Iter::value_type> && 
           MoveConstructible<Iter::value_type> && MoveAssignable<Iter::value_type>
  void inplace_merge(Iter @\farg{first}@,
                     Iter @\farg{middle}@,
                     Iter @\farg{last}@, Compare @\farg{comp}@);
\end{itemdecl}\color{black}

\begin{itemdescr}
\pnum
\effects\ 
Merges two sorted consecutive ranges
\range{\farg{first}}{\farg{middle}}\
and
\range{\farg{middle}}{\farg{last}},
putting the result of the merge into the range
\range{\farg{first}}{\farg{last}}.
The resulting range will be in non-decreasing order;
that is, for every iterator
\tcode{i}\
in
\range{\farg{first}}{\farg{last}}\
other than
\tcode{\farg{first}},
the condition
\tcode{*i < *(i - 1)}
or, respectively,
\tcode{\farg{comp}(*i, *(i - 1))}\
will be false.

\pnum
\removedConcepts{Requires:
The type of *first shall satisfy the
Swappable requirements (20.1.4),
the \mbox{\tcode{MoveConstructible}} requirements
(Table~33), and the
the \mbox{\tcode{MoveAssignable}} requirements
(Table~35).}

\pnum
\complexity\ 
When enough additional memory is available,
\tcode{(\farg{last}\ - \farg{first}) - 1}\
comparisons.
If no additional memory is available, an algorithm with complexity
$N \log(N)$
(where
\tcode{N}
is equal to
\tcode{\farg{last}\ - \farg{first}})\
may be used.

\pnum
\notes\ 
Stable.
\end{itemdescr}

\rSec2[alg.set.operations]{Set operations on sorted structures}

\pnum
This section defines all the basic set operations on sorted structures.
They also work with
\tcode{multiset}s~(\ref{lib.multiset})
containing multiple copies of equivalent elements.
The semantics of the set operations are generalized to
\tcode{multiset}s
in a standard way by defining
\tcode{set_union()}\
to contain the maximum number of occurrences of every element,
\tcode{set_intersection()}\
to contain the minimum, and so on.

\rSec3[lib.includes]{\tcode{includes}}

\index{includes@\tcode{includes}}%
\color{addclr}\begin{itemdecl}
template<InputIterator Iter1, InputIterator Iter2>
  requires SameType<Iter1::value_type, Iter2::value_type> && LessThanComparable<Iter1::value_type>
  bool includes(Iter1 @\farg{first1}@, Iter1 @\farg{last1}@,
                Iter2 @\farg{first2}@, Iter2 @\farg{last2}@);

template<InputIterator Iter1, InputIterator Iter2, 
         Predicate<auto, Iter1::value_type, Iter2::value_type> Compare>
  requires SameType<Iter1::value_type, Iter2::value_type>
  bool includes(Iter1 @\farg{first1}@, Iter1 @\farg{last1}@,
                Iter2 @\farg{first2}@, Iter2 @\farg{last2}@,
                Compare @\farg{comp}@);
\end{itemdecl}\color{black}

\begin{itemdescr}
\pnum
\returns\ 
\tcode{true}\
if every element in the range
\range{\farg{first2}}{\farg{last2}}\
is contained in the range
\range{\farg{first1}}{\farg{last1}}.
Returns
\tcode{false}\
otherwise.

\textcolor{black}{\pnum}
\complexity\ 
At most
\tcode{2 * ((\farg{last1}\ - \farg{first1}) + (\farg{last2}\ - \farg{first2})) - 1}\
comparisons.
\end{itemdescr}

\rSec3[lib.set.union]{\tcode{set_union}}

\index{set_union@\tcode{set_union}}%
\color{addclr}\begin{itemdecl}
template<InputIterator InIter1, InputIterator InIter2,
         OutputIterator<auto, InIter1::value_type> OutIter>
  requires SameType<InIter1::value_type, InIter2::value_type> &&
        LessThanComparable<InIter1::value_type>
  OutIter set_union(InIter1 @\farg{first1}@, InIter1 @\farg{last1}@,
                    InIter2 @\farg{first2}@, InIter2 @\farg{last2}@,
                    OutIter @\farg{result}@);

template<InputIterator InIter1, InputIterator InIter2,
         OutputIterator<auto, InIter1::value_type> OutIter, 
         Predicate<auto, InIter1::value_type, InIter2::value_type> Compare>
  requires SameType<InIter1::value_type, InIter2::value_type>
  OutIter set_union(InIter1 @\farg{first1}@, InIter1 @\farg{last1}@,
                    InIter2 @\farg{first2}@, InIter2 @\farg{last2}@,
                    OutIter @\farg{result}@, Compare @\farg{comp}@);
\end{itemdecl}\color{black}

\begin{itemdescr}
\pnum
\effects\ 
Constructs a sorted union of the elements from the two ranges;
that is, the set of elements that are present in one or both of the ranges.

\pnum
\requires\ 
The resulting range shall not overlap with either of the original ranges.

\pnum
\returns\ 
The end of the constructed range.

\textcolor{black}{\pnum}
\complexity\ 
At most
\tcode{2 * ((\farg{last1}\ - \farg{first1}) + (\farg{last2}\ - \farg{first2})) - 1}\
comparisons.

\pnum
\notes\ If \range{first1}{last1}\ contains $m$ elements that are equivalent to
each other and \range{first2}{last2}\ contains $n$ elements that are equivalent
to them, then all $m$ elements from the first range shall be copied to the output
\textcolor{black}{}range, in order, and then $\max(n - m, 0)$ elements from the second range shall
be copied to the output range, in order.
\end{itemdescr}

\rSec3[lib.set.intersection]{\tcode{set_intersection}}

\index{set_intersection@\tcode{set_intersection}}%
\color{addclr}\begin{itemdecl}
template<InputIterator InIter1, InputIterator InIter2,
         OutputIterator<auto, InIter1::value_type> OutIter>
  requires SameType<InIter1::value_type, InIter2::value_type> &&
        LessThanComparable<InIter1::value_type>
  OutIter set_intersection(InIter1 @\farg{first1}@, InIter1 @\farg{last1}@,
                           InIter2 @\farg{first2}@, InIter2 @\farg{last2}@,
                           OutIter @\farg{result}@);

template<InputIterator InIter1, InputIterator InIter2,
         OutputIterator<auto, InIter1::value_type> OutIter, 
         Predicate<auto, InIter1::value_type, InIter2::value_type> Compare>
  requires SameType<InIter1::value_type, InIter2::value_type>
  OutIter set_intersection(InIter1 @\farg{first1}@, InIter1 @\farg{last1}@,
                           InIter2 @\farg{first2}@, InIter2 @\farg{last2}@,
                           OutIter @\farg{result}@, Compare @\farg{comp}@);
\end{itemdecl}\color{black}

\begin{itemdescr}
\pnum
\effects\ 
Constructs a sorted intersection of the elements from the two ranges;
that is, the set of elements that are present in both of the ranges.

\pnum
\requires\ 
The resulting range shall not overlap with either of the original ranges.

\pnum
\returns\ 
The end of the constructed range.

\pnum
\complexity\ 
At most
\tcode{2 * ((\farg{last1}\ - \farg{first1}) + (\farg{last2}\ - \farg{first2})) - 1}\
comparisons.

\pnum
\notes\ If \range{first1}{last1}\ contains $m$ elements that are equivalent to
each other and \range{first2}{last2}\ contains $n$ elements that are equivalent
to them, the first $\min(m, n)$ elements shall be copied from the first range
to the output range, in order.
\end{itemdescr}

\rSec3[lib.set.difference]{\tcode{set_difference}}

\index{set_difference@\tcode{set_difference}}%
\color{addclr}\begin{itemdecl}
template<InputIterator InIter1, InputIterator InIter2,
         OutputIterator<auto, InIter1::value_type> OutIter>
  requires SameType<InIter1::value_type, InIter2::value_type> &&
           @\textcolor{addclr}{LessThanComparable}@<InIter1::value_type>
  OutIter set_difference(InIter1 @\farg{first1}@, InIter1 @\farg{last1}@,
                         InIter2 @\farg{first2}@, InIter2 @\farg{last2}@,
                         @\textcolor{addclr}{OutIter}@ @\farg{result}@);

template<InputIterator InIter1, InputIterator InIter2,
         OutputIterator<auto, InIter1::value_type> OutIter, 
         Predicate<auto, InIter1::value_type, InIter2::value_type> Compare>
  requires SameType<InIter1::value_type, InIter2::value_type>
  OutIter set_difference(InIter1 @\farg{first1}@, InIter1 @\farg{last1}@,
                         @\textcolor{addclr}{}@InIter2 @\farg{first2}@, InIter2 @\farg{last2}@,
                         OutIter @\farg{result}@, Compare @\farg{comp}@);
\end{itemdecl}\color{black}

\begin{itemdescr}
\pnum
\effects\ 
Copies the elements of the range
\range{\farg{first1}}{\farg{last1}}\
which are not present in the range
\range{\farg{first2}}{\farg{last2}}\
to the range beginning at
\tcode{result}.
The elements in the constructed range are sorted.

\pnum
\requires\ 
The resulting range shall not overlap with either of the original ranges.

\pnum
\returns\ 
The end of the constructed range.

\pnum
\complexity\ 
At most
\tcode{2 * ((\farg{last1}\ - \farg{first1}) + (\farg{last2}\ - \farg{first2})) - 1}\
comparisons.

\pnum
\notes\
If
\range{first1}{last1}\
contains $m$
elements that are equivalent to each other and
\range{first2}{last2}\
contains $n$
elements that are equivalent to them, the last
$\max(m - n, 0)$
elements from
\range{first1}{last1}\
shall be \color{black}copied to the output range.
\end{itemdescr}

\rSec3[lib.set.symmetric.difference]{\tcode{set_symmetric_difference}}

\index{set_symmetric_difference@\tcode{set_symmetric_difference}}%
\color{addclr}\begin{itemdecl}
template<InputIterator InIter1, InputIterator InIter2,
         OutputIterator<auto, InIter1::value_type> OutIter>
  requires SameType<InIter1::value_type, InIter2::value_type> &&
        LessThanComparable<InIter1::value_type>
  OutIter set_symmetric_difference(InIter1 @\farg{first1}@, InIter1 @\farg{last1}@,
                                   InIter2 @\farg{first2}@, InIter2 @\farg{last2}@,
                                   OutIter @\farg{result}@);

template<InputIterator InIter1, InputIterator InIter2,
         OutputIterator<auto, InIter1::value_type> OutIter, 
         Predicate<auto, InIter1::value_type, InIter2::value_type> Compare>
  requires SameType<InIter1::value_type, InIter2::value_type>
  OutIter set_symmetric_difference(InIter1 @\farg{first1}@, InIter1 @\farg{last1}@,
                                   InIter2 @\farg{first2}@, InIter2 @\farg{last2}@,
                                   OutIter @\farg{result}@, Compare @\farg{comp}@);
\end{itemdecl}\color{black}

\begin{itemdescr}
\pnum
\effects\ 
Copies the elements of the range
\range{\farg{first1}}{\farg{last1}}\
which are not present in the range
\range{\farg{first2}}{\farg{last2}},
and the elements of the range
\range{\farg{first2}}{\farg{last2}}\
which are not present in the range
\range{\farg{first1}}{\farg{last1}}\
to the range beginning at
\tcode{result}.
The elements in the constructed range are sorted.

\pnum
\requires\ 
The resulting range shall not overlap with either of the original ranges.

\pnum
\returns\ 
The end of the constructed range.

\textcolor{black}{\pnum}
\complexity\ 
At most
\tcode{2 * ((\farg{last1}\ - \farg{first1}) + (\farg{last2}\ - \farg{first2})) - 1}\
comparisons.

\pnum
\notes\
If \range{first1}{last1}\ contains $m$ elements that are equivalent to each other and
\range{first2}{last2}\ contains $n$ elements that are equivalent to them, then
$|m - n|$ of those elements shall be copied to the output range: the last
$m - n$ of these elements from \range{first1}{last1}\ if $m > n$, and the last 
$n - m$ of these elements from \range{first2}{last2}\ if $m < n$.
\end{itemdescr}

\rSec2[alg.heap.operations]{Heap operations}

\pnum
A
\techterm{heap}\ 
is a particular organization of elements in a range between two random access iterators
\range{a}{b}.
Its two key properties are:

\begin{description}
\item{(1)} There is no element greater than
\tcode{*a}\
in the range and
\item{(2)} \tcode{*a}\
may be removed by
\tcode{pop_heap()},
or a new element added by
\tcode{push_heap()},
in
$\mathcal{O}(\log(N))$
time.
\end{description}

\pnum
These properties make heaps useful as priority queues.

\pnum
\tcode{make_heap()}\
converts a range into a heap and
\tcode{sort_heap()}\
turns a heap into a sorted sequence.

\rSec3[lib.push.heap]{\tcode{push_heap}}

\index{push_heap@\tcode{push_heap}}%
\color{addclr}\begin{itemdecl}
template<MutableRandomAccessIterator Iter>
  requires MoveConstructible<Iter::value_type> && MoveAssignable<Iter::value_type> &&
           LessThanComparable<Iter::value_type>
  void push_heap(Iter @\farg{first}@, Iter @\farg{last}@);

template<MutableRandomAccessIterator Iter, Predicate<auto, Iter::value_type, Iter::value_type> Compare>
  requires MoveConstructible<Iter::value_type> && MoveAssignable<Iter::value_type>
  void push_heap(Iter @\farg{first}@, Iter @\farg{last}@,
                 Compare @\farg{comp}@);
\end{itemdecl}\color{black}

\begin{itemdescr}
\pnum
\effects\ 
Places the value in the location
\tcode{\farg{last} - 1}\
into the resulting heap
\range{\farg{first}}{\farg{last}}.

\pnum
\requires\ 
The range
\range{\farg{first}}{\farg{last} - 1}\
shall be a valid heap.
\removedConcepts{The type of \mbox{\tcode{*first}} shall satisfy
the \mbox{\tcode{MoveConstructible}} requirements
(Table~\mbox{\ref{moveconstructible}}) and the
the \mbox{\tcode{MoveAssignable}} requirements
(Table~\mbox{\ref{moveassignable}}).}

\pnum
\complexity\ 
At most
\tcode{log(\farg{last}\ - \farg{first})}\
comparisons.
\end{itemdescr}

\rSec3[lib.pop.heap]{\tcode{pop_heap}}

\index{pop_heap@\tcode{pop_heap}}%
\color{addclr}\begin{itemdecl}
template<MutableRandomAccessIterator Iter>
  requires MoveConstructible<Iter::value_type> && MoveAssignable<Iter::value_type> &&
           Swappable<Iter::value_type> && LessThanComparable<Iter::value_type>
  void pop_heap(Iter @\farg{first}@, Iter @\farg{last}@);

template<MutableRandomAccessIterator Iter, Predicate<auto, Iter::value_type, Iter::value_type> Compare>
  requires MoveConstructible<Iter::value_type> && MoveAssignable<Iter::value_type> &&
           Swappable<Iter::value_type>
  void pop_heap(Iter @\farg{first}@, Iter @\farg{last}@,
                Compare @\farg{comp}@);
\end{itemdecl}\color{black}

\begin{itemdescr}
\pnum
\effects\ 
Swaps the value in the location \farg{first}\
with the value in the location
\tcode{\farg{last} - 1}\
and makes
\range{\farg{first}}{\farg{last} - 1}\
into a heap.

\pnum
\requires\ 
The range
\range{\farg{first}}{\farg{last}}\
shall be a valid heap.
\removedConcepts{
The type of *first shall satisfy the
Swappable requirements (20.1.4),
the \mbox{\tcode{MoveConstructible}} requirements
(Table~\mbox{\ref{moveconstructible}}), and the
the \mbox{\tcode{MoveAssignable}} requirements
(Table~\mbox{\ref{moveassignable}}).}

\pnum
\textcolor{black}{}\complexity\ 
At most
\tcode{2 * log(\farg{last}\ - \farg{first})}\
comparisons.
\end{itemdescr}

\rSec3[lib.make.heap]{\tcode{make_heap}}

\index{make_heap@\tcode{make_heap}}%
\color{addclr}\begin{itemdecl}
template<MutableRandomAccessIterator Iter>
  requires MoveConstructible<Iter::value_type> && MoveAssignable<Iter::value_type> &&
           LessThanComparable<Iter::value_type>
  void make_heap(Iter @\farg{first}@, Iter @\farg{last}@);

@\color{addclr}@template<MutableRandomAccessIterator Iter, Predicate<auto, Iter::value_type, Iter::value_type> Compare>
  requires MoveConstructible<Iter::value_type> && MoveAssignable<Iter::value_type>
  void make_heap(Iter @\farg{first}@, Iter @\farg{last}@,
                 Compare @\farg{comp}@);
\end{itemdecl}\color{black}

\begin{itemdescr}
\pnum
\effects\ 
Constructs a heap out of the range
\range{\farg{first}}{\farg{last}}.

\pnum
\removedConcepts{\mbox{\requires} The type of \mbox{\tcode{*first}} shall satisfy
the \mbox{\tcode{MoveConstructible}} requirements
(Table~\mbox{\ref{moveconstructible}}) and the
the \mbox{\tcode{MoveAssignable}} requirements
(Table~\mbox{\ref{moveassignable}}).}

\pnum
\complexity\ 
At most
\tcode{3 * (\farg{last}\ - \farg{first})}\
comparisons.
\end{itemdescr}

\rSec3[lib.sort.heap]{\tcode{sort_heap}}

\index{sort_heap@\tcode{sort_heap}}%
\color{addclr}\begin{itemdecl}
template<MutableRandomAccessIterator Iter>
  requires MoveConstructible<Iter::value_type> && MoveAssignable<Iter::value_type> &&
           Swappable<Iter::value_type> && LessThanComparable<Iter::value_type>
  void sort_heap(Iter @\farg{first}@, Iter @\farg{last}@);

template<MutableRandomAccessIterator Iter, Predicate<auto, Iter::value_type, Iter::value_type> Compare>
  requires MoveConstructible<Iter::value_type> && MoveAssignable<Iter::value_type> &&
           Swappable<Iter::value_type>
  void sort_heap(Iter @\farg{first}@, Iter @\farg{last}@,
                 Compare @\farg{comp}@);
\end{itemdecl}\color{black}

\begin{itemdescr}
\pnum
\effects\ 
Sorts elements in the heap
\range{\farg{first}}{\farg{last}}.

\pnum
\removedConcepts{Requires:
The type of *first shall satisfy the
Swappable requirements (20.1.4),
the \mbox{\tcode{MoveConstructible}} requirements
(Table~\mbox{\ref{moveconstructible}}), and the
the \mbox{\tcode{MoveAssignable}} requirements
(Table~\mbox{\ref{moveassignable}}).}

\pnum
\complexity\ 
At most $N \log(N)$
comparisons (where
\tcode{N == \farg{last}\ - \farg{first}}).
\end{itemdescr}

\rSec3[is.heap]{\tcode{is_heap}}

\index{is_heap@\tcode{is_heap}}%
\color{addclr}
\begin{itemdecl}
template<RandomAccessIterator Iter>
  requires LessThanComparable<Iter::value_type>
  bool is_heap(Iter first, Iter last);
\end{itemdecl}
\color{black}

\begin{itemdescr}
\pnum
\addedD{\mbox{\returns} \mbox{\tcode{is_heap_until(first, last) == last}}}
\end{itemdescr}

\color{addclr}
\begin{itemdecl}
template<RandomAccessIterator Iter, Predicate<auto, Iter::value_type, Iter::value_type> Compare>
  bool is_heap(Iter first, Iter last, Compare comp);
\end{itemdecl}
\color{black}

\begin{itemdescr}
\pnum
\addedD{\mbox{\returns} \mbox{\tcode{is_heap_until(first, last, comp) == last}}}
\end{itemdescr}

\color{addclr}
\begin{itemdecl}
template<RandomAccessIterator Iter>
  Iter is_heap_until(Iter first, Iter last);
template<RandomAccessIterator Iter, Predicate<auto, Iter::value_type, Iter::value_type> Compare>
  Iter is_heap_until(Iter first, Iter last,
                     Compare comp);
\end{itemdecl}
\color{black}

\begin{itemdescr}
\pnum
\addedD{\mbox{\returns} If \mbox{\tcode{distance(first, last) < 2}}, returns
\mbox{\tcode{last}}. Otherwise, returns
the last iterator \mbox{\tcode{i}} in \mbox{\crange{first}{last}} for which the
range \mbox{\range{first}{i}} is a heap.}

\pnum
\addedD{\mbox{\complexity} Linear.}
\end{itemdescr}

\rSec2[alg.min.max]{Minimum and maximum}

\index{min@\tcode{min}}%
\color{addclr}\begin{itemdecl}
template<LessThanComparable T> const T& min(const T& @\farg{a}@, const T& @\farg{b}@);
template<class T, Predicate<auto, T, T> Compare>
  @\addedCC{requires !SameType<T, Compare>}@
  const T& min(const T& @\farg{a}@, const T& @\farg{b}@, Compare @\farg{comp}@);
\end{itemdecl}\color{black}

\begin{itemdescr}
\pnum
\removedConcepts{Requires:
Type
T
is
LessThanComparable~(20.1.2).}

\pnum
\returns\ 
The smaller value.

\pnum
\notes\ 
Returns the first argument when the arguments are equivalent.
\end{itemdescr}

\begin{itemdecl}
@\removedCC{template<LessThanComparable T> const T\& min(const T\& \mbox{\farg{a}}, const T\& \mbox{\farg{b}}, const T\& \mbox{\farg{c}});}@
@\addedCC{template<LessThanComparable T, LessThanComparable... Args>}@
  @\addedCC{requires SameType<T, Args>...}@
  @\addedCC{const T\& min(const T\& \mbox{\farg{a}}, const Args\&... \mbox{\farg{args}});}@
\end{itemdecl}

\begin{itemdescr}
\pnum
\removedCC{\mbox{\requires} Type \mbox{\tcode{T}} is
  \mbox{\tcode{LessThanComparable}}, and all types forming
  \mbox{\tcode{Args...}} are the same as \mbox{\tcode{T}}.}

\pnum
\mbox{\returns} The smallest value in the set of all
  arguments.

\pnum
\mbox{\notes} Returns the leftmost argument when several
  arguments are equivalent to the smallest. Returns \mbox{\farg{a}} if
  \mbox{\tcode{sizeof...(Args)}} is 0.
\end{itemdescr}

\begin{itemdecl}
template<class T, class U, class... Args>
  const T& min(const T& @\farg{a}@, const U& @\farg{b}@, const Args&... @\farg{args}@);
\end{itemdecl}

\editorial{At present, we do not know how to write concept constraints
  for this variant of \mbox{\tcode{min}}, because we cannot directly
  express the idea of splitting \mbox{\tcode{Args}} into its first N-1
  arguments (all of which are the same at \mbox{\tcode{T}}) and its
  Nth argument (the binary predicate). Most likely this is possible
  with what is effectively metaprogramming of the concept system to
  ``walk'' through the arguments, but such an implementation would
  make a poor specification. For now, we leave this version
  unconstrained.}

\begin{itemdescr}
\pnum
\requires\
The types of all arguments except the last one are the same as \tcode{T}.
The last argument is a binary predicate over \tcode{T}.

\pnum
\returns\
The first element in a partial ordering of all the arguments except
the last one, where the ordering is defined by the predicate.

\pnum
\notes\
Returns the leftmost argument when several arguments are equivalent to
the first element in the ordering. Returns \farg{a} if
\tcode{sizeof...(Args)} is 0.
\end{itemdescr}

\index{max@\tcode{max}}%
\color{addclr}\begin{itemdecl}
template<LessThanComparable T> const T& max(const T& @\farg{a}@, const T& @\farg{b}@);
template<class T, Predicate<auto, T, T> Compare>
  @\addedCC{requires !SameType<T, Compare>}@
  const T& max(const T& @\farg{a}@, const T& @\farg{b}@, Compare @\farg{comp}@);
\end{itemdecl}\color{black}

\begin{itemdescr}
\pnum
\removedConcepts{Requires:
Type
T
is
LessThanComparable~(20.1.2).}

\pnum
\returns\ 
The larger value.

\pnum
\notes\ 
Returns the first argument when the arguments are equivalent.
\end{itemdescr}

\begin{itemdecl}
@\removedCC{template<LessThanComparable T> const T\& max(const T\& \mbox{\farg{a}}, const T\& \mbox{\farg{b}}, const T\& \mbox{\farg{c}});}@
@\addedCC{template<LessThanComparable T, LessThanComparable... Args>}@
  @\addedCC{requires SameType<T, Args>...}@
  @\addedCC{const T\& max(const T\& \mbox{\farg{a}}, const Args\&... \mbox{\farg{args}});}@
\end{itemdecl}

\begin{itemdescr}
\pnum
\removedCC{\mbox{\requires} Type \mbox{\tcode{T}} is
  \mbox{\tcode{LessThanComparable}}, and all types forming
  \mbox{\tcode{Args...}} are the same as \mbox{\tcode{T}}.}

\pnum
\mbox{\returns} The largest value in the set of all
  arguments.

\pnum
\mbox{\notes} Returns the leftmost argument when several
  arguments are equivalent to the largest. Returns \mbox{\farg{a}} if
  \mbox{\tcode{sizeof...(Args)}} is 0.
\end{itemdescr}

\begin{itemdecl}
template<class T, class U, class... Args>
  const T& max(const T& @\farg{a}@, const U& @\farg{b}@, const Args&... @\farg{args}@);
\end{itemdecl}

\editorial{As with the corresponding \mbox{\tcode{min}} function, we
  do not know how to write the constraints, so we leave this version
  unconstrained.}

\begin{itemdescr}
\pnum
\requires\
The types of all arguments except the last one are the same as \tcode{T}.
The last argument is a binary predicate over \tcode{T}.

\pnum
\returns\
The last element in a partial ordering of all the arguments except
the last one, where the ordering is defined by the predicate.

\pnum
\notes\
Returns the leftmost argument when several arguments are equivalent to
the last element in the ordering. Returns \farg{a} if
\tcode{sizeof...(Args)} is 0.
\end{itemdescr}

\index{minmax@\tcode{minmax}}%
\color{addclr}
\begin{itemdecl}
template<LessThanComparable T> pair<const T&, const T&> minmax(const T& @\farg{a}@, const T& @\farg{b}@);
template<class T, Predicate<auto, T, T> Compare>
  @\addedCC{requires !SameType<T, Compare>}@
  @\textcolor{addclr}{pair}@<const T&, const T&> minmax(const T& @\farg{a}@, const T& @\farg{b}@, Compare @\farg{comp}@);
\end{itemdecl}
\color{black}

\begin{itemdescr}
\pnum
\removedConcepts{\mbox{\requires}
Type
\mbox{\tcode{T}}
shall be
\mbox{\tcode{LessThanComparable}}~(\mbox{\ref{lessthancomparable}}).}

\pnum
\addedB{\mbox{\returns}
\mbox{\tcode{pair<const T\&, const T\&>(b, a)}} if \mbox{\tcode{b}} is smaller
than \mbox{\tcode{a}}, and
\mbox{\tcode{pair<const T\&, const T\&>(a, b)}} otherwise.}

\pnum
\addedB{\mbox{\notes}
Returns \mbox{\tcode{<pair<const T\&, const T\&>(a, b)}} when the arguments are equivalent.}

\pnum
\addedB{\mbox{\complexity}
Exactly one comparison.}
\end{itemdescr}

\begin{itemdecl}
@\removedCC{template<LessThanComparable T>}@
  @\removedCC{pair<const T\&, const T\&> minmax(const T\& \mbox{\farg{a}}, const T\& \mbox{\farg{b}}, const T\& \mbox{\farg{c}});}@
@\addedCC{template<LessThanComparable T, LessThanComparable... Args>}@
  @\addedCC{requires SameType<T, Args>...}@
  @\addedCC{pair<const T\&, const T\&> minmax(const T\& \mbox{\farg{a}}, const Args\&... \mbox{\farg{args}});}@
\end{itemdecl}

\begin{itemdescr}
\pnum
\removedCC{\mbox{\requires} Type \mbox{\tcode{T}} is
  \mbox{\tcode{LessThanComparable}}, and all types forming
  \mbox{\tcode{Args...}} are the same as \mbox{\tcode{T}}.} 

\pnum
\returns\
\tcode{pair<const T\&, const T\&>(x, y)}\
where \tcode{x} is the first element and \tcode{y} the last element in
a partial ordering of all the arguments.

\pnum
\notes\
\tcode{x}\
is the leftmost argument when several arguments are
equivalent to the smallest. 
\tcode{y}\
is the rightmost argument when several arguments are equivalent to the
largest. 
Returns 
\tcode{pair<const T\&, const T\&>(a, a)}\
if 
\tcode{sizeof...(Args)}\
is 0.

\pnum
\complexity\
At most (3/2)\tcode{sizeof...(Args)} applications of the corresponding
predicate.
\end{itemdescr}

\begin{itemdecl}
@\textcolor{black}{template<class T, class U, class... Args>}@
  @\textcolor{black}{pair<const T\&, const T\&> minmax(const T\& \mbox{\farg{a}}, const U\& \mbox{\farg{b}}, const Args\&... \mbox{\farg{args}});}@
\end{itemdecl}

\editorial{As with the corresponding \mbox{\tcode{min}} and
  \mbox{\tcode{max}} functions, we
  do not know how to write the constraints, so we leave this version
  unconstrained.}

\begin{itemdescr}
\pnum
\requires\
The types of all arguments except the last one are the same as
\tcode{T}. The last argument is a binary predicate over \tcode{T}.

\pnum
\returns\
\tcode{pair<const T\&, const T\&>(x, y)}\
where \tcode{x} is the first element and \tcode{y} the last element in
a partial ordering of all the arguments defined by the predicate.

\pnum
\notes\
\tcode{x} is the leftmost argument when several arguments would order
equivalent as the first in the ordering. 
\tcode{y} is the rightmost argument when several arguments would order
equivalent as the last in the ordering.
Returns \tcode{pair<const T\&, const T\&>(a, a)} if
\tcode{sizeof...(Args)} is 0.

\pnum
\complexity\
At most (3/2)\tcode{sizeof...(Args)} applications of the corresponding
predicate.
\end{itemdescr}

\index{min_element@\tcode{min_element}}%
\color{addclr}\begin{itemdecl}
template<ForwardIterator Iter>
  requires LessThanComparable<Iter::reference>
  Iter min_element(Iter @\farg{first}@, Iter @\farg{last}@);

template<ForwardIterator Iter, Predicate<auto, Iter::reference, Iter::reference> Compare>
  Iter min_element(Iter @\farg{first}@, Iter @\farg{last}@,
                   Compare @\farg{comp}@);
\end{itemdecl}\color{black}

\begin{itemdescr}
\pnum
\returns\ 
The first iterator
\tcode{i}\
in the range
\range{\farg{first}}{\farg{last}}\
such that for any iterator
\tcode{j}\
in the range
\range{\farg{first}}{\farg{last}}\
the following corresponding conditions hold:
\tcode{!(*j < *i)}\
or
\tcode{\farg{comp}(*j, *i) == false}.
Returns
\tcode{\farg{last}}\
if
\tcode{\farg{first} == \farg{last}}.

\pnum
\complexity\ 
Exactly
\tcode{max((\farg{last} - \farg{first}) - 1, 0)}\
applications of the corresponding comparisons.
\end{itemdescr}

\index{max_element@\tcode{max_element}}%
\color{addclr}\begin{itemdecl}
template<ForwardIterator Iter>
  requires LessThanComparable<Iter::reference>
  Iter max_element(Iter @\farg{first}@, Iter @\farg{last}@);

template<ForwardIterator Iter, Predicate<auto, Iter::reference, Iter::reference> Compare>
  Iter max_element(Iter @\farg{first}@, Iter @\farg{last}@,
                   Compare @\farg{comp}@);
\end{itemdecl}\color{black}

\begin{itemdescr}
\pnum
\returns\ 
The first iterator
\tcode{i}\
in the range
\range{\farg{first}}{\farg{last}}\
such that for any iterator
\tcode{j}\
in the range
\range{\farg{first}}{\farg{last}}\
the following corresponding conditions hold:
\tcode{!(*i < *j)}
or
\tcode{\farg{comp}(*i, *j) == false}.
Returns
\tcode{\farg{last}}\
if
\tcode{\farg{first} == \farg{last}}.

\pnum
\complexity\ 
Exactly
\tcode{max((\farg{last}\ - \farg{first}) - 1, 0)}\
applications of the corresponding comparisons.
\end{itemdescr}

\index{minmax_element@\tcode{minmax_element}}%
\color{addclr}
\begin{itemdecl}
template<ForwardIterator Iter>
  requires LessThanComparable<Iter::value_type>
  pair<Iter, Iter>
    minmax_element(Iter @\farg{first}@, Iter @\farg{last}@);
template<ForwardIterator Iter, Predicate<auto, Iter::value_type, Iter::value_type> Compare>
  pair<Iter, Iter>
    minmax_element(Iter @\farg{first}@, Iter @\farg{last}@, Compare @\farg{comp}@);
\end{itemdecl}
\color{black}

\begin{itemdescr}
\pnum
\addedB{\mbox{\returns}
\mbox{\tcode{make_pair(m, M)}}, where \mbox{\tcode{m}} is
\mbox{\tcode{min_element(first, last)}}
or \mbox{\tcode{min_element(first, last, comp)}}
and \mbox{\tcode{M}} is \mbox{\tcode{max_element(first, last)}}
or \mbox{\tcode{max_element(first, last, comp)}}.}

\pnum
\addedB{\mbox{\complexity}
At most
\mbox{\tcode{max(2 * (\farg{last} - \farg{first}) - 2, 0)}}
applications of the corresponding comparisons.}
\end{itemdescr}

\rSec2[alg.lex.comparison]{Lexicographical comparison}

\index{lexicographical_compare@\tcode{lexicographical_compare}}%
\color{addclr}\begin{itemdecl}
template<InputIterator Iter1, InputIterator Iter2>
  requires LessThanComparable<Iter1::reference, Iter2::reference> &&
           LessThanComparable<Iter2::reference, Iter1::reference>
  bool lexicographical_compare(Iter1 @\farg{first1}@, Iter1 @\farg{last1}@,
                               Iter2 @\farg{first2}@, Iter2 @\farg{last2}@);

template<InputIterator Iter1, InputIterator Iter2, class Compare>
  requires Predicate<Compare, Iter1::reference, Iter2::reference> &&
           Predicate<Compare, Iter2::reference, Iter1::reference>
  bool lexicographical_compare(Iter1 @\farg{first1}@, Iter1 @\farg{last1}@,
                               Iter2 @\farg{first2}@, Iter2 @\farg{last2}@,
                               @\textcolor{addclr}{}@Compare @\farg{comp}@);
\end{itemdecl}\color{black}

\begin{itemdescr}
\pnum
\returns\ 
\tcode{true}\
if the sequence of elements defined by the range
\range{\farg{first1}}{\farg{last1}}\
is lexicographically less than the sequence of elements defined by the range
\range{\farg{first2}}{\farg{last2}}.

Returns
\tcode{false}\
otherwise.

\pnum
\complexity\ 
At most
\tcode{2*min((\farg{last1}\ - \farg{first1}), (\farg{last2}\ - \farg{first2}))}\
applications of the corresponding comparison.

\pnum
\notes\ 
If two sequences have the same number of elements and their corresponding
elements are equivalent, then neither sequence is lexicographically
less than the other.
If one sequence is a prefix of the other, then the shorter sequence is
lexicographically less than the longer sequence.
Otherwise, the lexicographical comparison of the sequences yields the same
result as the comparison of the first corresponding pair of
elements that are not equivalent.

\begin{codeblock}
for ( ; @\farg{first1}@ != @\farg{last1}@ && @\farg{first2}@ != @\farg{last2}@ ; @\farg{++first1}@, @\farg{++first2}@) {
  if (*@\farg{first1}@ < *@\farg{first2}@) return true;
  if (*@\farg{first2}@ < *@\farg{first1}@) return false;
}
return @\farg{first1}@ == @\farg{last1}@ && @\farg{first2}@ != @\farg{last2}@;
\end{codeblock}
\end{itemdescr}

\rSec2[alg.permutation.generators]{Permutation generators}

\index{next_permutation@\tcode{next_permutation}}%
\color{addclr}\begin{itemdecl}
template<MutableBidirectionalIterator Iter>
  requires SameType<Iter::value_type&, Iter::reference> && Swappable<Iter::value_type> &&
           LessThanComparable<Iter::reference>
  bool next_permutation(Iter @\farg{first}@, Iter @\farg{last}@);

template<MutableBidirectionalIterator Iter, Predicate<auto, Iter::reference, Iter::reference> Compare>
  requires SameType<Iter::value_type&, Iter::reference> && Swappable<Iter::value_type>
  @\textcolor{addclr}{bool}@ next_permutation(Iter @\farg{first}@, Iter @\farg{last}@, Compare @\farg{comp}@);
\end{itemdecl}\color{black}

\begin{itemdescr}
\pnum
\effects\ 
Takes a sequence defined by the range
\range{\farg{first}}{\farg{last}}\
and transforms it into the next permutation.
The next permutation is found by assuming that the set of all permutations is
lexicographically sorted with respect to
\tcode{operator<}\
or \farg{comp}.
If such a permutation exists, it returns
\tcode{true}.
Otherwise, it transforms the sequence into the smallest permutation,
that is, the ascendingly sorted one, and returns
\tcode{false}.

\pnum
\removedConcepts{Requires:
The type of first shall satisfy the
Swappable requirements (20.1.4).}

\pnum
\complexity\ 
At most
\tcode{(\farg{last}\ - \farg{first})/2}\
swaps.
\end{itemdescr}

\index{prev_permutation@\tcode{prev_permutation}}%
\color{addclr}\begin{itemdecl}
template<MutableBidirectionalIterator Iter>
  requires SameType<Iter::value_type&, Iter::reference> && Swappable<Iter::value_type> &&
        LessThanComparable<Iter::reference>
  bool prev_permutation(Iter @\farg{first}@, Iter @\farg{last}@);

template<MutableBidirectionalIterator Iter, Predicate<auto, Iter::reference, Iter::reference> Compare>
  requires SameType<Iter::value_type&, Iter::reference> && Swappable<Iter::value_type>
  bool prev_permutation(Iter @\farg{first}@, Iter @\farg{last}@, Compare @\farg{comp}@);
\end{itemdecl}\color{black}

\begin{itemdescr}
\pnum
\effects\ 
Takes a sequence defined by the range
\range{\farg{first}}{\farg{last}}\
and transforms it into the previous permutation.
The previous permutation is found by assuming that the set of all permutations is
lexicographically sorted with respect to
\tcode{operator<}\
or \farg{comp}.

\pnum
\returns\ 
\tcode{true}\
if such a permutation exists.
Otherwise, it transforms the sequence into the largest permutation,
that is, the descendingly sorted one, and returns
\tcode{false}.

\pnum
\removedConcepts{Requires:
The type of *first shall satisfy the
Swappable requirements (20.1.4).}

\pnum
\complexity\ 
At most
\tcode{(\farg{last}\ - \farg{first})/2}\
swaps.
\end{itemdescr}

\end{paras}

\bibliographystyle{plain}
\bibliography{local}

\end{document}