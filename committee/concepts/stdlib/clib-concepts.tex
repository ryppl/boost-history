\documentclass[american,twoside]{book}
\usepackage{refbib}
\usepackage{pdfsync}
% Definitions and redefinitions of special commands

\usepackage{babel}      % needed for iso dates
\usepackage{savesym}		% suppress duplicate macro definitions
\usepackage{fancyhdr}		% more flexible headers and footers
\usepackage{listings}		% code listings
\usepackage{longtable}	% auto-breaking tables
\usepackage{remreset}		% remove counters from reset list
\usepackage{booktabs}		% fancy tables
\usepackage{relsize}		% provide relative font size changes
\usepackage[htt]{hyphenat}	% hyphenate hyphenated words: conflicts with underscore
\savesymbol{BreakableUnderscore}	% suppress BreakableUnderscore defined in hyphenat
									                % (conflicts with underscore)
\usepackage{underscore}	% remove special status of '_' in ordinary text
\usepackage{verbatim}		% improved verbatim environment
\usepackage{parskip}		% handle non-indented paragraphs "properly"
\usepackage{array}			% new column definitions for tables
\usepackage[iso]{isodate} % use iso format for dates
\usepackage{soul}       % strikeouts and underlines for difference markups
\usepackage{color}      % define colors for strikeouts and underlines
\usepackage{amsmath}    % additional math symbols
\usepackage{mathrsfs}
\usepackage{multicol}

\usepackage[T1]{fontenc}
\usepackage{ae}
\usepackage{mathptmx}
\usepackage[scaled=.90]{helvet}

%% Difference markups
\definecolor{addclr}{rgb}{0,.4,.4}
\definecolor{remclr}{rgb}{1,0,0}
\newcommand{\added}[1]{\textcolor{addclr}{\ul{#1}}}
\newcommand{\removed}[1]{\textcolor{remclr}{\st{#1}}}
\newcommand{\changed}[2]{\removed{#1}\added{#2}}
\newcommand{\remfn}{\footnote{\removed{removed footnote}}}
\newcommand{\addfn}[1]{\footnote{\added{#1}}}
\newcommand{\remitem}[1]{\item\removed{#1}}
\newcommand{\additem}[1]{\item\added{#1}}

%% Added by JJ
\long\gdef\metacomment#1{[{\sc Editorial note:} \begingroup\sf\aftergroup] #1\endgroup}

%% October, 2005 changes
\newcommand{\addedA}[1]{#1}
\newcommand{\removedA}[1]{}
\newcommand{\changedA}[2]{#2}

%% April, 2006 changes
\newcommand{\addedB}[1]{#1}
\newcommand{\removedB}[1]{}
\newcommand{\changedB}[2]{#2}
\newcommand{\remfootnoteB}[1]{}
\newcommand{\marktr}{}
\newcommand\ptr{}

%% October, 2006 changes
%\newcommand{\addedC}[1]{\added{#1}}
%\newcommand{\removedC}[1]{\removed{#1}}
%\newcommand{\changedC}[2]{\changed{#1}{#2}}
%\newcommand{\remfootnoteC}[1]{\remfn}
%\newcommand{\addfootnoteC}[1]{\addfn{#1}}
%\newcommand{\remitemC}[1]{\remitem{#1}}
%\newcommand{\additemC}[1]{\additem{#1}}
%\newcommand{\remblockC}{}

%% November registration ballot
\newcommand{\addedC}[1]{#1}
\newcommand{\removedC}[1]{}
\newcommand{\changedC}[2]{#2}
\newcommand{\remfootnoteC}[1]{}
\newcommand{\addfootnoteC}[1]{\footnote{#1}}
\newcommand{\remitemC}[1]{}
\newcommand{\additemC}[1]{\item{#1}}
\newcommand{\remblockC}{\remov_this_block}

\newcommand{\addedD}[1]{#1}
\newcommand{\removedD}[1]{}
\newcommand{\changedD}[2]{#2}
\newcommand{\remfootnoteD}[1]{}
\newcommand{\addfootnoteD}[1]{\footnote{#1}}
\newcommand{\remitemD}[1]{}
\newcommand{\additemD}[1]{\item{#1}}
\newcommand{\remblockD}{\remov_this_block}

%% Variadic Templates changes
\newcommand{\addedVT}[1]{\textcolor{addclr}{\ul{#1}}}
\newcommand{\removedVT}[1]{\textcolor{remclr}{\st{#1}}}
\newcommand{\changedVT}[2]{\removed{#1}\added{#2}}

%% Concepts changes
\newcommand{\addedConcepts}[1]{\added{#1}}
\newcommand{\removedConcepts}[1]{\removed{#1}}
\newcommand{\changedConcepts}[2]{\changed{#1}{#2}}
\newcommand{\addedConceptsC}[1]{\textcolor{addclr}{\tcode{\ul{#1}}}}
\newcommand{\remitemConcepts}[1]{\remitem{#1}}
\newcommand{\additemConcepts}[1]{\additem{#1}}

%% Concepts changes since the last revision
\definecolor{ccadd}{rgb}{0,.6,0}
\newcommand{\addedCC}[1]{\textcolor{ccadd}{\ul{#1}}}
\newcommand{\removedCC}[1]{\textcolor{remclr}{\st{#1}}}
\newcommand{\changedCC}[2]{\removedCC{#1}\addedCC{#2}}
\newcommand{\remitemCC}[1]{\remitem{#1}}
\newcommand{\additemCC}[1]{\item\addedCC{#1}}
\newcommand{\changedCCC}[2]{\textcolor{ccadd}{\st{#1}}\addedCC{#2}}
\newcommand{\removedCCC}[1]{\textcolor{ccadd}{\st{#1}}}
\newcommand{\remitemCCC}[1]{\item\removedCCC{#1}}

%% Concepts changes for the next revision
\definecolor{zadd}{rgb}{0.8,0,0.8}
\newcommand{\addedZ}[1]{\textcolor{zadd}{\ul{#1}}}
\newcommand{\removedZ}[1]{\textcolor{remclr}{\st{#1}}}
\newcommand{\changedCZ}[2]{\textcolor{addclr}{\st{#1}}\addedZ{#2}}


%%--------------------------------------------------
%% Sectioning macros.  
% Each section has a depth, an automatically generated section 
% number, a name, and a short tag.  The depth is an integer in 
% the range [0,5].  (If it proves necessary, it wouldn't take much
% programming to raise the limit from 5 to something larger.)


% The basic sectioning command.  Example:
%    \Sec1[intro.scope]{Scope}
% defines a first-level section whose name is "Scope" and whose short
% tag is intro.scope.  The square brackets are mandatory.
\def\Sec#1[#2]#3{{%
\ifcase#1\let\s=\chapter
      \or\let\s=\section
      \or\let\s=\subsection
      \or\let\s=\subsubsection
      \or\let\s=\paragraph
      \or\let\s=\subparagraph
      \fi%
\s[#3]{#3\hfill[#2]}\relax\label{#2}}}

% A convenience feature (mostly for the convenience of the Project
% Editor, to make it easy to move around large blocks of text):
% the \rSec macro is just like the \Sec macro, except that depths 
% relative to a global variable, SectionDepthBase.  So, for example,
% if SectionDepthBase is 1,
%   \rSec1[temp.arg.type]{Template type arguments}
% is equivalent to
%   \Sec2[temp.arg.type]{Template type arguments}

\newcounter{SectionDepthBase}
\newcounter{scratch}

\def\rSec#1[#2]#3{{%
\setcounter{scratch}{#1}
\addtocounter{scratch}{\value{SectionDepthBase}}
\Sec{\arabic{scratch}}[#2]{#3}}}

% Change the way section headings are formatted.
\renewcommand{\chaptername}{}
\renewcommand{\appendixname}{Annex}

\makeatletter
\def\@makechapterhead#1{%
  \hrule\vspace*{1.5\p@}\hrule
  \vspace*{16\p@}%
  {\parindent \z@ \raggedright \normalfont
    \ifnum \c@secnumdepth >\m@ne
        \huge\bfseries \@chapapp\space \thechapter\space\space\space\space
    \fi
    \interlinepenalty\@M
    \huge \bfseries #1\par\nobreak
  \vspace*{16\p@}%
  \hrule\vspace*{1.5\p@}\hrule
  \vspace*{48\p@}
  }}

\renewcommand\section{\@startsection{section}{1}{0pt}%
                                   {-3.5ex plus -1ex minus -.2ex}%
                                   {.3ex plus .2ex}%
                                   {\normalfont\normalsize\bfseries}}
\renewcommand\section{\@startsection{section}{1}{0pt}%
                                   {2.5ex}% plus 1ex minus .2ex}%
                                   {.3ex}% plus .1ex minus .2 ex}%
                                   {\normalfont\normalsize\bfseries}}

\renewcommand\subsection{\@startsection{subsection}{2}{0pt}%
                                     {-3.25ex plus -1ex minus -.2ex}%
                                     {.3ex plus .2ex}%
                                     {\normalfont\normalsize\bfseries}}

\renewcommand\subsubsection{\@startsection{subsubsection}{3}{0pt}%
                                     {-3.25ex plus -1ex minus -.2ex}%
                                     {.3ex plus .2ex}%
                                     {\normalfont\normalsize\bfseries}}

\renewcommand\paragraph{\@startsection{paragraph}{4}{0pt}%
                                     {-3.25ex plus -1ex minus -.2ex}%
                                     {.3ex \@plus .2ex}%
                                     {\normalfont\normalsize\bfseries}}

\renewcommand\subparagraph{\@startsection{subparagraph}{5}{0pt}%
                                     {-3.25ex plus -1ex minus -.2ex}%
                                     {.3ex plus .2ex}%
                                     {\normalfont\normalsize\bfseries}}
\@removefromreset{footnote}{chapter}
\@removefromreset{table}{chapter}
\@removefromreset{figure}{chapter}
\makeatother

%%--------------------------------------------------
% Heading style for Annexes
\newcommand{\Annex}[3]{\chapter[#2]{\\(#3)\\#2\hfill[#1]}\relax\label{#1}}
\newcommand{\infannex}[2]{\Annex{#1}{#2}{informative}}
\newcommand{\normannex}[2]{\Annex{#1}{#2}{normative}}

\newcommand{\synopsis}[1]{\textbf{#1}}

%%--------------------------------------------------
% General code style
\newcommand{\CodeStyle}{\ttfamily}
\newcommand{\CodeStylex}[1]{\texttt{#1}}

% Code and definitions embedded in text.
\newcommand{\tcode}[1]{\CodeStylex{#1}}
\newcommand{\techterm}[1]{\textit{#1}}

%%--------------------------------------------------
%% allow line break if needed for justification
\newcommand{\brk}{\discretionary{}{}{}}
%  especially for scope qualifier
\newcommand{\colcol}{\brk::\brk}

%%--------------------------------------------------
%% Macros for funky text
%%!\newcommand{\Rplus}{\protect\nolinebreak\hspace{-.07em}\protect\raisebox{.25ex}{\small\textbf{+}}}
\newcommand{\Rplus}{+}
\newcommand{\Cpp}{C\Rplus\Rplus}
\newcommand{\opt}{$_\mathit{opt}$}
\newcommand{\shl}{<{<}}
\newcommand{\shr}{>{>}}
\newcommand{\dcr}{-{-}}
\newcommand{\bigohm}[1]{\mathscr{O}(#1)}
\newcommand{\bigoh}[1]{$\bigohm{#1}$}
\renewcommand{\tilde}{{\smaller$\sim$}}		% extra level of braces is necessary

%%--------------------------------------------------
%% States and operators

\newcommand{\state}[2]{\tcode{#1}\ensuremath{_{#2}}}
\newcommand{\bitand}{\ensuremath{\, \mathsf{bitand} \,}}
\newcommand{\bitor}{\ensuremath{\, \mathsf{bitor} \,}}
\newcommand{\xor}{\ensuremath{\, \mathsf{xor} \,}}
\newcommand{\rightshift}{\ensuremath{\, \mathsf{rshift} \,}}
\newcommand{\leftshift}{\ensuremath{\, \mathsf{lshift} \,}}

%% Notes and examples
\newcommand{\EnterBlock}[1]{[\,\textit{#1:}}
\newcommand{\ExitBlock}[1]{\textit{\ ---\,end #1}\,]}
\newcommand{\enternote}{\EnterBlock{Note}}
\newcommand{\exitnote}{\ExitBlock{note}}
\newcommand{\enterexample}{\EnterBlock{Example}}
\newcommand{\exitexample}{\ExitBlock{example}}

%% Library function descriptions
\newcommand{\Fundescx}[1]{\textit{#1}}
\newcommand{\Fundesc}[1]{\Fundescx{#1:}}
\newcommand{\required}{\Fundesc{Required behavior}}
\newcommand{\requires}{\Fundesc{Requires}}
\newcommand{\effects}{\Fundesc{Effects}}
\newcommand{\postconditions}{\Fundesc{Postconditions}}
\newcommand{\postcondition}{\Fundesc{Postcondition}}
\newcommand{\preconditions}{\Fundesc{Preconditions}}
\newcommand{\precondition}{\Fundesc{Precondition}}
\newcommand{\returns}{\Fundesc{Returns}}
\newcommand{\throws}{\Fundesc{Throws}}
\newcommand{\default}{\Fundesc{Default behavior}}
\newcommand{\complexity}{\Fundesc{Complexity}}
\newcommand{\note}{\Fundesc{Remark}}
\newcommand{\notes}{\Fundesc{Remarks}}
\newcommand{\implimits}{\Fundesc{Implementation limits}}
\newcommand{\replaceable}{\Fundesc{Replaceable}}
\newcommand{\exceptionsafety}{\Fundesc{Exception safety}}
\newcommand{\returntype}{\Fundesc{Return type}}

%% Cross reference
\newcommand{\xref}{\textsc{See also:}}

%% NTBS, etc.
\newcommand{\NTS}[1]{\textsc{#1}}
\newcommand{\ntbs}{\NTS{ntbs}}
\newcommand{\ntmbs}{\NTS{ntmbs}}
\newcommand{\ntwcs}{\NTS{ntwcs}}

%% Function argument
\newcommand{\farg}[1]{\texttt{\textit{#1}}}

%% Code annotations
\newcommand{\EXPO}[1]{\textbf{#1}}
\newcommand{\expos}{\EXPO{exposition only}}
\newcommand{\exposr}{\hfill\expos}
\newcommand{\exposrc}{\hfill// \expos}
\newcommand{\impdef}{\EXPO{implementation-defined}}
\newcommand{\notdef}{\EXPO{not defined}}

%% Double underscore
\newcommand{\unun}{\_\,\_}
\newcommand{\xname}[1]{\unun\,#1}
\newcommand{\mname}[1]{\tcode{\unun\,#1\,\unun}}

%% Ranges
\newcommand{\Range}[4]{\tcode{#1\brk{}#3,\brk{}#4\brk{}#2}}
\newcommand{\crange}[2]{\Range{[}{]}{#1}{#2}}
\newcommand{\orange}[2]{\Range{(}{)}{#1}{#2}}
\newcommand{\range}[2]{\Range{[}{)}{#1}{#2}}

%% Change descriptions
\newcommand{\diffdef}[1]{\hfill\break\textbf{#1:}}
\newcommand{\change}{\diffdef{Change}}
\newcommand{\rationale}{\diffdef{Rationale}}
\newcommand{\effect}{\diffdef{Effect on original feature}}
\newcommand{\difficulty}{\diffdef{Difficulty of converting}}
\newcommand{\howwide}{\diffdef{How widely used}}

%% Miscellaneous
\newcommand{\uniquens}{\textrm{\textit{\textbf{unique}}}}
\newcommand{\stage}[1]{\item{\textbf{Stage #1:}}}

%%--------------------------------------------------
%% Adjust markers
\renewcommand{\thetable}{\arabic{table}}
\renewcommand{\thefigure}{\arabic{figure}}
\renewcommand{\thefootnote}{\arabic{footnote})}

%% Change list item markers from box to dash
\renewcommand{\labelitemi}{---}
\renewcommand{\labelitemii}{---}
\renewcommand{\labelitemiii}{---}
\renewcommand{\labelitemiv}{---}

%%--------------------------------------------------
%% Environments for code listings.

% We use the 'listings' package, with some small customizations.  The
% most interesting customization: all TeX commands are available
% within comments.  Comments are set in italics, keywords and strings
% don't get special treatment.

\lstset{language=C++,
        basicstyle=\CodeStyle\small,
        keywordstyle=,
        stringstyle=,
        xleftmargin=1em,
        showstringspaces=false,
        commentstyle=\itshape\rmfamily,
        columns=flexible,
        keepspaces=true,
        texcl=true}

% Our usual abbreviation for 'listings'.  Comments are in 
% italics.  Arbitrary TeX commands can be used if they're 
% surrounded by @ signs.
\lstnewenvironment{codeblock}
{
 \lstset{escapechar=@}
 \renewcommand{\tcode}[1]{\textup{\CodeStyle##1}}
 \renewcommand{\techterm}[1]{\textit{##1}}
}
{
}

% Permit use of '@' inside codeblock blocks (don't ask)
\makeatletter
\newcommand{\atsign}{@}
\makeatother

%%--------------------------------------------------
%% Paragraph numbering
\newcounter{Paras}
\makeatletter
\@addtoreset{Paras}{chapter}
\@addtoreset{Paras}{section}
\@addtoreset{Paras}{subsection}
\@addtoreset{Paras}{subsubsection}
\@addtoreset{Paras}{paragraph}
\@addtoreset{Paras}{subparagraph}
\def\pnum{\addtocounter{Paras}{1}\noindent\llap{{\footnotesize\arabic{Paras}}\hspace{\@totalleftmargin}\quad}}
\makeatother

% For compatibility only.  We no longer need this environment.
\newenvironment{paras}{}{}

%%--------------------------------------------------
%% Indented text
\newenvironment{indented}
{\list{}{}\item\relax}
{\endlist}

%%--------------------------------------------------
%% Library item descriptions
\lstnewenvironment{itemdecl}
{
 \lstset{escapechar=@,
 xleftmargin=0em,
 aboveskip=2ex,
 belowskip=0ex	% leave this alone: it keeps these things out of the
				% footnote area
 }
}
{
}

\newenvironment{itemdescr}
{
 \begin{indented}}
{
 \end{indented}
}


%%--------------------------------------------------
%% Bnf environments
\newlength{\BnfIndent}
\setlength{\BnfIndent}{\leftmargini}
\newlength{\BnfInc}
\setlength{\BnfInc}{\BnfIndent}
\newlength{\BnfRest}
\setlength{\BnfRest}{2\BnfIndent}
\newcommand{\BnfNontermshape}{\rmfamily\itshape\small}
\newcommand{\BnfTermshape}{\ttfamily\upshape\small}
\newcommand{\nonterminal}[1]{{\BnfNontermshape #1}}

\newenvironment{bnfbase}
 {
 \newcommand{\terminal}[1]{{\BnfTermshape ##1}}
 \newcommand{\descr}[1]{\normalfont{##1}}
 \newcommand{\bnfindentfirst}{\BnfIndent}
 \newcommand{\bnfindentinc}{\BnfInc}
 \newcommand{\bnfindentrest}{\BnfRest}
 \begin{minipage}{.9\hsize}
 \newcommand{\br}{\hfill\\}
 }
 {
 \end{minipage}
 }

\newenvironment{BnfTabBase}[1]
{
 \begin{bnfbase}
 #1
 \begin{indented}
 \begin{tabbing}
 \hspace*{\bnfindentfirst}\=\hspace{\bnfindentinc}\=\hspace{.6in}\=\hspace{.6in}\=\hspace{.6in}\=\hspace{.6in}\=\hspace{.6in}\=\hspace{.6in}\=\hspace{.6in}\=\hspace{.6in}\=\hspace{.6in}\=\hspace{.6in}\=\kill%
}
{
 \end{tabbing}
 \end{indented}
 \end{bnfbase}
}

\newenvironment{bnfkeywordtab}
{
 \begin{BnfTabBase}{\BnfTermshape}
}
{
 \end{BnfTabBase}
}

\newenvironment{bnftab}
{
 \begin{BnfTabBase}{\BnfNontermshape}
}
{
 \end{BnfTabBase}
}

\newenvironment{simplebnf}
{
 \begin{bnfbase}
 \BnfNontermshape
 \begin{indented}
}
{
 \end{indented}
 \end{bnfbase}
}

\newenvironment{bnf}
{
 \begin{bnfbase}
 \list{}
	{
	\setlength{\leftmargin}{\bnfindentrest}
	\setlength{\listparindent}{-\bnfindentinc}
	\setlength{\itemindent}{\listparindent}
	}
 \BnfNontermshape
 \item\relax
}
{
 \endlist
 \end{bnfbase}
}

% non-copied versions of bnf environments
\newenvironment{ncbnftab}
{
 \begin{bnftab}
}
{
 \end{bnftab}
}

\newenvironment{ncsimplebnf}
{
 \begin{simplebnf}
}
{
 \end{simplebnf}
}

\newenvironment{ncbnf}
{
 \begin{bnf}
}
{
 \end{bnf}
}

%%--------------------------------------------------
%% Drawing environment
%
% usage: \begin{drawing}{UNITLENGTH}{WIDTH}{HEIGHT}{CAPTION}
\newenvironment{drawing}[4]
{
\begin{figure}[h]
\setlength{\unitlength}{#1}
\begin{center}
\begin{picture}(#2,#3)\thicklines
}
{
\end{picture}
\end{center}
%\caption{Directed acyclic graph}
\end{figure}
}

%%--------------------------------------------------
%% Table environments

% Base definitions for tables
\newenvironment{TableBase}
{
 \renewcommand{\tcode}[1]{{\CodeStyle##1}}
 \newcommand{\topline}{\hline}
 \newcommand{\capsep}{\hline\hline}
 \newcommand{\rowsep}{\hline}
 \newcommand{\bottomline}{\hline}

%% vertical alignment
 \newcommand{\rb}[1]{\raisebox{1.5ex}[0pt]{##1}}	% move argument up half a row

%% header helpers
 \newcommand{\hdstyle}[1]{\textbf{##1}}				% set header style
 \newcommand{\Head}[3]{\multicolumn{##1}{##2}{\hdstyle{##3}}}	% add title spanning multiple columns
 \newcommand{\lhdrx}[2]{\Head{##1}{|c}{##2}}		% set header for left column spanning #1 columns
 \newcommand{\chdrx}[2]{\Head{##1}{c}{##2}}			% set header for center column spanning #1 columns
 \newcommand{\rhdrx}[2]{\Head{##1}{c|}{##2}}		% set header for right column spanning #1 columns
 \newcommand{\ohdrx}[2]{\Head{##1}{|c|}{##2}}		% set header for only column spanning #1 columns
 \newcommand{\lhdr}[1]{\lhdrx{1}{##1}}				% set header for single left column
 \newcommand{\chdr}[1]{\chdrx{1}{##1}}				% set header for single center column
 \newcommand{\rhdr}[1]{\rhdrx{1}{##1}}				% set header for single right column
 \newcommand{\ohdr}[1]{\ohdrx{1}{##1}}
 \newcommand{\br}{\hfill\break}						% force newline within table entry

%% column styles
 \newcolumntype{x}[1]{>{\raggedright\let\\=\tabularnewline}p{##1}}	% word-wrapped ragged-right
 																	% column, width specified by #1
 \newcolumntype{m}[1]{>{\CodeStyle}l{##1}}							% variable width column, all entries in CodeStyle
}
{
}

% General Usage: TITLE is the title of the table, XREF is the
% cross-reference for the table. LAYOUT is a sequence of column
% type specifiers (e.g. cp{1.0}c), without '|' for the left edge
% or right edge.

% usage: \begin{floattablebase}{TITLE}{XREF}{COLUMNS}{PLACEMENT}
% produces floating table, location determined within limits
% by LaTeX.
\newenvironment{floattablebase}[4]
{
 \begin{TableBase}
 \begin{table}[#4]
 \caption{\label{#2}#1}
 \begin{center}
 \begin{tabular}{|#3|}
}
{
 \bottomline
 \end{tabular}
 \end{center}
 \end{table}
 \end{TableBase}
}

% usage: \begin{floattable}{TITLE}{XREF}{COLUMNS}
% produces floating table, location determined within limits
% by LaTeX.
\newenvironment{floattable}[3]
{
 \begin{floattablebase}{#1}{#2}{#3}{htbp}
}
{
 \end{floattablebase}
}

% usage: \begin{tokentable}{TITLE}{XREF}{HDR1}{HDR2}
% produces six-column table used for lists of replacement tokens;
% the columns are in pairs -- left-hand column has header HDR1,
% right hand column has header HDR2; pairs of columns are separated
% by vertical lines. Used in "trigraph sequences" table in standard.
\newenvironment{tokentable}[4]
{
 \begin{floattablebase}{#1}{#2}{cc|cc|cc}{htbp}
 \topline
 \textit{#3}   &   \textit{#4}    &
 \textit{#3}   &   \textit{#4}    &
 \textit{#3}   &   \textit{#4}    \\ \capsep
}
{
 \end{floattablebase}
}

% usage: \begin{libsumtabase}{TITLE}{XREF}{HDR1}{HDR2}
% produces two-column table with column headers HDR1 and HDR2.
% Used in "Library Categories" table in standard, and used as
% base for other library summary tables.
\newenvironment{libsumtabbase}[4]
{
 \begin{floattable}{#1}{#2}{ll}
 \topline
 \lhdr{#3}	&	\hdstyle{#4}	\\ \capsep
}
{
 \end{floattable}
}

% usage: \begin{libsumtab}{TITLE}{XREF}
% produces two-column table with column headers "Subclause" and "Header(s)".
% Used in "C++ Headers for Freestanding Implementations" table in standard.
\newenvironment{libsumtab}[2]
{
 \begin{libsumtabbase}{#1}{#2}{Subclause}{Header(s)}
}
{
 \end{libsumtabbase}
}

% usage: \begin{LibSynTab}{CAPTION}{TITLE}{XREF}{COUNT}{LAYOUT}
% produces table with COUNT columns. Used as base for
% C library description tables
\newcounter{LibSynTabCols}
\newcounter{LibSynTabWd}
\newenvironment{LibSynTabBase}[5]
{
 \setcounter{LibSynTabCols}{#4}
 \setcounter{LibSynTabWd}{#4}
 \addtocounter{LibSynTabWd}{-1}
 \newcommand{\centry}[1]{\textbf{##1}:}
 \newcommand{\macro}{\centry{Macro}}
 \newcommand{\macros}{\centry{Macros}}
 \newcommand{\function}{\centry{Function}}
 \newcommand{\functions}{\centry{Functions}}
 \newcommand{\templates}{\centry{Templates}}
 \newcommand{\type}{\centry{Type}}
 \newcommand{\types}{\centry{Types}}
 \newcommand{\values}{\centry{Values}}
 \newcommand{\struct}{\centry{Struct}}
 \newcommand{\cspan}[1]{\multicolumn{\value{LibSynTabCols}}{|l|}{##1}}
 \begin{floattable}{#1 \tcode{<#2>}\ synopsis}{#3}
 {#5}
 \topline
 \lhdr{Type}	&	\rhdrx{\value{LibSynTabWd}}{Name(s)}	\\ \capsep
}
{
 \end{floattable}
}

% usage: \begin{LibSynTab}{TITLE}{XREF}{COUNT}{LAYOUT}
% produces table with COUNT columns. Used as base for description tables
% for C library
\newenvironment{LibSynTab}[4]
{
 \begin{LibSynTabBase}{Header}{#1}{#2}{#3}{#4}
}
{
 \end{LibSynTabBase}
}

% usage: \begin{LibSynTabAdd}{TITLE}{XREF}{COUNT}{LAYOUT}
% produces table with COUNT columns. Used as base for description tables
% for additions to C library
\newenvironment{LibSynTabAdd}[4]
{
 \begin{LibSynTabBase}{Additions to header}{#1}{#2}{#3}{#4}
}
{
 \end{LibSynTabBase}
}

% usage: \begin{libsyntabN}{TITLE}{XREF}
%        \begin{libsyntabaddN}{TITLE}{XREF}
% produces a table with N columns for C library description tables
\newenvironment{libsyntab2}[2]
{
 \begin{LibSynTab}{#1}{#2}{2}{ll}
}
{
 \end{LibSynTab}
}

\newenvironment{libsyntab3}[2]
{
 \begin{LibSynTab}{#1}{#2}{3}{lll}
}
{
 \end{LibSynTab}
}

\newenvironment{libsyntab4}[2]
{
 \begin{LibSynTab}{#1}{#2}{4}{llll}
}
{
 \end{LibSynTab}
}

\newenvironment{libsyntab5}[2]
{
 \begin{LibSynTab}{#1}{#2}{5}{lllll}
}
{
 \end{LibSynTab}
}

\newenvironment{libsyntab6}[2]
{
 \begin{LibSynTab}{#1}{#2}{6}{llllll}
}
{
 \end{LibSynTab}
}

\newenvironment{libsyntabadd2}[2]
{
 \begin{LibSynTabAdd}{#1}{#2}{2}{ll}
}
{
 \end{LibSynTabAdd}
}

\newenvironment{libsyntabadd3}[2]
{
 \begin{LibSynTabAdd}{#1}{#2}{3}{lll}
}
{
 \end{LibSynTabAdd}
}

\newenvironment{libsyntabadd4}[2]
{
 \begin{LibSynTabAdd}{#1}{#2}{4}{llll}
}
{
 \end{LibSynTabAdd}
}

\newenvironment{libsyntabadd5}[2]
{
 \begin{LibSynTabAdd}{#1}{#2}{5}{lllll}
}
{
 \end{LibSynTabAdd}
}

\newenvironment{libsyntabadd6}[2]
{
 \begin{LibSynTabAdd}{#1}{#2}{6}{llllll}
}
{
 \end{LibSynTabAdd}
}

% usage: \begin{LongTable}{TITLE}{XREF}{LAYOUT}
% produces table that handles page breaks sensibly.
\newenvironment{LongTable}[3]
{
 \begin{TableBase}
 \begin{longtable}
 {|#3|}\caption{#1}\label{#2}
}
{
 \bottomline
 \end{longtable}
 \end{TableBase}
}

% usage: \begin{twocol}{TITLE}{XREF}
% produces a two-column breakable table. Used in
% "simple-type-specifiers and the types they specify" in the standard.
\newenvironment{twocol}[2]
{
 \begin{LongTable}
 {#1}{#2}
 {ll}
}
{
 \end{LongTable}
}

% usage: \begin{libreqtabN}{TITLE}{XREF}
% produces an N-column brekable table. Used in
% most of the library clauses for requirements tables.
% Example at "Position type requirements" in the standard.

\newenvironment{libreqtab1}[2]
{
 \begin{LongTable}
 {#1}{#2}
 {x{.55\hsize}}
}
{
 \end{LongTable}
}

\newenvironment{libreqtab2}[2]
{
 \begin{LongTable}
 {#1}{#2}
 {lx{.55\hsize}}
}
{
 \end{LongTable}
}

\newenvironment{libreqtab2a}[2]
{
 \begin{LongTable}
 {#1}{#2}
 {x{.30\hsize}x{.68\hsize}}
}
{
 \end{LongTable}
}

\newenvironment{libreqtab3}[2]
{
 \begin{LongTable}
 {#1}{#2}
 {x{.28\hsize}x{.18\hsize}x{.43\hsize}}
}
{
 \end{LongTable}
}

\newenvironment{libreqtab3a}[2]
{
 \begin{LongTable}
 {#1}{#2}
 {x{.28\hsize}x{.33\hsize}x{.29\hsize}}
}
{
 \end{LongTable}
}

\newenvironment{libreqtab3b}[2]
{
 \begin{LongTable}
 {#1}{#2}
 {x{.40\hsize}x{.25\hsize}x{.25\hsize}}
}
{
 \end{LongTable}
}

\newenvironment{libreqtab3c}[2]
{
 \begin{LongTable}
 {#1}{#2}
 {x{.30\hsize}x{.25\hsize}x{.35\hsize}}
}
{
 \end{LongTable}
}

\newenvironment{libreqtab3d}[2]
{
 \begin{LongTable}
 {#1}{#2}
 {x{.32\hsize}x{.27\hsize}x{.36\hsize}}
}
{
 \end{LongTable}
}

\newenvironment{libreqtab3e}[2]
{
 \begin{LongTable}
 {#1}{#2}
 {x{.38\hsize}x{.27\hsize}x{.25\hsize}}
}
{
 \end{LongTable}
}

\newenvironment{libreqtab3f}[2]
{
 \begin{LongTable}
 {#1}{#2}
 {x{.40\hsize}x{.22\hsize}x{.31\hsize}}
}
{
 \end{LongTable}
}

\newenvironment{libreqtab4}[2]
{
 \begin{LongTable}
 {#1}{#2}
}
{
 \end{LongTable}
}

\newenvironment{libreqtab4a}[2]
{
 \begin{LongTable}
 {#1}{#2}
 {x{.14\hsize}x{.30\hsize}x{.30\hsize}x{.14\hsize}}
}
{
 \end{LongTable}
}

\newenvironment{libreqtab4b}[2]
{
 \begin{LongTable}
 {#1}{#2}
 {x{.13\hsize}x{.15\hsize}x{.29\hsize}x{.27\hsize}}
}
{
 \end{LongTable}
}

\newenvironment{libreqtab4c}[2]
{
 \begin{LongTable}
 {#1}{#2}
 {x{.16\hsize}x{.21\hsize}x{.21\hsize}x{.30\hsize}}
}
{
 \end{LongTable}
}

\newenvironment{libreqtab4d}[2]
{
 \begin{LongTable}
 {#1}{#2}
 {x{.22\hsize}x{.22\hsize}x{.30\hsize}x{.15\hsize}}
}
{
 \end{LongTable}
}

\newenvironment{libreqtab5}[2]
{
 \begin{LongTable}
 {#1}{#2}
 {x{.14\hsize}x{.14\hsize}x{.20\hsize}x{.20\hsize}x{.14\hsize}}
}
{
 \end{LongTable}
}

% usage: \begin{libtab2}{TITLE}{XREF}{LAYOUT}{HDR1}{HDR2}
% produces two-column table with column headers HDR1 and HDR2.
% Used in "seekoff positioning" in the standard.
\newenvironment{libtab2}[5]
{
 \begin{floattable}
 {#1}{#2}{#3}
 \topline
 \lhdr{#4}	&	\rhdr{#5}	\\ \capsep
}
{
 \end{floattable}
}

% usage: \begin{longlibtab2}{TITLE}{XREF}{LAYOUT}{HDR1}{HDR2}
% produces two-column table with column headers HDR1 and HDR2.
\newenvironment{longlibtab2}[5]
{
 \begin{LongTable}{#1}{#2}{#3}
 \\ \topline
 \lhdr{#4}	&	\rhdr{#5}	\\ \capsep
}
{
  \end{LongTable}
}

% usage: \begin{LibEffTab}{TITLE}{XREF}{HDR2}{WD2}
% produces a two-column table with left column header "Element"
% and right column header HDR2, right column word-wrapped with
% width specified by WD2.
\newenvironment{LibEffTab}[4]
{
 \begin{libtab2}{#1}{#2}{lp{#4}}{Element}{#3}
}
{
 \end{libtab2}
}

% Same as LibEffTab except that it uses a long table.
\newenvironment{longLibEffTab}[4]
{
 \begin{longlibtab2}{#1}{#2}{lp{#4}}{Element}{#3}
}
{
 \end{longlibtab2}
}

% usage: \begin{libefftab}{TITLE}{XREF}
% produces a two-column effects table with right column
% header "Effect(s) if set", width 4.5 in. Used in "fmtflags effects"
% table in standard.
\newenvironment{libefftab}[2]
{
 \begin{LibEffTab}{#1}{#2}{Effect(s) if set}{4.5in}
}
{
 \end{LibEffTab}
}

% Same as libefftab except that it uses a long table.
\newenvironment{longlibefftab}[2]
{
 \begin{longLibEffTab}{#1}{#2}{Effect(s) if set}{4.5in}
}
{
 \end{longLibEffTab}
}

% usage: \begin{libefftabmean}{TITLE}{XREF}
% produces a two-column effects table with right column
% header "Meaning", width 4.5 in. Used in "seekdir effects"
% table in standard.
\newenvironment{libefftabmean}[2]
{
 \begin{LibEffTab}{#1}{#2}{Meaning}{4.5in}
}
{
 \end{LibEffTab}
}

% Same as libefftabmean except that it uses a long table.
\newenvironment{longlibefftabmean}[2]
{
 \begin{longLibEffTab}{#1}{#2}{Meaning}{4.5in}
}
{
 \end{longLibEffTab}
}

% usage: \begin{libefftabvalue}{TITLE}{XREF}
% produces a two-column effects table with right column
% header "Value", width 3 in. Used in "basic_ios::init() effects"
% table in standard.
\newenvironment{libefftabvalue}[2]
{
 \begin{LibEffTab}{#1}{#2}{Value}{3in}
}
{
 \end{LibEffTab}
}

% Same as libefftabvalue except that it uses a long table and a
% slightly wider column.
\newenvironment{longlibefftabvalue}[2]
{
 \begin{longLibEffTab}{#1}{#2}{Value}{3.5in}
}
{
 \end{longLibEffTab}
}

% usage: \begin{liberrtab}{TITLE}{XREF} produces a two-column table
% with left column header ``Value'' and right header "Error
% condition", width 4.5 in. Used in regex clause in the TR.

\newenvironment{liberrtab}[2]
{
 \begin{libtab2}{#1}{#2}{lp{4.5in}}{Value}{Error condition}
}
{
 \end{libtab2}
}

% Like liberrtab except that it uses a long table.
\newenvironment{longliberrtab}[2]
{
 \begin{longlibtab2}{#1}{#2}{lp{4.5in}}{Value}{Error condition}
}
{
 \end{longlibtab2}
}

% enumerate with lowercase letters
\newenvironment{enumeratea}
{
 \renewcommand{\labelenumi}{\alph{enumi})}
 \begin{enumerate}
}
{
 \end{enumerate}
}

% enumerate with arabic numbers
\newenvironment{enumeraten}
{
 \renewcommand{\labelenumi}{\arabic{enumi})}
 \begin{enumerate}
}
{
 \end{enumerate}
}

%%--------------------------------------------------
%% Definitions section
% usage: \definition{name}{xref}
%\newcommand{\definition}[2]{\rSec2[#2]{#1}}
% for ISO format, use:
\newcommand{\definition}[2]
 {\hfill\vspace{.25ex plus .5ex minus .2ex}\\
 \addtocounter{subsection}{1}%
 \textbf{\thesubsection\hfill\relax[#2]}\\
 \textbf{#1}\label{#2}\\
 }


%%--------------------------------------------------
%% PDF

\usepackage[pdftex,
            pdftitle={Core Concepts for the C++0x Standard Library},
            pdfsubject={C++ International Standard Proposal},
            pdfcreator={Douglas Gregor},
            bookmarks=true,
            bookmarksnumbered=true,
            pdfpagelabels=true,
            pdfpagemode=UseOutlines,
            pdfstartview=FitH,
            linktocpage=true,
            colorlinks=true,
            linkcolor=blue,
            plainpages=false
           ]{hyperref}

%%--------------------------------------------------
%% Set section numbering limit, toc limit
\setcounter{secnumdepth}{5}
\setcounter{tocdepth}{1}

%%--------------------------------------------------
%% Parameters that govern document appearance
\setlength{\oddsidemargin}{0pt}
\setlength{\evensidemargin}{0pt}
\setlength{\textwidth}{6.6in}

%%--------------------------------------------------
%% Handle special hyphenation rules
\hyphenation{tem-plate ex-am-ple in-put-it-er-a-tor}

% Do not put blank pages after chapters that end on odd-numbered pages.
\def\cleardoublepage{\clearpage\if@twoside%
  \ifodd\c@page\else\hbox{}\thispagestyle{empty}\newpage%
  \if@twocolumn\hbox{}\newpage\fi\fi\fi}

\newsavebox\rebindbox

\begin{document}
\raggedbottom

\begin{titlepage}
\begin{center}
\huge
Foundational Concepts for the C++0x Standard Library\\
(Revision 3)
\vspace{0.25in}
\end{center}

\normalsize
\vspace{0.25in}
\par\noindent Authors: 
\begin{tabular}[t]{l}
Douglas Gregor, Indiana University \\
Mat Marcus, Adobe Systems, Inc.\\
Thomas Witt, Zephyr Associates, Inc.\\
Andrew Lumsdaine, Indiana University
\end{tabular}\vspace{-6pt}

\par\noindent Document number: N2677=08-0187\vspace{-6pt}
\par\noindent Revises document number: N2621=08-0131\vspace{-6pt}
\par\noindent Date: \today\vspace{-6pt}
\par\noindent Project: Programming Language \Cpp{}, Library Working Group\vspace{-6pt}
\par\noindent Reply-to: Douglas Gregor $<$\href{mailto:dgregor@osl.iu.edu}{dgregor@osl.iu.edu}$>$\vspace{-6pt}

\section*{Introduction}
This document proposes basic support for concepts in the \Cpp0x
Standard Library. It describes a new header \tcode{<concepts>} that
contains concepts that require compiler support (such as
\tcode{SameType} and \tcode{ObjectType}) and concepts that describe
common type behaviors likely to be used in many templates, including
those in the Standard Library (such as \tcode{CopyConstructible} and
\tcode{EqualityComparable}). 

Within the proposed wording, text that has been added
\textcolor{addclr}{will be presented in blue} \addedConcepts{and
  underlined when possible}. Text that has been removed will be
presented \textcolor{remclr}{in red},\removedConcepts{with
  strike-through when possible}. Removals from the previous draft
\removedCCC{strike out text in green}, additions are
\addedCC{underlined in green}.

\editorial{Purely editorial comments will be written in a separate,
  shaded box.}

\end{titlepage}

%%--------------------------------------------------
%% Headers and footers
\pagestyle{fancy}
\fancyhead[LE,RO]{\textbf{\rightmark}}
\fancyhead[RE]{\textbf{\leftmark\hspace{1em}\thepage}}
\fancyhead[LO]{\textbf{\thepage\hspace{1em}\leftmark}}
\fancyfoot[C]{Draft}

\fancypagestyle{plain}{
\renewcommand{\headrulewidth}{0in}
\fancyhead[LE,RO]{}
\fancyhead[RE,LO]{}
\fancyfoot{}
}

\renewcommand{\sectionmark}[1]{\markright{\thesection\hspace{1em}#1}}
\renewcommand{\chaptermark}[1]{\markboth{#1}{}}

\setcounter{chapter}{19}
\rSec0[utilities]{General utilities library}
\setcounter{Paras}{1}
\textcolor{black}{\pnum}
The following clauses describe utility \removedConcepts{and allocator} \changedConcepts{requirements}{concepts}, utility
components, \addedB{tuples, type traits templates,} function objects, dynamic
memory management utilities, and date/time utilities, as summarized in
Table~\ref{tab:util.lib.summary}.

\setcounter{table}{29}
\begin{libsumtab}{General utilities library summary}{tab:util.lib.summary}
\ref{utility.concepts}
\changedConcepts{Requirements}{Concepts}    &         \addedConcepts{\ttfamily <concepts>}                                          \\ \rowsep
\ref{utility} Utility components            &       \tcode{<utility>}       \\ \rowsep
\ref{tuple} \addedB{Tuples}         &       \tcode{\addedB{<tuple>}}        \\ \rowsep
\ref{meta} \addedB{Type traits}             &       \tcode{\addedB{<type_traits>}}  \\ \rowsep
\ref{function.objects} Function objects     &       \tcode{<functional>}\\ \rowsep
                                                                                        &       \tcode{<memory>}        \\
\ref{memory} Memory                                         &       \tcode{<cstdlib>}       \\
                                                                                        &       \tcode{<cstring>}       \\ \rowsep
\ref{date.time} Date and time                       &       \tcode{<ctime>}         \\
\end{libsumtab}

\noindent\editorial{Replace the section [utility.requirements] with
  the following section [utility.concepts]}

\color{addclr}
\rSec1[utility.concepts]{Concepts}

\pnum The \tcode{<concepts>} header describes requirements on template
arguments used throughout the \Cpp\ Standard Library.

\synopsis{Header \tcode{<concepts>}\ synopsis}
\begin{codeblock}
namespace std {
  // \ref{concept.support}, support concepts:
  concept Returnable<typename T> { }
  concept PointeeType<typename T> { }
  @\addedConcepts{concept MemberPointeeType<typename T> \mbox{\textit{see below}};}@
  concept ReferentType<typename T> { }
  concept VariableType<typename T> { }
  concept ObjectType<typename T> @\textit{see below}@;
  concept ClassType<typename T> @\textit{see below}@;
  concept Class<typename T> @\textit{see below}@;
  concept Union<typename T> @\textit{see below}@;
  concept TrivialType<typename T> @\textit{see below}@;
  concept StandardLayoutType<typename T> @\textit{see below}@;
  concept LiteralType<typename T> @\textit{see below}@;
  concept ScalarType<typename T> @\textit{see below}@;
  concept NonTypeTemplateParameterType<typename T> @\textit{see below}@;
  concept IntegralConstantExpressionType<typename T> @\textit{see below}@;
  concept IntegralType<typename T> @\textit{see below}@;
  concept EnumerationType<typename T> @\textit{see below}@;
  concept SameType<typename T, typename U> {  }
  concept DerivedFrom<typename Derived, typename Base> { }

  // \ref{concept.true}, true:
  concept True<bool> { }
  concept_map True<true> { }

  // \ref{concept.operator}, operator concepts:
  auto concept HasPlus<typename T, typename U@\removedCCC{= T}@> @\textit{see below}@;
  auto concept HasMinus<typename T, typename U@\removedCCC{= T}@> @\textit{see below}@;
  auto concept HasMultiply<typename T, typename U@\removedCCC{= T}@> @\textit{see below}@;
  auto concept HasDivide<typename T, typename U@\removedCCC{= T}@> @\textit{see below}@;
  auto concept HasModulus<typename T, typename U@\removedCCC{= T}@> @\textit{see below}@;
  auto concept HasUnaryPlus<typename T> @\textit{see below}@;
  auto concept HasNegate<typename T> @\textit{see below}@;
  auto concept HasLess<typename T, typename U@\removedCCC{= T}@> @\textit{see below}@;
  @\addedCC{auto concept HasGreater<typename T, typename U> \textit{see below};}@
  @\addedCC{auto concept HasLessEqual<typename T, typename U> \textit{see below};}@
  @\addedCC{auto concept HasGreaterEqual<typename T, typename U> \textit{see below};}@
  auto concept HasEqualTo<typename T, typename U@\removedCCC{= T}@> @\textit{see below}@;
  @\addedCC{auto concept HasNotEqualTo<typename T, typename U> \textit{see below};}@
  auto concept HasLogicalAnd<typename T, typename U@\removedCCC{= T}@> @\textit{see below}@;
  auto concept HasLogicalOr<typename T, typename U@\removedCCC{= T}@> @\textit{see below}@;
  auto concept HasLogicalNot<typename T> @\textit{see below}@;
  auto concept HasBitAnd<typename T, typename U@\removedCCC{= T}@> @\textit{see below}@;
  auto concept HasBitOr<typename T, typename U@\removedCCC{= T}@> @\textit{see below}@;
  auto concept HasBitXor<typename T, typename U@\removedCCC{= T}@> @\textit{see below}@;
  auto concept HasComplement<typename T> @\textit{see below}@;
  auto concept HasLeftShift<typename T, typename U@\removedCCC{= T}@> @\textit{see below}@;
  auto concept HasRightShift<typename T, typename U@\removedCCC{= T}@> @\textit{see below}@;
  auto concept @\addedCC{Has}@Dereference@\removedCCC{able}@<typename T> @\textit{see below}@;
  auto concept @\changedCCC{Addressable}{HasAddressOf}@<typename T> @\textit{see below}@;
  auto concept Callable<typename F, typename... Args> @\textit{see below}@;
  @\removedCCC{auto concept HasMoveAssign<typename T, typename U = T> \textit{see below};}@
  @\removedCCC{auto concept HasCopyAssign<typename T, typename U = T> \textit{see below};}@
  @\addedCC{auto concept HasAssign<typename T, typename U> \textit{see below};}@
  auto concept HasPlusAssign<typename T, typename U@\removedCCC{= T}@> @\textit{see below}@;
  auto concept HasMinusAssign<typename T, typename U@\removedCCC{= T}@> @\textit{see below}@;
  auto concept HasMultiplyAssign<typename T, typename U@\removedCCC{= T}@> @\textit{see below}@;
  auto concept HasDivideAssign<typename T, typename U@\removedCCC{= T}@> @\textit{see below}@;
  auto concept HasModulusAssign<typename T, typename U@\removedCCC{= T}@> @\textit{see below}@;
  auto concept HasBitAndAssign<typename T, typename U@\removedCCC{= T}@> @\textit{see below}@;
  auto concept HasBitOrAssign<typename T, typename U@\removedCCC{= T}@> @\textit{see below}@;
  auto concept HasBitXorAssign<typename T, typename U@\removedCCC{= T}@> @\textit{see below}@;
  auto concept HasLeftShiftAssign<typename T, typename U@\removedCCC{= T}@> @\textit{see below}@;
  auto concept HasRightShiftAssign<typename T, typename U@\removedCCC{= T}@> @\textit{see below}@;
  @\addedCC{auto concept HasPreincrement<typename T> \textit{see below};}@
  @\addedCC{auto concept HasPostincrement<typename T> \textit{see below};}@
  @\addedCC{auto concept HasPredecrement<typename T> \textit{see below};}@
  @\addedCC{auto concept HasPostdecrement<typename T> \textit{see below};}@
  @\addedCC{auto concept HasComma<typename T, typename U> \textit{see below};}@

  // \ref{concept.predicate}, predicates:
  auto concept Predicate<typename F, typename... Args> @\textit{see below}@;

  // \ref{concept.comparison}, comparisons:
  auto concept LessThanComparable<typename T> @\textit{see below}@;
  auto concept EqualityComparable<typename T> @\textit{see below}@;
  concept TriviallyEqualityComparable<typename T> @\textit{see below}@;
  @\addedCC{auto concept StrictWeakOrder<typename F, typename T> \mbox{\textit{see below}};}@

  // \ref{concept.construct}, construction:
  auto concept HasConstructor<typename T, typename... Args> @\textit{see below}@;
  auto concept DefaultConstructible<typename T> @\textit{see below}@;
  concept TriviallyDefaultConstructible<typename T> @\textit{see below}@;

  // \ref{concept.destruct}, destruction:
  auto concept @\changedCC{Destructible}{HasDestructor}@<typename T> @\textit{see below}@;
  @\addedCC{auto concept NothrowDestructible<typename T> \textit{see below};}@
  concept TriviallyDestructible<typename T> @\textit{see below}@;

  // \ref{concept.copymove}, copy and move:
  auto concept MoveConstructible<typename T> @\textit{see below}@;
  auto concept CopyConstructible<typename T> @\textit{see below}@;
  concept TriviallyCopyConstructible<typename T> @\textit{see below}@;
  auto concept MoveAssignable<typename T> @\textit{see below}@;
  auto concept CopyAssignable<typename T> @\textit{see below}@;
  concept TriviallyCopyAssignable<typename T> @\textit{see below}@;
  @\addedCC{auto concept HasSwap<typename T, typename U> \textit{see below};}@
  auto concept Swappable<typename T> @\textit{see below}@;

  // \ref{concept.memory}, memory allocation:
  @\addedCC{auto concept HasPlacementNew<typename T> \textit{see below};}@
  auto concept FreeStoreAllocatable<typename T> @\textit{see below}@;

  // \ref{concept.regular}, regular types:
  auto concept Semiregular<typename T> @\textit{see below}@;
  auto concept Regular<typename T> @\textit{see below}@;

  // \ref{concept.convertible}, convertibility:
  auto concept ExplicitlyConvertible<typename T, typename U> @\textit{see below}@;
  auto concept Convertible<typename T, typename U> @\textit{see below}@;


  // \ref{concept.arithmetic}, arithmetic concepts:
  concept ArithmeticLike<typename T> @\textit{see below}@;
  concept IntegralLike<typename T> @\textit{see below}@;
  concept SignedIntegralLike<typename T> @\textit{see below}@;
  concept UnsignedIntegralLike<typename T> @\textit{see below}@;
  concept FloatingPointLike<typename T> @\textit{see below}@;
}
\end{codeblock}

\rSec2[concept.support]{Support concepts}
\pnum
The concepts in [concept.support] provide the ability to state
template requirements for C++ type classifications ([basic.types]) and
type relationships that cannot be expressed directly with concepts
([concept]). Concept maps for these concepts are implicitly defined.
A program shall not provide concept maps for any concept in
[concept.support].

\begin{itemdecl}
concept Returnable<typename T> { }
\end{itemdecl}

\begin{itemdescr}
\pnum
\addedConcepts{\mbox{\reallynote} Describes types that can be used as the
  return type of a function.}

\pnum
\addedConcepts{\mbox{\requires} for every}
\addedConcepts{non-array}
\addedConcepts{type \mbox{\tcode{T}} that is
  \mbox{\techterm{cv}} \mbox{\tcode{void}} or that meets the
  requirement \mbox{\tcode{MoveConstructible<T>}}
  (\mbox{\ref{concept.copymove}}), the concept map
  \mbox{\tcode{Returnable<T>}} shall be implicitly defined in
  namespace \mbox{\tcode{std}}.}
\end{itemdescr}

\begin{itemdecl}
concept PointeeType<typename T> { }
\end{itemdecl}

\begin{itemdescr}
\pnum
\addedConcepts{\mbox{\reallynote}
describes types to which a pointer can be
created.}

\pnum 
\addedConcepts{\mbox{\requires}
for every type \mbox{\tcode{T}} that is an object type,
  function type, or \mbox{\techterm{cv}} \mbox{\tcode{void}}, a
  concept map \mbox{\tcode{PointeeType}} shall be implicitly defined
  in namespace \mbox{\tcode{std}}.}
\end{itemdescr}

\begin{itemdecl}
@\addedConcepts{concept MemberPointeeType<typename T> : PointeeType<T> \{ \}}@
\end{itemdecl}

\begin{itemdescr}
\pnum
\addedConcepts{\mbox{\reallynote}
describes types to which a pointer-to-member can be
created.}

\pnum 
\addedConcepts{\mbox{\requires}
for every type \mbox{\tcode{T}} that is an object type or
  function type, a
  concept map \mbox{\tcode{MemberPointeeType}} shall be implicitly defined
  in namespace \mbox{\tcode{std}}.}
\end{itemdescr}

\begin{itemdecl}
concept ReferentType<typename T> { }
\end{itemdecl}

\begin{itemdescr}
\pnum
\mbox{\reallynote}
describes types to which a reference can be
created\addedCC{, including reference types (since references to
  references can be formed during substitution of template
  arguments)}.

\pnum 
\mbox{\requires}
for every type \mbox{\tcode{T}} that is an object type, a
  function type, or a reference type, a
  concept map \mbox{\tcode{ReferentType}} shall be implicitly defined
  in namespace \mbox{\tcode{std}}.
\end{itemdescr}

\begin{itemdecl}
concept VariableType<typename T> { }
\end{itemdecl}

\begin{itemdescr}
\pnum
\addedConcepts{\mbox{\reallynote} describes types that can be used to
  declare a variable.}

\pnum
\addedConcepts{\mbox{\requires} for every type \mbox{\tcode{T}} that
  is an object type or reference type, a concept map
  \mbox{\tcode{VariableType<T>}} shall be implicitly defined in 
namespace \mbox{\tcode{std}}.}
\end{itemdescr}

\begin{itemdecl}
concept ObjectType<typename T> : VariableType<T>@\addedConcepts{, MemberPointeeType<T>} { }
\end{itemdecl}

\begin{itemdescr}
\pnum
\addedConcepts{\mbox{\reallynote} describes object types ([basic.types]),
  for which storage can be allocated.}

\pnum 
\addedConcepts{\mbox{\requires} for every type \mbox{\tcode{T}} that
  is an object type, a concept map
  \mbox{\tcode{ObjectType<T>}} shall be implicitly defined in
  namespace \mbox{\tcode{std}}.}
\end{itemdescr}


\begin{itemdecl}
concept ClassType<typename T> : ObjectType<T> { }
\end{itemdecl}

\begin{itemdescr}
\pnum
\addedConcepts{\mbox{\reallynote} describes class types (i.e., unions,
  classes, and structs).}

\pnum 
\addedConcepts{\mbox{\requires} for every type \mbox{\tcode{T}} that
  is a class type ([class]), a concept map \mbox{\tcode{ClassType<T>}}
  shall be implicitly defined in namespace \mbox{\tcode{std}}.}
\end{itemdescr}

\begin{itemdecl}
concept Class<typename T> : ClassType<T> { }
\end{itemdecl}

\begin{itemdescr}
\pnum
\addedConcepts{\mbox{\reallynote} describes classes and structs ([class]).}

\pnum
\addedConcepts{\mbox{\requires} for every type \mbox{\tcode{T}} that
  is a class or struct, a concept map
  \mbox{\tcode{Class<T>}} shall be implicitly defined in namespace
  \mbox{\tcode{std}}.}
\end{itemdescr}

\begin{itemdecl}
concept Union<typename T> : ClassType<T> { }
\end{itemdecl}

\begin{itemdescr}
\pnum
\addedConcepts{\mbox{\reallynote} describes union types ([class.union]).}

\pnum
\addedConcepts{\mbox{\requires} for every type \mbox{\tcode{T}} that
  is a union, a concept map \mbox{\tcode{Union<T>}}
  shall be implicitly defined in namespace \mbox{\tcode{std}}.}
\end{itemdescr}

\begin{itemdecl}
concept TrivialType<typename T> : ObjectType<T> { }
\end{itemdecl}

\begin{itemdescr}
\pnum
\addedConcepts{\mbox{\reallynote} describes trivial types ([basic.types]).}

\pnum
\addedConcepts{\mbox{\requires} for every type \mbox{\tcode{T}} that
  is a trivial type, a concept map \mbox{\tcode{TrivialType<T>}} shall
  be implicitly defined in namespace \mbox{\tcode{std}}.}
\end{itemdescr}

\begin{itemdecl}
concept StandardLayoutType<typename T> : ObjectType<T> { }
\end{itemdecl}

\begin{itemdescr}
\pnum
\addedConcepts{\mbox{\reallynote} describes standard-layout types ([basic.types]).}

\pnum
\addedConcepts{\mbox{\requires} for every type \mbox{\tcode{T}} that
  is a standard-layout type, a concept map
  \mbox{\tcode{StandardLayoutType<T>}} shall be implicitly defined in 
namespace \mbox{\tcode{std}}.}
\end{itemdescr}

\begin{itemdecl}
concept LiteralType<typename T> : ObjectType<T> { }
\end{itemdecl}

\begin{itemdescr}
\pnum
\addedConcepts{\mbox{\reallynote} describes literal types ([basic.types]).}

\pnum
\addedConcepts{\mbox{\requires} for every type \mbox{\tcode{T}} that
  is a literal type, a concept map \mbox{\tcode{LiteralType<T>}} shall
  be implicitly defined in namespace \mbox{\tcode{std}}.}
\end{itemdescr}

\begin{itemdecl}
concept ScalarType<typename T> 
  : TrivialType<T>, LiteralType<T>, StandardLayoutType<T> { }
\end{itemdecl}

\begin{itemdescr}
\pnum
\addedConcepts{\mbox{\reallynote} describes scalar types ([basic.types]).}

\pnum
\addedConcepts{\mbox{\requires} for every type \mbox{\tcode{T}} that
  is a scalar type, a concept map \mbox{\tcode{ScalarType<T>}} shall
  be implicitly defined in namespace \mbox{\tcode{std}}.}
\end{itemdescr}

\begin{itemdecl}
concept NonTypeTemplateParameterType<typename T> : VariableType<T> { }
\end{itemdecl}

\begin{itemdescr}
\pnum 
\addedConcepts{\mbox{\reallynote} describes type that can
be used as the type of a non-type template parameter ([temp.param]).}

\pnum
\addedConcepts{\mbox{\requires}
for every type \mbox{\tcode{T}} that can be the type of a non-type
\mbox{\techterm{template-parameter}} ([temp.param]), a concept map
\mbox{\tcode{NonTypeTemplateParameterType<T>}} shall be implicitly defined in
namespace \mbox{\tcode{std}}.}
\end{itemdescr}

\begin{itemdecl}
concept IntegralConstantExpressionType<typename T> 
  : ScalarType<T>, NonTypeTemplateParameterType<T> { }
\end{itemdecl}

\begin{itemdescr}
\pnum
\addedConcepts{\mbox{\reallynote} describes types that
can be the type of an integral constant expression ([expr.const]).} 

\pnum
\addedConcepts{\mbox{\requires} for every type \mbox{\tcode{T}} that
  is an integral type or enumeration type, a concept map}\\
  \addedConcepts{\mbox{\tcode{IntegralConstantExpressionType<T>}} shall be implicitly
  defined in namespace \mbox{\tcode{std}}.}
\end{itemdescr}

\begin{itemdecl}
concept IntegralType<typename T> : IntegralConstantExpressionType<T> { }
\end{itemdecl}

\begin{itemdescr}
\pnum
\addedConcepts{\mbox{\reallynote} describes integral types
([basic.fundamental]).}

\pnum
\addedConcepts{\mbox{\requires}
for every type \mbox{\tcode{T}} that is an integral type, a concept map
\mbox{\tcode{IntegralType<T>}} shall be implicitly defined in
namespace \mbox{\tcode{std}}.}
\end{itemdescr}

\begin{itemdecl}
concept EnumerationType<typename T> : IntegralConstantExpressionType<T> { }
\end{itemdecl}

\begin{itemdescr}
\pnum
\addedConcepts{\mbox{\reallynote} describes enumeration types
([dcl.enum]).} 

\pnum
\addedConcepts{\mbox{\requires}
for every type \mbox{\tcode{T}} that is an enumeration type, a concept map
\mbox{\tcode{EnumerationType<T>}} shall be implicitly defined in namespace
\mbox{\tcode{std}}.}
\end{itemdescr}

\begin{itemdecl}
concept SameType<typename T, typename U> { }
\end{itemdecl}

\begin{itemdescr}
\pnum
\addedConcepts{\mbox{\reallynote} describes a same-type requirement
  ([temp.req]).}
\end{itemdescr}

\begin{itemdecl}
concept DerivedFrom<typename Derived, typename Base> { }
\end{itemdecl}

\begin{itemdescr}
\pnum
\mbox{\requires}
for every pair of class types (\mbox{\tcode{T}}, \mbox{\tcode{U}}),
such that \mbox{\tcode{T}} is either the same as or publicly and
unambiguously derived from \mbox{\tcode{U}}, a concept map
\mbox{\tcode{DerivedFrom<T, U>}} shall be implicitly defined in namespace
\mbox{\tcode{std}}.
\end{itemdescr}

\rSec2[concept.true]{True}

\begin{itemdecl}
concept True<bool> { }
concept_map True<true> { }
\end{itemdecl}

\begin{itemdescr}
\pnum
\addedConcepts{\mbox{\reallynote} used to express the requirement that a
  particular integral constant expression evaluate true.}

\pnum
\addedConcepts{\mbox{\requires} a program shall not provide a concept map for the
\mbox{\tcode{True}} concept.}
\end{itemdescr}

\rSec2[concept.operator]{Operator concepts}
\begin{itemdecl}
auto concept HasPlus<typename T, typename U@\removedCCC{= T}@> {
  typename result_type;
  result_type operator+(const T&, const U&);
}
\end{itemdecl}

\begin{itemdescr}
\pnum
\addedConcepts{\mbox{\reallynote} describes types with a binary \mbox{\tcode{operator+}}.}
\end{itemdescr}

\begin{itemdecl}
auto concept HasMinus<typename T, typename U@\removedCCC{= T}@> {
  typename result_type;
  result_type operator-(const T&, const U&);
}
\end{itemdecl}

\begin{itemdescr}
\pnum
\addedConcepts{\mbox{\reallynote} describes types with a binary \mbox{\tcode{operator-}}.}
\end{itemdescr}

\begin{itemdecl}
auto concept HasMultiply<typename T, typename U@\removedCCC{= T}@> {
  typename result_type;
  result_type operator*(const T&, const U&);
}
\end{itemdecl}

\begin{itemdescr}
\pnum
\addedConcepts{\mbox{\reallynote} describes types with a binary \mbox{\tcode{operator*}}.}
\end{itemdescr}

\begin{itemdecl}
auto concept HasDivide<typename T, typename U@\removedCCC{= T}@> {
  typename result_type;
  result_type operator/(const T&, const U&);
}
\end{itemdecl}

\begin{itemdescr}
\pnum
\addedConcepts{\mbox{\reallynote} describes types with an \mbox{\tcode{operator/}}.}
\end{itemdescr}

\begin{itemdecl}
auto concept HasModulus<typename T, typename U@\removedCCC{= T}@> {
  typename result_type;
  result_type operator%(const T&, const U&);
}
\end{itemdecl}

\begin{itemdescr}
\pnum
\addedConcepts{\mbox{\reallynote} describes types with an \mbox{\tcode{operator\%}}.}
\end{itemdescr}

\begin{itemdecl}
auto concept HasUnaryPlus<typename T> {
  typename result_type;
  result_type operator+(const T&);
}
\end{itemdecl}

\begin{itemdescr}
\pnum
\mbox{\reallynote} describes types with a unary \mbox{\tcode{operator+}}.
\end{itemdescr}

\begin{itemdecl}
auto concept HasNegate<typename T> {
  typename result_type;
  result_type operator-(const T&);
}
\end{itemdecl}

\begin{itemdescr}
\pnum
\addedConcepts{\mbox{\reallynote} describes types with a unary \mbox{\tcode{operator-}}.}
\end{itemdescr}

\begin{itemdecl}
auto concept HasLess<typename T, typename U@\removedCCC{= T}@> {
  bool operator<(const T& a, const U& b);
}
\end{itemdecl}

\begin{itemdescr}
\pnum
\addedConcepts{\mbox{\reallynote} describes types with an
  \mbox{\tcode{operator<}}.}
\end{itemdescr}

\begin{itemdecl}
@\addedCC{auto concept HasGreater<typename T, typename U> \{}@
  @\addedCC{bool operator>(const T\& a, const U\& b);}@
@\addedCC{\}}@
\end{itemdecl}

\begin{itemdescr}
\pnum
\addedCC{\mbox{\reallynote} describes types with an
  \mbox{\tcode{operator>}}.}
\end{itemdescr}

\begin{itemdecl}
@\addedCC{auto concept HasLessEqual<typename T, typename U> \{}@
  @\addedCC{bool operator<=(const T\& a, const U\& b);}@
@\addedCC{\}}@
\end{itemdecl}

\begin{itemdescr}
\pnum
\addedCC{\mbox{\reallynote} describes types with an
  \mbox{\tcode{operator<=}}.}
\end{itemdescr}

\begin{itemdecl}
@\addedCC{auto concept HasGreaterEqual<typename T, typename U> \{}@
  @\addedCC{bool operator>=(const T\& a, const U\& b);}@
@\addedCC{\}}@
\end{itemdecl}

\begin{itemdescr}
\pnum
\addedCC{\mbox{\reallynote} describes types with an
  \mbox{\tcode{operator>=}}.}
\end{itemdescr}

\begin{itemdecl}
auto concept HasEqualTo<typename T, typename U@\removedCCC{= T}@> {
  bool operator==(const T& a, const U& b);
}
\end{itemdecl}

\begin{itemdescr}
\pnum
\addedConcepts{\mbox{\reallynote} describes types with an
  \mbox{\tcode{operator==}}.}
\end{itemdescr}

\begin{itemdecl}
@\addedCC{auto concept HasNotEqualTo<typename T, typename U> \{}@
  @\addedCC{bool operator!=(const T\& a, const U\& b);}@
@\addedCC{\}}@
\end{itemdecl}

\begin{itemdescr}
\pnum
\addedCC{\mbox{\reallynote} describes types with an
  \mbox{\tcode{operator!=}}.}
\end{itemdescr}

\begin{itemdecl}
auto concept HasLogicalAnd<typename T, typename U@\removedCCC{= T}@> {
  bool operator&&(const T&, const U&);
}
\end{itemdecl}

\begin{itemdescr}
\pnum
\addedConcepts{\mbox{\reallynote} describes types with a logical conjunction operator.}
\end{itemdescr}

\begin{itemdecl}
auto concept HasLogicalOr<typename T, typename U@\removedCCC{= T}@> {
  bool operator||(const T&, const U&);
}
\end{itemdecl}

\begin{itemdescr}
\pnum
\addedConcepts{\mbox{\reallynote} describes types with a logical disjunction operator.}
\end{itemdescr}

\begin{itemdecl}
auto concept HasLogicalNot<typename T> {
  bool operator!(const T&);
}
\end{itemdecl}

\begin{itemdescr}
\pnum
\addedConcepts{\mbox{\reallynote} describes types with a logical negation operator.}
\end{itemdescr}

\begin{itemdecl}
auto concept HasBitAnd<typename T, typename U@\removedCCC{= T}@> {
  typename result_type;
  result_type operator&(const T&, const U&);
}
\end{itemdecl}

\begin{itemdescr}
\pnum
\addedConcepts{\mbox{\reallynote} describes types with a binary \mbox{\tcode{operator\&}}.}
\end{itemdescr}

\begin{itemdecl}
auto concept HasBitOr<typename T, typename U@\removedCCC{= T}@> {
  typename result_type;
  result_type operator|(const T&, const U&);
}
\end{itemdecl}

\begin{itemdescr}
\pnum
\addedConcepts{\mbox{\reallynote} describes types with an \mbox{\tcode{operator|}}.}
\end{itemdescr}

\begin{itemdecl}
auto concept HasBitXor<typename T, typename U@\removedCCC{= T}@> {
  typename result_type;
  result_type operator^(const T&, const U&);
}
\end{itemdecl}

\begin{itemdescr}
\pnum
\addedConcepts{\mbox{\reallynote} describes types with an \mbox{\tcode{operator\^}}.}
\end{itemdescr}

\begin{itemdecl}
auto concept HasComplement<typename T> {
  typename result_type;
  result_type operator~(const T&);
}
\end{itemdecl}

\begin{itemdescr}
\pnum
\addedConcepts{\mbox{\reallynote} describes types with an
  \mbox{\tcode{operator\~}}.}
\end{itemdescr}

\begin{itemdecl}
auto concept HasLeftShift<typename T, typename U@\removedCCC{= T}@> {
  typename result_type;
  result_type operator<<(const T&, const U&);
}
\end{itemdecl}

\begin{itemdescr}
\pnum
\mbox{\reallynote} describes types with an \mbox{\tcode{operator$<<$}}.
\end{itemdescr}

\begin{itemdecl}
auto concept HasRightShift<typename T, typename U@\removedCCC{= T}@> {
  typename result_type;
  result_type operator>>(const T&, const U&);
}
\end{itemdecl}

\begin{itemdescr}
\pnum
\mbox{\reallynote} describes types with an \mbox{\tcode{operator$>>$}}.
\end{itemdescr}

\begin{itemdecl}
auto concept @\addedCC{Has}@Dereference@\removedCCC{able}@<typename T> {
  typename @\changedCCC{reference}{result_type}@;
  @\changedCCC{reference}{result_type}@ operator*(@\addedCC{const}@ T@\addedCC{\&}@);
}
\end{itemdecl}

\begin{itemdescr}
\pnum
\addedConcepts{\mbox{\reallynote} describes types with a dereferencing \mbox{\tcode{operator*}}.}
\end{itemdescr}

\begin{itemdecl}
auto concept @\changedCCC{Addressable}{HasAddressOf}@<typename T> {
  typename @\changedCCC{pointer}{result_type}@;
  @\changedCCC{pointer}{result_type}@ operator&(T&);
}
\end{itemdecl}

\begin{itemdescr}
\pnum
\addedConcepts{\mbox{\reallynote} describes types with an address-of \mbox{\tcode{operator\&}}.}
\end{itemdescr}

\begin{itemdecl}
auto concept Callable<typename F, typename... Args> {
  typename result_type;
  result_type operator()(F&@\addedCC{\&}@, Args...);
}
\end{itemdecl}

\begin{itemdescr}
\pnum 
\addedConcepts{\mbox{\reallynote} describes function object types 
callable given arguments of types \mbox{\tcode{Args...}}.}
\end{itemdescr}

\begin{itemdecl}
auto concept Has@\removedCCC{Move}@Assign<typename T, typename U@\removedCCC{= T}@> {
  typename result_type;
  result_type T::operator=(U@\removedCCC{\&\&}@);
}
\end{itemdecl}

\begin{itemdescr}
\pnum
\addedConcepts{\mbox{\reallynote} describes types with}
\changedCCC{the ability
  to assign to an object from an rvalue (which may have a different
  type), potentially altering the rvalue.}{an assignment operator.}
\end{itemdescr}

\begin{itemdecl}
@\removedCCC{auto concept HasCopyAssign<typename T, typename U = T> : HasMoveAssign<T, U> \{}@
  @\removedCCC{result_type T::operator=(const U\&);}@
@\removedCCC{\}}
\end{itemdecl}

\begin{itemdescr}
\pnum
\addedConcepts{\mbox{\reallynote} describes types with the ability to assign to an
object (which may have a different type).}
\end{itemdescr}

\begin{itemdecl}
auto concept HasPlusAssign<typename T, typename U@\removedCCC{= T}@> {
  typename result_type;
  result_type operator+=(T&, @\removedCCC{const}@ U@\removedCCC{\&}@);
}
\end{itemdecl}

\begin{itemdescr}
\pnum
\mbox{\reallynote} describes types with an \mbox{\tcode{operator$+=$}}.
\end{itemdescr}

\begin{itemdecl}
auto concept HasMinusAssign<typename T, typename U@\removedCCC{= T}@> {
  typename result_type;
  result_type operator-=(T&, @\removedCCC{const}@ U@\removedCCC{\&}@);
}
\end{itemdecl}

\begin{itemdescr}
\pnum
\mbox{\reallynote} describes types with an \mbox{\tcode{operator$-=$}}.
\end{itemdescr}

\begin{itemdecl}
auto concept HasMultiplyAssign<typename T, typename U@\removedCCC{= T}@> {
  typename result_type;
  result_type operator*=(T&, @\removedCCC{const}@ U@\removedCCC{\&}@);
}
\end{itemdecl}

\begin{itemdescr}
\pnum
\mbox{\reallynote} describes types with an \mbox{\tcode{operator$*=$}}.
\end{itemdescr}

\begin{itemdecl}
auto concept HasDivideAssign<typename T, typename U@\removedCCC{= T}@> {
  typename result_type;
  result_type operator/=(T&, @\removedCCC{const}@ U@\removedCCC{\&}@);
}
\end{itemdecl}

\begin{itemdescr}
\pnum
\mbox{\reallynote} describes types with an \mbox{\tcode{operator$/=$}}.
\end{itemdescr}

\begin{itemdecl}
auto concept HasModulusAssign<typename T, typename U@\removedCCC{= T}@> {
  typename result_type;
  result_type operator%=(T&, @\removedCCC{const}@ U@\removedCCC{\&}@);
}
\end{itemdecl}

\begin{itemdescr}
\pnum
\mbox{\reallynote} describes types with an \mbox{\tcode{operator$\%=$}}.
\end{itemdescr}

\begin{itemdecl}
auto concept HasBitAndAssign<typename T, typename U@\removedCCC{= T}@> {
  typename result_type;
  result_type operator&=(T&, @\removedCCC{const}@ U@\removedCCC{\&}@);
}
\end{itemdecl}

\begin{itemdescr}
\pnum
\mbox{\reallynote} describes types with an \mbox{\tcode{operator$\&=$}}.
\end{itemdescr}

\begin{itemdecl}
auto concept HasBitOrAssign<typename T, typename U@\removedCCC{= T}@> {
  typename result_type;
  result_type operator|=(T&, @\removedCCC{const}@ U@\removedCCC{\&}@);
}
\end{itemdecl}

\begin{itemdescr}
\pnum
\mbox{\reallynote} describes types with an \mbox{\tcode{operator$|=$}}.
\end{itemdescr}

\begin{itemdecl}
auto concept HasBitXorAssign<typename T, typename U@\removedCCC{= T}@> {
  typename result_type;
  result_type operator^=(T&, @\removedCCC{const}@ U@\removedCCC{\&}@);
}
\end{itemdecl}

\begin{itemdescr}
\pnum
\mbox{\reallynote} describes types with an \mbox{\tcode{operator\^{}=}}.
\end{itemdescr}

\begin{itemdecl}
auto concept HasLeftShiftAssign<typename T, typename U@\removedCCC{= T}@> {
  typename result_type;
  result_type operator<<=(T&, @\removedCCC{const}@ U@\removedCCC{\&}@);
}
\end{itemdecl}

\begin{itemdescr}
\pnum
\mbox{\reallynote} describes types with an \mbox{\tcode{operator$<<=$}}.
\end{itemdescr}

\begin{itemdecl}
auto concept HasRightShiftAssign<typename T, typename U@\removedCCC{= T}@> {
  typename result_type;
  result_type operator>>=(T&, @\removedCCC{const}@ U@\removedCCC{\&}@);
}
\end{itemdecl}

\begin{itemdescr}
\pnum
\mbox{\reallynote} describes types with an \mbox{\tcode{operator$>>=$}}.
\end{itemdescr}

\begin{itemdecl}
@\addedCC{auto concept HasPreincrement<typename T> \{}@
  @\addedCC{typename result_type;}@
  @\addedCC{result_type operator++(T\&);}@
@\addedCC{\}}@
\end{itemdecl}

\begin{itemdescr}
\pnum
\addedCC{\mbox{\reallynote} describes types with a pre-increment operator.}
\end{itemdescr}

\begin{itemdecl}
@\addedCC{auto concept HasPostincrement<typename T> \{}@
  @\addedCC{typename result_type;}@
  @\addedCC{result_type operator++(T\&, int);}@
@\addedCC{\}}@
\end{itemdecl}

\begin{itemdescr}
\pnum
\addedCC{\mbox{\reallynote} describes types with a post-increment operator.}
\end{itemdescr}

\begin{itemdecl}
@\addedCC{auto concept HasPredecrement<typename T> \{}@
  @\addedCC{typename result_type;}@
  @\addedCC{result_type operator-{}-(T\&);}@
@\addedCC{\}}@
\end{itemdecl}

\begin{itemdescr}
\pnum
\addedCC{\mbox{\reallynote} describes types with a pre-decrement operator.}
\end{itemdescr}

\begin{itemdecl}
@\addedCC{auto concept HasPostdecrement<typename T> \{}@
  @\addedCC{typename result_type;}@
  @\addedCC{result_type operator-{}-(T\&, int);}@
@\addedCC{\}}@
\end{itemdecl}

\begin{itemdescr}
\pnum
\addedCC{\mbox{\reallynote} describes types with a post-decrement operator.}
\end{itemdescr}

\begin{itemdecl}
@\addedCC{auto concept HasComma<typename T, typename U> \{}@
  @\addedCC{typename result_type}@
  @\addedCC{result_type operator,(const T\&, const U\&);}@
@\addedCC{\}}@
\end{itemdecl}

\begin{itemdescr}
\pnum
\addedCC{\mbox{\reallynote} describes types with a comma operator.}
\end{itemdescr}

\rSec2[concept.predicate]{Predicates}

\begin{itemdecl}
auto concept Predicate<typename F, typename... Args> : Callable<F, @\addedCC{const}@ Args@\addedCC{\&}@...> {
  requires Convertible<result_type, bool>;
}
\end{itemdecl}

\begin{itemdescr}
\pnum
\addedConcepts{\mbox{\reallynote} describes function objects 
callable with some set of arguments, the result of which can be used in a
context that requires a \mbox{\tcode{bool}}.}

\pnum
\addedConcepts{\mbox{\requires} 
predicate function objects shall not apply any non-constant function
through the predicate arguments.}
\end{itemdescr}

\rSec2[concept.comparison]{Comparisons}
\begin{itemdecl}
auto concept LessThanComparable<typename T> : HasLess<T@\addedCC{, T}@> {
  bool operator>(const T& a, const T& b) { return b < a; }
  bool operator<=(const T& a, const T& b) { return !(b < a); }
  bool operator>=(const T& a, const T& b) { return !(a < b); }

  axiom Consistency(T a, T b) {
    (a > b) == (b < a);
    (a <= b) == !(b < a);
    (a >= b) == !(a < b);
  }

  axiom Irreflexivity(T a) { (a < a) == false; }

  axiom Antisymmetry(T a, T b) { 
    if (a < b) 
      (b < a) == false;
  }

  axiom Transitivity(T a, T b, T c) {
    if (a < b && b < c) 
      (a < c) == true;
  }

  axiom TransitivityOfEquivalence(T a, T b, T c) {
    if (!(a < b) && !(b < a) && !(b < c) && !(c < b))
      (!(a < c) && !(c < a)) == true;
  } 
}
\end{itemdecl}

\begin{itemdescr}
\pnum 
\addedConcepts{\mbox{\reallynote} describes types whose values can be
  ordered, where \mbox{\tcode{operator<}}
  is a strict weak ordering relation (\mbox{\ref{alg.sorting}}).}
\end{itemdescr}

\color{ccadd}
\begin{itemdecl}
auto concept StrictWeakOrder<typename F, typename T> : Predicate<F, T, T> {

  axiom Irreflexivity(F f, T a) { f(a, a) == false; }

  axiom Antisymmetry(F f, T a, T b) { 
    if (f(a, b)) 
      f(b, a) == false;
  }

  axiom Transitivity(F f, T a, T b, T c) {
    if (f(a, b) && f(b, c)) 
      f(a, c) == true;
  }

  axiom TransitivityOfEquivalence(F f, T a, T b, T c) {
    if (!f(a, b) && !f(b, a) && !f(b, c) && !f(c, b))
      (!f(a, c) && !f(c, a)) == true;
  }
}
\end{itemdecl}
\color{addclr}

\begin{itemdescr}
\pnum 
\addedCC{\mbox{\reallynote} describes a strict weak ordering
  relation (\mbox{\ref{alg.sorting}}), \mbox{\tcode{F}}, on a type \mbox{\tcode{T}}.}
\end{itemdescr}

\begin{itemdecl}
auto concept EqualityComparable<typename T> : HasEqualTo<T@\addedCC{, T}@> {
  bool operator!=(const T& a, const T& b) { return !(a == b); }

  axiom Consistency(T a, T b) {
    (a == b) == !(a != b);
  }

  axiom Reflexivity(T a) { a == a; }

  axiom Symmetry(T a, T b) { 
    if (a == b) 
      b == a; 
  }

  axiom Transitivity(T a, T b, T c) {
    if (a == b && b == c) 
      a == c;
  }
}
\end{itemdecl}

\begin{itemdescr}
\pnum
\addedConcepts{\mbox{\reallynote} describes types whose values can be
compared for equality with \mbox{\tcode{operator==}}, which is an
equivalence relation.}
\end{itemdescr}

\begin{itemdecl}
concept TriviallyEqualityComparable<typename T> : EqualityComparable<T> { }
\end{itemdecl}

\begin{itemdescr}
\pnum 
\addedConcepts{\mbox{\reallynote} describes types whose equality comparison
  operators (\mbox{\tcode{==}}, \mbox{\tcode{!=}}) can be implemented
  via a bitwise equality comparison, as with \mbox{\tcode{memcmp}}.
\mbox{\enternote} such types should not have
padding, i.e. the size of the type is the sum of the sizes of its
elements. If padding exists, the comparison may provide false
negatives, but never false positives. \mbox{\exitnote}}

\pnum
\addedConcepts{\mbox{\requires} for every integral type
  \mbox{\tcode{T}} and pointer type, a concept map
  \mbox{\tcode{TriviallyEqualityComparable<T>}} shall be 
  defined in namespace \mbox{\tcode{std}}.}
\end{itemdescr}

\rSec2[concept.construct]{Construction}
\begin{itemdecl}
auto concept HasConstructor<typename T, typename... Args> @\removedCCC{: Destructible<T>}@ {
  T::T(Args...);
}
\end{itemdecl}

\begin{itemdescr}
\pnum
\addedConcepts{\mbox{\reallynote} describes types that can be constructed
  from a given set of arguments.}
\end{itemdescr}

\begin{itemdecl}
auto concept DefaultConstructible<typename T> : HasConstructor<T> { }
\end{itemdecl}

\begin{itemdescr}
\pnum
\addedConcepts{\mbox{\reallynote} describes types for which an object
  can be constructed without initializing the object to any particular
  value.}
\end{itemdescr}

\begin{itemdecl}
@\addedCC{concept TriviallyDefaultConstructible<typename T> : DefaultConstructible<T> \{\}}@
\end{itemdecl}

\begin{itemdescr}
\pnum
\addedCC{\mbox{\reallynote} describes types whose default constructor is trivial.}

\pnum \addedCC{\mbox{\requires} for every type \mbox{\tcode{T}} that is
  a trivial type (\mbox{\ref{basic.types}}) or a class type with a
  trivial default constructor (\mbox{\ref{class.ctor}}), a concept map
  \mbox{\tcode{TriviallyDefaultConstructible<T>}} shall be implicitly
  defined in namespace \mbox{\tcode{std}}.}
\end{itemdescr}

\rSec2[concept.destruct]{Destruction}
\begin{itemdecl}
auto concept @\changedCCC{Destructible}{HasDestructor}@<typename T> : VariableType<T> {
  T::~T();
}
\end{itemdecl}

\begin{itemdescr}
\pnum
\addedConcepts{\mbox{\reallynote} describes types that can be destroyed}\changedCCC{,
  including}{. These are} 
\addedConcepts{scalar types, references, and class types with a public}
\addedCC{non-deleted}
  \addedConcepts{destructor.}

\pnum
\removedCCC{\mbox{\requires} following destruction of an object, all resources owned by the object are reclaimed.}
\end{itemdescr}

\begin{itemdecl}
@\addedCC{auto concept NothrowDestructible<typename T> : HasDestructor<T> \{ \}}@
\end{itemdecl}

\begin{itemdescr}
\begin{codeblock}
T::~T() @\addedCC{// inherited from HasDestructor<T>}@
\end{codeblock}
\pnum
\addedCC{\mbox{\requires} no exception is propagated.}
\end{itemdescr}

\begin{itemdecl}
concept TriviallyDestructible<typename T> : @\addedCC{Nothrow}@Destructible<T> { }
\end{itemdecl}

\begin{itemdescr}
\pnum
\addedConcepts{\mbox{\reallynote} describes types whose
destructors do not need to be executed when the object is destroyed.}

\pnum
\addedConcepts{\mbox{\requires} for every type \mbox{\tcode{T}} that
  is a trivial type ([basic.types]), reference, or class type with a
  trivial destructor ([class.dtor]), a concept map
  \mbox{\tcode{TriviallyDestructible<T>}} shall be implicitly 
  defined in namespace \mbox{\tcode{std}}.}
\end{itemdescr}

\rSec2[concept.copymove]{Copy and move}
\begin{itemdecl}
auto concept MoveConstructible<typename T> : HasConstructor<T, T&&> { }
\end{itemdecl}

\begin{itemdescr}
\pnum
\addedConcepts{\mbox{\reallynote} 
describes types that can move-construct an
object from a value of the same type, possibly altering that
value.}
\end{itemdescr}

\begin{itemdecl}
T::T(T&& rv); // note: inherited from HasConstructor<T, T\&\&>
\end{itemdecl}

\begin{itemdescr}
\pnum
\addedConcepts{\mbox{\postcondition}
the constructed \mbox{\tcode{T}} object is equivalent to the value of
\mbox{\tcode{rv}} before the construction. 
\mbox{\enternote} there is no requirement on the value of
\mbox{\tcode{rv}} after the construction. \mbox{\exitnote}}
\end{itemdescr}

\begin{itemdecl}
auto concept CopyConstructible<typename T> : MoveConstructible<T>, HasConstructor<T, const T&> {
  axiom CopyPreservation(T x) {
    T(x) == x;
  }
}
\end{itemdecl}

\begin{itemdescr}
\pnum 
\addedConcepts{\mbox{\reallynote} describes types with a public copy constructor.}
\end{itemdescr}

\begin{itemdecl}
concept TriviallyCopyConstructible<typename T> : CopyConstructible<T> { }
\end{itemdecl}

\begin{itemdescr}
\pnum 
\addedConcepts{\mbox{\reallynote} describes types whose copy
constructor is equivalent to \mbox{\tcode{memcpy}}.}

\pnum
\addedConcepts{\mbox{\requires} for every type \mbox{\tcode{T}} that
  is a trivial type ([basic.types]), a reference, or a class type with
  a trivial copy constructor ([class.copy]), a concept map
  \mbox{\tcode{TriviallyCopyConstructible<T>}} 
shall be implicitly defined in namespace \mbox{\tcode{std}}.}
\end{itemdescr}

\begin{itemdecl}
auto concept MoveAssignable<typename T> : Has@\removedCCC{Move}@Assign<T@\addedCC{, T\&\&}@> { }
\end{itemdecl}

\begin{itemdescr}
\pnum
\addedConcepts{\mbox{\reallynote} describes types with the ability
  to assign to an object from an rvalue, potentially altering the rvalue.}
\end{itemdescr}

\begin{itemdecl}
result_type T::operator=(T&& rv); // inherited from \changedCCC{HasMoveAssign}{HasAssign<T, T\&\&>}
\end{itemdecl}

\begin{itemdescr}
\pnum
\addedConcepts{\mbox{\postconditions}
the constructed \mbox{\tcode{T}} object is equivalent to the value of
\mbox{\tcode{rv}} before the assignment. 
\mbox{\enternote} there is no requirement on the value of
\mbox{\tcode{rv}} after 
the assignment. \mbox{\exitnote}}
\end{itemdescr}

\begin{itemdecl}
auto concept CopyAssignable<typename T> : @\addedCC{HasAssign<T, const T\&>, }@MoveAssignable<T> {
  axiom CopyPreservation(T& x, T y) {
    (x = y, x) == y;
  }
}
\end{itemdecl}

\begin{itemdescr}
\pnum
\addedConcepts{\mbox{\reallynote} describes types with the ability to assign to an
object.}
\end{itemdescr}

\editorial{\textcolor{black}{The CopyAssignable requirements in N2461 specify that
  \tcode{operator=} must return a \tcode{T\&}. This is too strong a
  requirement for most of the uses of \tcode{CopyAssignable}, so we have
  weakened \tcode{CopyAssignable} to not require anything of its return
  type. When we need a \tcode{T\&}, we'll add that as an explicit
  requirement. See, e.g., the \tcode{IntegralLike} concept.}}

\begin{itemdecl}
concept TriviallyCopyAssignable<typename T> : CopyAssignable<T> { }
\end{itemdecl}

\begin{itemdescr}
\pnum
\addedConcepts{\mbox{\reallynote} describes types whose copy-assignment
  operator is equivalent to \mbox{\tcode{memcpy}}.}

\pnum
\addedConcepts{\mbox{\requires} for every type \mbox{\tcode{T}} that
  is a trivial type ([basic.types]) or a 
class type with a trivial copy assignment operator ([class.copy]), a
concept map \mbox{\tcode{TriviallyCopyAssignable<T>}} shall be implicitly
defined in namespace \mbox{\tcode{std}}.}
\end{itemdescr}

\begin{itemdecl}
@\addedCC{auto concept HasSwap<typename T, typename U> \{}@
  @\addedCC{void swap(T, U);}@
@\addedCC{\}}@
\end{itemdecl}

\begin{itemdescr}
\pnum
\addedCC{\mbox{\reallynote} describes types that have a swap operation.}
\end{itemdescr}

\begin{itemdecl}
auto concept Swappable<typename T> @\addedCC{: HasSwap<T\&, T\&>}@ {
  @\removedCCC{void swap(T\&, T\&);}@
}
\end{itemdecl}

\begin{itemdescr}
\pnum
\addedConcepts{\mbox{\reallynote} describes types for which two values of
  that type can be swapped.}
\end{itemdescr}

\begin{itemdecl}
void swap(T& t, T& u); @\addedCC{// inherited from HasSwap<T, T>}@
\end{itemdecl}

\begin{itemdescr}
\pnum
\addedConcepts{\mbox{\postconditions}
\mbox{\tcode{t}} has the value originally held by \mbox{\tcode{u}},
and \mbox{\tcode{u}} has the value originally held by \mbox{\tcode{t}}.}
\end{itemdescr}

\rSec2[concept.memory]{Memory allocation}
\begin{itemdecl}
@\addedCC{auto concept HasPlacementNew<typename T> \{}@
  @\addedCC{void* T::operator new(size_t size, void*);}@
@\addedCC{\}}@
\end{itemdecl}

\begin{itemdescr}
\pnum
\addedCC{\mbox{\reallynote} Describes types that have a placement new.}
\end{itemdescr}

\begin{itemdecl}
auto concept FreeStoreAllocatable<typename T> {
  void* T::operator new(size_t size);
  @\removedCCC{void* T::operator new(size_t size, void*);}@
  void* T::operator new[](size_t size);
  void T::operator delete(void*);
  void T::operator delete[](void*);

  @\addedCC{void* T::operator new(size_t size, const nothrow_t\&) \{}@
    @\addedCC{try \{}@
      @\addedCC{return T::operator new(size);}@
    @\addedCC{\} catch(...) \{}@
      @\addedCC{return 0;}@
    @\addedCC{\}}@
  @\addedCC{\}}@

  @\addedCC{void* T::operator new[](size_t size, const nothrow_t\&) \{}@
    @\addedCC{try \{}@
      @\addedCC{return T::operator new[](size);}@
    @\addedCC{\} catch(...) \{}@
      @\addedCC{return 0;}@
    @\addedCC{\}}@
  @\addedCC{\}}@

  @\addedCC{void T::operator delete(void* ptr, const nothrow_t\&) \{}@
    @\addedCC{T::operator delete(ptr);}@
  @\addedCC{\}}@

  @\addedCC{void T::operator delete[](void* ptr, const nothrow_t\&) \{}@
    @\addedCC{T::operator delete[](ptr);}@
  @\addedCC{\}}@
}
\end{itemdecl}

\begin{itemdescr}
\pnum
\addedConcepts{\mbox{\reallynote} describes types for which objects and
arrays of objects can be allocated on or freed from the free store with
\mbox{\tcode{new}} and \mbox{\tcode{delete}}.}
\end{itemdescr}

\rSec2[concept.regular]{Regular types}

\begin{itemdecl}
auto concept Semiregular<typename T> 
  : @\addedCC{NothrowDestructible<T>, }@CopyConstructible<T>, CopyAssignable<T>, FreeStoreAllocatable<T> { 
  requires SameType<CopyAssignable<T>::result_type, T&>;
}
\end{itemdecl}

\begin{itemdescr}
\pnum 
\addedConcepts{\mbox{\reallynote} collects several common
requirements supported by most types.}
\end{itemdescr}

\begin{itemdecl}
auto concept Regular<typename T> 
  : Semiregular<T>, DefaultConstructible<T>, EqualityComparable<T> { }
\end{itemdecl}

\begin{itemdescr}
\pnum
\addedConcepts{\mbox{\reallynote} describes semi-regular types that are default
constructible and have equality comparison operators.}
\end{itemdescr}

\rSec2[concept.convertible]{Convertibility}

\begin{itemdecl}
auto concept ExplicitlyConvertible<typename T, typename U> : VariableType<T> {
  explicit operator U(const T&);
}
\end{itemdecl}

\begin{itemdescr}
\pnum
\addedConcepts{\mbox{\reallynote} describes types with a conversion (explicit
or implicit) from}
\addedConcepts{\mbox{\tcode{T} to \tcode{U}}.}
\end{itemdescr}

\begin{itemdecl}
auto concept Convertible<typename T, typename U> : ExplicitlyConvertible<T, U> {
  operator U(const T&);
}
\end{itemdecl}

\begin{itemdescr}
\pnum
\addedConcepts{\mbox{\reallynote} describes types with an implicit conversion from \mbox{\tcode{T} to \tcode{U}}.}
\end{itemdescr}

\rSec2[concept.arithmetic]{Arithmetic concepts}

\begin{itemdecl}
concept ArithmeticLike<typename T> 
  : Regular<T>, @\removedCCC{LessThanComparable<T>,}@ HasUnaryPlus<T>, HasNegate<T>@\addedCC{,}@
    HasPlus<T@\addedCC{, T}@>, HasMinus<T@\addedCC{, T}@>, HasMultiply<T@\addedCC{, T}@>, HasDivide<T@\addedCC{, T}@>, 
    @\addedCC{HasLess<T, T>, HasGreater<T, T>, HasLessEqual<T, T>, HasGreaterEqual<T, T>}@,
    @\addedCC{HasPreincrement<T>, HasPostincrement<T>, HasPredecrement<T>, HasPostdecrement<T>}@,
    @\addedCC{HasPlusAssign<T, const T\&>, HasMinusAssign<T, const T\&>,}@
    @\addedCC{HasMultiplyAssign<T, const T\&>, HasDivideAssign<T, const T\&>}@ {
  T::T(intmax_t);
  @\addedCC{T::T(uintmax_t);}@
  T::T(long double);

  @\removedCCC{T\& operator++(T\&);}@
  @\removedCCC{T operator++(T\& t, int) \{ T tmp(t); ++t; return tmp; \}}@
  @\removedCCC{T\& operator-{}-(T\&);}@
  @\removedCCC{T operator-{}-(T\& t, int) \{ T tmp(t); --t; return tmp; \}}@

  requires Convertible<HasUnaryPlus<T>::result_type, T>
        && Convertible<HasNegate<T>::result_type, T>
        && Convertible<HasPlus<T@\addedCC{, T}@>::result_type, T>
        && Convertible<HasMinus<T@\addedCC{, T}@>::result_type, T>
        && Convertible<HasMultiply<T@\addedCC{, T}@>::result_type, T>
        && Convertible<HasDivide<T@\addedCC{, T}@>::result_type, T>@\addedCC{,}@
        @\addedCC{\&\& SameType<HasPreincrement<T>::result_type, T\&>,}@
        @\addedCC{\&\& SameType<HasPostincrement<T>::result_type, T>,}@
        @\addedCC{\&\& SameType<HasPredecrement<T>::result_type, T\&>,}@
        @\addedCC{\&\& SameType<HasPostdecrement<T>::result_type, T>,}@
        @\addedCC{\&\& SameType<HasPlusAssign<T, const T\&>::result_type, T\&>,}@
        @\addedCC{\&\& SameType<HasMinusAssign<T, const T\&>::result_type, T\&>,}@
        @\addedCC{\&\& SameType<HasMultiplyAssign<T, const T\&>::result_type, T\&>,}@
        @\addedCC{\&\& SameType<HasDivideAssign<T, const T\&>::result_type, T\&>}@;

  @\removedCCC{T\& operator*=(T\&, T);}@
  @\removedCCC{T\& operator/=(T\&, T);}@
  @\removedCCC{T\& operator+=(T\&, T);}@
  @\removedCCC{T\& operator-=(T\&, T);}@
}
\end{itemdecl}

\begin{itemdescr}
\pnum
\addedConcepts{\mbox{\reallynote} describes types that provide all of the
  operations available on arithmetic types ([basic.fundamental]).}
\end{itemdescr}

\begin{itemdecl}
concept IntegralLike<typename T> 
  : ArithmeticLike<T>, @\addedCC{LessThanComparable<T>,}@
    HasComplement<T>, HasModulus<T@\addedCC{, T}@>, HasBitAnd<T@\addedCC{, T}@>, HasBitXor<T@\addedCC{, T}@>, HasBitOr<T@\addedCC{, T}@>,
    HasLeftShift<T@\addedCC{, T}@>, HasRightShift<T@\addedCC{, T}@> 
    @\addedCC{HasModulusAssign<T, const T\&>, HasLeftShiftAssign<T, const T\&>, HasRightShiftAssign<T, const T\&>}@
    @\addedCC{HasBitAndAssign<T, const T\&>, HasBitXorAssign<T, const T\&>, HasBitOrAssign<T, const T\&>}@ {
  requires Convertible<HasComplement<T>::result_type, T>
        && Convertible<HasModulus<T@\addedCC{, T}@>::result_type, T>
        && Convertible<HasBitAnd<T@\addedCC{, T}@>::result_type, T>
        && Convertible<HasBitXor<T@\addedCC{, T}@>::result_type, T>
        && Convertible<HasBitOr<T@\addedCC{, T}@>::result_type, T>
        && Convertible<HasLeftShift<T@\addedCC{, T}@>::result_type, T>
        && Convertible<HasRightShift<T@\addedCC{, T}@>::result_type, T>@\addedCC{,}@
        @\addedCC{\&\& SameType<HasModulusAssign<T, const T\&>::result_type, T\&>,}@
        @\addedCC{\&\& SameType<HasLeftShiftAssign<T, const T\&>::result_type, T\&>,}@
        @\addedCC{\&\& SameType<HasRightShiftAssign<T, const T\&>::result_type, T\&>,}@
        @\addedCC{\&\& SameType<HasBitAndAssign<T, const T\&>::result_type, T\&>,}@
        @\addedCC{\&\& SameType<HasBitXorAssign<T, const T\&>::result_type, T\&>,}@
        @\addedCC{\&\& SameType<HasBitOrAssign<T, const T\&>::result_type, T\&>}@;

  @\removedCCC{T\& operator\%=(T\&, T);}@
  @\removedCCC{T\& operator\&=(T\&, T);}@
  @\removedCCC{T\& operator\^=(T\&, T);}@
  @\removedCCC{T\& operator|=(T\&, T);}@
  @\removedCCC{T\& operator$\langle\langle$=(T\&, T);}@
  @\removedCCC{T\& operator$\rangle\rangle$=(T\&, T);}@
}
\end{itemdecl}

\begin{itemdescr}
\pnum 
\addedConcepts{\mbox{\reallynote} describes types that provide all of the operations
  available on integral types.}
\end{itemdescr}

\begin{itemdecl}
concept SignedIntegralLike<typename T> : IntegralLike<T> { }
\end{itemdecl}

\begin{itemdescr}
\pnum
\addedConcepts{\mbox{\reallynote} describes types that provide all of the
  operations available on signed integral types.}

\pnum
\addedConcepts{\mbox{\requires} for every signed integral type
  \mbox{\tcode{T}} ([basic.fundamental]), including signed extended
  integral types, an empty concept map
  \mbox{\tcode{SignedIntegralLike<T>}} shall be defined in namespace
  \mbox{\tcode{std}}.}
\end{itemdescr}

\begin{itemdecl}
concept UnsignedIntegralLike<typename T> : IntegralLike<T> { }
\end{itemdecl}

\begin{itemdescr}
\pnum
\addedConcepts{\mbox{\reallynote} describes types that provide all of the
  operations available on unsigned integral types.}

\pnum
\addedConcepts{\mbox{\requires} for every unsigned integral type
  \mbox{\tcode{T}} ([basic.fundamental]), including unsigned extended
  integral types, an empty concept map
  \mbox{\tcode{UnsignedIntegralLike<T>}} shall be defined in namespace
  \mbox{\tcode{std}}.}
\end{itemdescr}

\begin{itemdecl}
concept FloatingPointLike<typename T> : ArithmeticLike<T> { }
\end{itemdecl}

\begin{itemdescr}
\pnum
\addedConcepts{\mbox{\reallynote} describes floating-point types.}

\pnum
\addedConcepts{\mbox{\requires}
for every floating point type \mbox{\tcode{T}} ([basic.fundamental]), 
an empty concept map \mbox{\tcode{FloatingPointLike<T>}} shall be defined
in namespace \mbox{\tcode{std}}.}
\end{itemdescr}

\section*{Acknowledgments}
Daniel Kr\"ugler made many valuable suggestions that helped improve
this document. 

\end{document}
