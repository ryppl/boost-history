\documentclass[american,twoside]{book}
\usepackage{refbib}
\usepackage{pdfsync}
% Definitions and redefinitions of special commands

\usepackage{babel}      % needed for iso dates
\usepackage{savesym}		% suppress duplicate macro definitions
\usepackage{fancyhdr}		% more flexible headers and footers
\usepackage{listings}		% code listings
\usepackage{longtable}	% auto-breaking tables
\usepackage{remreset}		% remove counters from reset list
\usepackage{booktabs}		% fancy tables
\usepackage{relsize}		% provide relative font size changes
\usepackage[htt]{hyphenat}	% hyphenate hyphenated words: conflicts with underscore
\savesymbol{BreakableUnderscore}	% suppress BreakableUnderscore defined in hyphenat
									                % (conflicts with underscore)
\usepackage{underscore}	% remove special status of '_' in ordinary text
\usepackage{verbatim}		% improved verbatim environment
\usepackage{parskip}		% handle non-indented paragraphs "properly"
\usepackage{array}			% new column definitions for tables
\usepackage[iso]{isodate} % use iso format for dates
\usepackage{soul}       % strikeouts and underlines for difference markups
\usepackage{color}      % define colors for strikeouts and underlines
\usepackage{amsmath}    % additional math symbols
\usepackage{mathrsfs}
\usepackage{multicol}

\usepackage[T1]{fontenc}
\usepackage{ae}
\usepackage{mathptmx}
\usepackage[scaled=.90]{helvet}

%% Difference markups
\definecolor{addclr}{rgb}{0,.4,.4}
\definecolor{remclr}{rgb}{1,0,0}
\newcommand{\added}[1]{\textcolor{addclr}{\ul{#1}}}
\newcommand{\removed}[1]{\textcolor{remclr}{\st{#1}}}
\newcommand{\changed}[2]{\removed{#1}\added{#2}}
\newcommand{\remfn}{\footnote{\removed{removed footnote}}}
\newcommand{\addfn}[1]{\footnote{\added{#1}}}
\newcommand{\remitem}[1]{\item\removed{#1}}
\newcommand{\additem}[1]{\item\added{#1}}

%% Added by JJ
\long\gdef\metacomment#1{[{\sc Editorial note:} \begingroup\sf\aftergroup] #1\endgroup}

%% October, 2005 changes
\newcommand{\addedA}[1]{#1}
\newcommand{\removedA}[1]{}
\newcommand{\changedA}[2]{#2}

%% April, 2006 changes
\newcommand{\addedB}[1]{#1}
\newcommand{\removedB}[1]{}
\newcommand{\changedB}[2]{#2}
\newcommand{\remfootnoteB}[1]{}
\newcommand{\marktr}{}
\newcommand\ptr{}

%% October, 2006 changes
%\newcommand{\addedC}[1]{\added{#1}}
%\newcommand{\removedC}[1]{\removed{#1}}
%\newcommand{\changedC}[2]{\changed{#1}{#2}}
%\newcommand{\remfootnoteC}[1]{\remfn}
%\newcommand{\addfootnoteC}[1]{\addfn{#1}}
%\newcommand{\remitemC}[1]{\remitem{#1}}
%\newcommand{\additemC}[1]{\additem{#1}}
%\newcommand{\remblockC}{}

%% November registration ballot
\newcommand{\addedC}[1]{#1}
\newcommand{\removedC}[1]{}
\newcommand{\changedC}[2]{#2}
\newcommand{\remfootnoteC}[1]{}
\newcommand{\addfootnoteC}[1]{\footnote{#1}}
\newcommand{\remitemC}[1]{}
\newcommand{\additemC}[1]{\item{#1}}
\newcommand{\remblockC}{\remov_this_block}

\newcommand{\addedD}[1]{#1}
\newcommand{\removedD}[1]{}
\newcommand{\changedD}[2]{#2}
\newcommand{\remfootnoteD}[1]{}
\newcommand{\addfootnoteD}[1]{\footnote{#1}}
\newcommand{\remitemD}[1]{}
\newcommand{\additemD}[1]{\item{#1}}
\newcommand{\remblockD}{\remov_this_block}

%% Variadic Templates changes
\newcommand{\addedVT}[1]{\textcolor{addclr}{\ul{#1}}}
\newcommand{\removedVT}[1]{\textcolor{remclr}{\st{#1}}}
\newcommand{\changedVT}[2]{\removed{#1}\added{#2}}

%% Concepts changes
\newcommand{\addedConcepts}[1]{\added{#1}}
\newcommand{\removedConcepts}[1]{\removed{#1}}
\newcommand{\changedConcepts}[2]{\changed{#1}{#2}}
\newcommand{\addedConceptsC}[1]{\textcolor{addclr}{\tcode{\ul{#1}}}}
\newcommand{\remitemConcepts}[1]{\remitem{#1}}
\newcommand{\additemConcepts}[1]{\additem{#1}}

%% Concepts changes since the last revision
\definecolor{ccadd}{rgb}{0,.6,0}
\newcommand{\addedCC}[1]{\textcolor{ccadd}{\ul{#1}}}
\newcommand{\removedCC}[1]{\textcolor{remclr}{\st{#1}}}
\newcommand{\changedCC}[2]{\removedCC{#1}\addedCC{#2}}
\newcommand{\remitemCC}[1]{\remitem{#1}}
\newcommand{\additemCC}[1]{\item\addedCC{#1}}
\newcommand{\changedCCC}[2]{\textcolor{ccadd}{\st{#1}}\addedCC{#2}}
\newcommand{\removedCCC}[1]{\textcolor{ccadd}{\st{#1}}}
\newcommand{\remitemCCC}[1]{\item\removedCCC{#1}}

%% Concepts changes for the next revision
\definecolor{zadd}{rgb}{0.8,0,0.8}
\newcommand{\addedZ}[1]{\textcolor{zadd}{\ul{#1}}}
\newcommand{\removedZ}[1]{\textcolor{remclr}{\st{#1}}}
\newcommand{\changedCZ}[2]{\textcolor{addclr}{\st{#1}}\addedZ{#2}}


%%--------------------------------------------------
%% Sectioning macros.  
% Each section has a depth, an automatically generated section 
% number, a name, and a short tag.  The depth is an integer in 
% the range [0,5].  (If it proves necessary, it wouldn't take much
% programming to raise the limit from 5 to something larger.)


% The basic sectioning command.  Example:
%    \Sec1[intro.scope]{Scope}
% defines a first-level section whose name is "Scope" and whose short
% tag is intro.scope.  The square brackets are mandatory.
\def\Sec#1[#2]#3{{%
\ifcase#1\let\s=\chapter
      \or\let\s=\section
      \or\let\s=\subsection
      \or\let\s=\subsubsection
      \or\let\s=\paragraph
      \or\let\s=\subparagraph
      \fi%
\s[#3]{#3\hfill[#2]}\relax\label{#2}}}

% A convenience feature (mostly for the convenience of the Project
% Editor, to make it easy to move around large blocks of text):
% the \rSec macro is just like the \Sec macro, except that depths 
% relative to a global variable, SectionDepthBase.  So, for example,
% if SectionDepthBase is 1,
%   \rSec1[temp.arg.type]{Template type arguments}
% is equivalent to
%   \Sec2[temp.arg.type]{Template type arguments}

\newcounter{SectionDepthBase}
\newcounter{scratch}

\def\rSec#1[#2]#3{{%
\setcounter{scratch}{#1}
\addtocounter{scratch}{\value{SectionDepthBase}}
\Sec{\arabic{scratch}}[#2]{#3}}}

% Change the way section headings are formatted.
\renewcommand{\chaptername}{}
\renewcommand{\appendixname}{Annex}

\makeatletter
\def\@makechapterhead#1{%
  \hrule\vspace*{1.5\p@}\hrule
  \vspace*{16\p@}%
  {\parindent \z@ \raggedright \normalfont
    \ifnum \c@secnumdepth >\m@ne
        \huge\bfseries \@chapapp\space \thechapter\space\space\space\space
    \fi
    \interlinepenalty\@M
    \huge \bfseries #1\par\nobreak
  \vspace*{16\p@}%
  \hrule\vspace*{1.5\p@}\hrule
  \vspace*{48\p@}
  }}

\renewcommand\section{\@startsection{section}{1}{0pt}%
                                   {-3.5ex plus -1ex minus -.2ex}%
                                   {.3ex plus .2ex}%
                                   {\normalfont\normalsize\bfseries}}
\renewcommand\section{\@startsection{section}{1}{0pt}%
                                   {2.5ex}% plus 1ex minus .2ex}%
                                   {.3ex}% plus .1ex minus .2 ex}%
                                   {\normalfont\normalsize\bfseries}}

\renewcommand\subsection{\@startsection{subsection}{2}{0pt}%
                                     {-3.25ex plus -1ex minus -.2ex}%
                                     {.3ex plus .2ex}%
                                     {\normalfont\normalsize\bfseries}}

\renewcommand\subsubsection{\@startsection{subsubsection}{3}{0pt}%
                                     {-3.25ex plus -1ex minus -.2ex}%
                                     {.3ex plus .2ex}%
                                     {\normalfont\normalsize\bfseries}}

\renewcommand\paragraph{\@startsection{paragraph}{4}{0pt}%
                                     {-3.25ex plus -1ex minus -.2ex}%
                                     {.3ex \@plus .2ex}%
                                     {\normalfont\normalsize\bfseries}}

\renewcommand\subparagraph{\@startsection{subparagraph}{5}{0pt}%
                                     {-3.25ex plus -1ex minus -.2ex}%
                                     {.3ex plus .2ex}%
                                     {\normalfont\normalsize\bfseries}}
\@removefromreset{footnote}{chapter}
\@removefromreset{table}{chapter}
\@removefromreset{figure}{chapter}
\makeatother

%%--------------------------------------------------
% Heading style for Annexes
\newcommand{\Annex}[3]{\chapter[#2]{\\(#3)\\#2\hfill[#1]}\relax\label{#1}}
\newcommand{\infannex}[2]{\Annex{#1}{#2}{informative}}
\newcommand{\normannex}[2]{\Annex{#1}{#2}{normative}}

\newcommand{\synopsis}[1]{\textbf{#1}}

%%--------------------------------------------------
% General code style
\newcommand{\CodeStyle}{\ttfamily}
\newcommand{\CodeStylex}[1]{\texttt{#1}}

% Code and definitions embedded in text.
\newcommand{\tcode}[1]{\CodeStylex{#1}}
\newcommand{\techterm}[1]{\textit{#1}}

%%--------------------------------------------------
%% allow line break if needed for justification
\newcommand{\brk}{\discretionary{}{}{}}
%  especially for scope qualifier
\newcommand{\colcol}{\brk::\brk}

%%--------------------------------------------------
%% Macros for funky text
%%!\newcommand{\Rplus}{\protect\nolinebreak\hspace{-.07em}\protect\raisebox{.25ex}{\small\textbf{+}}}
\newcommand{\Rplus}{+}
\newcommand{\Cpp}{C\Rplus\Rplus}
\newcommand{\opt}{$_\mathit{opt}$}
\newcommand{\shl}{<{<}}
\newcommand{\shr}{>{>}}
\newcommand{\dcr}{-{-}}
\newcommand{\bigohm}[1]{\mathscr{O}(#1)}
\newcommand{\bigoh}[1]{$\bigohm{#1}$}
\renewcommand{\tilde}{{\smaller$\sim$}}		% extra level of braces is necessary

%%--------------------------------------------------
%% States and operators

\newcommand{\state}[2]{\tcode{#1}\ensuremath{_{#2}}}
\newcommand{\bitand}{\ensuremath{\, \mathsf{bitand} \,}}
\newcommand{\bitor}{\ensuremath{\, \mathsf{bitor} \,}}
\newcommand{\xor}{\ensuremath{\, \mathsf{xor} \,}}
\newcommand{\rightshift}{\ensuremath{\, \mathsf{rshift} \,}}
\newcommand{\leftshift}{\ensuremath{\, \mathsf{lshift} \,}}

%% Notes and examples
\newcommand{\EnterBlock}[1]{[\,\textit{#1:}}
\newcommand{\ExitBlock}[1]{\textit{\ ---\,end #1}\,]}
\newcommand{\enternote}{\EnterBlock{Note}}
\newcommand{\exitnote}{\ExitBlock{note}}
\newcommand{\enterexample}{\EnterBlock{Example}}
\newcommand{\exitexample}{\ExitBlock{example}}

%% Library function descriptions
\newcommand{\Fundescx}[1]{\textit{#1}}
\newcommand{\Fundesc}[1]{\Fundescx{#1:}}
\newcommand{\required}{\Fundesc{Required behavior}}
\newcommand{\requires}{\Fundesc{Requires}}
\newcommand{\effects}{\Fundesc{Effects}}
\newcommand{\postconditions}{\Fundesc{Postconditions}}
\newcommand{\postcondition}{\Fundesc{Postcondition}}
\newcommand{\preconditions}{\Fundesc{Preconditions}}
\newcommand{\precondition}{\Fundesc{Precondition}}
\newcommand{\returns}{\Fundesc{Returns}}
\newcommand{\throws}{\Fundesc{Throws}}
\newcommand{\default}{\Fundesc{Default behavior}}
\newcommand{\complexity}{\Fundesc{Complexity}}
\newcommand{\note}{\Fundesc{Remark}}
\newcommand{\notes}{\Fundesc{Remarks}}
\newcommand{\implimits}{\Fundesc{Implementation limits}}
\newcommand{\replaceable}{\Fundesc{Replaceable}}
\newcommand{\exceptionsafety}{\Fundesc{Exception safety}}
\newcommand{\returntype}{\Fundesc{Return type}}

%% Cross reference
\newcommand{\xref}{\textsc{See also:}}

%% NTBS, etc.
\newcommand{\NTS}[1]{\textsc{#1}}
\newcommand{\ntbs}{\NTS{ntbs}}
\newcommand{\ntmbs}{\NTS{ntmbs}}
\newcommand{\ntwcs}{\NTS{ntwcs}}

%% Function argument
\newcommand{\farg}[1]{\texttt{\textit{#1}}}

%% Code annotations
\newcommand{\EXPO}[1]{\textbf{#1}}
\newcommand{\expos}{\EXPO{exposition only}}
\newcommand{\exposr}{\hfill\expos}
\newcommand{\exposrc}{\hfill// \expos}
\newcommand{\impdef}{\EXPO{implementation-defined}}
\newcommand{\notdef}{\EXPO{not defined}}

%% Double underscore
\newcommand{\unun}{\_\,\_}
\newcommand{\xname}[1]{\unun\,#1}
\newcommand{\mname}[1]{\tcode{\unun\,#1\,\unun}}

%% Ranges
\newcommand{\Range}[4]{\tcode{#1\brk{}#3,\brk{}#4\brk{}#2}}
\newcommand{\crange}[2]{\Range{[}{]}{#1}{#2}}
\newcommand{\orange}[2]{\Range{(}{)}{#1}{#2}}
\newcommand{\range}[2]{\Range{[}{)}{#1}{#2}}

%% Change descriptions
\newcommand{\diffdef}[1]{\hfill\break\textbf{#1:}}
\newcommand{\change}{\diffdef{Change}}
\newcommand{\rationale}{\diffdef{Rationale}}
\newcommand{\effect}{\diffdef{Effect on original feature}}
\newcommand{\difficulty}{\diffdef{Difficulty of converting}}
\newcommand{\howwide}{\diffdef{How widely used}}

%% Miscellaneous
\newcommand{\uniquens}{\textrm{\textit{\textbf{unique}}}}
\newcommand{\stage}[1]{\item{\textbf{Stage #1:}}}

%%--------------------------------------------------
%% Adjust markers
\renewcommand{\thetable}{\arabic{table}}
\renewcommand{\thefigure}{\arabic{figure}}
\renewcommand{\thefootnote}{\arabic{footnote})}

%% Change list item markers from box to dash
\renewcommand{\labelitemi}{---}
\renewcommand{\labelitemii}{---}
\renewcommand{\labelitemiii}{---}
\renewcommand{\labelitemiv}{---}

%%--------------------------------------------------
%% Environments for code listings.

% We use the 'listings' package, with some small customizations.  The
% most interesting customization: all TeX commands are available
% within comments.  Comments are set in italics, keywords and strings
% don't get special treatment.

\lstset{language=C++,
        basicstyle=\CodeStyle\small,
        keywordstyle=,
        stringstyle=,
        xleftmargin=1em,
        showstringspaces=false,
        commentstyle=\itshape\rmfamily,
        columns=flexible,
        keepspaces=true,
        texcl=true}

% Our usual abbreviation for 'listings'.  Comments are in 
% italics.  Arbitrary TeX commands can be used if they're 
% surrounded by @ signs.
\lstnewenvironment{codeblock}
{
 \lstset{escapechar=@}
 \renewcommand{\tcode}[1]{\textup{\CodeStyle##1}}
 \renewcommand{\techterm}[1]{\textit{##1}}
}
{
}

% Permit use of '@' inside codeblock blocks (don't ask)
\makeatletter
\newcommand{\atsign}{@}
\makeatother

%%--------------------------------------------------
%% Paragraph numbering
\newcounter{Paras}
\makeatletter
\@addtoreset{Paras}{chapter}
\@addtoreset{Paras}{section}
\@addtoreset{Paras}{subsection}
\@addtoreset{Paras}{subsubsection}
\@addtoreset{Paras}{paragraph}
\@addtoreset{Paras}{subparagraph}
\def\pnum{\addtocounter{Paras}{1}\noindent\llap{{\footnotesize\arabic{Paras}}\hspace{\@totalleftmargin}\quad}}
\makeatother

% For compatibility only.  We no longer need this environment.
\newenvironment{paras}{}{}

%%--------------------------------------------------
%% Indented text
\newenvironment{indented}
{\list{}{}\item\relax}
{\endlist}

%%--------------------------------------------------
%% Library item descriptions
\lstnewenvironment{itemdecl}
{
 \lstset{escapechar=@,
 xleftmargin=0em,
 aboveskip=2ex,
 belowskip=0ex	% leave this alone: it keeps these things out of the
				% footnote area
 }
}
{
}

\newenvironment{itemdescr}
{
 \begin{indented}}
{
 \end{indented}
}


%%--------------------------------------------------
%% Bnf environments
\newlength{\BnfIndent}
\setlength{\BnfIndent}{\leftmargini}
\newlength{\BnfInc}
\setlength{\BnfInc}{\BnfIndent}
\newlength{\BnfRest}
\setlength{\BnfRest}{2\BnfIndent}
\newcommand{\BnfNontermshape}{\rmfamily\itshape\small}
\newcommand{\BnfTermshape}{\ttfamily\upshape\small}
\newcommand{\nonterminal}[1]{{\BnfNontermshape #1}}

\newenvironment{bnfbase}
 {
 \newcommand{\terminal}[1]{{\BnfTermshape ##1}}
 \newcommand{\descr}[1]{\normalfont{##1}}
 \newcommand{\bnfindentfirst}{\BnfIndent}
 \newcommand{\bnfindentinc}{\BnfInc}
 \newcommand{\bnfindentrest}{\BnfRest}
 \begin{minipage}{.9\hsize}
 \newcommand{\br}{\hfill\\}
 }
 {
 \end{minipage}
 }

\newenvironment{BnfTabBase}[1]
{
 \begin{bnfbase}
 #1
 \begin{indented}
 \begin{tabbing}
 \hspace*{\bnfindentfirst}\=\hspace{\bnfindentinc}\=\hspace{.6in}\=\hspace{.6in}\=\hspace{.6in}\=\hspace{.6in}\=\hspace{.6in}\=\hspace{.6in}\=\hspace{.6in}\=\hspace{.6in}\=\hspace{.6in}\=\hspace{.6in}\=\kill%
}
{
 \end{tabbing}
 \end{indented}
 \end{bnfbase}
}

\newenvironment{bnfkeywordtab}
{
 \begin{BnfTabBase}{\BnfTermshape}
}
{
 \end{BnfTabBase}
}

\newenvironment{bnftab}
{
 \begin{BnfTabBase}{\BnfNontermshape}
}
{
 \end{BnfTabBase}
}

\newenvironment{simplebnf}
{
 \begin{bnfbase}
 \BnfNontermshape
 \begin{indented}
}
{
 \end{indented}
 \end{bnfbase}
}

\newenvironment{bnf}
{
 \begin{bnfbase}
 \list{}
	{
	\setlength{\leftmargin}{\bnfindentrest}
	\setlength{\listparindent}{-\bnfindentinc}
	\setlength{\itemindent}{\listparindent}
	}
 \BnfNontermshape
 \item\relax
}
{
 \endlist
 \end{bnfbase}
}

% non-copied versions of bnf environments
\newenvironment{ncbnftab}
{
 \begin{bnftab}
}
{
 \end{bnftab}
}

\newenvironment{ncsimplebnf}
{
 \begin{simplebnf}
}
{
 \end{simplebnf}
}

\newenvironment{ncbnf}
{
 \begin{bnf}
}
{
 \end{bnf}
}

%%--------------------------------------------------
%% Drawing environment
%
% usage: \begin{drawing}{UNITLENGTH}{WIDTH}{HEIGHT}{CAPTION}
\newenvironment{drawing}[4]
{
\begin{figure}[h]
\setlength{\unitlength}{#1}
\begin{center}
\begin{picture}(#2,#3)\thicklines
}
{
\end{picture}
\end{center}
%\caption{Directed acyclic graph}
\end{figure}
}

%%--------------------------------------------------
%% Table environments

% Base definitions for tables
\newenvironment{TableBase}
{
 \renewcommand{\tcode}[1]{{\CodeStyle##1}}
 \newcommand{\topline}{\hline}
 \newcommand{\capsep}{\hline\hline}
 \newcommand{\rowsep}{\hline}
 \newcommand{\bottomline}{\hline}

%% vertical alignment
 \newcommand{\rb}[1]{\raisebox{1.5ex}[0pt]{##1}}	% move argument up half a row

%% header helpers
 \newcommand{\hdstyle}[1]{\textbf{##1}}				% set header style
 \newcommand{\Head}[3]{\multicolumn{##1}{##2}{\hdstyle{##3}}}	% add title spanning multiple columns
 \newcommand{\lhdrx}[2]{\Head{##1}{|c}{##2}}		% set header for left column spanning #1 columns
 \newcommand{\chdrx}[2]{\Head{##1}{c}{##2}}			% set header for center column spanning #1 columns
 \newcommand{\rhdrx}[2]{\Head{##1}{c|}{##2}}		% set header for right column spanning #1 columns
 \newcommand{\ohdrx}[2]{\Head{##1}{|c|}{##2}}		% set header for only column spanning #1 columns
 \newcommand{\lhdr}[1]{\lhdrx{1}{##1}}				% set header for single left column
 \newcommand{\chdr}[1]{\chdrx{1}{##1}}				% set header for single center column
 \newcommand{\rhdr}[1]{\rhdrx{1}{##1}}				% set header for single right column
 \newcommand{\ohdr}[1]{\ohdrx{1}{##1}}
 \newcommand{\br}{\hfill\break}						% force newline within table entry

%% column styles
 \newcolumntype{x}[1]{>{\raggedright\let\\=\tabularnewline}p{##1}}	% word-wrapped ragged-right
 																	% column, width specified by #1
 \newcolumntype{m}[1]{>{\CodeStyle}l{##1}}							% variable width column, all entries in CodeStyle
}
{
}

% General Usage: TITLE is the title of the table, XREF is the
% cross-reference for the table. LAYOUT is a sequence of column
% type specifiers (e.g. cp{1.0}c), without '|' for the left edge
% or right edge.

% usage: \begin{floattablebase}{TITLE}{XREF}{COLUMNS}{PLACEMENT}
% produces floating table, location determined within limits
% by LaTeX.
\newenvironment{floattablebase}[4]
{
 \begin{TableBase}
 \begin{table}[#4]
 \caption{\label{#2}#1}
 \begin{center}
 \begin{tabular}{|#3|}
}
{
 \bottomline
 \end{tabular}
 \end{center}
 \end{table}
 \end{TableBase}
}

% usage: \begin{floattable}{TITLE}{XREF}{COLUMNS}
% produces floating table, location determined within limits
% by LaTeX.
\newenvironment{floattable}[3]
{
 \begin{floattablebase}{#1}{#2}{#3}{htbp}
}
{
 \end{floattablebase}
}

% usage: \begin{tokentable}{TITLE}{XREF}{HDR1}{HDR2}
% produces six-column table used for lists of replacement tokens;
% the columns are in pairs -- left-hand column has header HDR1,
% right hand column has header HDR2; pairs of columns are separated
% by vertical lines. Used in "trigraph sequences" table in standard.
\newenvironment{tokentable}[4]
{
 \begin{floattablebase}{#1}{#2}{cc|cc|cc}{htbp}
 \topline
 \textit{#3}   &   \textit{#4}    &
 \textit{#3}   &   \textit{#4}    &
 \textit{#3}   &   \textit{#4}    \\ \capsep
}
{
 \end{floattablebase}
}

% usage: \begin{libsumtabase}{TITLE}{XREF}{HDR1}{HDR2}
% produces two-column table with column headers HDR1 and HDR2.
% Used in "Library Categories" table in standard, and used as
% base for other library summary tables.
\newenvironment{libsumtabbase}[4]
{
 \begin{floattable}{#1}{#2}{ll}
 \topline
 \lhdr{#3}	&	\hdstyle{#4}	\\ \capsep
}
{
 \end{floattable}
}

% usage: \begin{libsumtab}{TITLE}{XREF}
% produces two-column table with column headers "Subclause" and "Header(s)".
% Used in "C++ Headers for Freestanding Implementations" table in standard.
\newenvironment{libsumtab}[2]
{
 \begin{libsumtabbase}{#1}{#2}{Subclause}{Header(s)}
}
{
 \end{libsumtabbase}
}

% usage: \begin{LibSynTab}{CAPTION}{TITLE}{XREF}{COUNT}{LAYOUT}
% produces table with COUNT columns. Used as base for
% C library description tables
\newcounter{LibSynTabCols}
\newcounter{LibSynTabWd}
\newenvironment{LibSynTabBase}[5]
{
 \setcounter{LibSynTabCols}{#4}
 \setcounter{LibSynTabWd}{#4}
 \addtocounter{LibSynTabWd}{-1}
 \newcommand{\centry}[1]{\textbf{##1}:}
 \newcommand{\macro}{\centry{Macro}}
 \newcommand{\macros}{\centry{Macros}}
 \newcommand{\function}{\centry{Function}}
 \newcommand{\functions}{\centry{Functions}}
 \newcommand{\templates}{\centry{Templates}}
 \newcommand{\type}{\centry{Type}}
 \newcommand{\types}{\centry{Types}}
 \newcommand{\values}{\centry{Values}}
 \newcommand{\struct}{\centry{Struct}}
 \newcommand{\cspan}[1]{\multicolumn{\value{LibSynTabCols}}{|l|}{##1}}
 \begin{floattable}{#1 \tcode{<#2>}\ synopsis}{#3}
 {#5}
 \topline
 \lhdr{Type}	&	\rhdrx{\value{LibSynTabWd}}{Name(s)}	\\ \capsep
}
{
 \end{floattable}
}

% usage: \begin{LibSynTab}{TITLE}{XREF}{COUNT}{LAYOUT}
% produces table with COUNT columns. Used as base for description tables
% for C library
\newenvironment{LibSynTab}[4]
{
 \begin{LibSynTabBase}{Header}{#1}{#2}{#3}{#4}
}
{
 \end{LibSynTabBase}
}

% usage: \begin{LibSynTabAdd}{TITLE}{XREF}{COUNT}{LAYOUT}
% produces table with COUNT columns. Used as base for description tables
% for additions to C library
\newenvironment{LibSynTabAdd}[4]
{
 \begin{LibSynTabBase}{Additions to header}{#1}{#2}{#3}{#4}
}
{
 \end{LibSynTabBase}
}

% usage: \begin{libsyntabN}{TITLE}{XREF}
%        \begin{libsyntabaddN}{TITLE}{XREF}
% produces a table with N columns for C library description tables
\newenvironment{libsyntab2}[2]
{
 \begin{LibSynTab}{#1}{#2}{2}{ll}
}
{
 \end{LibSynTab}
}

\newenvironment{libsyntab3}[2]
{
 \begin{LibSynTab}{#1}{#2}{3}{lll}
}
{
 \end{LibSynTab}
}

\newenvironment{libsyntab4}[2]
{
 \begin{LibSynTab}{#1}{#2}{4}{llll}
}
{
 \end{LibSynTab}
}

\newenvironment{libsyntab5}[2]
{
 \begin{LibSynTab}{#1}{#2}{5}{lllll}
}
{
 \end{LibSynTab}
}

\newenvironment{libsyntab6}[2]
{
 \begin{LibSynTab}{#1}{#2}{6}{llllll}
}
{
 \end{LibSynTab}
}

\newenvironment{libsyntabadd2}[2]
{
 \begin{LibSynTabAdd}{#1}{#2}{2}{ll}
}
{
 \end{LibSynTabAdd}
}

\newenvironment{libsyntabadd3}[2]
{
 \begin{LibSynTabAdd}{#1}{#2}{3}{lll}
}
{
 \end{LibSynTabAdd}
}

\newenvironment{libsyntabadd4}[2]
{
 \begin{LibSynTabAdd}{#1}{#2}{4}{llll}
}
{
 \end{LibSynTabAdd}
}

\newenvironment{libsyntabadd5}[2]
{
 \begin{LibSynTabAdd}{#1}{#2}{5}{lllll}
}
{
 \end{LibSynTabAdd}
}

\newenvironment{libsyntabadd6}[2]
{
 \begin{LibSynTabAdd}{#1}{#2}{6}{llllll}
}
{
 \end{LibSynTabAdd}
}

% usage: \begin{LongTable}{TITLE}{XREF}{LAYOUT}
% produces table that handles page breaks sensibly.
\newenvironment{LongTable}[3]
{
 \begin{TableBase}
 \begin{longtable}
 {|#3|}\caption{#1}\label{#2}
}
{
 \bottomline
 \end{longtable}
 \end{TableBase}
}

% usage: \begin{twocol}{TITLE}{XREF}
% produces a two-column breakable table. Used in
% "simple-type-specifiers and the types they specify" in the standard.
\newenvironment{twocol}[2]
{
 \begin{LongTable}
 {#1}{#2}
 {ll}
}
{
 \end{LongTable}
}

% usage: \begin{libreqtabN}{TITLE}{XREF}
% produces an N-column brekable table. Used in
% most of the library clauses for requirements tables.
% Example at "Position type requirements" in the standard.

\newenvironment{libreqtab1}[2]
{
 \begin{LongTable}
 {#1}{#2}
 {x{.55\hsize}}
}
{
 \end{LongTable}
}

\newenvironment{libreqtab2}[2]
{
 \begin{LongTable}
 {#1}{#2}
 {lx{.55\hsize}}
}
{
 \end{LongTable}
}

\newenvironment{libreqtab2a}[2]
{
 \begin{LongTable}
 {#1}{#2}
 {x{.30\hsize}x{.68\hsize}}
}
{
 \end{LongTable}
}

\newenvironment{libreqtab3}[2]
{
 \begin{LongTable}
 {#1}{#2}
 {x{.28\hsize}x{.18\hsize}x{.43\hsize}}
}
{
 \end{LongTable}
}

\newenvironment{libreqtab3a}[2]
{
 \begin{LongTable}
 {#1}{#2}
 {x{.28\hsize}x{.33\hsize}x{.29\hsize}}
}
{
 \end{LongTable}
}

\newenvironment{libreqtab3b}[2]
{
 \begin{LongTable}
 {#1}{#2}
 {x{.40\hsize}x{.25\hsize}x{.25\hsize}}
}
{
 \end{LongTable}
}

\newenvironment{libreqtab3c}[2]
{
 \begin{LongTable}
 {#1}{#2}
 {x{.30\hsize}x{.25\hsize}x{.35\hsize}}
}
{
 \end{LongTable}
}

\newenvironment{libreqtab3d}[2]
{
 \begin{LongTable}
 {#1}{#2}
 {x{.32\hsize}x{.27\hsize}x{.36\hsize}}
}
{
 \end{LongTable}
}

\newenvironment{libreqtab3e}[2]
{
 \begin{LongTable}
 {#1}{#2}
 {x{.38\hsize}x{.27\hsize}x{.25\hsize}}
}
{
 \end{LongTable}
}

\newenvironment{libreqtab3f}[2]
{
 \begin{LongTable}
 {#1}{#2}
 {x{.40\hsize}x{.22\hsize}x{.31\hsize}}
}
{
 \end{LongTable}
}

\newenvironment{libreqtab4}[2]
{
 \begin{LongTable}
 {#1}{#2}
}
{
 \end{LongTable}
}

\newenvironment{libreqtab4a}[2]
{
 \begin{LongTable}
 {#1}{#2}
 {x{.14\hsize}x{.30\hsize}x{.30\hsize}x{.14\hsize}}
}
{
 \end{LongTable}
}

\newenvironment{libreqtab4b}[2]
{
 \begin{LongTable}
 {#1}{#2}
 {x{.13\hsize}x{.15\hsize}x{.29\hsize}x{.27\hsize}}
}
{
 \end{LongTable}
}

\newenvironment{libreqtab4c}[2]
{
 \begin{LongTable}
 {#1}{#2}
 {x{.16\hsize}x{.21\hsize}x{.21\hsize}x{.30\hsize}}
}
{
 \end{LongTable}
}

\newenvironment{libreqtab4d}[2]
{
 \begin{LongTable}
 {#1}{#2}
 {x{.22\hsize}x{.22\hsize}x{.30\hsize}x{.15\hsize}}
}
{
 \end{LongTable}
}

\newenvironment{libreqtab5}[2]
{
 \begin{LongTable}
 {#1}{#2}
 {x{.14\hsize}x{.14\hsize}x{.20\hsize}x{.20\hsize}x{.14\hsize}}
}
{
 \end{LongTable}
}

% usage: \begin{libtab2}{TITLE}{XREF}{LAYOUT}{HDR1}{HDR2}
% produces two-column table with column headers HDR1 and HDR2.
% Used in "seekoff positioning" in the standard.
\newenvironment{libtab2}[5]
{
 \begin{floattable}
 {#1}{#2}{#3}
 \topline
 \lhdr{#4}	&	\rhdr{#5}	\\ \capsep
}
{
 \end{floattable}
}

% usage: \begin{longlibtab2}{TITLE}{XREF}{LAYOUT}{HDR1}{HDR2}
% produces two-column table with column headers HDR1 and HDR2.
\newenvironment{longlibtab2}[5]
{
 \begin{LongTable}{#1}{#2}{#3}
 \\ \topline
 \lhdr{#4}	&	\rhdr{#5}	\\ \capsep
}
{
  \end{LongTable}
}

% usage: \begin{LibEffTab}{TITLE}{XREF}{HDR2}{WD2}
% produces a two-column table with left column header "Element"
% and right column header HDR2, right column word-wrapped with
% width specified by WD2.
\newenvironment{LibEffTab}[4]
{
 \begin{libtab2}{#1}{#2}{lp{#4}}{Element}{#3}
}
{
 \end{libtab2}
}

% Same as LibEffTab except that it uses a long table.
\newenvironment{longLibEffTab}[4]
{
 \begin{longlibtab2}{#1}{#2}{lp{#4}}{Element}{#3}
}
{
 \end{longlibtab2}
}

% usage: \begin{libefftab}{TITLE}{XREF}
% produces a two-column effects table with right column
% header "Effect(s) if set", width 4.5 in. Used in "fmtflags effects"
% table in standard.
\newenvironment{libefftab}[2]
{
 \begin{LibEffTab}{#1}{#2}{Effect(s) if set}{4.5in}
}
{
 \end{LibEffTab}
}

% Same as libefftab except that it uses a long table.
\newenvironment{longlibefftab}[2]
{
 \begin{longLibEffTab}{#1}{#2}{Effect(s) if set}{4.5in}
}
{
 \end{longLibEffTab}
}

% usage: \begin{libefftabmean}{TITLE}{XREF}
% produces a two-column effects table with right column
% header "Meaning", width 4.5 in. Used in "seekdir effects"
% table in standard.
\newenvironment{libefftabmean}[2]
{
 \begin{LibEffTab}{#1}{#2}{Meaning}{4.5in}
}
{
 \end{LibEffTab}
}

% Same as libefftabmean except that it uses a long table.
\newenvironment{longlibefftabmean}[2]
{
 \begin{longLibEffTab}{#1}{#2}{Meaning}{4.5in}
}
{
 \end{longLibEffTab}
}

% usage: \begin{libefftabvalue}{TITLE}{XREF}
% produces a two-column effects table with right column
% header "Value", width 3 in. Used in "basic_ios::init() effects"
% table in standard.
\newenvironment{libefftabvalue}[2]
{
 \begin{LibEffTab}{#1}{#2}{Value}{3in}
}
{
 \end{LibEffTab}
}

% Same as libefftabvalue except that it uses a long table and a
% slightly wider column.
\newenvironment{longlibefftabvalue}[2]
{
 \begin{longLibEffTab}{#1}{#2}{Value}{3.5in}
}
{
 \end{longLibEffTab}
}

% usage: \begin{liberrtab}{TITLE}{XREF} produces a two-column table
% with left column header ``Value'' and right header "Error
% condition", width 4.5 in. Used in regex clause in the TR.

\newenvironment{liberrtab}[2]
{
 \begin{libtab2}{#1}{#2}{lp{4.5in}}{Value}{Error condition}
}
{
 \end{libtab2}
}

% Like liberrtab except that it uses a long table.
\newenvironment{longliberrtab}[2]
{
 \begin{longlibtab2}{#1}{#2}{lp{4.5in}}{Value}{Error condition}
}
{
 \end{longlibtab2}
}

% enumerate with lowercase letters
\newenvironment{enumeratea}
{
 \renewcommand{\labelenumi}{\alph{enumi})}
 \begin{enumerate}
}
{
 \end{enumerate}
}

% enumerate with arabic numbers
\newenvironment{enumeraten}
{
 \renewcommand{\labelenumi}{\arabic{enumi})}
 \begin{enumerate}
}
{
 \end{enumerate}
}

%%--------------------------------------------------
%% Definitions section
% usage: \definition{name}{xref}
%\newcommand{\definition}[2]{\rSec2[#2]{#1}}
% for ISO format, use:
\newcommand{\definition}[2]
 {\hfill\vspace{.25ex plus .5ex minus .2ex}\\
 \addtocounter{subsection}{1}%
 \textbf{\thesubsection\hfill\relax[#2]}\\
 \textbf{#1}\label{#2}\\
 }


%%--------------------------------------------------
%% PDF

\usepackage[pdftex,
            pdftitle={Iterator Concepts for the C++0x Standard Library},
            pdfsubject={C++ International Standard Proposal},
            pdfcreator={Douglas Gregor},
            bookmarks=true,
            bookmarksnumbered=true,
            pdfpagelabels=true,
            pdfpagemode=UseOutlines,
            pdfstartview=FitH,
            linktocpage=true,
            colorlinks=true,
            linkcolor=blue,
            plainpages=false
           ]{hyperref}

%%--------------------------------------------------
%% Set section numbering limit, toc limit
\setcounter{secnumdepth}{5}
\setcounter{tocdepth}{3}

%%--------------------------------------------------
%% Parameters that govern document appearance
\setlength{\oddsidemargin}{0pt}
\setlength{\evensidemargin}{0pt}
\setlength{\textwidth}{6.6in}

\newcommand{\resetcolor}{\textcolor{addclr}{}}

%%--------------------------------------------------
%% Handle special hyphenation rules
\hyphenation{tem-plate ex-am-ple in-put-it-er-a-tor}

% Do not put blank pages after chapters that end on odd-numbered pages.
\def\cleardoublepage{\clearpage\if@twoside%
  \ifodd\c@page\else\hbox{}\thispagestyle{empty}\newpage%
  \if@twocolumn\hbox{}\newpage\fi\fi\fi}

\begin{document}
\raggedbottom

\begin{titlepage}
\begin{center}
\huge
Iterator Concepts for the C++0x Standard Library\\
(Revision 5)
\vspace{0.5in}

\normalsize
Douglas Gregor, Jeremy Siek and Andrew Lumsdaine \\
\href{mailto:dgregor@osl.iu.edu}{dgregor@osl.iu.edu}, \href{mailto:jeremy.siek@colorado.edu}{jeremy.siek@colorado.edu}, \href{mailto:lums@osl.iu.edu}{lums@osl.iu.edu}
\end{center}

\vspace{1in}
\par\noindent Document number: DRAFT \vspace{-6pt}
\par\noindent Revises document number: N2739=08-0249 \vspace{-6pt}
\par\noindent Date: \today\vspace{-6pt}
\par\noindent Project: Programming Language \Cpp{}, Library Working Group\vspace{-6pt}
\par\noindent Reply-to: Douglas Gregor $<$\href{mailto:dgregor@osl.iu.edu}{dgregor@osl.iu.edu}$>$\vspace{-6pt}

\section*{Introduction}
This document proposes new iterator concepts in the \Cpp0x Standard
Library. It describes a new header \tcode{<iterator_concepts>} that
contains these concepts, along with concept maps and
\tcode{iterator_traits} specializations that provide backward
compatibility for existing iterators and generic algorithms.

Within the proposed wording, text that has been added
\textcolor{addclr}{will be presented in blue} \addedConcepts{and
underlined when possible}. Text that has been removed will be
presented \textcolor{remclr}{in red},\removedConcepts{with
strike-through when possible}. 

\editorial{Purely editorial comments will be written in a separate,
  shaded box. These comments are not intended to be included in the
  working paper.}

\paragraph*{About the new iterator concept taxonomy}
At the Library Working Group's request, we sought to determine whether
we could eliminate the mutable iterator concepts from the iterator
taxonomy. The observation made in Sophia-Antipolis was that the
mutable iterator concepts were used very rarely, and in those places
where they were used, we were able to discern simpler
requirements. As a result of this investigation, we have eliminated
the mutable iterator concepts and designed an improved iterator
taxonomy that better describes the various kinds of iterators usable
with the C++0x standard library.

The fundamental problem with the mutable iterator concepts is that
they were initially ill-defined within C++98/03, mentioned only
casually as iterators for which one could write a value to the result
of dereferencing an iterator. This requirement was taken to mean a
\tcode{CopyAssignable} requirement (which works reasonably well for
C++98/03 forward iterators and above), but that fails in two important
ways for C++0x:

\begin{itemize}
\item We can now construct sequences with types that are
  move-assignable but not copy-assignable. If we merely change the
  mutable iterator requirement to \tcode{MoveAssignable}, our
  definition of mutable iterator has changed from C++98/03. Besides,
  even this is incorrect: one can mutate values that aren't even
  move-assignable, e.g., by swapping values or acting on lvalues.

\item As part of improving the iterator concepts, we have intended to
  better support proxy iterators (like the infamous
  \tcode{vector<bool>} iterator, although there exist many more
  important examples of such iterators) throughout the C++0x standard
  library. The simple notion of a copy-assignable value type does not
  match with a proxy reference that supports writing.
\end{itemize}

Thus, the most important realization is that there are multiple forms
of mutability used within the C++0x standard library, and that these
mutations involve both the value type of the iterator (e.g., the type
actually stored in the container the iterator references) and the
reference type of the iterator (which may be an lvalue reference,
rvalue reference, or a proxy class). The new iterator taxonomy
captures these forms of mutability through two iterator concepts:
\tcode{OutputIterator} and \tcode{ShuffleIterator}.

The \tcode{OutputIterator} concept (\ref{output.iterators}) is a
faithful representation of a C++98/03 output iterator. Output
iterators are an odd kind of iterator, because it does not make sense
to say that a type \tcode{X} is an output iterator. Rather, one must
say that \tcode{X} is an output iterator \textit{for a value type
  \tcode{T}}. Moreover, a given type \tcode{X} can be an output
iterator for a whole family of types, e.g., all types that can be
printed, and can permit specific parameter-passing conventions. For
example, \tcode{X} could support writing only rvalues of type
\tcode{T}, or both lvalues and rvalues of type \tcode{T}. Thus, the
\tcode{OutputIterator} concept is a two-parameter concept, one
parameter for \tcode{X} and another for \tcode{T} (called
\tcode{Value}). This is not new; however, the updated iterator
taxonomy encodes the parameter-passing convention into the
\tcode{Value} template parameter, so that the user of the output
iterator can distinguish between writing lvalues and writing
rvalues. For example, the \tcode{copy} algorithm provides the
\tcode{reference} type of the input iterator as the output type of the
output iterator:

\begin{codeblock}
template<InputIterator InIter, 
         OutputIterator<auto, InIter::reference> OutIter>
  inline OutIter
  copy(InIter first, InIter last, OutIter result)
  {
    for (; first != last; ++result, ++first)
      *result = *first;
    return result;
  }
\end{codeblock}
  
With this scheme, a typical input iterator that returns an lvalue
reference will pass that lvalue reference on to the output iterator
(as in C++98/03). More interesting, however, is when the input
iterator is actually a \tcode{move_iterator}, whose reference type is
an rvalue reference. In this case, we're writing rvalue-references to
the output iterator, and therefore moving values from the input to the
output sequence. The use of the \tcode{reference} type as the output
type for the output iterator also copes with the transfer of values
via proxies.

We have already noted that a single type \tcode{X} can be an output
iterator for multiple, different value types. However, further study
of the standard library algorithms illustrates that a type parameter
might meet the \tcode{OutputIterator} requirements in multiple ways
\textit{within a single algorithm}. For example, in the
\tcode{replace_copy} algorithm the \tcode{OutIter} parameter acts as
an output iterator for both the input iterator's reference type
(allowing moves rather than copies) and the type of the replacement
value:

\begin{codeblock}
template<InputIterator InIter, typename OutIter, typename _Tp>
  requires OutputIterator<OutIter, InIter::reference>
        && OutputIterator<OutIter, const _Tp&>
        && HasEqualTo<InIter::value_type, _Tp>
  OutIter
  replace_copy(InIter first, InIter last,
               OutIter result,
               const _Tp& old_value, const _Tp& new_value)
  {
    for ( ; first != last; ++first, ++result)
      if (*first == old_value)
        *result = new_value;
      else
        *result = *first;
    return result;
  }
\end{codeblock}


This formulation of \tcode{replace_copy} uncovered an interesting
issue. Since both of the \tcode{OutputIterator} requirements have
different argument types, each contains dereference
(\tcode{operator*}) and increment (\tcode{operator+})
operators. Through concept maps, it is conceivable (however unlikely)
that these operators could be different for one requirement than the
other, and therefore each use of \tcode{*} or \tcode{++} that applies
to an instance of the \tcode{OutIter} type within \tcode{replace_copy}
returns an ambiguity. The ambiguity is a result of incomplete concept
analysis, and was solved by refactoring the requirements of
\tcode{OutputIterator} into two concepts: \tcode{Iterator<X>}, which
provides the syntax of \tcode{operator*} and \tcode{operator++} (and
is also refined by the \tcode{InputIterator<X>} concept), and the
\tcode{OutputIterator<X, Value>} concept, which adds assignability
requirements from \tcode{Value} to the reference type of the output
iterator. Thus, the requirements on \tcode{replace_copy} now say that
there is only one \tcode{operator*}, but the reference type that it
returns can be used in multiple, different ways. Moreover, we now have
an actual root to our iterator hierarchy, the \tcode{Iterator<X>}
concept, which provides only increment and dereference---the basics of
moving through a sequence of values---but does not provide any read or
write capabilities.

The \tcode{ShuffleIterator} concept (\ref{shuffle.iterators}) is a new
kind of iterator that captures the requirements needed to shuffle
values within a sequence using moves and
swaps. \tcode{ShuffleIterator} allows one to
move-construct or move-assign from an element in the sequence into a
variable of the iterator's value type. This permits, for example, the
pivot element in a quicksort to be extracted from the sequence and
placed into a variable. Additionally, values can be move-assigned from a
variable of the iterator's value type (e.g., moving the pivot back
from the temporary into the sorted sequence at the right time). This
concept is the proxy-aware conceptualization of the
Swappable+MoveConstructible+MoveAssignable set of concepts from the
reflector discussion starting with c++std-lib-21212. The
\tcode{ShuffleIterator} concept is used sparingly, in those cases
where the sequence cannot be efficiently reordered within simple swap
operations. 

Changes to the iterator taxonomy should not be taken lightly; even the
apparently simple iterators in C++98/03 turned out to be surprisingly
complicated, and have resulted in numerous defect reports and several
attempts at revisions. C++0x iterators are decidedly more complex, due
to the introduction of rvalue references and the desire to provide
support for proxy iterators throughout the standard library. To verify
the iterator concepts presented in this document, we have fully
implemented these concepts in ConceptGCC, applied them to nearly every
algorithm in the standard library, and tested the result against the
full libstdc++ test suite to ensure backward compatibility with
existing iterators. Indeed, many of the observations that drove this
refactoring came from implementation experience: the \tcode{operator*}
ambiguity, for example, was initially detected by ConceptGCC. 

\paragraph*{Changes from N2739}
\begin{itemize}
\item Moved the dereferenceability requirement on \tcode{operator*}
  from the \tcode{InputIterator} concept to the \tcode{Iterator}
  concept, since it applies to all iterators.
\item Translated the \tcode{BackwardTraversal} axiom of the
  \tcode{BidirectionalIterator} concept back into normative text,
  since we can't easily express the preconditions of the operations
  used.
\item Fixed the associated function default implementations in
  \tcode{RandomAccessIterator}.
\end{itemize}

\paragraph*{Changes from N2695}
\begin{itemize}
\item Replaced the \tcode{HasStdMove} concept (and its uses) with
  \tcode{RvalueOf}, which is now part of the foundational concepts.
\item In the \tcode{ShuffleIterator} concept, the
  \tcode{HasConstructor} requirement on the \tcode{value_type} has
  been replaced with the equivalent \tcode{Constructible} requirement;
  the explicit \tcode{NothrowDestructible} requirement has
  subsequently been removed.
\item Added move-constructibility of the value type into the
  \tcode{ShuffleIterator} concept.
\item We permit the \tcode{iterator_traits} typedefs for output
  iterators to be \tcode{void}. This behavior is permitted by
  \Cpp{}03, and was previously thought to cause problems with the
  backward-compatibility concept maps for output iterators (which
  relied on non-\tcode{void} typedefs even for output
  iterators). However, we determined that the concepts mechanism can
  deduce these types for output iterators, therefore ignoring the
  types specified in \tcode{iterator_traits} and allowing us to
  re-instate this user leeway. In [iterator.backward]p2, we clarify
  that we perform this deduction for output iterators.
\item Added the \tcode{subscript_reference} associated type to the
  \tcode{RandomAccessIterator} concept, to capture the result of
  \tcode{operator[]}. This type may need to be a proxy that is
  different from the \tcode{reference} type, for, e.g., a
  \tcode{counting_iterator} that meets the \Cpp03
  \tcode{RandomAccessIterator} requirements. See the reflector thread
  starting at c++std-lib-22126 for more information.
\end{itemize}


\end{titlepage}

\section*{Proposed Wording}
\paragraph*{Issues resolved by concepts}
The following LWG are resolved by concepts. These issues should be
resolved as NAD following the application of this proposal to the
wording paper:
\begin{description}
\item[Issue 299. Incorrect return types for iterator dereference.]
  Concepts specify precise return types for the iterator 
  operations, including \tcode{operator[]}.
\item[Issue 258. 24.1.5 contains unintented limitation for operator-.]
  Concepts now specify that the difference type of an iterator is a
  signed integral type.
\item[Issue 484. Convertible to T.]
With concepts, the iterator requirements also specify
"convertible to T", and this conversion will automatically be used
within constrained templates as necessary, so that the overload that
will be selected becomes clear from the requirements of the template.
\item[Issue 742. Enabling swap for proxy iterators.]  The concepts
  proposal provides a two-parameter swap that is available when
  \tcode{swap(w, v)} is valid or when the types of \tcode{w} and
  \tcode{v} are the same and that type is \tcode{MoveAssignable} and
  \tcode{MoveConstructible}, per the \tcode{std::swap} algorithm. The
  use of this \tcode{HasSwap} concept in the iterator concepts and in
  algorithms makes proxy iterators viable throughout the standard
  library.
\end{description}

%% --------------------------------------------------
%% Headers and footers
\pagestyle{fancy}
\fancyhead[LE,RO]{\textbf{\rightmark}}
\fancyhead[RE]{\textbf{\leftmark\hspace{1em}\thepage}}
\fancyhead[LO]{\textbf{\thepage\hspace{1em}\leftmark}}
\fancyfoot[C]{Draft}

\fancypagestyle{plain}{
\renewcommand{\headrulewidth}{0in}
\fancyhead[LE,RO]{}
\fancyhead[RE,LO]{}
\fancyfoot{}
}

\renewcommand{\sectionmark}[1]{\markright{\thesection\hspace{1em}#1}}
\renewcommand{\chaptermark}[1]{\markboth{#1}{}}

\setcounter{chapter}{23}
\rSec0[iterators]{Iterators library}

\begin{paras}

\setcounter{Paras}{1}

\textcolor{black}{\pnum}
The following subclauses describe
iterator \changedConcepts{requirements}{concepts}, and
components for
iterator primitives,
predefined iterators,
and stream iterators,
as summarized in Table~\ref{tab:iterators.lib.summary}.

\begin{libsumtab}{Iterators library summary}{tab:iterators.lib.summary}
\ref{iterator.concepts} \changedConcepts{Requirements}{Concepts}            &      \addedConcepts{\tt <iterator_concepts>}                                   \\ \rowsep
\ref{depr.lib.iterator.primitives} Iterator primitives       &       \tcode{<iterator>}              \\
\ref{predef.iterators} Predefined iterators         &                                                       \\
\ref{stream.iterators} Stream iterators                     &                                                       \\
\end{libsumtab}

\editorial{The following section has been renamed from ``Iterator requirements'' to ``Iterator concepts''.}
\rSec1[iterator.concepts]{Iterator concepts}

\pnum 
\addedConcepts{The \mbox{\tcode{<iterator_concepts>}} header describes requirements on iterators.}

\color{addclr}
\synopsis{Header \tcode{<iterator_concepts>}\ synopsis}
\begin{codeblock}
namespace std {
  concept Iterator<typename X> @\textit{see below}@;

  // \ref{input.iterators}, input iterators:
  concept InputIterator<typename X> @\textit{see below}@;

  // \ref{output.iterators}, output iterators:
  auto concept OutputIterator<typename X, typename Value> @\textit{see below}@;

  // \ref{forward.iterators}, forward iterators:
  concept ForwardIterator<typename X> @\textit{see below}@;

  @\textcolor{addclr}{}@// \ref{bidirectional.iterators}, bidirectional iterators:
  concept BidirectionalIterator<typename X> @\textit{see below}@;

  // \ref{random.access.iterators}, random access iterators:
  concept RandomAccessIterator<typename X> @\textit{see below}@;
  template<ObjectType T> concept_map RandomAccessIterator<T*> @\textit{see below}@;
  template<ObjectType T> concept_map RandomAccessIterator<const T*> @\textit{see below}@;

  @\addedConcepts{// \mbox{\ref{shuffle.iterators}}, shuffle iterators:}@
  auto concept ShuffleIterator<typename X> @\textit{see below}@;
}
\end{codeblock}
\color{black}

\pnum
\index{requirements!iterator}%
Iterators are a generalization of pointers that allow a \Cpp\ program to work with different data structures
(containers) in a uniform manner.
To be able to construct template algorithms that work correctly and
efficiently on different types of data structures, the library formalizes not just the interfaces but also the
semantics and complexity assumptions of iterators.
\addedConcepts{All iterators meet the requirements of the
  \mbox{\tcode{Iterator}} concept.}
All input iterators
\tcode{i}\
support the expression
\tcode{*i},
resulting in a value of some class, enumeration, or built-in type
\tcode{T},
called the
\techterm{value type}\ 
of the iterator.
All output iterators support the expression
\tcode{*i = o}\
where
\tcode{o}\
is a value of some type that is in the set of types that are
\techterm{writable}\
to the particular iterator type of
\tcode{i}.
All iterators
\tcode{i}\
for which the expression
\tcode{(*i).m}\
is well-defined, support the expression
\tcode{i->m}\
with the same semantics as
\tcode{(*i).m}.
For every iterator type
\tcode{X}\
for which
equality is defined, there is a corresponding signed integral type called the
\techterm{difference type}\ 
of the iterator.

\pnum
Since iterators are an abstraction of pointers, their semantics is
a generalization of most of the semantics of pointers in \Cpp.
This ensures that every
function template
that takes iterators
works as well with regular pointers.
This International Standard defines
\changedConcepts{five categories of iterators}{several iterator concepts}, according to the operations
defined on them:
\techterm{input iterators},
\techterm{output iterators},
\techterm{forward iterators}, 
\techterm{bidirectional iterators}, \removedConcepts{and}
\techterm{random access iterators}\addedConcepts{,} 
\addedConcepts{and shuffle iterators},
as shown in Table~\ref{tab:iterators.relations}.

\begin{floattable}{Relations among iterator \changedConcepts{categories}{concepts}}{tab:iterators.relations}
{ccccc}
\topline
\textbf{Random Access}                  &       $\rightarrow$ \textbf{Bidirectional}    &
$\rightarrow$ \textbf{Forward}  &       $\rightarrow$ \textbf{Input}  & \addedConcepts{\mbox{$\rightarrow$} \mbox{\textbf{Iterator}}}                   \\ 
 &  &  & \addedConcepts{\mbox{$\uparrow$}} & \addedConcepts{\mbox{$\uparrow$}} \\
   &        &     & \addedConcepts{\mbox{\textbf{Shuffle}}} &   \addedConcepts{\mbox{$\rightarrow$}} \textbf{Output}                   \\
\end{floattable}

\pnum
Forward iterators satisfy all the requirements of the input
\removedConcepts{and output} iterators and can be used whenever 
\changedConcepts{either kind}{an input iterator} is specified.
Bidirectional iterators also satisfy all the requirements of the
forward iterators and can be used whenever a forward iterator is specified.
Random access iterators also satisfy all the requirements of bidirectional
iterators and can be used whenever a bidirectional iterator is specified.

\pnum
\changedConcepts{
Besides its category, a forward, bidirectional, or random access iterator
can also be
mutable
or
constant
depending on
whether the result of the expression
*i
behaves as a reference or as a reference to a constant.
Constant iterators do not satisfy the requirements for output iterators,
and the result of the expression
*i
(for constant iterator
i)
cannot be used in an expression where an lvalue is required.}{
Iterators that meet the requirements of the
\mbox{\tcode{OutputIterator}} concept are called mutable
iterators. Non-mutable iterators are referred to as constant
iterators.}

\pnum
\resetcolor{}Just as a regular pointer to an array guarantees that there is a pointer value pointing past the last element
of the array, so \textcolor{black}{}for any iterator type there is an iterator value that points past the last element of a
corresponding container.
These values are called
\techterm{past-the-end}\ 
values.
Values of an iterator
\tcode{i}\
for which the expression
\tcode{*i}\
is defined are called
\techterm{dereferenceable}.
The library never assumes that past-the-end values are dereferenceable.
Iterators can also have singular values that are not associated with any
container.
\enterexample\ 
After the declaration of an uninitialized pointer
\tcode{x}\
(as with
\tcode{int* x;}),
\tcode{x}\
must always be assumed to have a singular value of a pointer.
\exitexample\ 
Results of most expressions are undefined for singular values;
the only exceptions are destroying an iterator that holds a singular value
and the assignment of a non-singular value to
an iterator that holds a singular value.
In this case the singular
value is overwritten the same way as any other value.
Dereferenceable
values are always non-singular.

\pnum
An iterator
\tcode{j}\
is called
\techterm{reachable}\ 
from an iterator
\tcode{i}\
if and only if there is a finite sequence of applications of
the expression
\tcode{++i}\
that makes
\tcode{i == j}.
If
\tcode{j}\
is reachable from
\tcode{i},
they refer to the same container.

\pnum
Most of the library's algorithmic templates that operate on data structures have interfaces that use ranges.
A
\techterm{range}\ 
is a pair of iterators that designate the beginning and end of the computation.
A range \range{i}{i}\
is an empty range;
in general, a range \range{i}{j}\
refers to the elements in the data structure starting with the one
pointed to by
\tcode{i}\
and up to but not including the one pointed to by
\tcode{j}.
Range \range{i}{j}\
is valid if and only if
\tcode{j}\
is reachable from
\tcode{i}.
The result of the application of functions in the library to invalid ranges is
undefined.

\pnum
All the \changedConcepts{categories of iterators}{iterator concepts} require only those functions that are realizable \removedConcepts{for a given category} in
constant time (amortized).
\removedConcepts{Therefore, requirement tables for the iterators do not have a complexity column.}

\pnum
Destruction of an iterator may invalidate pointers and references
previously obtained from that iterator.

\pnum
An
\techterm{invalid}\
iterator is an iterator that may be singular.%
\footnote{This definition applies to pointers, since pointers are iterators.
The effect of dereferencing an iterator that has been invalidated
is undefined.
}

\pnum
\removedConcepts{
In the following sections,
a
and
b
denote values of type
const X,
n
denotes a value of the difference type
Distance,
u,
tmp,
and
m
denote identifiers,
r
denotes a value of
X\&,
t
denotes a value of value type
T,
o
denotes a value of some type that is writable to the output iterator.}

\color{ccadd}
\begin{codeblock}
concept Iterator<typename X> : Semiregular<X> {
  MoveConstructible reference = typename X::reference;  
  MoveConstructible postincrement_result;

  requires HasDereference<postincrement_result>;

  reference operator*(X&&);
  X& operator++(X&);
  postincrement_result operator++(X&, int);   
}
\end{codeblock}
\color{black}

\pnum
\addedConcepts{The \mbox{\tcode{Iterator}} concept forms the basis of the
  iterator concept taxonomy, and every iterator meets the requirements
  of the \mbox{\tcode{Iterator}} concept. This concept specifies
  operations for dereferencing and incrementing the iterator, but
  provides no way to manipulate values. Most
  algorithms will require addition operations to read
  (\mbox{\ref{input.iterators}}) or write
  (\mbox{\ref{output.iterators}}) values, or to provide a richer set
  of iterator movements (\mbox{\ref{forward.iterators}},
  \mbox{\ref{bidirectional.iterators}},
  \mbox{\ref{random.access.iterators}}).}

\editorial{Of particular interest in this concept is the dereference
  operator, which accepts an rvalue reference to an iterator. This
  permits non-\tcode{const} lvalues and rvalues of iterators to be
  dereferenced, but it represents a minor break from C++98/03 where
  one could dereference a const iterator (not an iterator-to-const;
  those are unaffected). We expect the impact to be minimal, given
  that one cannot increment const iterators, and if there were an
  algorithm that dereferences a const iterator, it could just make a
  non-const copy of the iterator to dereference. We have verified this
  change with ConceptGCC and found no ill effects.}

\color{addclr}
\begin{itemdecl}
reference operator*(X&& @\farg{a}@);
\end{itemdecl}

\pnum
\addedConcepts{\mbox{\requires}
\mbox{\tcode{\farg{a}}}
is dereferenceable.}

\begin{itemdecl}
postincrement_result operator++(X& r, int);
\end{itemdecl}
\color{black}

\pnum
\effects\
equivalent to \mbox{\tcode{\{ X tmp = r; ++r; return tmp; \}}}.

\rSec2[input.iterators]{Input iterators}

\pnum
A class or a built-in type
\tcode{X}\
satisfies the requirements of an input iterator for the value type
\tcode{T}\
if \changedConcepts{the following expressions are valid,
where
U
is the type of any specified member of type
T,
as shown in Table~95.}{it meets the syntactic and semantic
requirements of the }\addedConcepts{\tt InputIterator}\addedConcepts{ concept.}

\color{addclr}
\begin{codeblock}
concept InputIterator<typename X> : Iterator<X>, EqualityComparable<X> {
  ObjectType value_type = typename X::value_type;
  MoveConstructible pointer = typename X::pointer;

  SignedIntegralLike difference_type = typename X::difference_type;

  requires IntegralType<difference_type>
        && Convertible<reference, const value_type &>;
        @\textcolor{addclr}{}@&& Convertible<pointer, const value_type*>;

  requires Convertible<HasDereference<postincrement_result>::result_type, const value_type&>;

  pointer operator->(const X&);
}
\end{codeblock}
\color{black}

\pnum
\changedConcepts{In Table~95}{In the }\addedConcepts{\tt
  InputIterator}\addedConcepts{ concept}, the term
\techterm{the domain of \tcode{==}}\
is used in the ordinary mathematical sense to denote
the set of values over which
\tcode{==}\ is (required to be) defined.
This set can change over time.
Each algorithm places additional requirements on the domain of
\tcode{==}\ for the iterator values it uses.
These requirements can be inferred from the uses that algorithm
makes of \tcode{==}\ and \tcode{!=}.
\enterexample
the call \tcode{find(a,b,x)}\
is defined only if the value of \tcode{a}\
has the property \textit{p}\
defined as follows:
\tcode{b}\ has property \textit{p}\
and a value \tcode{i}\
has property \textit{p}\
if
\tcode{(*i==x)}\
or if
\tcode{(*i!=x}\
and
\tcode{++i}\
has property
\tcode{p}).
\exitexample\

\eremove{Remove Table 96: Input iterator requirements}

\pnum
\enternote\ 
For input iterators,
\tcode{a == b}\
does not imply
\tcode{++a == ++b}.
(Equality does not guarantee the substitution property or referential transparency.)
Algorithms on input iterators should never attempt to pass through the same iterator twice.
\resetcolor{}They should be
\techterm{single pass}\ 
algorithms.
\removedConcepts{Value type T is not required to be an Assignable type (23.1).}\
These algorithms can be used with istreams as the source of the input data through the
\tcode{istream_iterator}\
class.
\exitnote\ 

\color{addclr}
\begin{itemdecl}
reference operator*(X&& @\farg{a}@); // inherited from Iterator<X>
\end{itemdecl}

\pnum
\addedConcepts{\mbox{\returns}
the value referenced by the iterator}

\pnum
\addedConcepts{\mbox{\notes}
If \mbox{\tcode{b}} is a value of type \mbox{\tcode{X}},
\mbox{\tcode{a == b}} and 
\mbox{\tcode{(a, b)}} is in the domain of \mbox{\tcode{==}}
then \mbox{\tcode{*a}} is equivalent to \mbox{\tcode{*b}}.}

\begin{itemdecl}
pointer operator->(const X& a);
\end{itemdecl}

\pnum
\addedConcepts{\mbox{\returns}
 a pointer to the value referenced by
  the iterator}

\begin{itemdecl}
bool operator==(const X& a, const X& b); // inherited from EqualityComparable<X>
\end{itemdecl}

\pnum
\addedConcepts{If two iterators \mbox{\tcode{a}} and \mbox{\tcode{b}}
  of the same type are equal, then either \mbox{\tcode{a}} and
  \mbox{\tcode{b}} 
are both dereferenceable
or else neither is dereferenceable.}

\begin{itemdecl}
X& operator++(X& r);
\end{itemdecl}

\pnum
\addedConcepts{\mbox{\precondition}
\mbox{\tcode{r}} is dereferenceable}

\pnum
\addedConcepts{\mbox{\postcondition}
\mbox{\tcode{r}} is dereferenceable or \mbox{\tcode{r}} is
past-the-end. Any copies 
of the previous value of \mbox{\tcode{r}} are no longer required either to be
dereferenceable or in the domain of \mbox{\tcode{==}}.}
\end{paras}

\rSec2[output.iterators]{Output iterators}

\pnum
A class or a built-in type
\tcode{X}\
satisfies the requirements of an output iterator
if
\changedConcepts{X
is a CopyConstructible (20.1.3)
and Assignable type (23.1) and also
the following expressions are
valid, as shown in Table~96}{meets the syntactic and semantic requirements of the \mbox{\tcode{OutputIterator}} concept.}

\eremove{Remove Table 97: Output iterator requirements}

\pnum
\enternote\ 
The only valid use of an
\tcode{operator*}\
is on the left side of the assignment statement.
\textit{Assignment through the same value of the iterator happens only once.}\ 
Algorithms on output iterators should never attempt to pass through the same iterator twice.
They should be
\techterm{single pass}\ 
algorithms.
Equality and inequality might not be defined.
Algorithms that take output iterators can be used with ostreams as the destination
for placing data through the
\tcode{ostream_iterator}\
class as well as with insert iterators and insert pointers.
\exitnote\ 

\pnum
The \tcode{OutputIterator} concept describes an output iterator that
may permit output of many different value types.

\color{addclr}
\begin{itemdecl}
auto concept OutputIterator<typename X, typename Value> {
  requires Iterator<X>;

  typename reference = Iterator<X>::reference;
  typename postincrement_result = Iterator<X>::postincrement_result;
  requires SameType<reference, Iterator<X>::reference>
        && SameType<postincrement_result, Iterator<X>::postincrement_result>
        && Convertible<postincrement_result, const X&>
        && HasAssign<reference, Value>
        && HasAssign<HasDereference<postincrement_result>::result_type, Value>;
}
\end{itemdecl}
\color{black}

\pnum
\addedConcepts{\enternote\ Any iterator that meets the additional requirements
  specified by \mbox{\tcode{OutputIterator}} for a given
  \mbox{\tcode{Value}} type is considered an output iterator. \exitnote}

\color{addclr}
\begin{itemdecl}
X& operator++(X& r); // from Iterator<X>
\end{itemdecl}

\pnum
\addedConcepts{\mbox{\postcondition}
\mbox{\tcode{\&\farg{r} == \&++\farg{r}}}}

\color{black}

\rSec2[forward.iterators]{Forward iterators}

\pnum
A class or a built-in type
\tcode{X}\
satisfies the requirements of a forward iterator if 
\changedConcepts{the following expressions are
valid, as shown in Table~97.}{it meets the syntactic and semantic
requirements of the ForwardIterator concept.}

\eremove{Remove Table 98: Forward iterator requirements.}

\color{addclr}
\begin{itemdecl}
concept ForwardIterator<typename X> : InputIterator<X>, Regular<X> {
  requires Convertible<postincrement_result, const X&>;

  axiom MultiPass(X a, X b) {
    if (a == b) *a == *b;
    if (a == b) ++a == ++b;
  }
}
\end{itemdecl}
\color{black}

\editorial{The \tcode{ForwardIterator} concept here provides weaker
  requirements on the \tcode{reference} and \tcode{pointer} types than
  the associated requirements table in C++03, because these types do
  not need to be true references or pointers to
  \tcode{value_type}. This change weakens the concept, meaning that
  C++03 iterators (which meet the stronger requirements) still meet
  these requirements, but algorithms that relied on these stricter
  requirements will no longer work just with the iterator
  requirements: they will need to specify true references or pointers
  as additional requirements. By weakening the requirements, however,
  we permit proxy iterators to model the forward, bidirectional, and
  random access iterator concepts.}

\begin{itemdecl}
X::X(); // inherited from Regular<X>
\end{itemdecl}

\begin{itemdescr}
  \pnum \addedConcepts{\mbox{\reallynote} the constructed object might have
    a singular value.}
\end{itemdescr}

\textcolor{black}{}\pnum
\enternote\ 
The \changedConcepts{condition}{axiom} that
\tcode{a == b}\
implies
\tcode{++a == ++b}\
(which is not true for input and output iterators)
and the removal of the restrictions on the number of the assignments through the iterator
(which applies to output iterators)
allows the use of multi-pass one-directional algorithms with forward iterators.
\exitnote\ 

\begin{itemdecl}
X& operator++(X& r); // inherited from InputIterator<X>
\end{itemdecl}

\begin{itemdescr}
\pnum
\addedConcepts{\mbox{\postcondition} \mbox{\tcode{\&r == \&++r}}.}
\end{itemdescr}

\rSec2[bidirectional.iterators]{Bidirectional iterators}

\pnum
A class or a built-in type
\tcode{X}\
satisfies the requirements of a bidirectional iterator if
\changedConcepts{,
in addition to satisfying the requirements for forward iterators,
the following expressions are valid as shown in
Table~98.}{it meets the
syntactic and semantic requirements of the
BidirectionalIterator concept.}

\eremove{Remove Table 99: Bidirectional iterator requirements.}

\color{addclr}
\begin{itemdecl}
concept BidirectionalIterator<typename X> : ForwardIterator<X> {
  MoveConstructible postdecrement_result;
  requires HasDereference<postdecrement_result>
        &&   Convertible<HasDereference<postdecrement_result>::result_type, const value_type&>
        &&   Convertible<postdecrement_result, const X&>;

  X& operator--(X&);
  postdecrement_result operator--(X&, int);
}
\end{itemdecl}
\color{black}

\pnum
\enternote\ 
Bidirectional iterators allow algorithms to move iterators backward as well as forward.
\exitnote\ 

\color{addclr}
\begin{itemdecl}  
X& operator--(X& r);
\end{itemdecl}

\pnum
\addedConcepts{\mbox{\precondition}
there exists \mbox{\tcode{s}} such that \mbox{\tcode{r == ++s}}.}

\pnum
\addedConcepts{\mbox{\requires} 
\mbox{\tcode{-{}-(++r) == r}} and, 
given lvalues \mbox{\tcode{a}} and \mbox{\tcode{b}} of type \mbox{\tcode{X}},
\mbox{\tcode{-{}-a == -{}-b}} implies \mbox{\tcode{a == b}}}

\pnum
\addedConcepts{\mbox{\postcondition}
\mbox{\tcode{r}} is dereferenceable.}
\addedConcepts{\mbox{\tcode{\&r == \&{-}{-}r}}.}

\begin{itemdecl}
postdecrement_result operator--(X& r, int);
\end{itemdecl}

\pnum
\addedConcepts{\mbox{\effects}
equivalent to}
\begin{codeblock}
{ X tmp = r;
--r;
return tmp; }
\end{codeblock}
\color{black}

\rSec2[random.access.iterators]{Random access iterators}

\pnum
A class or a built-in type
\tcode{X}\
satisfies the requirements of a random access iterator if
\changedConcepts{,
in addition to satisfying the requirements for bidirectional iterators,
the following expressions are valid as shown in Table~99.}
{it meets the syntactic and semantic requirements of the
\mbox{\tcode{RandomAccessIterator}} concept.}

\color{addclr}
\begin{itemdecl}
concept RandomAccessIterator<typename X> : BidirectionalIterator<X>, LessThanComparable<X> {
  MoveConstructible subscript_reference;
  requires Convertible<subscript_reference, const value_type&>;

  X& operator+=(X&, difference_type);
  X  operator+ (const X& x, difference_type n) { X tmp(x); tmp += n; return tmp; }
  X  operator+ (difference_type n, const X& x) { X tmp(x); tmp += n; return tmp; }
  X& operator-=(X&, difference_type);
  X  operator- (const X& x, difference_type n) { X tmp(x); tmp -= n; return tmp; }

  difference_type operator-(const X&, const X&);
  subscript_reference operator[](const X& x, difference_type n);
}
\end{itemdecl}
\color{black}

\eremove{Remove Table 100: Random access iterator requirements.}

\color{addclr}
\begin{itemdecl}
X& operator+=(X& r, difference_type n);
\end{itemdecl}

\pnum
\addedConcepts{\mbox{\effects}
equivalent to}
\begin{codeblock}
{ difference_type m = n;
  if (m >= 0) while (m--) ++r;
  else while (m++) --r;
  return r; }
\end{codeblock}

\begin{itemdecl}
X operator+(const X& a, difference_type n);
X operator+(difference_type n, const X& a);
\end{itemdecl}

\pnum
\addedConcepts{\mbox{\effects}
equivalent to}
\begin{codeblock}
{ X tmp = a;
return tmp += n; }
\end{codeblock}

\pnum
\addedConcepts{\mbox{\postcondition}
\mbox{\tcode{a + n == n + a}}}

\begin{itemdecl}
@\textcolor{addclr}{X}@& operator-=(X& r, difference_type n);
\end{itemdecl}

\pnum
\addedConcepts{\mbox{\returns}
\mbox{\tcode{r += -n}}}

\begin{itemdecl}
X operator-(const X& a, difference_type n);
\end{itemdecl}

\pnum
\addedConcepts{\mbox{\effects}
equivalent to}
\begin{codeblock}
{ X tmp = a;
  return tmp -= n; }
\end{codeblock}

\begin{itemdecl}
difference_type operator-(const X& a, const X& b);
\end{itemdecl}

\pnum
\addedConcepts{\mbox{\precondition}
there exists a value \mbox{\tcode{n}} of \mbox{\tcode{difference_type}}
  such that \mbox{\tcode{a + n == b}}.}

\pnum
\addedConcepts{\mbox{\effects}
\mbox{\tcode{b == a + (b - a)}}}

\pnum
\addedConcepts{\mbox{\returns}
\mbox{\tcode{(a < b) ? distance(a,b) : -distance(b,a)}}}

\pnum
\addedConcepts{Pointers are random access iterators with the following
concept map}

\begin{codeblock}
namespace std {
  template<ObjectType T> concept_map RandomAccessIterator<T*> {
    typedef T value_type;
    typedef ptrdiff_t difference_type;
    typedef T& reference;
    typedef T* pointer;
  }
}
\end{codeblock}

\addedConcepts{and pointers to const are random access iterators}

\begin{codeblock}
namespace std {
  template<ObjectType T> concept_map RandomAccessIterator<const T*> {
    typedef T value_type;
    typedef ptrdiff_t difference_type;
    typedef const T& reference;
    typedef const T* pointer;
  }
}
\end{codeblock}

\pnum
\addedConcepts{\mbox{\enternote}
If there is an additional pointer type
\mbox{\tcode{\,\xname{far}}}
such that the difference of two
\mbox{\tcode{\,\xname{far}}}
is of type
\mbox{\tcode{long}},
an implementation may define}

\color{addclr}
\begin{codeblock}
  template <ObjectType T> concept_map RandomAccessIterator<T @\xname{far}@*> {
    typedef long difference_type;
    typedef T value_type;
    typedef T @\xname{far}@* pointer;
    typedef T @\xname{far}@& reference;
  }

  template <ObjectType T> concept_map RandomAccessIterator<const T @\xname{far}@*> {
    typedef long difference_type;
    typedef T value_type;
    typedef const T @\xname{far}@* pointer;
    typedef const T @\xname{far}@& reference;
  }
\end{codeblock}
\textcolor{addclr}{\exitnote}
\color{black}

\rSec2[shuffle.iterators]{Shuffle iterators}
\pnum
\addedConcepts{A class or built-in type \mbox{\tcode{X}} satisfies the
  requirements of a shuffle iterator if it meets the syntactic and
  semantic requirements of the \mbox{\tcode{ShuffleIterator}} concept.}

\color{ccadd}
\begin{codeblock}
auto concept ShuffleIterator<typename X> {
  requires InputIterator<X>
        && OutputIterator<X, RvalueOf<InputIterator<X>::value_type>::type>
        && OutputIterator<X, RvalueOf<InputIterator<X>::reference>::type>
        && Constructible<InputIterator<X>::value_type, 
                         RvalueOf<InputIterator<X>::reference>::type>
        && MoveConstructible<InputIterator<X>::value_type>
        && HasAssign<InputIterator<X>::value_type, 
                     RvalueOf<InputIterator<X>::reference>::type>
        && HasSwap<InputIterator<X>::reference, InputIterator<X>::reference>;
}
\end{codeblock}
\color{black}

\pnum
\addedConcepts{A shuffle iterator is a form of input and output iterator
  that allows values to be moved into or out of a sequence, along with
  permitting efficient swapping of values within the sequence. Shuffle
  iterators are typically used in algorithms that need to rearrange
  the elements within a sequence in a way that cannot be performed
  efficiently with swaps alone.}

\pnum
\addedConcepts{\enternote\ Any iterator that meets the additional requirements
  specified by \mbox{\tcode{ShuffleIterator}} is considered a shuffle
  iterator. \exitnote}

\appendix
\setcounter{chapter}{3}
\normannex{depr}{Compatibility features}
\begin{paras}

\setcounter{section}{9}
\rSec1[depr.lib.iterator.primitives]{Iterator primitives}

\textcolor{black}{\pnum}
To simplify the \changedConcepts{task of defining iterators}{use of
  iterators and provide backward compatibility with previous C++
  Standard Libraries}, the library provides
several classes and functions. 

\pnum
\addedConcepts{The \mbox{\tcode{iterator_traits}} and supporting
  facilities described 
in this section are deprecated. \mbox{\enternote} the iterator
concepts (\mbox{\ref{iterator.concepts}}) provide the equivalent
functionality using the concept mechanism. \mbox{\exitnote}}

\end{paras}

\rSec2[iterator.traits]{Iterator traits}

\pnum
\changedConcepts{
To implement algorithms only in terms of iterators, it is often necessary to
determine the value and
difference types that correspond to a particular iterator type.
Accordingly, it is required that if}
{Iterator traits provide an auxiliary mechanism for
accessing the associated types of an iterator. If}
\tcode{Iterator}\
is the type of an iterator,
the types

\begin{codeblock}
iterator_traits<Iterator>::difference_type
iterator_traits<Iterator>::value_type
iterator_traits<Iterator>::iterator_category
\end{codeblock}

\addedConcepts{shall} be defined as the iterator's difference type, value type and iterator
category \addedConcepts{(24.3.3)}, respectively.
In addition, the types

\begin{codeblock}
iterator_traits<Iterator>::reference
iterator_traits<Iterator>::pointer
\end{codeblock}

shall be defined as the iterator's reference and pointer types, that is, for an
iterator object \tcode{a}, the same type as the type of \tcode{*a} and \tcode{a->},
respectively. In the case of an output iterator, the types

\begin{codeblock}
iterator_traits<Iterator>::difference_type
iterator_traits<Iterator>::value_type
iterator_traits<Iterator>::reference
iterator_traits<Iterator>::pointer
\end{codeblock}

may be defined as \tcode{void}.

\editorial{Paragraphs 2--5 of this section are unchanged.}

\setcounter{Paras}{5}
\pnum
\addedConcepts{For each iterator category, a partial specializations of the
  \mbox{\tcode{iterator_traits}} class template provide appropriate
  type definitions for programs that use the deprecated iterator
  traits mechanism. These partial specializations provide backward
  compatibility for unconstrained templates using iterators as
  specified by the corresponding requirements tables of ISO/IEC
  14882:2003.}
\color{ccadd}
\begin{codeblock}
concept IsReference<typename T> { } // exposition only
template<typename T> concept_map IsReference<T&> { }

concept IsPointer<typename T> { } // exposition only
template<typename T> concept_map IsPointer<T*> { }

template<Iterator Iter> struct iterator_traits<Iter> {
  typedef void                             difference_type;
  typedef void                             value_type;
  typedef void                             pointer;
  typedef void                             reference;
  typedef output_iterator_tag              iterator_category;
};

template<InputIterator Iter> struct iterator_traits<Iter> {
  typedef Iter::difference_type            difference_type;
  typedef Iter::value_type                 value_type;
  typedef Iter::pointer                    pointer;
  typedef Iter::reference                  reference;
  typedef input_iterator_tag                   iterator_category;
};

template<ForwardIterator Iter> 
  requires IsReference<Iter::reference> && IsPointer<Iter::pointer>
  struct iterator_traits<Iter> {
    typedef Iter::difference_type            difference_type;
    typedef Iter::value_type                 value_type;
    typedef Iter::pointer                    pointer;
    typedef Iter::reference                  reference;
    typedef forward_iterator_tag                 iterator_category;
  };

template<BidirectionalIterator Iter> 
  requires IsReference<Iter::reference> && IsPointer<Iter::pointer>
  struct iterator_traits<Iter> {
    typedef Iter::difference_type            difference_type;
    typedef Iter::value_type                 value_type;
    typedef Iter::pointer                    pointer;
    typedef Iter::reference                  reference;
    typedef bidirectional_iterator_tag       iterator_category;
  };

template<RandomAccessIterator Iter> 
  requires IsReference<Iter::reference> && IsPointer<Iter::pointer>
  struct iterator_traits<Iter> {
    typedef Iter::difference_type            difference_type;
    typedef Iter::value_type                 value_type;
    typedef Iter::pointer                    pointer;
    typedef Iter::reference                  reference;
    typedef random_access_iterator_tag       iterator_category;
  };
\end{codeblock} 
\exitnote\
\color{black}

\rSec2[iterator.basic]{Basic iterator}

\editorial{We deprecated the basic \tcode{iterator} template because
  it isn't really the right way to specify iterators any more. Even
  when using this template, users should write concept maps so that
  (1) their iterators will work when \tcode{iterator_traits} and the
  backward-compatibility models go away, and (2) so that their
  iterators will be checked against the iterator concepts as early as
  possible.}

\pnum
The
\tcode{iterator}
template may be used as a base class to ease the definition of required types
for new iterators.

\begin{codeblock}
namespace std {
  template<class Category, class T, class Distance = ptrdiff_t,
           class Pointer = T*, class Reference = T&>
  struct iterator {
        typedef T         value_type;
        typedef Distance  difference_type;
        typedef Pointer   pointer;
        typedef Reference reference;
        typedef Category  iterator_category;
  };
}
\end{codeblock}

\rSec2[std.iterator.tags]{Standard iterator tags}

\pnum
\index{input_iterator_tag@\tcode{input_iterator_tag}}%
\index{output_iterator_tag@\tcode{output_iterator_tag}}%
\index{forward_iterator_tag@\tcode{forward_iterator_tag}}%
\index{bidirectional_iterator_tag@\tcode{bidirectional_iterator_tag}}%
\index{random_access_iterator_tag@\tcode{random_access_iterator_tag}}%
\changedConcepts{It is often desirable for a
function template specialization
to find out what is the most specific category of its iterator
argument, so that the function can select the most efficient algorithm at compile time.
To facilitate this, the}{The}
library \textcolor{black}{}introduces
\techterm{category tag}\ 
classes which are used as compile time tags
\changedConcepts{for algorithm selection.}{to distinguish the
  different iterator concepts when using the \mbox{\tcode{iterator_traits}} mechanism.}
They are:
\tcode{input_iterator_tag},
\tcode{output_iterator_tag},
\tcode{forward_iterator_tag},
\tcode{bidirectional_iterator_tag}\
and
\tcode{random_access_iterator_tag}.
For every iterator of type
\tcode{Iterator},
\tcode{iterator_traits<Iterator>::it\-er\-a\-tor_ca\-te\-go\-ry}
shall be defined to be the most specific category tag that describes the
iterator's behavior.

\begin{codeblock}
namespace std {
  struct input_iterator_tag {};
  struct output_iterator_tag {};
  @\color{black}@struct forward_iterator_tag: public input_iterator_tag {};
  struct bidirectional_iterator_tag: public forward_iterator_tag {};
  struct random_access_iterator_tag: public bidirectional_iterator_tag {};
}
\end{codeblock}

\pnum 
\eremove{Remove this paragraph: It gives an example using
  \tcode{iterator_traits}, which we no longer encourage.}

\color{addclr}
\rSec2[iterator.backward]{Iterator backward compatibility}

\pnum
\addedConcepts{The library provides concept maps that allow iterators
  specified with \mbox{\tcode{iterator_traits}}
to interoperate with algorithms that require iterator
concepts. \enterexample}
\begin{codeblock}
struct random_iterator
{
  typedef std::input_iterator_tag iterator_category;
  typedef int                     value_type;
  typedef int                     difference_type;
  typedef int*                    pointer;
  typedef int                     reference;

  random_iterator(int remaining = 0) : remaining(remaining) { }

  int operator*() const { return std::rand(); }
  int* operator->() const { return 0; }
  
  random_iterator& operator++() { --remaining; return *this; }

  random_iterator operator++(int) { 
    random_iterator tmp(*this); ++(*this); return tmp;
  }

  int remaining;

  friend bool 
  operator==(const random_iterator& i, const random_iterator& j)
  {
    return i.remaining == j.remaining;
  }

  friend bool 
  operator!=(const random_iterator& i, const random_iterator& j)
  {
    return i.remaining != j.remaining;
  }
};

void f(random_iterator i, random_iterator j) {
  std::copy(i, j, std::ostream_iterator<int>(std::cout, " ")); // okay: standard library produces concept
                                                               // map InputIterator<random_iterator>
}
\end{codeblock}
\addedConcepts{\exitexample}

\pnum
\addedConcepts{For all iterator types except output iterators, the
associated types 
\mbox{\tcode{difference_type}},
\mbox{\tcode{value_type}},
\mbox{\tcode{pointer}}
and
\mbox{\tcode{reference}}
are given the same values as their counterparts in 
\mbox{\tcode{iterator_traits}}. For output iterators, the
\mbox{\tcode{reference}} type is deduced from the type of the output
iterator's dereference operator.}

\pnum
\color{addclr}
\addedConcepts{When the 
\mbox{\tcode{iterator_traits}}
specialization contains the nested types 
\mbox{\tcode{difference_type}},
\mbox{\tcode{value_type}},
\mbox{\tcode{pointer}},
\mbox{\tcode{reference}}
and
\mbox{\tcode{iterator_category}}, the
\mbox{\tcode{iterator_traits}} specialization is considered to be \mbox{\techterm{valid}}.}

\addedConcepts{\enterexample}
\addedConcepts{The following example is well-formed. The backward-compatbility
concept map for \mbox{\tcode{InputIterator}} does not match because
\mbox{\tcode{iterator_traits<int>}} is not valid.}
\begin{codeblock}
@\color{addclr}@template<IntegralLike T> void f(T);
template<InputIterator T> void f(T);

void g(int x) {
  f(x); // okay
}
\end{codeblock}
\addedConcepts{\exitexample}

\pnum
\addedConcepts{The library shall provide a concept map
\mbox{\tcode{Iterator<Iter>}}
for any type \mbox{\tcode{Iter}} with a valid \mbox{\tcode{iterator_traits<Iter>}}, an
\mbox{\tcode{iterator_traits<Iter>::it\-er\-a\-tor_ca\-te\-go\-ry}}
convertible to 
\mbox{\tcode{output_iterator_tag}}, and that meets the
  syntactic requirements of the \mbox{\tcode{Iterator}}
  concept.} 

\pnum
\addedConcepts{The library shall provide a concept map
\mbox{\tcode{InputIterator<Iter>}}
for any type \mbox{\tcode{Iter}} with a valid \mbox{\tcode{iterator_traits<Iter>}}, an \mbox{\tcode{iterator_traits<Iter>::it\-er\-a\-tor_ca\-te\-go\-ry}} convertible to
\mbox{\tcode{input_iterator_tag}}, and that meets the
syntactic requirements of the \mbox{\tcode{InputIterator}} concept.}

\pnum
\addedConcepts{The library shall provide a concept map
\mbox{\tcode{ForwardIterator<Iter>}}
for any type \mbox{\tcode{Iter}} with a valid
\mbox{\tcode{iterator_traits<Iter>}}, an \mbox{\tcode{iterator_traits<Iterator>::it\-er\-a\-tor_ca\-te\-go\-ry}}
convertible to
\mbox{\tcode{forward_iterator_tag}}, and that meets the
syntactic requirements of the \mbox{\tcode{ForwardIterator}} concept.}

\pnum
\addedConcepts{The library shall provide a concept map
\mbox{\tcode{BidirectionalIterator<Iter>}}
for any type \mbox{\tcode{Iter}} with a valid
\mbox{\tcode{iterator_traits<Iter>}}, an
\mbox{\tcode{iterator_traits<Iterator>::it\-er\-a\-tor_ca\-te\-go\-ry}}
convertible to
\mbox{\tcode{bidirectional_iterator_tag}}, and that meets the
syntactic requirements of the \mbox{\tcode{BidirectionalIterator}} concept.}

\pnum
\addedConcepts{The library shall provide a concept map
\mbox{\tcode{RandomAccessIterator<Iter>}}
for any type \mbox{\tcode{Iter}} with a valid
\mbox{\tcode{iterator_traits<Iter>}}, an
\mbox{\tcode{iterator_traits<Iterator>::it\-er\-a\-tor_ca\-te\-go\-ry}}
convertible to
\mbox{\tcode{random_access_iterator_tag}}, and that meets the
syntactic requirements of the \mbox{\tcode{RandomAccessIterator}} concept.}

\section*{Acknowledgments}
Thanks to Beman Dawes for alerting us to omissions from the iterator
concepts and Daniel Kr\"ugler for many helpful comments. Both Mat
Marcus and Jaakko J\"arvi were particularly helpful in the design of
the new iterator taxonomy.

\bibliographystyle{plain}
\bibliography{../local}

\end{document}