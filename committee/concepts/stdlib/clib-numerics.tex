\documentclass[american,twoside]{book}
\usepackage{refbib}
\usepackage{hyperref}
\usepackage{pdfsync}
% Definitions and redefinitions of special commands

\usepackage{babel}      % needed for iso dates
\usepackage{savesym}		% suppress duplicate macro definitions
\usepackage{fancyhdr}		% more flexible headers and footers
\usepackage{listings}		% code listings
\usepackage{longtable}	% auto-breaking tables
\usepackage{remreset}		% remove counters from reset list
\usepackage{booktabs}		% fancy tables
\usepackage{relsize}		% provide relative font size changes
\usepackage[htt]{hyphenat}	% hyphenate hyphenated words: conflicts with underscore
\savesymbol{BreakableUnderscore}	% suppress BreakableUnderscore defined in hyphenat
									                % (conflicts with underscore)
\usepackage{underscore}	% remove special status of '_' in ordinary text
\usepackage{verbatim}		% improved verbatim environment
\usepackage{parskip}		% handle non-indented paragraphs "properly"
\usepackage{array}			% new column definitions for tables
\usepackage[iso]{isodate} % use iso format for dates
\usepackage{soul}       % strikeouts and underlines for difference markups
\usepackage{color}      % define colors for strikeouts and underlines
\usepackage{amsmath}    % additional math symbols
\usepackage{mathrsfs}
\usepackage{multicol}

\usepackage[T1]{fontenc}
\usepackage{ae}
\usepackage{mathptmx}
\usepackage[scaled=.90]{helvet}

%% Difference markups
\definecolor{addclr}{rgb}{0,.4,.4}
\definecolor{remclr}{rgb}{1,0,0}
\newcommand{\added}[1]{\textcolor{addclr}{\ul{#1}}}
\newcommand{\removed}[1]{\textcolor{remclr}{\st{#1}}}
\newcommand{\changed}[2]{\removed{#1}\added{#2}}
\newcommand{\remfn}{\footnote{\removed{removed footnote}}}
\newcommand{\addfn}[1]{\footnote{\added{#1}}}
\newcommand{\remitem}[1]{\item\removed{#1}}
\newcommand{\additem}[1]{\item\added{#1}}

%% Added by JJ
\long\gdef\metacomment#1{[{\sc Editorial note:} \begingroup\sf\aftergroup] #1\endgroup}

%% October, 2005 changes
\newcommand{\addedA}[1]{#1}
\newcommand{\removedA}[1]{}
\newcommand{\changedA}[2]{#2}

%% April, 2006 changes
\newcommand{\addedB}[1]{#1}
\newcommand{\removedB}[1]{}
\newcommand{\changedB}[2]{#2}
\newcommand{\remfootnoteB}[1]{}
\newcommand{\marktr}{}
\newcommand\ptr{}

%% October, 2006 changes
%\newcommand{\addedC}[1]{\added{#1}}
%\newcommand{\removedC}[1]{\removed{#1}}
%\newcommand{\changedC}[2]{\changed{#1}{#2}}
%\newcommand{\remfootnoteC}[1]{\remfn}
%\newcommand{\addfootnoteC}[1]{\addfn{#1}}
%\newcommand{\remitemC}[1]{\remitem{#1}}
%\newcommand{\additemC}[1]{\additem{#1}}
%\newcommand{\remblockC}{}

%% November registration ballot
\newcommand{\addedC}[1]{#1}
\newcommand{\removedC}[1]{}
\newcommand{\changedC}[2]{#2}
\newcommand{\remfootnoteC}[1]{}
\newcommand{\addfootnoteC}[1]{\footnote{#1}}
\newcommand{\remitemC}[1]{}
\newcommand{\additemC}[1]{\item{#1}}
\newcommand{\remblockC}{\remov_this_block}

\newcommand{\addedD}[1]{#1}
\newcommand{\removedD}[1]{}
\newcommand{\changedD}[2]{#2}
\newcommand{\remfootnoteD}[1]{}
\newcommand{\addfootnoteD}[1]{\footnote{#1}}
\newcommand{\remitemD}[1]{}
\newcommand{\additemD}[1]{\item{#1}}
\newcommand{\remblockD}{\remov_this_block}

%% Variadic Templates changes
\newcommand{\addedVT}[1]{\textcolor{addclr}{\ul{#1}}}
\newcommand{\removedVT}[1]{\textcolor{remclr}{\st{#1}}}
\newcommand{\changedVT}[2]{\removed{#1}\added{#2}}

%% Concepts changes
\newcommand{\addedConcepts}[1]{\added{#1}}
\newcommand{\removedConcepts}[1]{\removed{#1}}
\newcommand{\changedConcepts}[2]{\changed{#1}{#2}}
\newcommand{\addedConceptsC}[1]{\textcolor{addclr}{\tcode{\ul{#1}}}}
\newcommand{\remitemConcepts}[1]{\remitem{#1}}
\newcommand{\additemConcepts}[1]{\additem{#1}}

%% Concepts changes since the last revision
\definecolor{ccadd}{rgb}{0,.6,0}
\newcommand{\addedCC}[1]{\textcolor{ccadd}{\ul{#1}}}
\newcommand{\removedCC}[1]{\textcolor{remclr}{\st{#1}}}
\newcommand{\changedCC}[2]{\removedCC{#1}\addedCC{#2}}
\newcommand{\remitemCC}[1]{\remitem{#1}}
\newcommand{\additemCC}[1]{\item\addedCC{#1}}
\newcommand{\changedCCC}[2]{\textcolor{ccadd}{\st{#1}}\addedCC{#2}}
\newcommand{\removedCCC}[1]{\textcolor{ccadd}{\st{#1}}}
\newcommand{\remitemCCC}[1]{\item\removedCCC{#1}}

%% Concepts changes for the next revision
\definecolor{zadd}{rgb}{0.8,0,0.8}
\newcommand{\addedZ}[1]{\textcolor{zadd}{\ul{#1}}}
\newcommand{\removedZ}[1]{\textcolor{remclr}{\st{#1}}}
\newcommand{\changedCZ}[2]{\textcolor{addclr}{\st{#1}}\addedZ{#2}}


%%--------------------------------------------------
%% Sectioning macros.  
% Each section has a depth, an automatically generated section 
% number, a name, and a short tag.  The depth is an integer in 
% the range [0,5].  (If it proves necessary, it wouldn't take much
% programming to raise the limit from 5 to something larger.)


% The basic sectioning command.  Example:
%    \Sec1[intro.scope]{Scope}
% defines a first-level section whose name is "Scope" and whose short
% tag is intro.scope.  The square brackets are mandatory.
\def\Sec#1[#2]#3{{%
\ifcase#1\let\s=\chapter
      \or\let\s=\section
      \or\let\s=\subsection
      \or\let\s=\subsubsection
      \or\let\s=\paragraph
      \or\let\s=\subparagraph
      \fi%
\s[#3]{#3\hfill[#2]}\relax\label{#2}}}

% A convenience feature (mostly for the convenience of the Project
% Editor, to make it easy to move around large blocks of text):
% the \rSec macro is just like the \Sec macro, except that depths 
% relative to a global variable, SectionDepthBase.  So, for example,
% if SectionDepthBase is 1,
%   \rSec1[temp.arg.type]{Template type arguments}
% is equivalent to
%   \Sec2[temp.arg.type]{Template type arguments}

\newcounter{SectionDepthBase}
\newcounter{scratch}

\def\rSec#1[#2]#3{{%
\setcounter{scratch}{#1}
\addtocounter{scratch}{\value{SectionDepthBase}}
\Sec{\arabic{scratch}}[#2]{#3}}}

% Change the way section headings are formatted.
\renewcommand{\chaptername}{}
\renewcommand{\appendixname}{Annex}

\makeatletter
\def\@makechapterhead#1{%
  \hrule\vspace*{1.5\p@}\hrule
  \vspace*{16\p@}%
  {\parindent \z@ \raggedright \normalfont
    \ifnum \c@secnumdepth >\m@ne
        \huge\bfseries \@chapapp\space \thechapter\space\space\space\space
    \fi
    \interlinepenalty\@M
    \huge \bfseries #1\par\nobreak
  \vspace*{16\p@}%
  \hrule\vspace*{1.5\p@}\hrule
  \vspace*{48\p@}
  }}

\renewcommand\section{\@startsection{section}{1}{0pt}%
                                   {-3.5ex plus -1ex minus -.2ex}%
                                   {.3ex plus .2ex}%
                                   {\normalfont\normalsize\bfseries}}
\renewcommand\section{\@startsection{section}{1}{0pt}%
                                   {2.5ex}% plus 1ex minus .2ex}%
                                   {.3ex}% plus .1ex minus .2 ex}%
                                   {\normalfont\normalsize\bfseries}}

\renewcommand\subsection{\@startsection{subsection}{2}{0pt}%
                                     {-3.25ex plus -1ex minus -.2ex}%
                                     {.3ex plus .2ex}%
                                     {\normalfont\normalsize\bfseries}}

\renewcommand\subsubsection{\@startsection{subsubsection}{3}{0pt}%
                                     {-3.25ex plus -1ex minus -.2ex}%
                                     {.3ex plus .2ex}%
                                     {\normalfont\normalsize\bfseries}}

\renewcommand\paragraph{\@startsection{paragraph}{4}{0pt}%
                                     {-3.25ex plus -1ex minus -.2ex}%
                                     {.3ex \@plus .2ex}%
                                     {\normalfont\normalsize\bfseries}}

\renewcommand\subparagraph{\@startsection{subparagraph}{5}{0pt}%
                                     {-3.25ex plus -1ex minus -.2ex}%
                                     {.3ex plus .2ex}%
                                     {\normalfont\normalsize\bfseries}}
\@removefromreset{footnote}{chapter}
\@removefromreset{table}{chapter}
\@removefromreset{figure}{chapter}
\makeatother

%%--------------------------------------------------
% Heading style for Annexes
\newcommand{\Annex}[3]{\chapter[#2]{\\(#3)\\#2\hfill[#1]}\relax\label{#1}}
\newcommand{\infannex}[2]{\Annex{#1}{#2}{informative}}
\newcommand{\normannex}[2]{\Annex{#1}{#2}{normative}}

\newcommand{\synopsis}[1]{\textbf{#1}}

%%--------------------------------------------------
% General code style
\newcommand{\CodeStyle}{\ttfamily}
\newcommand{\CodeStylex}[1]{\texttt{#1}}

% Code and definitions embedded in text.
\newcommand{\tcode}[1]{\CodeStylex{#1}}
\newcommand{\techterm}[1]{\textit{#1}}

%%--------------------------------------------------
%% allow line break if needed for justification
\newcommand{\brk}{\discretionary{}{}{}}
%  especially for scope qualifier
\newcommand{\colcol}{\brk::\brk}

%%--------------------------------------------------
%% Macros for funky text
%%!\newcommand{\Rplus}{\protect\nolinebreak\hspace{-.07em}\protect\raisebox{.25ex}{\small\textbf{+}}}
\newcommand{\Rplus}{+}
\newcommand{\Cpp}{C\Rplus\Rplus}
\newcommand{\opt}{$_\mathit{opt}$}
\newcommand{\shl}{<{<}}
\newcommand{\shr}{>{>}}
\newcommand{\dcr}{-{-}}
\newcommand{\bigohm}[1]{\mathscr{O}(#1)}
\newcommand{\bigoh}[1]{$\bigohm{#1}$}
\renewcommand{\tilde}{{\smaller$\sim$}}		% extra level of braces is necessary

%%--------------------------------------------------
%% States and operators

\newcommand{\state}[2]{\tcode{#1}\ensuremath{_{#2}}}
\newcommand{\bitand}{\ensuremath{\, \mathsf{bitand} \,}}
\newcommand{\bitor}{\ensuremath{\, \mathsf{bitor} \,}}
\newcommand{\xor}{\ensuremath{\, \mathsf{xor} \,}}
\newcommand{\rightshift}{\ensuremath{\, \mathsf{rshift} \,}}
\newcommand{\leftshift}{\ensuremath{\, \mathsf{lshift} \,}}

%% Notes and examples
\newcommand{\EnterBlock}[1]{[\,\textit{#1:}}
\newcommand{\ExitBlock}[1]{\textit{\ ---\,end #1}\,]}
\newcommand{\enternote}{\EnterBlock{Note}}
\newcommand{\exitnote}{\ExitBlock{note}}
\newcommand{\enterexample}{\EnterBlock{Example}}
\newcommand{\exitexample}{\ExitBlock{example}}

%% Library function descriptions
\newcommand{\Fundescx}[1]{\textit{#1}}
\newcommand{\Fundesc}[1]{\Fundescx{#1:}}
\newcommand{\required}{\Fundesc{Required behavior}}
\newcommand{\requires}{\Fundesc{Requires}}
\newcommand{\effects}{\Fundesc{Effects}}
\newcommand{\postconditions}{\Fundesc{Postconditions}}
\newcommand{\postcondition}{\Fundesc{Postcondition}}
\newcommand{\preconditions}{\Fundesc{Preconditions}}
\newcommand{\precondition}{\Fundesc{Precondition}}
\newcommand{\returns}{\Fundesc{Returns}}
\newcommand{\throws}{\Fundesc{Throws}}
\newcommand{\default}{\Fundesc{Default behavior}}
\newcommand{\complexity}{\Fundesc{Complexity}}
\newcommand{\note}{\Fundesc{Remark}}
\newcommand{\notes}{\Fundesc{Remarks}}
\newcommand{\implimits}{\Fundesc{Implementation limits}}
\newcommand{\replaceable}{\Fundesc{Replaceable}}
\newcommand{\exceptionsafety}{\Fundesc{Exception safety}}
\newcommand{\returntype}{\Fundesc{Return type}}

%% Cross reference
\newcommand{\xref}{\textsc{See also:}}

%% NTBS, etc.
\newcommand{\NTS}[1]{\textsc{#1}}
\newcommand{\ntbs}{\NTS{ntbs}}
\newcommand{\ntmbs}{\NTS{ntmbs}}
\newcommand{\ntwcs}{\NTS{ntwcs}}

%% Function argument
\newcommand{\farg}[1]{\texttt{\textit{#1}}}

%% Code annotations
\newcommand{\EXPO}[1]{\textbf{#1}}
\newcommand{\expos}{\EXPO{exposition only}}
\newcommand{\exposr}{\hfill\expos}
\newcommand{\exposrc}{\hfill// \expos}
\newcommand{\impdef}{\EXPO{implementation-defined}}
\newcommand{\notdef}{\EXPO{not defined}}

%% Double underscore
\newcommand{\unun}{\_\,\_}
\newcommand{\xname}[1]{\unun\,#1}
\newcommand{\mname}[1]{\tcode{\unun\,#1\,\unun}}

%% Ranges
\newcommand{\Range}[4]{\tcode{#1\brk{}#3,\brk{}#4\brk{}#2}}
\newcommand{\crange}[2]{\Range{[}{]}{#1}{#2}}
\newcommand{\orange}[2]{\Range{(}{)}{#1}{#2}}
\newcommand{\range}[2]{\Range{[}{)}{#1}{#2}}

%% Change descriptions
\newcommand{\diffdef}[1]{\hfill\break\textbf{#1:}}
\newcommand{\change}{\diffdef{Change}}
\newcommand{\rationale}{\diffdef{Rationale}}
\newcommand{\effect}{\diffdef{Effect on original feature}}
\newcommand{\difficulty}{\diffdef{Difficulty of converting}}
\newcommand{\howwide}{\diffdef{How widely used}}

%% Miscellaneous
\newcommand{\uniquens}{\textrm{\textit{\textbf{unique}}}}
\newcommand{\stage}[1]{\item{\textbf{Stage #1:}}}

%%--------------------------------------------------
%% Adjust markers
\renewcommand{\thetable}{\arabic{table}}
\renewcommand{\thefigure}{\arabic{figure}}
\renewcommand{\thefootnote}{\arabic{footnote})}

%% Change list item markers from box to dash
\renewcommand{\labelitemi}{---}
\renewcommand{\labelitemii}{---}
\renewcommand{\labelitemiii}{---}
\renewcommand{\labelitemiv}{---}

%%--------------------------------------------------
%% Environments for code listings.

% We use the 'listings' package, with some small customizations.  The
% most interesting customization: all TeX commands are available
% within comments.  Comments are set in italics, keywords and strings
% don't get special treatment.

\lstset{language=C++,
        basicstyle=\CodeStyle\small,
        keywordstyle=,
        stringstyle=,
        xleftmargin=1em,
        showstringspaces=false,
        commentstyle=\itshape\rmfamily,
        columns=flexible,
        keepspaces=true,
        texcl=true}

% Our usual abbreviation for 'listings'.  Comments are in 
% italics.  Arbitrary TeX commands can be used if they're 
% surrounded by @ signs.
\lstnewenvironment{codeblock}
{
 \lstset{escapechar=@}
 \renewcommand{\tcode}[1]{\textup{\CodeStyle##1}}
 \renewcommand{\techterm}[1]{\textit{##1}}
}
{
}

% Permit use of '@' inside codeblock blocks (don't ask)
\makeatletter
\newcommand{\atsign}{@}
\makeatother

%%--------------------------------------------------
%% Paragraph numbering
\newcounter{Paras}
\makeatletter
\@addtoreset{Paras}{chapter}
\@addtoreset{Paras}{section}
\@addtoreset{Paras}{subsection}
\@addtoreset{Paras}{subsubsection}
\@addtoreset{Paras}{paragraph}
\@addtoreset{Paras}{subparagraph}
\def\pnum{\addtocounter{Paras}{1}\noindent\llap{{\footnotesize\arabic{Paras}}\hspace{\@totalleftmargin}\quad}}
\makeatother

% For compatibility only.  We no longer need this environment.
\newenvironment{paras}{}{}

%%--------------------------------------------------
%% Indented text
\newenvironment{indented}
{\list{}{}\item\relax}
{\endlist}

%%--------------------------------------------------
%% Library item descriptions
\lstnewenvironment{itemdecl}
{
 \lstset{escapechar=@,
 xleftmargin=0em,
 aboveskip=2ex,
 belowskip=0ex	% leave this alone: it keeps these things out of the
				% footnote area
 }
}
{
}

\newenvironment{itemdescr}
{
 \begin{indented}}
{
 \end{indented}
}


%%--------------------------------------------------
%% Bnf environments
\newlength{\BnfIndent}
\setlength{\BnfIndent}{\leftmargini}
\newlength{\BnfInc}
\setlength{\BnfInc}{\BnfIndent}
\newlength{\BnfRest}
\setlength{\BnfRest}{2\BnfIndent}
\newcommand{\BnfNontermshape}{\rmfamily\itshape\small}
\newcommand{\BnfTermshape}{\ttfamily\upshape\small}
\newcommand{\nonterminal}[1]{{\BnfNontermshape #1}}

\newenvironment{bnfbase}
 {
 \newcommand{\terminal}[1]{{\BnfTermshape ##1}}
 \newcommand{\descr}[1]{\normalfont{##1}}
 \newcommand{\bnfindentfirst}{\BnfIndent}
 \newcommand{\bnfindentinc}{\BnfInc}
 \newcommand{\bnfindentrest}{\BnfRest}
 \begin{minipage}{.9\hsize}
 \newcommand{\br}{\hfill\\}
 }
 {
 \end{minipage}
 }

\newenvironment{BnfTabBase}[1]
{
 \begin{bnfbase}
 #1
 \begin{indented}
 \begin{tabbing}
 \hspace*{\bnfindentfirst}\=\hspace{\bnfindentinc}\=\hspace{.6in}\=\hspace{.6in}\=\hspace{.6in}\=\hspace{.6in}\=\hspace{.6in}\=\hspace{.6in}\=\hspace{.6in}\=\hspace{.6in}\=\hspace{.6in}\=\hspace{.6in}\=\kill%
}
{
 \end{tabbing}
 \end{indented}
 \end{bnfbase}
}

\newenvironment{bnfkeywordtab}
{
 \begin{BnfTabBase}{\BnfTermshape}
}
{
 \end{BnfTabBase}
}

\newenvironment{bnftab}
{
 \begin{BnfTabBase}{\BnfNontermshape}
}
{
 \end{BnfTabBase}
}

\newenvironment{simplebnf}
{
 \begin{bnfbase}
 \BnfNontermshape
 \begin{indented}
}
{
 \end{indented}
 \end{bnfbase}
}

\newenvironment{bnf}
{
 \begin{bnfbase}
 \list{}
	{
	\setlength{\leftmargin}{\bnfindentrest}
	\setlength{\listparindent}{-\bnfindentinc}
	\setlength{\itemindent}{\listparindent}
	}
 \BnfNontermshape
 \item\relax
}
{
 \endlist
 \end{bnfbase}
}

% non-copied versions of bnf environments
\newenvironment{ncbnftab}
{
 \begin{bnftab}
}
{
 \end{bnftab}
}

\newenvironment{ncsimplebnf}
{
 \begin{simplebnf}
}
{
 \end{simplebnf}
}

\newenvironment{ncbnf}
{
 \begin{bnf}
}
{
 \end{bnf}
}

%%--------------------------------------------------
%% Drawing environment
%
% usage: \begin{drawing}{UNITLENGTH}{WIDTH}{HEIGHT}{CAPTION}
\newenvironment{drawing}[4]
{
\begin{figure}[h]
\setlength{\unitlength}{#1}
\begin{center}
\begin{picture}(#2,#3)\thicklines
}
{
\end{picture}
\end{center}
%\caption{Directed acyclic graph}
\end{figure}
}

%%--------------------------------------------------
%% Table environments

% Base definitions for tables
\newenvironment{TableBase}
{
 \renewcommand{\tcode}[1]{{\CodeStyle##1}}
 \newcommand{\topline}{\hline}
 \newcommand{\capsep}{\hline\hline}
 \newcommand{\rowsep}{\hline}
 \newcommand{\bottomline}{\hline}

%% vertical alignment
 \newcommand{\rb}[1]{\raisebox{1.5ex}[0pt]{##1}}	% move argument up half a row

%% header helpers
 \newcommand{\hdstyle}[1]{\textbf{##1}}				% set header style
 \newcommand{\Head}[3]{\multicolumn{##1}{##2}{\hdstyle{##3}}}	% add title spanning multiple columns
 \newcommand{\lhdrx}[2]{\Head{##1}{|c}{##2}}		% set header for left column spanning #1 columns
 \newcommand{\chdrx}[2]{\Head{##1}{c}{##2}}			% set header for center column spanning #1 columns
 \newcommand{\rhdrx}[2]{\Head{##1}{c|}{##2}}		% set header for right column spanning #1 columns
 \newcommand{\ohdrx}[2]{\Head{##1}{|c|}{##2}}		% set header for only column spanning #1 columns
 \newcommand{\lhdr}[1]{\lhdrx{1}{##1}}				% set header for single left column
 \newcommand{\chdr}[1]{\chdrx{1}{##1}}				% set header for single center column
 \newcommand{\rhdr}[1]{\rhdrx{1}{##1}}				% set header for single right column
 \newcommand{\ohdr}[1]{\ohdrx{1}{##1}}
 \newcommand{\br}{\hfill\break}						% force newline within table entry

%% column styles
 \newcolumntype{x}[1]{>{\raggedright\let\\=\tabularnewline}p{##1}}	% word-wrapped ragged-right
 																	% column, width specified by #1
 \newcolumntype{m}[1]{>{\CodeStyle}l{##1}}							% variable width column, all entries in CodeStyle
}
{
}

% General Usage: TITLE is the title of the table, XREF is the
% cross-reference for the table. LAYOUT is a sequence of column
% type specifiers (e.g. cp{1.0}c), without '|' for the left edge
% or right edge.

% usage: \begin{floattablebase}{TITLE}{XREF}{COLUMNS}{PLACEMENT}
% produces floating table, location determined within limits
% by LaTeX.
\newenvironment{floattablebase}[4]
{
 \begin{TableBase}
 \begin{table}[#4]
 \caption{\label{#2}#1}
 \begin{center}
 \begin{tabular}{|#3|}
}
{
 \bottomline
 \end{tabular}
 \end{center}
 \end{table}
 \end{TableBase}
}

% usage: \begin{floattable}{TITLE}{XREF}{COLUMNS}
% produces floating table, location determined within limits
% by LaTeX.
\newenvironment{floattable}[3]
{
 \begin{floattablebase}{#1}{#2}{#3}{htbp}
}
{
 \end{floattablebase}
}

% usage: \begin{tokentable}{TITLE}{XREF}{HDR1}{HDR2}
% produces six-column table used for lists of replacement tokens;
% the columns are in pairs -- left-hand column has header HDR1,
% right hand column has header HDR2; pairs of columns are separated
% by vertical lines. Used in "trigraph sequences" table in standard.
\newenvironment{tokentable}[4]
{
 \begin{floattablebase}{#1}{#2}{cc|cc|cc}{htbp}
 \topline
 \textit{#3}   &   \textit{#4}    &
 \textit{#3}   &   \textit{#4}    &
 \textit{#3}   &   \textit{#4}    \\ \capsep
}
{
 \end{floattablebase}
}

% usage: \begin{libsumtabase}{TITLE}{XREF}{HDR1}{HDR2}
% produces two-column table with column headers HDR1 and HDR2.
% Used in "Library Categories" table in standard, and used as
% base for other library summary tables.
\newenvironment{libsumtabbase}[4]
{
 \begin{floattable}{#1}{#2}{ll}
 \topline
 \lhdr{#3}	&	\hdstyle{#4}	\\ \capsep
}
{
 \end{floattable}
}

% usage: \begin{libsumtab}{TITLE}{XREF}
% produces two-column table with column headers "Subclause" and "Header(s)".
% Used in "C++ Headers for Freestanding Implementations" table in standard.
\newenvironment{libsumtab}[2]
{
 \begin{libsumtabbase}{#1}{#2}{Subclause}{Header(s)}
}
{
 \end{libsumtabbase}
}

% usage: \begin{LibSynTab}{CAPTION}{TITLE}{XREF}{COUNT}{LAYOUT}
% produces table with COUNT columns. Used as base for
% C library description tables
\newcounter{LibSynTabCols}
\newcounter{LibSynTabWd}
\newenvironment{LibSynTabBase}[5]
{
 \setcounter{LibSynTabCols}{#4}
 \setcounter{LibSynTabWd}{#4}
 \addtocounter{LibSynTabWd}{-1}
 \newcommand{\centry}[1]{\textbf{##1}:}
 \newcommand{\macro}{\centry{Macro}}
 \newcommand{\macros}{\centry{Macros}}
 \newcommand{\function}{\centry{Function}}
 \newcommand{\functions}{\centry{Functions}}
 \newcommand{\templates}{\centry{Templates}}
 \newcommand{\type}{\centry{Type}}
 \newcommand{\types}{\centry{Types}}
 \newcommand{\values}{\centry{Values}}
 \newcommand{\struct}{\centry{Struct}}
 \newcommand{\cspan}[1]{\multicolumn{\value{LibSynTabCols}}{|l|}{##1}}
 \begin{floattable}{#1 \tcode{<#2>}\ synopsis}{#3}
 {#5}
 \topline
 \lhdr{Type}	&	\rhdrx{\value{LibSynTabWd}}{Name(s)}	\\ \capsep
}
{
 \end{floattable}
}

% usage: \begin{LibSynTab}{TITLE}{XREF}{COUNT}{LAYOUT}
% produces table with COUNT columns. Used as base for description tables
% for C library
\newenvironment{LibSynTab}[4]
{
 \begin{LibSynTabBase}{Header}{#1}{#2}{#3}{#4}
}
{
 \end{LibSynTabBase}
}

% usage: \begin{LibSynTabAdd}{TITLE}{XREF}{COUNT}{LAYOUT}
% produces table with COUNT columns. Used as base for description tables
% for additions to C library
\newenvironment{LibSynTabAdd}[4]
{
 \begin{LibSynTabBase}{Additions to header}{#1}{#2}{#3}{#4}
}
{
 \end{LibSynTabBase}
}

% usage: \begin{libsyntabN}{TITLE}{XREF}
%        \begin{libsyntabaddN}{TITLE}{XREF}
% produces a table with N columns for C library description tables
\newenvironment{libsyntab2}[2]
{
 \begin{LibSynTab}{#1}{#2}{2}{ll}
}
{
 \end{LibSynTab}
}

\newenvironment{libsyntab3}[2]
{
 \begin{LibSynTab}{#1}{#2}{3}{lll}
}
{
 \end{LibSynTab}
}

\newenvironment{libsyntab4}[2]
{
 \begin{LibSynTab}{#1}{#2}{4}{llll}
}
{
 \end{LibSynTab}
}

\newenvironment{libsyntab5}[2]
{
 \begin{LibSynTab}{#1}{#2}{5}{lllll}
}
{
 \end{LibSynTab}
}

\newenvironment{libsyntab6}[2]
{
 \begin{LibSynTab}{#1}{#2}{6}{llllll}
}
{
 \end{LibSynTab}
}

\newenvironment{libsyntabadd2}[2]
{
 \begin{LibSynTabAdd}{#1}{#2}{2}{ll}
}
{
 \end{LibSynTabAdd}
}

\newenvironment{libsyntabadd3}[2]
{
 \begin{LibSynTabAdd}{#1}{#2}{3}{lll}
}
{
 \end{LibSynTabAdd}
}

\newenvironment{libsyntabadd4}[2]
{
 \begin{LibSynTabAdd}{#1}{#2}{4}{llll}
}
{
 \end{LibSynTabAdd}
}

\newenvironment{libsyntabadd5}[2]
{
 \begin{LibSynTabAdd}{#1}{#2}{5}{lllll}
}
{
 \end{LibSynTabAdd}
}

\newenvironment{libsyntabadd6}[2]
{
 \begin{LibSynTabAdd}{#1}{#2}{6}{llllll}
}
{
 \end{LibSynTabAdd}
}

% usage: \begin{LongTable}{TITLE}{XREF}{LAYOUT}
% produces table that handles page breaks sensibly.
\newenvironment{LongTable}[3]
{
 \begin{TableBase}
 \begin{longtable}
 {|#3|}\caption{#1}\label{#2}
}
{
 \bottomline
 \end{longtable}
 \end{TableBase}
}

% usage: \begin{twocol}{TITLE}{XREF}
% produces a two-column breakable table. Used in
% "simple-type-specifiers and the types they specify" in the standard.
\newenvironment{twocol}[2]
{
 \begin{LongTable}
 {#1}{#2}
 {ll}
}
{
 \end{LongTable}
}

% usage: \begin{libreqtabN}{TITLE}{XREF}
% produces an N-column brekable table. Used in
% most of the library clauses for requirements tables.
% Example at "Position type requirements" in the standard.

\newenvironment{libreqtab1}[2]
{
 \begin{LongTable}
 {#1}{#2}
 {x{.55\hsize}}
}
{
 \end{LongTable}
}

\newenvironment{libreqtab2}[2]
{
 \begin{LongTable}
 {#1}{#2}
 {lx{.55\hsize}}
}
{
 \end{LongTable}
}

\newenvironment{libreqtab2a}[2]
{
 \begin{LongTable}
 {#1}{#2}
 {x{.30\hsize}x{.68\hsize}}
}
{
 \end{LongTable}
}

\newenvironment{libreqtab3}[2]
{
 \begin{LongTable}
 {#1}{#2}
 {x{.28\hsize}x{.18\hsize}x{.43\hsize}}
}
{
 \end{LongTable}
}

\newenvironment{libreqtab3a}[2]
{
 \begin{LongTable}
 {#1}{#2}
 {x{.28\hsize}x{.33\hsize}x{.29\hsize}}
}
{
 \end{LongTable}
}

\newenvironment{libreqtab3b}[2]
{
 \begin{LongTable}
 {#1}{#2}
 {x{.40\hsize}x{.25\hsize}x{.25\hsize}}
}
{
 \end{LongTable}
}

\newenvironment{libreqtab3c}[2]
{
 \begin{LongTable}
 {#1}{#2}
 {x{.30\hsize}x{.25\hsize}x{.35\hsize}}
}
{
 \end{LongTable}
}

\newenvironment{libreqtab3d}[2]
{
 \begin{LongTable}
 {#1}{#2}
 {x{.32\hsize}x{.27\hsize}x{.36\hsize}}
}
{
 \end{LongTable}
}

\newenvironment{libreqtab3e}[2]
{
 \begin{LongTable}
 {#1}{#2}
 {x{.38\hsize}x{.27\hsize}x{.25\hsize}}
}
{
 \end{LongTable}
}

\newenvironment{libreqtab3f}[2]
{
 \begin{LongTable}
 {#1}{#2}
 {x{.40\hsize}x{.22\hsize}x{.31\hsize}}
}
{
 \end{LongTable}
}

\newenvironment{libreqtab4}[2]
{
 \begin{LongTable}
 {#1}{#2}
}
{
 \end{LongTable}
}

\newenvironment{libreqtab4a}[2]
{
 \begin{LongTable}
 {#1}{#2}
 {x{.14\hsize}x{.30\hsize}x{.30\hsize}x{.14\hsize}}
}
{
 \end{LongTable}
}

\newenvironment{libreqtab4b}[2]
{
 \begin{LongTable}
 {#1}{#2}
 {x{.13\hsize}x{.15\hsize}x{.29\hsize}x{.27\hsize}}
}
{
 \end{LongTable}
}

\newenvironment{libreqtab4c}[2]
{
 \begin{LongTable}
 {#1}{#2}
 {x{.16\hsize}x{.21\hsize}x{.21\hsize}x{.30\hsize}}
}
{
 \end{LongTable}
}

\newenvironment{libreqtab4d}[2]
{
 \begin{LongTable}
 {#1}{#2}
 {x{.22\hsize}x{.22\hsize}x{.30\hsize}x{.15\hsize}}
}
{
 \end{LongTable}
}

\newenvironment{libreqtab5}[2]
{
 \begin{LongTable}
 {#1}{#2}
 {x{.14\hsize}x{.14\hsize}x{.20\hsize}x{.20\hsize}x{.14\hsize}}
}
{
 \end{LongTable}
}

% usage: \begin{libtab2}{TITLE}{XREF}{LAYOUT}{HDR1}{HDR2}
% produces two-column table with column headers HDR1 and HDR2.
% Used in "seekoff positioning" in the standard.
\newenvironment{libtab2}[5]
{
 \begin{floattable}
 {#1}{#2}{#3}
 \topline
 \lhdr{#4}	&	\rhdr{#5}	\\ \capsep
}
{
 \end{floattable}
}

% usage: \begin{longlibtab2}{TITLE}{XREF}{LAYOUT}{HDR1}{HDR2}
% produces two-column table with column headers HDR1 and HDR2.
\newenvironment{longlibtab2}[5]
{
 \begin{LongTable}{#1}{#2}{#3}
 \\ \topline
 \lhdr{#4}	&	\rhdr{#5}	\\ \capsep
}
{
  \end{LongTable}
}

% usage: \begin{LibEffTab}{TITLE}{XREF}{HDR2}{WD2}
% produces a two-column table with left column header "Element"
% and right column header HDR2, right column word-wrapped with
% width specified by WD2.
\newenvironment{LibEffTab}[4]
{
 \begin{libtab2}{#1}{#2}{lp{#4}}{Element}{#3}
}
{
 \end{libtab2}
}

% Same as LibEffTab except that it uses a long table.
\newenvironment{longLibEffTab}[4]
{
 \begin{longlibtab2}{#1}{#2}{lp{#4}}{Element}{#3}
}
{
 \end{longlibtab2}
}

% usage: \begin{libefftab}{TITLE}{XREF}
% produces a two-column effects table with right column
% header "Effect(s) if set", width 4.5 in. Used in "fmtflags effects"
% table in standard.
\newenvironment{libefftab}[2]
{
 \begin{LibEffTab}{#1}{#2}{Effect(s) if set}{4.5in}
}
{
 \end{LibEffTab}
}

% Same as libefftab except that it uses a long table.
\newenvironment{longlibefftab}[2]
{
 \begin{longLibEffTab}{#1}{#2}{Effect(s) if set}{4.5in}
}
{
 \end{longLibEffTab}
}

% usage: \begin{libefftabmean}{TITLE}{XREF}
% produces a two-column effects table with right column
% header "Meaning", width 4.5 in. Used in "seekdir effects"
% table in standard.
\newenvironment{libefftabmean}[2]
{
 \begin{LibEffTab}{#1}{#2}{Meaning}{4.5in}
}
{
 \end{LibEffTab}
}

% Same as libefftabmean except that it uses a long table.
\newenvironment{longlibefftabmean}[2]
{
 \begin{longLibEffTab}{#1}{#2}{Meaning}{4.5in}
}
{
 \end{longLibEffTab}
}

% usage: \begin{libefftabvalue}{TITLE}{XREF}
% produces a two-column effects table with right column
% header "Value", width 3 in. Used in "basic_ios::init() effects"
% table in standard.
\newenvironment{libefftabvalue}[2]
{
 \begin{LibEffTab}{#1}{#2}{Value}{3in}
}
{
 \end{LibEffTab}
}

% Same as libefftabvalue except that it uses a long table and a
% slightly wider column.
\newenvironment{longlibefftabvalue}[2]
{
 \begin{longLibEffTab}{#1}{#2}{Value}{3.5in}
}
{
 \end{longLibEffTab}
}

% usage: \begin{liberrtab}{TITLE}{XREF} produces a two-column table
% with left column header ``Value'' and right header "Error
% condition", width 4.5 in. Used in regex clause in the TR.

\newenvironment{liberrtab}[2]
{
 \begin{libtab2}{#1}{#2}{lp{4.5in}}{Value}{Error condition}
}
{
 \end{libtab2}
}

% Like liberrtab except that it uses a long table.
\newenvironment{longliberrtab}[2]
{
 \begin{longlibtab2}{#1}{#2}{lp{4.5in}}{Value}{Error condition}
}
{
 \end{longlibtab2}
}

% enumerate with lowercase letters
\newenvironment{enumeratea}
{
 \renewcommand{\labelenumi}{\alph{enumi})}
 \begin{enumerate}
}
{
 \end{enumerate}
}

% enumerate with arabic numbers
\newenvironment{enumeraten}
{
 \renewcommand{\labelenumi}{\arabic{enumi})}
 \begin{enumerate}
}
{
 \end{enumerate}
}

%%--------------------------------------------------
%% Definitions section
% usage: \definition{name}{xref}
%\newcommand{\definition}[2]{\rSec2[#2]{#1}}
% for ISO format, use:
\newcommand{\definition}[2]
 {\hfill\vspace{.25ex plus .5ex minus .2ex}\\
 \addtocounter{subsection}{1}%
 \textbf{\thesubsection\hfill\relax[#2]}\\
 \textbf{#1}\label{#2}\\
 }


%%--------------------------------------------------
%% Set section numbering limit, toc limit
\setcounter{secnumdepth}{5}
\setcounter{tocdepth}{1}

%%--------------------------------------------------
%% Parameters that govern document appearance
\setlength{\oddsidemargin}{0pt}
\setlength{\evensidemargin}{0pt}
\setlength{\textwidth}{6.6in}

%%--------------------------------------------------
%% Handle special hyphenation rules
\hyphenation{tem-plate ex-am-ple in-put-it-er-a-tor}

% Do not put blank pages after chapters that end on odd-numbered pages.
\def\cleardoublepage{\clearpage\if@twoside%
  \ifodd\c@page\else\hbox{}\thispagestyle{empty}\newpage%
  \if@twocolumn\hbox{}\newpage\fi\fi\fi}

\begin{document}
\raggedbottom

\begin{titlepage}
\begin{center}
\huge
Concepts for the C++0x Standard Library: Numerics\\
(Revision 1)

\vspace{0.5in}

\normalsize
Douglas Gregor, and Andrew Lumsdaine \\
Open Systems Laboratory \\
Indiana University \\
Bloomington, IN\ \  47405 \\
\{\href{mailto:dgregor@osl.iu.edu}{dgregor}, \href{mailto:lums@osl.iu.edu}{lums}\}@osl.iu.edu
\end{center}

\vspace{1in}
\par\noindent Document number: N2574=08-0084\vspace{-6pt}
\par\noindent Revises document number: N2041=06-0111\vspace{-6pt}
\par\noindent Date: \today\vspace{-6pt}
\par\noindent Project: Programming Language \Cpp{}, Library Working Group\vspace{-6pt}
\par\noindent Reply-to: Douglas Gregor $<$\href{mailto:dgregor@osl.iu.edu}{dgregor@osl.iu.edu}$>$\vspace{-6pt}

\section*{Introduction}
\libintrotext{Chapter 26}

\paragraph*{Changes from N2041}
\begin{itemize}
\item Update to the latest concepts syntax and the appropriate names
  of the core concepts.
\end{itemize}

\end{titlepage}

%%--------------------------------------------------
%% Headers and footers
\pagestyle{fancy}
\fancyhead[LE,RO]{\textbf{\rightmark}}
\fancyhead[RE]{\textbf{\leftmark\hspace{1em}\thepage}}
\fancyhead[LO]{\textbf{\thepage\hspace{1em}\leftmark}}
\fancyfoot[C]{Draft}

\fancypagestyle{plain}{
\renewcommand{\headrulewidth}{0in}
\fancyhead[LE,RO]{}
\fancyhead[RE,LO]{}
\fancyfoot{}
}

\renewcommand{\sectionmark}[1]{\markright{\thesection\hspace{1em}#1}}
\renewcommand{\chaptermark}[1]{\markboth{#1}{}}

\color{black}

\setcounter{chapter}{25}
\rSec0[lib.numerics]{Numerics library}
\begin{paras}

\setcounter{section}{5}

\rSec1[lib.numeric.ops]{Generalized numeric operations}

\synopsis{Header \tcode{<numeric>}\ synopsis}
\index{numeric@\tcode{<numeric>}}%

\color{addclr}
\begin{codeblock}
namespace std {
  template <InputIterator Iter, HasPlus<auto, Iter::reference> T>
    requires CopyAssignable<T, T::result_type>
    T accumulate(Iter @\farg{first}@, Iter @\farg{last}@, T @\farg{init}@);
  template <InputIterator Iter, class T, Callable<auto, T, Iter::reference> BinaryOperation>
    requires CopyAssignable<T, BinaryOperation::result_type>
    T accumulate(Iter @\farg{first}@, Iter @\farg{last}@, T @\farg{init}@,
  	         BinaryOperation @\farg{binary_op}@);
  template <InputIterator Iter1, InputIterator Iter2, class T>
    requires HasMultiply<Iter1::reference, Iter2::reference> &&
             HasPlus<T, HasMultiply<Iter1::reference, Iter2::reference>::result_type> &&
             CopyAssignable<
               T, 
               HasPlus<T, 
                       HasMultiply<Iter1::reference, Iter2::reference>::result_type>::result_type>
    T inner_product(Iter1 @\farg{first1}@, Iter1 @\farg{last1}@,
  		    Iter2 @\farg{first2}@, T @\farg{init}@);
  template <InputIterator Iter1, InputIterator Iter2, class T,
  	    class BinaryOperation1, Callable<auto, Iter1::reference, Iter2::reference> BinaryOperation2>
    requires Callable<BinaryOperation1, T, BinaryOperation2::result_type> &&
             CopyAssignable<T, BinaryOperation1::result_type>
    T inner_product(Iter1 @\farg{first1}@, Iter1 @\farg{last1}@,
  		    Iter2 @\farg{first2}@, T @\farg{init}@,
  		    BinaryOperation1 @\farg{binary_op1}@,
  		    BinaryOperation2 @\farg{binary_op2}@);
  template <InputIterator InIter, OutputIterator<auto, InIter::value_type> OutIter>
    requires HasPlus<InIter::value_type> && 
             CopyAssignable<InIter::value_type, HasPlus<InIter::value_type>::result_type> &&
             CopyConstructible<InIter::value_type>
    OutIter partial_sum(InIter @\farg{first}@, InIter @\farg{last}@,
  	  	        OutIter @\farg{result}@);
  template<InputIterator InIter, OutputIterator<auto, InIter::value_type> OutIter, 
           Callable<auto, InIter::value_type, InIter::value_type> BinaryOperation>
    requires CopyAssignable<InIter::value_type, BinaryOperation::result_type> && 
             CopyConstructible<InIter::value_type>
    OutIter partial_sum(InIter @\farg{first}@, InIter @\farg{last}@,
      		        OutIter @\farg{result}@, BinaryOperation @\farg{binary_op}@);
  template <InputIterator InIter, OutputIterator<auto, InIter::value_type> OutIter>
    requires HasMinus<InIter::value_type, InIter::value_type> &&
             CopyAssignable<OutIter, HasMinus<InIter::value_type, InIter::value_type>::result_type> &&
             CopyConstructible<InIter::value_type> && CopyAssignable<InIter::value_type> 
    OutIter adjacent_difference(InIter @\farg{first}@, InIter @\farg{last}@,
     			        OutIter @\farg{result}@);
  template <InputIterator InIter, OutputIterator<auto, InIter::value_type> OutIter, 
            Callable<auto, InIter::value_type, InIter::value_type> BinaryOperation>
    requires CopyAssignable<OutIter::reference, BinaryOperation::result_type> &&
             CopyConstructible<InIter::value_type> && CopyAssignable<InIter::value_type>
    OutIter adjacent_difference(InIter @\farg{first}@, InIter @\farg{last}@,
    			        OutIter @\farg{result}@,
    			        BinaryOperation @\farg{binary_op}@);
}
\end{codeblock}
\color{black}

\pnum
The requirements on the types of algorithms' arguments that are
described in the introduction to clause \ref{lib.algorithms}\ also
apply to the following algorithms.

\rSec2[lib.accumulate]{Accumulate}

\color{addclr}
\index{accumulate@\tcode{accumulate}}%
\begin{itemdecl}
template <InputIterator Iter, HasPlus<auto, Iter::reference> T>
  requires CopyAssignable<T, T::result_type>
  T accumulate(Iter @\farg{first}@, Iter @\farg{last}@, T @\farg{init}@);
template <InputIterator Iter, class T, Callable<auto, T, Iter::reference> BinaryOperation>
  requires CopyAssignable<T, BinaryOperation::result_type>
  T accumulate(Iter @\farg{first}@, Iter @\farg{last}@, T @\farg{init}@,
               BinaryOperation @\farg{binary_op}@);
\end{itemdecl}
\color{black}

\begin{itemdescr}
\pnum
\effects\ 
Computes its result by initializing the accumulator
\tcode{acc}\
with the initial value
\tcode{init}\
and then modifies it with
\tcode{acc = acc + *i}\
or
\tcode{acc = binary_op(acc, *i)}\
for every iterator
\tcode{i}\
in the range \range{first}{last}\
in order.%
\footnote{
\tcode{accumulate}\
is similar to the APL reduction operator and Common Lisp reduce function, but it avoids the
difficulty of defining the result of reduction on an empty sequence by always requiring an initial value.
}

\pnum
\requires\ 
\removedConcepts{T shall meet the requirements of CopyConstructible (20.1.3)
and Assignable (21.3) types.}
In the range
\crange{first}{last},
\tcode{binary_op}\
shall neither modify elements nor invalidate iterators or subranges.%
\footnote{The use of fully closed ranges is intentional
}
\end{itemdescr}

\rSec2[lib.inner.product]{Inner product}
\index{inner_product@\tcode{inner_product}}%

\color{addclr}
\begin{itemdecl}
template <InputIterator Iter1, InputIterator Iter2, class T>
  requires HasMultiply<Iter1::reference, Iter2::reference> &&
           HasPlus<T, HasMultiply<Iter1::reference, Iter2::reference>::result_type> &&
           CopyAssignable<
             T, 
             HasPlus<T, 
                     HasMultiply<Iter1::reference, Iter2::reference>::result_type>::result_type>
  @\textcolor{addclr}{}@T inner_product(Iter1 @\farg{first1}@, Iter1 @\farg{last1}@,
		  Iter2 @\farg{first2}@, T @\farg{init}@);
template <InputIterator Iter1, InputIterator Iter2, class T,
	    class BinaryOperation1, Callable<auto, Iter1::reference, Iter2::reference> BinaryOperation2>
  requires Callable<BinaryOperation1, T, BinaryOperation2::result_type> &&
           CopyAssignable<T, BinaryOperation1::result_type>
  T inner_product(Iter1 @\farg{first1}@, Iter1 @\farg{last1}@,
		  Iter2 @\farg{first2}@, T @\farg{init}@,
		  BinaryOperation1 @\farg{binary_op1}@,
		  BinaryOperation2 @\farg{binary_op2}@);
\end{itemdecl}
\color{black}

\begin{itemdescr}
\pnum
\effects\ 
Computes its result by initializing the accumulator
\tcode{acc}\
with the initial value
\tcode{init}\
and then modifying it with
\tcode{acc = acc + (*i1) * (*i2)}\
or
\tcode{acc = binary_op1(acc, binary_op2(*i1, *i2))}\
for every iterator
\tcode{i1}\
in the range \range{first}{last}\
and iterator
\tcode{i2}\
in the range
\range{first2}{first2 + (last - first)}
in order.

\pnum
\requires\ 
\removedConcepts{T shall meet the requirements of CopyConstructible (20.1.3)
and Assignable (21.3) types.}
In the ranges
\crange{first}{last}\
and
\crange{first2}{first2 + (last - first)}\
\tcode{binary_op1}\
and
\tcode{binary_op2}\
shall neither modify elements nor invalidate iterators or subranges.%
\footnote{The use of fully closed ranges is intentional
}
\end{itemdescr}

\rSec2[lib.partial.sum]{Partial sum}
\index{partial_sum@\tcode{partial_sum}}%
\color{addclr}
\begin{itemdecl}
template <InputIterator InIter, OutputIterator<auto, InIter::value_type> OutIter>
  requires HasPlus<InIter::value_type> && 
           CopyAssignable<InIter::value_type, HasPlus<InIter::value_type>::result_type> &&
           CopyConstructible<InIter::value_type>
  OutIter partial_sum(InIter @\farg{first}@, InIter @\farg{last}@,
	  	      OutIter @\farg{result}@);
template<InputIterator InIter, OutputIterator<auto, InIter::value_type> OutIter, 
         Callable<auto, InIter::value_type, InIter::value_type> BinaryOperation>
  requires CopyAssignable<InIter::value_type, BinaryOperation::result_type> && 
           CopyConstructible<InIter::value_type>
  OutIter partial_sum(InIter @\farg{first}@, InIter @\farg{last}@,
    		      OutIter @\farg{result}@, BinaryOperation @\farg{binary_op}@);
\end{itemdecl}
\color{black}

\begin{itemdescr}
\pnum
\effects\ 
Assigns to every element referred to by iterator
\tcode{i}\
in the range
\range{result}{result + (last - first)}
a value
correspondingly equal to

\begin{codeblock}
((...(*first + *(first + 1)) + ...) + *(first + (i - result)))
\end{codeblock}

or

\begin{codeblock}
binary_op(binary_op(...,
    binary_op(*first, *(first + 1)),...), *(first + (i - result)))
\end{codeblock}

\pnum
\returns\ 
\tcode{result + (last - first)}.

\pnum
\complexity\ 
Exactly
\tcode{(last - first) - 1}\
applications of
\tcode{binary_op}.

\pnum
\requires\ 
In the ranges
\crange{first}{last}\
and
\crange{result}{result + (last - first)}\
\tcode{binary_op}\
shall neither modify elements nor invalidate iterators or subranges.%
\footnote{The use of fully closed ranges is intentional.
}

\pnum
\notes\ 
\tcode{result}\
may be equal to
\tcode{first}.
\end{itemdescr}

\rSec2[lib.adjacent.difference]{Adjacent difference}

\index{adjacent_difference@\tcode{adjacent_difference}}%
\color{addclr}
\begin{itemdecl}
template <InputIterator InIter, OutputIterator<auto, InIter::value_type> OutIter>
  requires HasMinus<InIter::value_type, InIter::value_type> &&
           CopyAssignable<OutIter, HasMinus<InIter::value_type, InIter::value_type>::result_type> &&
           CopyConstructible<InIter::value_type> && CopyAssignable<InIter::value_type> 
  OutIter adjacent_difference(InIter @\farg{first}@, InIter @\farg{last}@,
   			      OutIter @\farg{result}@);
template <InputIterator InIter, OutputIterator<auto, InIter::value_type> OutIter, 
          Callable<auto, InIter::value_type, InIter::value_type> BinaryOperation>
  requires CopyAssignable<OutIter::reference, BinaryOperation::result_type> &&
           CopyConstructible<InIter::value_type> && CopyAssignable<InIter::value_type>
  OutIter adjacent_difference(InIter @\farg{first}@, InIter @\farg{last}@,
			      OutIter @\farg{result}@,
  			      BinaryOperation @\farg{binary_op}@);
\end{itemdecl}
\color{black}

\begin{itemdescr}
\pnum
\effects\ 
Assigns to every element referred to by iterator
\tcode{i}\
in the range
\range{result + 1}{result + (last - first)}
a value correspondingly equal to

\begin{codeblock}
*(first + (i - result)) - *(first + (i - result) - 1)
\end{codeblock}

or

\begin{codeblock}
binary_op(*(first + (i - result)), *(first + (i - result) - 1)).
\end{codeblock}

\tcode{result}
gets the value of
\tcode{*first}.

\pnum
\requires\ 
In the ranges
\crange{first}{last}\
and
\crange{result}{result + (last - first)},
\tcode{binary_op}\
shall neither modify elements nor invalidate iterators or subranges.%
\footnote{The use of fully closed ranges is intentional.
}

\pnum
\notes\ 
\tcode{result}\
may be equal to
\tcode{first}.

\pnum
\returns\ 
\tcode{result + (last - first)}.

\pnum
\complexity\ 
Exactly
\tcode{(last - first) - 1}\
applications of
\tcode{binary_op}.
\end{itemdescr}

\end{paras}

\bibliographystyle{plain}
\bibliography{local}

\end{document}

\setcounter{section}{4}
\rSec1[numarray]{Numeric arrays}

\rSec2[valarray.synopsis]{Header \tcode{<valarray>}\ synopsis}
\index{valarray@\tcode{<valarray>}}%
\begin{codeblock}
namespace std {
  template<@\changedCC{class}{Regular}@ T> class valarray;         // An array of type \tcode{T}
  class slice;                              // a BLAS-like slice out of an array
  template<@\changedCC{class}{Regular}@ T> class slice_array;
  class gslice;                             // a generalized slice out of an array
  template<@\changedCC{class}{Regular}@ T> class gslice_array;
  template<@\changedCC{class}{Regular}@ T> class mask_array;       // a masked array
  template<@\changedCC{class}{Regular}@ T> class indirect_array;   // an indirected array

  @\addedD{template<@\changedCC{class}{Regular}@ T> void swap(valarray<T>\&, valarray<T>\&);}@
  @\addedD{template<@\changedCC{class}{Regular}@ T> void swap(valarray<T>\&\&, valarray<T>\&);}@
  @\addedD{template<@\changedCC{class}{Regular}@ T> void swap(valarray<T>\&, valarray<T>\&\&);}@

  template<class T> 
    @\addedCC{requires HasMultiply<T> \&\& Convertible<T::result_type, T>}@
    valarray<T> operator* (const valarray<T>&, const valarray<T>&);
  template<class T> 
    @\addedCC{requires HasMultiply<T> \&\& Convertible<T::result_type, T>}@
    valarray<T> operator* (const valarray<T>&, const T&);
  template<class T> 
    @\addedCC{requires HasMultiply<T> \&\& Convertible<T::result_type, T>}@
    valarray<T> operator* (const T&, const valarray<T>&);

  template<class T> 
    @\addedCC{requires HasDivide<T> \&\& Convertible<T::result_type, T>}@
    valarray<T> operator/ (const valarray<T>&, const valarray<T>&);
  template<class T> 
    @\addedCC{requires HasDivide<T> \&\& Convertible<T::result_type, T>}@
    valarray<T> operator/ (const valarray<T>&, const T&);
  template<class T> 
    @\addedCC{requires HasDivide<T> \&\& Convertible<T::result_type, T>}@
    valarray<T> operator/ (const T&, const valarray<T>&);

  template<class T> 
    @\addedCC{requires HasModulus<T> \&\& Convertible<T::result_type, T>}@
    valarray<T> operator% (const valarray<T>&, const valarray<T>&);
  template<class T> 
    @\addedCC{requires HasModulus<T> \&\& Convertible<T::result_type, T>}@
    valarray<T> operator% (const valarray<T>&, const T&);
  template<class T> 
    @\addedCC{requires HasModulus<T> \&\& Convertible<T::result_type, T>}@
    valarray<T> operator% (const T&, const valarray<T>&);

  template<class T> 
    @\addedCC{requires HasPlus<T> \&\& Convertible<T::result_type, T>}@
    valarray<T> operator+ (const valarray<T>&, const valarray<T>&);
  template<class T> 
    @\addedCC{requires HasPlus<T> \&\& Convertible<T::result_type, T>}@
    valarray<T> operator+ (const valarray<T>&, const T&);
  template<class T> 
    @\addedCC{requires HasPlus<T> \&\& Convertible<T::result_type, T>}@
    valarray<T> operator+ (const T&, const valarray<T>&);

  template<class T> 
    @\addedCC{requires HasMinus<T> \&\& Convertible<T::result_type, T>}@
    valarray<T> operator- (const valarray<T>&, const valarray<T>&);
  template<class T> 
    @\addedCC{requires HasMinus<T> \&\& Convertible<T::result_type, T>}@
    valarray<T> operator- (const valarray<T>&, const T&);
  template<class T> 
    @\addedCC{requires HasMinus<T> \&\& Convertible<T::result_type, T>}@
    valarray<T> operator- (const T&, const valarray<T>&);

  template<class T> 
    @\addedCC{requires HasBitXor<T> \&\& Convertible<T::result_type, T>}@
    valarray<T> operator^ (const valarray<T>&, const valarray<T>&);
  template<class T> 
    @\addedCC{requires HasBitXor<T> \&\& Convertible<T::result_type, T>}@
    valarray<T> operator^ (const valarray<T>&, const T&);
  template<class T> 
    @\addedCC{requires HasBitXor<T> \&\& Convertible<T::result_type, T>}@
    valarray<T> operator^ (const T&, const valarray<T>&);

  template<class T> 
    @\addedCC{requires HasBitAnd<T> \&\& Convertible<T::result_type, T>}@
    valarray<T> operator& (const valarray<T>&, const valarray<T>&);
  template<class T> 
    @\addedCC{requires HasBitAnd<T> \&\& Convertible<T::result_type, T>}@
    valarray<T> operator& (const valarray<T>&, const T&);
  template<class T> 
    @\addedCC{requires HasBitAnd<T> \&\& Convertible<T::result_type, T>}@
    valarray<T> operator& (const T&, const valarray<T>&);

  template<class T> 
    @\addedCC{requires HasBitOr<T> \&\& Convertible<T::result_type, T>}@
    valarray<T> operator| (const valarray<T>&, const valarray<T>&);
  template<class T> 
    @\addedCC{requires HasBitOr<T> \&\& Convertible<T::result_type, T>}@
    valarray<T> operator| (const valarray<T>&, const T&);
  template<class T> 
    @\addedCC{requires HasBitOr<T> \&\& Convertible<T::result_type, T>}@
    valarray<T> operator| (const T&, const valarray<T>&);

  template<class T> 
    @\addedCC{requires HasLeftShift<T> \&\& Convertible<T::result_type, T>}@
    valarray<T> operator<<(const valarray<T>&, const valarray<T>&);
  template<class T> 
    @\addedCC{requires HasLeftShift<T> \&\& Convertible<T::result_type, T>}@
    valarray<T> operator<<(const valarray<T>&, const T&);
  template<class T> 
    @\addedCC{requires HasLeftShift<T> \&\& Convertible<T::result_type, T>}@
    valarray<T> operator<<(const T&, const valarray<T>&);

  template<class T> 
    @\addedCC{requires HasRightShift<T> \&\& Convertible<T::result_type, T>}@
    valarray<T> operator>>(const valarray<T>&, const valarray<T>&);
  template<class T> 
    @\addedCC{requires HasRightShift<T> \&\& Convertible<T::result_type, T>}@
    valarray<T> operator>>(const valarray<T>&, const T&);
  template<class T> 
    @\addedCC{requires HasRightShift<T> \&\& Convertible<T::result_type, T>}@
    valarray<T> operator>>(const T&, const valarray<T>&);

  template<class T> 
    @\addedCC{requires HasLogicalAnd<T> \&\& Convertible<T::result_type, T>}@
    valarray<bool> operator&&(const valarray<T>&, const valarray<T>&);
  template<class T> 
    @\addedCC{requires HasLogicalAnd<T> \&\& Convertible<T::result_type, T>}@
    valarray<bool> operator&&(const valarray<T>&, const T&);
  template<class T> 
    @\addedCC{requires HasLogicalAnd<T> \&\& Convertible<T::result_type, T>}@
    valarray<bool> operator&&(const T&, const valarray<T>&);

  template<class T> 
    @\addedCC{requires HasLogicalOr<T> \&\& Convertible<T::result_type, T>}@
    valarray<bool> operator||(const valarray<T>&, const valarray<T>&);
  template<class T> 
    @\addedCC{requires HasLogicalOr<T> \&\& Convertible<T::result_type, T>}@
    valarray<bool> operator||(const valarray<T>&, const T&);
  template<class T> 
    @\addedCC{requires HasLogicalOr<T> \&\& Convertible<T::result_type, T>}@
    valarray<bool> operator||(const T&, const valarray<T>&);

  template<class T>
    valarray<bool> operator==(const valarray<T>&, const valarray<T>&);
  template<class T> valarray<bool> operator==(const valarray<T>&, const T&);
  template<class T> valarray<bool> operator==(const T&, const valarray<T>&);
  template<class T>
    valarray<bool> operator!=(const valarray<T>&, const valarray<T>&);
  template<class T> valarray<bool> operator!=(const valarray<T>&, const T&);
  template<class T> valarray<bool> operator!=(const T&, const valarray<T>&);

  template<class T>
    valarray<bool> operator< (const valarray<T>&, const valarray<T>&);
  template<class T> valarray<bool> operator< (const valarray<T>&, const T&);
  template<class T> valarray<bool> operator< (const T&, const valarray<T>&);
  template<class T>
    valarray<bool> operator> (const valarray<T>&, const valarray<T>&);
  template<class T> valarray<bool> operator> (const valarray<T>&, const T&);
  template<class T> valarray<bool> operator> (const T&, const valarray<T>&);
  template<class T>
    valarray<bool> operator<=(const valarray<T>&, const valarray<T>&);
  template<class T> valarray<bool> operator<=(const valarray<T>&, const T&);
  template<class T> valarray<bool> operator<=(const T&, const valarray<T>&);
  template<class T>
    valarray<bool> operator>=(const valarray<T>&, const valarray<T>&);
  template<class T> valarray<bool> operator>=(const valarray<T>&, const T&);
  template<class T> valarray<bool> operator>=(const T&, const valarray<T>&);

  template<class T> valarray<T> abs  (const valarray<T>&);
  template<class T> valarray<T> acos (const valarray<T>&);
  template<class T> valarray<T> asin (const valarray<T>&);
  template<class T> valarray<T> atan (const valarray<T>&);

  template<class T> valarray<T> atan2
    (const valarray<T>&, const valarray<T>&);
  template<class T> valarray<T> atan2(const valarray<T>&, const T&);
  template<class T> valarray<T> atan2(const T&, const valarray<T>&);

  template<class T> valarray<T> cos  (const valarray<T>&);
  template<class T> valarray<T> cosh (const valarray<T>&);
  template<class T> valarray<T> exp  (const valarray<T>&);
  template<class T> valarray<T> log  (const valarray<T>&);
  template<class T> valarray<T> log10(const valarray<T>&);

  template<class T> valarray<T> pow(const valarray<T>&, const valarray<T>&);
  template<class T> valarray<T> pow(const valarray<T>&, const T&);
  template<class T> valarray<T> pow(const T&, const valarray<T>&);

  template<class T> valarray<T> sin  (const valarray<T>&);
  template<class T> valarray<T> sinh (const valarray<T>&);
  template<class T> valarray<T> sqrt (const valarray<T>&);
  template<class T> valarray<T> tan  (const valarray<T>&);
  template<class T> valarray<T> tanh (const valarray<T>&);
}
\end{codeblock}

\pnum
The header
\tcode{<valarray>}\
defines five
class templates
(\tcode{valarray},
\tcode{slice_array},
\tcode{gslice_array},
\tcode{mask_array},
and
\tcode{indirect_array}),
two classes (\tcode{slice}\
and
\tcode{gslice}),
and a series of related
function templates
for representing
and manipulating arrays of values.

\pnum
The
\tcode{valarray}\
array classes
are defined to be free of certain forms of aliasing, thus allowing
operations on these classes to be optimized.

\pnum
Any function returning a
\tcode{valarray<T>}\
is permitted to return an object of another type, provided all the
const member functions of
\tcode{valarray<T>}\
are also applicable to this type.
This return type shall not add
more than two levels of template nesting over the most deeply nested
argument type.%
\footnote{
Clause \ref{limits}\ recommends a minimum number of recursively nested template
instantiations.
This requirement thus indirectly suggests a minimum
allowable complexity for valarray expressions.
}

\pnum
Implementations introducing such replacement types shall provide
additional functions and operators as follows:
\begin{itemize}
\item
for every function taking a
\tcode{const valarray<T>\&},
identical functions taking the replacement types shall be added;
\item
for every function taking two
\tcode{const valarray<T>\&}\
arguments, identical functions taking every combination of
\tcode{const valarray<T>\&}\
and replacement types shall be added.
\end{itemize}

\pnum
In particular, an implementation shall allow a
\tcode{valarray<T>}\
to be constructed from such replacement types and shall allow assignments
and computed assignments of such types to
\tcode{valarray<T>},
\tcode{slice_array<T>},
\tcode{gslice_array<T>},
\tcode{mask_array<T>}
and
\tcode{indirect_array<T>}
objects.

\pnum
These library functions are permitted to throw a
\tcode{bad_alloc}\
(\ref{bad.alloc}) exception if there are not sufficient resources available
to carry out the operation.
Note that the exception is not mandated.

\rSec2[template.valarray]{Class template \tcode{valarray}}

\index{valarray@\tcode{valarray}}%
\begin{codeblock}
namespace std {
  template<class @\farg{T}@> class valarray {
  public:
    typedef T value_type;

    // \ref{valarray.cons} construct/destroy:
    valarray();
    explicit valarray(size_t);
    valarray(const T&, size_t);
    valarray(const T*, size_t);
    valarray(const valarray&);
    @\addedD{valarray(valarray\&\&);}@
    valarray(const slice_array<T>&);
    valarray(const gslice_array<T>&);
    valarray(const mask_array<T>&);
    valarray(const indirect_array<T>&);
   ~valarray();

    // \ref{valarray.assign} assignment:
    valarray<T>& operator=(const valarray<T>&);
    @\addedD{valarray<T>\& operator=(valarray<T>\&\&);}@
    valarray<T>& operator=(const T&);
    valarray<T>& operator=(const slice_array<T>&);
    valarray<T>& operator=(const gslice_array<T>&);
    valarray<T>& operator=(const mask_array<T>&);
    valarray<T>& operator=(const indirect_array<T>&);

    // \ref{valarray.access} element access:
    const T&          operator[](size_t) const;
    T&                operator[](size_t);

    // \ref{valarray.sub} subset operations:
    valarray<T>       operator[](slice) const;
    slice_array<T>    operator[](slice);
    valarray<T>       operator[](const gslice&) const;
    gslice_array<T>   operator[](const gslice&);
    valarray<T>       operator[](const valarray<bool>&) const;
    mask_array<T>     operator[](const valarray<bool>&);
    valarray<T>       operator[](const valarray<size_t>&) const;
    indirect_array<T> operator[](const valarray<size_t>&);

    // \ref{valarray.unary} unary operators:
    valarray<T> operator+() const;
    valarray<T> operator-() const;
    valarray<T> operator~() const;
    valarray<bool> operator!() const;

    // \ref{valarray.cassign} computed assignment:
    valarray<T>& operator*= (const T&);
    valarray<T>& operator/= (const T&);
    valarray<T>& operator%= (const T&);
    valarray<T>& operator+= (const T&);
    valarray<T>& operator-= (const T&);
    valarray<T>& operator^= (const T&);
    valarray<T>& operator&= (const T&);
    valarray<T>& operator|= (const T&);
    valarray<T>& operator<<=(const T&);
    valarray<T>& operator>>=(const T&);

    valarray<T>& operator*= (const valarray<T>&);
    valarray<T>& operator/= (const valarray<T>&);
    valarray<T>& operator%= (const valarray<T>&);
    valarray<T>& operator+= (const valarray<T>&);
    valarray<T>& operator-= (const valarray<T>&);
    valarray<T>& operator^= (const valarray<T>&);
    valarray<T>& operator|= (const valarray<T>&);
    valarray<T>& operator&= (const valarray<T>&);
    valarray<T>& operator<<=(const valarray<T>&);
    valarray<T>& operator>>=(const valarray<T>&);

    // \ref{valarray.members} member functions:
    @\addedD{void swap(valarray\&\&);}@

    size_t size() const;

    T    sum() const;
    T    min() const;
    T    max() const;

    valarray<T> shift (int) const;
    valarray<T> cshift(int) const;
    valarray<T> apply(T func(T)) const;
    valarray<T> apply(T func(const T&)) const;
    void resize(size_t sz, T c = T());
  };
}
\end{codeblock}

\pnum
The
class template
\tcode{valarray<\farg{T}>}\
is a
one-dimensional smart array, with elements numbered sequentially from zero.
It is a representation of the mathematical concept
of an ordered set of values.
The illusion of higher dimensionality
may be produced by the familiar idiom of computed indices, together
with the powerful subsetting capabilities provided
by the generalized subscript operators.%
\footnote{
The intent is to specify an array template that has the minimum functionality
necessary to address aliasing ambiguities and the proliferation of
temporaries.
Thus, the
\tcode{valarray}
template is neither a
matrix class nor a field class.
However, it is a very useful building block for designing such classes.
}

\pnum
An implementation is permitted to qualify any of the functions declared in
\tcode{<valarray>}\
as
\tcode{inline}.

\rSec3[valarray.cons]{\tcode{valarray}\ constructors}

\index{valarray@\tcode{valarray}!\tcode{valarray}}%
\begin{itemdecl}
valarray();
\end{itemdecl}

\begin{itemdescr}
\pnum
\effects\ 
Constructs an object of class
\tcode{valarray<\farg{T}>},%
\footnote{
For convenience, such objects are referred to as ``arrays'' throughout the
remainder of \ref{numarray}.
}
which has zero length until it is passed into a library function as a
modifiable lvalue or through a non-constant
\tcode{this}\
pointer.%
\footnote{
This default constructor is essential,
since arrays of
\tcode{valarray}\
are likely to prove useful.
There shall also be a way to change the size of an array after
initialization; this is supplied by the semantics of the
\tcode{resize}\
member function.
}
\end{itemdescr}

\begin{itemdecl}
explicit valarray(size_t);
\end{itemdecl}

\begin{itemdescr}
\pnum
The array created by this constructor has a length equal to the value of the argument.
The elements of the array are constructed using the default constructor for the
instantiating type \farg{T}.
\end{itemdescr}

\begin{itemdecl}
valarray(const T&, size_t);
\end{itemdecl}

\begin{itemdescr}
\pnum
The array created by this constructor has a length equal to the second
argument.
The elements of the array are initialized with the value of the first argument.
\end{itemdescr}

\begin{itemdecl}
valarray(const T*, size_t);
\end{itemdecl}

\begin{itemdescr}
\pnum
The array created by this constructor has a length equal to the second
argument
\tcode{n}.
The values of the elements of the array are initialized with the
first
\tcode{n}\
values pointed to by the first argument.%
\footnote{
This constructor is the preferred method for converting a C array to a
\tcode{valarray}
object.
}
If the value of the second argument is greater than the number of values
pointed to by the first argument, the behavior is undefined.%
\index{undefined}
\end{itemdescr}

\begin{itemdecl}
valarray(const valarray<T>&);
\end{itemdecl}

\begin{itemdescr}
\pnum
The array created by this constructor has the same length as the argument
array.
The elements are initialized with the values of the corresponding
elements of the argument array.%
\footnote{
This copy constructor creates a distinct array rather than an alias.
Implementations in which arrays share storage are permitted, but they
shall implement a copy-on-reference mechanism to ensure that arrays are
conceptually distinct.
}
\end{itemdescr}

\begin{itemdecl}
@\addedD{valarray(valarray<T>\&\&);}@
\end{itemdecl}

\begin{itemdescr}
\pnum
\addedD{The array created by this constructor has the same length as the argument
array.
The elements are initialized with the values of the corresponding
elements of the argument array. After construction,
\mbox{\tcode{v}} is in a valid but unspecified state.}

\pnum
\addedD{\mbox{\complexity} Constant.}

\pnum
\addedD{\mbox{\throws} Nothing.}
\end{itemdescr}

\begin{itemdecl}
valarray(const slice_array<T>&);
valarray(const gslice_array<T>&);
valarray(const mask_array<T>&);
valarray(const indirect_array<T>&);
\end{itemdecl}

\begin{itemdescr}
\pnum
These conversion constructors convert one of the four reference templates
to a
\tcode{valarray}.
\end{itemdescr}

\index{\~valarray@\tcode{$\sim$valarray}!valarray@\tcode{valarray}}%
\begin{itemdecl}
~valarray();
\end{itemdecl}

\begin{itemdescr}
\pnum
The destructor is applied to every element of
\tcode{*this};
an implementation may return all allocated memory.
\end{itemdescr}

\rSec3[valarray.assign]{\tcode{valarray}\ assignment}

\index{operator=@\tcode{operator=}!\tcode{valarray}}%
\begin{itemdecl}
valarray<T>& operator=(const valarray<T>&);
\end{itemdecl}

\begin{itemdescr}
\pnum
Each element of the
\tcode{*this}\
array is assigned the value of the corresponding element of the argument
array.
The resulting behavior is undefined if the length of the argument
array is not equal to the length of the
\tcode{*this}\
array.
\end{itemdescr}

\index{operator=@\tcode{operator=}!\tcode{valarray}}%
\begin{itemdecl}
@\addedD{valarray<T>\& operator=(valarray<T>\&\&);}@
\end{itemdecl}

\begin{itemdescr}
\pnum
\addedD{\mbox{\effects} \mbox{\tcode{*this}} obtains the value of \mbox{\tcode{v}}.
After the assignment, \mbox{\tcode{v}} is in a valid but unspecified state.}

\pnum
\addedD{\mbox{\complexity} Constant.}

\pnum
\addedD{\mbox{\throws} Nothing.}
\end{itemdescr}

\index{operator=@\tcode{operator=}!\tcode{valarray}}%
\begin{itemdecl}
valarray<T>& operator=(const T&);
\end{itemdecl}

\begin{itemdescr}
\pnum
The scalar assignment operator causes each element of the
\tcode{*this}\
array to be assigned the value of the argument.
\end{itemdescr}

\begin{itemdecl}
valarray<T>& operator=(const slice_array<T>&);
valarray<T>& operator=(const gslice_array<T>&);
valarray<T>& operator=(const mask_array<T>&);
valarray<T>& operator=(const indirect_array<T>&);
\end{itemdecl}

\begin{itemdescr}
\pnum
These operators allow the results of a generalized subscripting operation
to be assigned directly to a
\tcode{valarray}.

\pnum
If the value of an element in the left-hand side of a valarray assignment
operator depends on the value of another element in that left-hand side,
the resulting behavior is undefined.
\end{itemdescr}

\rSec3[valarray.access]{\tcode{valarray}\ element access}

\index{operator[]@\tcode{operator[]}!valarray@\tcode{valarray}}%
\begin{itemdecl}
const T&  operator[](size_t) const;
T& operator[](size_t);
\end{itemdecl}

\begin{itemdescr}
\pnum
When applied to a constant array, the subscript operator returns the value
of the corresponding element of the array.
When applied to a non-constant array, the subscript operator
returns a reference to the corresponding element of the array.

\pnum
Thus, the expression
\tcode{(a[i] = q, a[i]) == q}\
evaluates as true for any non-constant
\tcode{valarray<T> a},
any
\tcode{T q},
and for any
\tcode{size_t i}\
such that the value of
\tcode{i}\
is less than the length of
\tcode{a}.

\pnum
The expression
\tcode{\&a[i+j] == \&a[i] + j}\
evaluates as true for all
\tcode{size_t i}\
and
\tcode{size_t j}\
such that
\tcode{i+j}\
is less than the length of the non-constant
array
\tcode{a}.

\pnum
Likewise, the expression
\tcode{\&a[i] != \&b[j]}\
evaluates as
\tcode{true}\
for any two non-constant arrays
\tcode{a}\
and
\tcode{b}\
and for any
\tcode{size_t i}\
and
\tcode{size_t j}\
such that
\tcode{i}\
is less than the length of
\tcode{a}\
and
\tcode{j}\
is less than the length of
\tcode{b}.
This property indicates an absence of aliasing and may be used to
advantage by optimizing compilers.%
\footnote{
Compilers may take advantage of inlining, constant propagation, loop fusion,
tracking of pointers obtained from
\tcode{operator new},
and other
techniques to generate efficient
\tcode{valarray}s.
}

\pnum
The reference returned by the subscript operator for a non-constant array is
guaranteed to be valid until the member function
\tcode{resize(size_t, T)}\
(\ref{valarray.members}) is called for that array or until the lifetime of
that array ends, whichever happens first.

\pnum
If the subscript operator
is invoked with a
\tcode{size_t}\
argument whose value is not
less than the length of the array, the behavior is undefined.%
\index{undefined}
\end{itemdescr}

\rSec3[valarray.sub]{\tcode{valarray}\ subset operations}

\index{operator[]@\tcode{operator[]}!valarray@\tcode{valarray}}%
\begin{itemdecl}
valarray<T>       operator[](slice) const;
slice_array<T>    operator[](slice);
valarray<T>       operator[](const gslice&) const;
gslice_array<T>   operator[](const gslice&);
valarray<T>       operator[](const valarray<bool>&) const;
mask_array<T>     operator[](const valarray<bool>&);
valarray<T>       operator[](const valarray<size_t>&) const;
indirect_array<T> operator[](const valarray<size_t>&);
\end{itemdecl}

\begin{itemdescr}
\pnum
Each of these operations returns a subset of the array.
The
\tcode{const}-qualified
versions return this subset as a new
\tcode{valarray}.
The
non-\tcode{const}
versions return a
class template object which has reference semantics to the original array.
\end{itemdescr}

\rSec3[valarray.unary]{\tcode{valarray}\ unary operators}

\index{operator+@\tcode{operator+}!\tcode{valarray}}%
\index{operator-@\tcode{operator-}!\tcode{valarray}}%
\index{operator\~{}@\tcode{operator\~{}}!\tcode{valarray}}%
\index{operator"!@\tcode{operator"!}!\tcode{valarray}}%
\begin{itemdecl}
valarray<T> operator+() const;
valarray<T> operator-() const;
valarray<T> operator~() const;
valarray<bool> operator!() const;
\end{itemdecl}

\begin{itemdescr}
\pnum
Each of these operators may only be instantiated for a type \farg{T}\ 
to which the indicated operator can be applied and for which the indicated
operator returns a value which is of type \farg{T}\ (\farg{bool}\ for
\farg{operator!}) or which may be unambiguously converted to type
\farg{T}\ (\farg{bool} for \farg{operator!}).

\pnum
Each of these operators returns an array whose length is equal to the length
of the array.
Each element of the returned array is initialized with the result of
applying the indicated operator to the corresponding element of the array.
\end{itemdescr}

\rSec3[valarray.cassign]{\tcode{valarray}\ computed assignment}

\index{operator*=@\tcode{operator*=}!valarray@\tcode{valarray}}%
\index{operator/=@\tcode{operator/=}!valarray@\tcode{valarray}}%
\index{operator\%=@\tcode{operator\%=}!valarray@\tcode{valarray}}%
\index{operator+=@\tcode{operator+=}!valarray@\tcode{valarray}}%
\index{operator-=@\tcode{operator-=}!valarray@\tcode{valarray}}%
\index{operator\^{}=@\tcode{operator\^{}=}!valarray@\tcode{valarray}}%
\index{operator\&=@\tcode{operator\&=}!valarray@\tcode{valarray}}%
\index{operator"|=@\tcode{operator"|=}!valarray@\tcode{valarray}}%
\index{operator<<=@\tcode{operator\shl=}!valarray@\tcode{valarray}}%
\index{operator>>=@\tcode{operator\shr=}!valarray@\tcode{valarray}}%
\begin{itemdecl}
valarray<T>& operator*= (const valarray<T>&);
valarray<T>& operator/= (const valarray<T>&);
valarray<T>& operator%= (const valarray<T>&);
valarray<T>& operator+= (const valarray<T>&);
valarray<T>& operator-= (const valarray<T>&);
valarray<T>& operator^= (const valarray<T>&);
valarray<T>& operator&= (const valarray<T>&);
valarray<T>& operator|= (const valarray<T>&);
valarray<T>& operator<<=(const valarray<T>&);
valarray<T>& operator>>=(const valarray<T>&);
\end{itemdecl}

\begin{itemdescr}
\pnum
Each of these operators may only be instantiated for a type \farg{T}\ 
to which the indicated operator can be applied.
Each of these operators
performs the indicated operation on each of its elements and the
corresponding element of the argument array.

\pnum
The array is then returned by reference.

\pnum
If the array and the
argument array do not have the same length, the behavior is undefined.%
\index{undefined}\
The appearance of an array on the left-hand side of a computed assignment
does \farg{not}\ invalidate references or pointers.

\pnum
If the value of an element in the left-hand side of a valarray computed
assignment operator depends on the value of another element in that left
hand side, the resulting behavior is undefined.
\end{itemdescr}

\begin{itemdecl}
valarray<T>& operator*= (const T&);
valarray<T>& operator/= (const T&);
valarray<T>& operator%= (const T&);
valarray<T>& operator+= (const T&);
valarray<T>& operator-= (const T&);
valarray<T>& operator^= (const T&);
valarray<T>& operator&= (const T&);
valarray<T>& operator|= (const T&);
valarray<T>& operator<<=(const T&);
valarray<T>& operator>>=(const T&);
\end{itemdecl}

\begin{itemdescr}
\pnum
Each of these operators may only be instantiated for a type \farg{T}\ 
to which the indicated operator can be applied.

\pnum
Each of these operators applies the indicated operation to each element
of the array and the non-array argument.

\pnum
The array is then returned by reference.

\pnum
The appearance of an array on the left-hand side of a computed assignment
does
\textit{not}\ 
invalidate references or pointers to the elements of the array.
\end{itemdescr}

\rSec3[valarray.members]{\tcode{valarray}\ member functions}

\begin{itemdecl}
@\addedD{void swap(valarray\&\& v);}@
\end{itemdecl}

\begin{itemdescr}
\pnum
\addedD{\mbox{\effects} \mbox{\tcode{*this}} obtains the value of
\mbox{\tcode{v}}. \mbox{\tcode{v}} obtains the value of \mbox{\tcode{*this}}.}

\pnum
\addedD{\mbox{\complexity} Constant.}

\pnum
\addedD{\mbox{\throws} Nothing.}
\end{itemdescr}

\index{length@\tcode{length}!valarray@\tcode{valarray}}%
\begin{itemdecl}
size_t size() const;
\end{itemdecl}

\begin{itemdescr}
\pnum
This function returns the number of elements in the array.
\end{itemdescr}

\index{sum@\tcode{sum}!valarray@\tcode{valarray}}%
\begin{itemdecl}
T sum() const;
\end{itemdecl}

\begin{itemdescr}
This function may only be instantiated for a type \farg{T}\ to which
\tcode{operator+=}\
can be applied.
This function returns the sum of all the elements of the array.

\pnum
If the array has length 0, the behavior is undefined.%
\index{undefined}
If the array has length 1,
\tcode{sum()}\
returns the value of element 0.
Otherwise, the returned value is calculated by applying
\tcode{operator+=}\
to a copy of an element of the array and
all other elements of the array in an unspecified order.%
\index{unspecified behavior}
\end{itemdescr}

\index{min@\tcode{min}!valarray@\tcode{valarray}}%
\begin{itemdecl}
T min() const;
\end{itemdecl}

\begin{itemdescr}
\pnum
This function returns the minimum value contained in
\tcode{*this}.
The value returned for an array of length 0 is undefined.
For an array
of length 1, the value of element 0 is returned.
For all other array
lengths, the determination is made using
\tcode{operator<}.
\end{itemdescr}

\index{max@\tcode{max}!valarray@\tcode{valarray}}%
\begin{itemdecl}
T max() const;
\end{itemdecl}

\begin{itemdescr}
\pnum
This function returns the maximum value contained in
\tcode{*this}.
The value returned for an array of length 0 is undefined.
For an array
of length 1, the value of element 0 is returned.
For all other array
lengths, the determination is made using
\tcode{operator<}.
\end{itemdescr}

\index{shift@\tcode{shift}!valarray@\tcode{valarray}}%
\begin{itemdecl}
valarray<T> shift(int @\farg{n}@) const;
\end{itemdecl}

\begin{itemdescr}
\pnum
This function returns an object of class
\tcode{valarray<T>}\
of length
\tcode{size()},
each of whose elements
\textit{I}\ 
is
\tcode{(*this)[\farg{I}\ + \farg{n}]}
if
\tcode{\farg{I}\ + \farg{n}}
is non-negative and less than
\tcode{size()},
otherwise
\tcode{T()}.
Thus if element zero is taken as the leftmost element,
a positive value of \farg{n}\ shifts the elements left \farg{n}\
places, with zero fill.

\pnum
\enterexample\ 
If the argument has the value -2,
the first two elements of the result will be constructed using the default
constructor; the third element of the result will be assigned the value
of the first element of the argument; etc.
\exitexample\ 
\index{cshift@\tcode{cshift}!valarray@\tcode{valarray}}%
\begin{itemdecl}
valarray<T> cshift(int @\farg{n}@) const;
\end{itemdecl}

\pnum
This function returns an object of class
\tcode{valarray<T>},
of length
\tcode{size()},
each of whose elements
\textit{I}\ 
is
\tcode{(*this)[(\farg{I}\ + \farg{n}) \% size()]}.
Thus, if element zero is taken as the leftmost element,
a positive value of \farg{n}\ shifts the elements circularly
left \farg{n}\ places.
\end{itemdescr}

\index{apply@\tcode{apply}!valarray@\tcode{valarray}}%
\begin{itemdecl}
valarray<T> apply(T func(T)) const;
valarray<T> apply(T func(const T&)) const;
\end{itemdecl}

\begin{itemdescr}
\pnum
These functions return an array whose length is equal to the array.
Each element of the returned array is assigned
the value returned by applying the argument function to the
corresponding element of the array.

\index{resize@\tcode{resize}!valarray@\tcode{valarray}}%
\begin{itemdecl}
void resize(size_t sz, T c = T());
\end{itemdecl}

\pnum
This member function changes the length of the
\tcode{*this}\
array to
\tcode{sz}\
and then assigns to each element the value of the second argument.
Resizing invalidates all pointers and references to elements in the array.
\end{itemdescr}

\rSec2[valarray.nonmembers]{\tcode{valarray} non-member operations}

\rSec3[valarray.binary]{\tcode{valarray}\ binary operators}

\index{operator*@\tcode{operator*}!valarray@\tcode{valarray}}%
\index{operator/@\tcode{operator/}!valarray@\tcode{valarray}}%
\index{operator\%@\tcode{operator\%}!valarray@\tcode{valarray}}%
\index{operator+@\tcode{operator+}!valarray@\tcode{valarray}}%
\index{operator-@\tcode{operator-}!valarray@\tcode{valarray}}%
\index{operator\^{}@\tcode{operator\^{}}!valarray@\tcode{valarray}}%
\index{operator\&@\tcode{operator\&}!valarray@\tcode{valarray}}%
\index{operator"|@\tcode{operator"|}!valarray@\tcode{valarray}}%
\index{operator<<@\tcode{operator\shl}!valarray@\tcode{valarray}}%
\index{operator>>@\tcode{operator\shr}!valarray@\tcode{valarray}}%
\index{operator\&\&@\tcode{operator\&\&}!valarray@\tcode{valarray}}%
\index{operator||@\tcode{operator||}!valarray@\tcode{valarray}}%
\begin{itemdecl}
template<class T> valarray<T> operator*
    (const valarray<T>&, const valarray<T>&);
template<class T> valarray<T> operator/
    (const valarray<T>&, const valarray<T>&);
template<class T> valarray<T> operator%
    (const valarray<T>&, const valarray<T>&);
template<class T> valarray<T> operator+
    (const valarray<T>&, const valarray<T>&);
template<class T> valarray<T> operator-
    (const valarray<T>&, const valarray<T>&);
template<class T> valarray<T> operator^
    (const valarray<T>&, const valarray<T>&);
template<class T> valarray<T> operator&
    (const valarray<T>&, const valarray<T>&);
template<class T> valarray<T> operator|
    (const valarray<T>&, const valarray<T>&);
template<class T> valarray<T> operator<<
    (const valarray<T>&, const valarray<T>&);
template<class T> valarray<T> operator>>
    (const valarray<T>&, const valarray<T>&);
\end{itemdecl}

\begin{itemdescr}
\pnum
Each of these operators may only be instantiated for a type \farg{T}\ 
to which the indicated operator can be applied and for which the indicated
operator returns a value which is of type \farg{T}\ or which
can be unambiguously converted to type \farg{T}.

\pnum
Each of these operators returns an array whose length is equal to the
lengths of the argument arrays.
Each element of the returned array is
initialized with the result of applying the indicated operator to the
corresponding elements of the argument arrays.

\pnum
If the argument arrays do not have the same length, the behavior is undefined.%
\index{undefined}
\end{itemdescr}

\begin{itemdecl}
template<class T> valarray<T> operator* (const valarray<T>&, const T&);
template<class T> valarray<T> operator* (const T&, const valarray<T>&);
template<class T> valarray<T> operator/ (const valarray<T>&, const T&);
template<class T> valarray<T> operator/ (const T&, const valarray<T>&);
template<class T> valarray<T> operator% (const valarray<T>&, const T&);
template<class T> valarray<T> operator% (const T&, const valarray<T>&);
template<class T> valarray<T> operator+ (const valarray<T>&, const T&);
template<class T> valarray<T> operator+ (const T&, const valarray<T>&);
template<class T> valarray<T> operator- (const valarray<T>&, const T&);
template<class T> valarray<T> operator- (const T&, const valarray<T>&);
template<class T> valarray<T> operator^ (const valarray<T>&, const T&);
template<class T> valarray<T> operator^ (const T&, const valarray<T>&);
template<class T> valarray<T> operator& (const valarray<T>&, const T&);
template<class T> valarray<T> operator& (const T&, const valarray<T>&);
template<class T> valarray<T> operator| (const valarray<T>&, const T&);
template<class T> valarray<T> operator| (const T&, const valarray<T>&);
template<class T> valarray<T> operator<<(const valarray<T>&, const T&);
template<class T> valarray<T> operator<<(const T&, const valarray<T>&);
template<class T> valarray<T> operator>>(const valarray<T>&, const T&);
template<class T> valarray<T> operator>>(const T&, const valarray<T>&);
\end{itemdecl}

\begin{itemdescr}
\pnum
Each of these operators may only be instantiated for a type \farg{T}\ 
to which the indicated operator can be applied and for which
the indicated operator returns a value which is of type \farg{T}\ 
or which can be unambiguously converted to type \farg{T}.

\pnum
Each of these operators returns an array whose length is equal to the
length of the array argument.
Each element of the returned array is
initialized with the result of applying the indicated operator to the
corresponding element of the array argument and the non-array argument.
\end{itemdescr}

\rSec3[valarray.comparison]{\tcode{valarray}\ logical operators}

\index{operator==@\tcode{operator==}!valarray@\tcode{valarray}}%
\index{operator"!=@\tcode{operator"!=}!valarray@\tcode{valarray}}%
\index{operator<@\tcode{operator<}!valarray@\tcode{valarray}}%
\index{operator>@\tcode{operator>}!valarray@\tcode{valarray}}%
\index{operator<=@\tcode{operator<=}!valarray@\tcode{valarray}}%
\index{operator>=@\tcode{operator>=}!valarray@\tcode{valarray}}%
\index{operator\&\&@\tcode{operator\&\&}!valarray@\tcode{valarray}}%
\index{operator||@\tcode{operator||}!valarray@\tcode{valarray}}%
\begin{itemdecl}
template<class T> valarray<bool> operator==
    (const valarray<T>&, const valarray<T>&);
template<class T> valarray<bool> operator!=
    (const valarray<T>&, const valarray<T>&);
template<class T> valarray<bool> operator<
    (const valarray<T>&, const valarray<T>&);
template<class T> valarray<bool> operator>
    (const valarray<T>&, const valarray<T>&);
template<class T> valarray<bool> operator<=
    (const valarray<T>&, const valarray<T>&);
template<class T> valarray<bool> operator>=
    (const valarray<T>&, const valarray<T>&);
template<class T> valarray<bool> operator&&
    (const valarray<T>&, const valarray<T>&);
template<class T> valarray<bool> operator||
    (const valarray<T>&, const valarray<T>&);
\end{itemdecl}

\begin{itemdescr}
\pnum
Each of these operators may only be instantiated for a type \farg{T}\ 
to which the indicated operator can be applied and for which
the indicated operator returns a value which is of type \farg{bool}\ 
or which can be unambiguously converted to type \farg{bool}.

\pnum
Each of these operators returns a \farg{bool}\ array whose length
is equal to the length of the array arguments.
Each element of the returned
array is initialized with the result of applying the indicated
operator to the corresponding elements of the argument arrays.

\pnum
If the two array arguments do not have the same length,
the behavior is undefined.%
\index{undefined}
\end{itemdescr}

\begin{itemdecl}
template<class T> valarray<bool> operator==(const valarray<T>&, const T&);
template<class T> valarray<bool> operator==(const T&, const valarray<T>&);
template<class T> valarray<bool> operator!=(const valarray<T>&, const T&);
template<class T> valarray<bool> operator!=(const T&, const valarray<T>&);
template<class T> valarray<bool> operator< (const valarray<T>&, const T&);
template<class T> valarray<bool> operator< (const T&, const valarray<T>&);
template<class T> valarray<bool> operator> (const valarray<T>&, const T&);
template<class T> valarray<bool> operator> (const T&, const valarray<T>&);
template<class T> valarray<bool> operator<=(const valarray<T>&, const T&);
template<class T> valarray<bool> operator<=(const T&, const valarray<T>&);
template<class T> valarray<bool> operator>=(const valarray<T>&, const T&);
template<class T> valarray<bool> operator>=(const T&, const valarray<T>&);
template<class T> valarray<bool> operator&&(const valarray<T>&, const T&);
template<class T> valarray<bool> operator&&(const T&, const valarray<T>&);
template<class T> valarray<bool> operator||(const valarray<T>&, const T&);
template<class T> valarray<bool> operator||(const T&, const valarray<T>&);
\end{itemdecl}

\begin{itemdescr}
\pnum
Each of these operators may only be instantiated for a type \farg{T}\ 
to which the indicated operator can be applied and for which
the indicated operator returns a value which is of type \farg{bool}\ 
or which can be unambiguously converted to type \farg{bool}.

\pnum
Each of these operators returns a \farg{bool}\ array whose
length is equal to the length of the array argument.
Each element
of the returned array is initialized with the result of applying the
indicated operator to the corresponding element of the array and the non-array argument.
\end{itemdescr}

\rSec3[valarray.transcend]{\tcode{valarray}\ transcendentals}

\index{abs@\tcode{abs}}%
\index{acos@\tcode{acos}}%
\index{asin@\tcode{asin}}%
\index{atan@\tcode{atan}}%
\index{atan2@\tcode{atan2}}%
\index{cos@\tcode{cos}}%
\index{cosh@\tcode{cosh}}%
\index{exp@\tcode{exp}}%
\index{log@\tcode{log}}%
\index{log10@\tcode{log10}}%
\index{pow@\tcode{pow}}%
\index{sin@\tcode{sin}}%
\index{sinh@\tcode{sinh}}%
\index{sqrt@\tcode{sqrt}}%
\index{tan@\tcode{tan}}%
\index{tanh@\tcode{tanh}}%
\begin{itemdecl}
template<class T> valarray<T> abs  (const valarray<T>&);
template<class T> valarray<T> acos (const valarray<T>&);
template<class T> valarray<T> asin (const valarray<T>&);
template<class T> valarray<T> atan (const valarray<T>&);
template<class T> valarray<T> atan2
    (const valarray<T>&, const valarray<T>&);
template<class T> valarray<T> atan2(const valarray<T>&, const T&);
template<class T> valarray<T> atan2(const T&, const valarray<T>&);
template<class T> valarray<T> cos  (const valarray<T>&);
template<class T> valarray<T> cosh (const valarray<T>&);
template<class T> valarray<T> exp  (const valarray<T>&);
template<class T> valarray<T> log  (const valarray<T>&);
template<class T> valarray<T> log10(const valarray<T>&);
template<class T> valarray<T> pow
    (const valarray<T>&, const valarray<T>&);
template<class T> valarray<T> pow  (const valarray<T>&, const T&);
template<class T> valarray<T> pow  (const T&, const valarray<T>&);
template<class T> valarray<T> sin  (const valarray<T>&);
template<class T> valarray<T> sinh (const valarray<T>&);
template<class T> valarray<T> sqrt (const valarray<T>&);
template<class T> valarray<T> tan  (const valarray<T>&);
template<class T> valarray<T> tanh (const valarray<T>&);
\end{itemdecl}

\begin{itemdescr}
\pnum
Each of these functions may only be instantiated for a type \farg{T}\ 
to which a unique function with the indicated name can be applied (unqualified).
This function shall return a value which is of type \farg{T}\ 
or which can be unambiguously converted to type \farg{T}.
\end{itemdescr}

\rSec3[valarray.special]{\tcode{valarray}\ specialized algorithms}

\begin{itemdecl}
@\addedD{template <class T> void swap(valarray<T>\& x, valarray<T>\& y);}@
@\addedD{template <class T> void swap(valarray<T>\&\& x, valarray<T>\& y);}@
@\addedD{template <class T> void swap(valarray<T>\& x, valarray<T>\&\& y);}@
\end{itemdecl}

\begin{itemdescr}
\pnum
\addedD{\mbox{\effects} \mbox{\tcode{x.swap(y)}}.}
\end{itemdescr}


\rSec2[class.slice]{Class \tcode{slice}}

\index{slice@\tcode{slice}}%
\begin{codeblock}
namespace std {
  class slice {
  public:
    slice();
    slice(size_t, size_t, size_t);

    size_t start() const;
    size_t size() const;
    size_t stride() const;
  };
}
\end{codeblock}

\pnum
The
\tcode{slice}\
class represents a BLAS-like slice from an array.
Such a slice is specified by a starting index, a length, and a stride.%
\footnote{
BLAS stands for
\textit{Basic Linear Algebra Subprograms.}\ 
\Cpp\ programs may instantiate this class.
See, for example,
Dongarra, Du Croz, Duff, and Hammerling:
\textit{A set of Level 3 Basic Linear Algebra Subprograms};
Technical Report MCS-P1-0888,
Argonne National Laboratory (USA),
Mathematics and Computer Science Division,
August, 1988.
}

\rSec3[cons.slice]{\tcode{slice}\ constructors}

\index{slice@\tcode{slice}!\tcode{slice}}%
\begin{itemdecl}
slice();
slice(size_t @\farg{start}@, size_t @\farg{length}@, size_t @\farg{stride}@);
slice(const slice&);
\end{itemdecl}

\begin{itemdescr}
\pnum
\changedD{The default constructor for
\mbox{\tcode{slice}}
creates a
\mbox{\tcode{slice}}
which specifies no elements.}
{The default constructor is equivalent to \mbox{\tcode{slice(0, 0, 0)}}.}
A default constructor is provided only to permit the declaration of arrays of slices.
The constructor with arguments for a slice takes a start, length, and stride
parameter.

\pnum
\enterexample\ 
\tcode{slice(3, 8, 2)}\
constructs a slice which selects elements 3, 5, 7, ... 17 from an array.
\exitexample\ 
\end{itemdescr}

\rSec3[slice.access]{\tcode{slice}\ access functions}
\index{start@\tcode{start}!\tcode{slice}}%
\index{size@\tcode{size}!\tcode{slice}}%
\index{stride@\tcode{stride}!\tcode{slice}}%
\begin{itemdecl}
size_t start() const;
size_t size() const;
size_t stride() const;
\end{itemdecl}

\begin{itemdescr}
\pnum
These functions return the start, length, or stride specified by a
\tcode{slice}\ object.
\end{itemdescr}

\rSec2[template.slice.array]{Class template \tcode{slice_array}}

\index{slice_array@\tcode{slice_array}}%
\begin{codeblock}
namespace std {
  template <class T> class slice_array {
  public:
    typedef T value_type;

    void operator=  (const valarray<T>&) const;
    void operator*= (const valarray<T>&) const;
    void operator/= (const valarray<T>&) const;
    void operator%= (const valarray<T>&) const;
    void operator+= (const valarray<T>&) const;
    void operator-= (const valarray<T>&) const;
    void operator^= (const valarray<T>&) const;
    void operator&= (const valarray<T>&) const;
    void operator|= (const valarray<T>&) const;
    void operator<<=(const valarray<T>&) const;
    void operator>>=(const valarray<T>&) const;

    slice_array(const slice_array&);
   ~slice_array();
    slice_array& operator=(const slice_array&);
	void operator=(const T&) const;
  private:
    slice_array();
  };
}
\end{codeblock}

\pnum
The
\tcode{slice_array}\
template is a helper template used by the
\tcode{slice}\
subscript operator

\begin{codeblock}
slice_array<T> valarray<T>::operator[](slice);
\end{codeblock}

It has reference semantics to a subset of an array specified by a
\tcode{slice}
object.

\pnum
\enterexample\ 
The expression
\tcode{a[slice(1, 5, 3)] = b;}\
has the effect of assigning the elements of
\tcode{b}\
to a slice of the elements in
\tcode{a}.
For the slice shown, the elements
selected from
\tcode{a}\
are 1, 4, ..., 13.
\exitexample\ 

\rSec3[cons.slice.arr]{\tcode{slice_array}\ constructors}

\index{slice_array@\tcode{slice_array}!\tcode{slice_array}}%
\begin{itemdecl}
slice_array();
\end{itemdecl}

\begin{itemdescr}
\pnum
This constructor is declared to be private.
This constructor need not be defined.
\end{itemdescr}

\rSec3[slice.arr.assign]{\tcode{slice_array}\ assignment}

\index{operator=@\tcode{operator=}!\tcode{slice_array}}%
\begin{itemdecl}
void         operator=(const valarray<T>&) const;
slice_array& operator=(const slice_array&);
\end{itemdecl}

\begin{itemdescr}
\pnum
These assignment operators have reference semantics,
assigning the values of the argument array elements to selected
elements of the
\tcode{valarray<T>}\
object to which the
\tcode{slice_array}\
object refers.
\end{itemdescr}

\rSec3[slice.arr.comp.assign]{\tcode{slice_array}\ computed assignment}

\index{operator*=@\tcode{operator*=}!\tcode{slice_array}}%
\index{operator/=@\tcode{operator/=}!\tcode{slice_array}}%
\index{operator\%=@\tcode{operator\%=}!\tcode{slice_array}}%
\index{operator+=@\tcode{operator+=}!\tcode{slice_array}}%
\index{operator-=@\tcode{operator-=}!\tcode{slice_array}}%
\index{operator\^{}=@\tcode{operator\^{}=}!\tcode{slice_array}}%
\index{operator\&=@\tcode{operator\&=}!\tcode{slice_array}}%
\index{operator"|=@\tcode{operator"|=}!\tcode{slice_array}}%
\index{operator<<=@\tcode{operator\shl=}!\tcode{slice_array}}%
\index{operator>>=@\tcode{operator\shr=}!\tcode{slice_array}}%
\begin{itemdecl}
void operator*= (const valarray<T>&) const;
void operator/= (const valarray<T>&) const;
void operator%= (const valarray<T>&) const;
void operator+= (const valarray<T>&) const;
void operator-= (const valarray<T>&) const;
void operator^= (const valarray<T>&) const;
void operator&= (const valarray<T>&) const;
void operator|= (const valarray<T>&) const;
void operator<<=(const valarray<T>&) const;
void operator>>=(const valarray<T>&) const;
\end{itemdecl}

\begin{itemdescr}
\pnum
These computed assignments have reference semantics, applying the
indicated operation to the elements of the argument array
and selected elements of the
\tcode{valarray<T>}\
object to which the
\tcode{slice_array}\
object refers.
\end{itemdescr}

\rSec3[slice.arr.fill]{\tcode{slice_array}\ fill function}

\index{fill@\tcode{fill}!\tcode{slice_array}}%
\begin{itemdecl}
void operator=(const T&) const;
\end{itemdecl}

\begin{itemdescr}
\pnum
This function has reference semantics, assigning the value of its argument
to the elements of the
\tcode{valarray<T>}\
object to which the
\tcode{slice_array}\
object refers.
\end{itemdescr}

\rSec2[class.gslice]{The \tcode{gslice}\ class}

\index{gslice@\tcode{gslice}!class}%
\begin{codeblock}
namespace std {
  class gslice {
  public:
    gslice();
    gslice(size_t s, const valarray<size_t>& l, const valarray<size_t>& d);

    size_t           start() const;
    valarray<size_t> size() const;
    valarray<size_t> stride() const;
  };
}
\end{codeblock}

\pnum
This class represents a generalized slice out of an array.
A
\tcode{gslice}\
is defined by a starting offset ($s$),
a set of lengths ($l_j$),
and a set of strides ($d_j$).
The number of lengths shall equal the number of strides.

\pnum
A
\tcode{gslice}\
represents a mapping from a set of indices ($i_j$),
equal in number to the number of strides, to a single index $k$.
It is useful for building multidimensional array classes using
the
\tcode{valarray}\
template, which is one-dimensional.
The set of one-dimensional index values specified by a
\tcode{gslice}\
are $$k = s + \sum_ji_jd_j$$
% \$k = s + sum from j \{ i sub j d sub j \}\$
where the multidimensional indices $i_j$ range in value from
0 to $l_{ij} - 1$.

\pnum
\enterexample\ 
The
\tcode{gslice}\
specification
\begin{codeblock}
start  = 3
length = {2, 4, 3}
stride = {19, 4, 1}

\end{codeblock}
yields the sequence of one-dimensional indices

$$k = 3 + (0,1) \times 19 + (0,1,2,3) \times 4 + (0,1,2) \times 1$$

which are ordered as shown in the following table:

\begin{tabbing}
\hspace{.5in}\=\hspace{.4in}\=\kill%
\>$(i_0,\quad i_1,\quad i_2,\quad k)\quad =$\\
\>\>$(0,\quad 0,\quad 0,\quad 3$),		\\
\>\>$(0,\quad 0,\quad 1,\quad 4$),		\\
\>\>$(0,\quad 0,\quad 2,\quad 5$),		\\
\>\>$(0,\quad 1,\quad 0,\quad 7$),		\\
\>\>$(0,\quad 1,\quad 1,\quad 8$),		\\
\>\>$(0,\quad 1,\quad 2,\quad 9$),		\\
\>\>$(0,\quad 2,\quad 0,\quad 11$),	\\
\>\>$(0,\quad 2,\quad 1,\quad 12$),	\\
\>\>$(0,\quad 2,\quad 2,\quad 13$),	\\
\>\>$(0,\quad 3,\quad 0,\quad 15$),	\\
\>\>$(0,\quad 3,\quad 1,\quad 16$),	\\
\>\>$(0,\quad 3,\quad 2,\quad 17$),	\\
\>\>$(1,\quad 0,\quad 0,\quad 22$),	\\
\>\>$(1,\quad 0,\quad 1,\quad 23$),	\\
\>\>$\ldots$			\\
\>\>$(1,\quad 3,\quad 2,\quad 36$)
\end{tabbing}

That is, the highest-ordered index turns fastest.
\exitexample\ 

\pnum
It is possible to have degenerate generalized slices in which an address
is repeated.

\pnum
\enterexample\ 
If the stride parameters in the previous
example are changed to \{1, 1, 1\}, the first few elements of the
resulting sequence of indices will be

\begin{tabbing}
\hspace{.9in}\=\kill%
\>$(0,\quad 0,\quad 0,\quad 3$),	\\
\>$(0,\quad 0,\quad 1,\quad 4$),	\\
\>$(0,\quad 0,\quad 2,\quad 5$),	\\
\>$(0,\quad 1,\quad 0,\quad 4$),	\\
\>$(0,\quad 1,\quad 1,\quad 5$),	\\
\>$(0,\quad 1,\quad 2,\quad 6$),	\\
\>$\ldots$
\end{tabbing}
\exitexample\ 

\pnum
If a degenerate slice is used as the argument to the
non-\tcode{const}
version of
\tcode{operator[](const gslice\&)},
the resulting behavior is undefined.
\index{undefined}%

\rSec3[gslice.cons]{\tcode{gslice}\ constructors}

\index{gslice@\tcode{gslice}!\tcode{gslice}}%
\begin{itemdecl}
gslice();
gslice(size_t @\farg{start}@, const valarray<size_t>& @\farg{lengths}@,
			   const valarray<size_t>& @\farg{strides}@);
gslice(const gslice&);
\end{itemdecl}

\begin{itemdescr}
\pnum
\changedD{The default constructor creates a
\mbox{\tcode{gslice}}
which specifies no elements.}
{The default constructor is equivalent to
\mbox{\tcode{gslice(0, valarray<size_t>(), valarray<size_t>())}}.}
The constructor with arguments builds a
\tcode{gslice}\
based on a specification of start, lengths, and strides, as explained
in the previous section.
\end{itemdescr}

\rSec3[gslice.access]{\tcode{gslice}\ access functions}

\index{start@\tcode{start}!\tcode{gslice}}%
\index{size@\tcode{size}!\tcode{gslice}}%
\index{stride@\tcode{stride}!\tcode{gslice}}%
\begin{itemdecl}
size_t           start()  const;
valarray<size_t> size() const;
valarray<size_t> stride() const;
\end{itemdecl}

\begin{itemdescr}
\pnum
These access functions return the representation of the start, lengths, or
strides specified for the
\tcode{gslice}.
\end{itemdescr}

\rSec2[template.gslice.array]{Class template \tcode{gslice_array}}

\index{gslice_array@\tcode{gslice_array}}%
\begin{codeblock}
namespace std {
  template <class T> class gslice_array {
  public:
    typedef T value_type;

    void operator=  (const valarray<T>&) const;
    void operator*= (const valarray<T>&) const;
    void operator/= (const valarray<T>&) const;
    void operator%= (const valarray<T>&) const;
    void operator+= (const valarray<T>&) const;
    void operator-= (const valarray<T>&) const;
    void operator^= (const valarray<T>&) const;
    void operator&= (const valarray<T>&) const;
    void operator|= (const valarray<T>&) const;
    void operator<<=(const valarray<T>&) const;
    void operator>>=(const valarray<T>&) const;

    gslice_array(const gslice_array&);
   ~gslice_array();
    gslice_array& operator=(const gslice_array&);
    void operator=(const T&) const;
  private:
    gslice_array();
  };
}
\end{codeblock}

\pnum
This template is a helper template used by the
\tcode{slice}\
subscript operator

\index{gslice_array@\tcode{gslice_array}}%
\index{valarray@\tcode{valarray}}%
\begin{itemdecl}
gslice_array<T> valarray<T>::operator[](const gslice&);
\end{itemdecl}

\begin{itemdescr}
\pnum
It has reference semantics to a subset of an array specified by a
\tcode{gslice}\
object.

\pnum
Thus, the expression
\tcode{a[gslice(1, length, stride)] = b}\
has the effect of assigning the elements of
\tcode{b}\
to a
generalized slice of the elements in
\tcode{a}.
\end{itemdescr}

\rSec3[gslice.array.cons]{\tcode{gslice_array}\ constructors}

\index{gslice_array@\tcode{gslice_array}!\tcode{gslice_array}}%
\begin{itemdecl}
gslice_array();
\end{itemdecl}

\begin{itemdescr}
\pnum
This constructor is declared to be private.
This constructor need not be defined.
\end{itemdescr}

\rSec3[gslice.array.assign]{\tcode{gslice_array}\ assignment}

\index{operator=@\tcode{operator=}!\tcode{gslice_array}}%
\begin{itemdecl}
void operator=(const valarray<T>&) const;
gslice_array& operator=(const gslice_array&);
\end{itemdecl}

\begin{itemdescr}
\pnum
These assignment operators have reference semantics, assigning the values
of the argument array elements to selected elements of the
\tcode{valarray<T>}\
object to which the
\tcode{gslice_array}\
refers.
\end{itemdescr}

\rSec3[gslice.array.comp.assign]{\tcode{gslice_array}}

\index{operator*=@\tcode{operator*=}!\tcode{gslice_array}}%
\index{operator/=@\tcode{operator/=}!\tcode{gslice_array}}%
\index{operator\%=@\tcode{operator\%=}!\tcode{gslice_array}}%
\index{operator+=@\tcode{operator+=}!\tcode{gslice_array}}%
\index{operator-=@\tcode{operator-=}!\tcode{gslice_array}}%
\index{operator\^{}=@\tcode{operator\^{}=}!\tcode{gslice_array}}%
\index{operator\&=@\tcode{operator\&=}!\tcode{gslice_array}}%
\index{operator"|=@\tcode{operator"|=}!\tcode{gslice_array}}%
\index{operator<<=@\tcode{operator\shl=}!\tcode{gslice_array}}%
\index{operator>>=@\tcode{operator\shr=}!\tcode{gslice_array}}%
\begin{itemdecl}
void operator*= (const valarray<T>&) const;
void operator/= (const valarray<T>&) const;
void operator%= (const valarray<T>&) const;
void operator+= (const valarray<T>&) const;
void operator-= (const valarray<T>&) const;
void operator^= (const valarray<T>&) const;
void operator&= (const valarray<T>&) const;
void operator|= (const valarray<T>&) const;
void operator<<=(const valarray<T>&) const;
void operator>>=(const valarray<T>&) const;
\end{itemdecl}

\begin{itemdescr}
\pnum
These computed assignments have reference semantics, applying the
indicated operation to the elements of the argument array and selected
elements of the
\tcode{valarray<T>}\
object to which the
\tcode{gslice_array}\
object refers.
\end{itemdescr}

\rSec3[gslice.array.fill]{\tcode{gslice_array}\ fill function}

\index{fill@\tcode{fill}!\tcode{gslice_array}}%
\begin{itemdecl}
void operator=(const T&) const;
\end{itemdecl}

\begin{itemdescr}
\pnum
This function has reference semantics, assigning the value of its argument
to the elements of the
\tcode{valarray<T>}\
object to which the
\tcode{gslice_array}\
object refers.
\end{itemdescr}

\rSec2[template.mask.array]{Class template \tcode{mask_array}}

\index{mask_array@\tcode{mask_array}}%
\begin{codeblock}
namespace std {
  template <class T> class mask_array {
  public:
    typedef T value_type;

    void operator=  (const valarray<T>&) const;
    void operator*= (const valarray<T>&) const;
    void operator/= (const valarray<T>&) const;
    void operator%= (const valarray<T>&) const;
    void operator+= (const valarray<T>&) const;
    void operator-= (const valarray<T>&) const;
    void operator^= (const valarray<T>&) const;
    void operator&= (const valarray<T>&) const;
    void operator|= (const valarray<T>&) const;
    void operator<<=(const valarray<T>&) const;
    void operator>>=(const valarray<T>&) const;

    mask_array(const mask_array&);
   ~mask_array();
    mask_array& operator=(const mask_array&);
    void operator=(const T&) const;
  private:
    mask_array();
  };
}
\end{codeblock}

\pnum
This template is a helper template used by the mask subscript operator:

\begin{itemdecl}
mask_array<T> valarray<T>::operator[](const valarray<bool>&).
\end{itemdecl}

\begin{itemdescr}
\pnum
It has reference semantics to a subset of an array specified by a boolean mask.
Thus, the expression
\tcode{a[mask] = b;}\
has the effect of assigning the elements of
\tcode{b}\
to the masked
elements in
\tcode{a}\
(those for which the corresponding element
in
\tcode{mask}\
is
\tcode{true}.)
\end{itemdescr}

\rSec3[mask.array.cons]{\tcode{mask_array}\ constructors}

\index{mask_array@\tcode{mask_array}!\tcode{mask_array}}%
\begin{itemdecl}
mask_array();
\end{itemdecl}

\begin{itemdescr}
\pnum
This constructor is declared to be private.
This constructor need not be defined.
\end{itemdescr}

\rSec3[mask.array.assign]{\tcode{mask_array}\ assignment}

\index{operator=@\tcode{operator=}!\tcode{mask_array}}%
\begin{itemdecl}
void operator=(const valarray<T>&) const;
mask_array& operator=(const mask_array&);
\end{itemdecl}

\begin{itemdescr}
\pnum
These assignment operators have reference semantics, assigning the values
of the argument array elements to selected elements of the
\tcode{valarray<T>}\
object to which it refers.
\end{itemdescr}

\rSec3[mask.array.comp.assign]{\tcode{mask_array}\ computed assignment}

\index{operator*=@\tcode{operator*=}!\tcode{mask_array}}%
\index{operator/=@\tcode{operator/=}!\tcode{mask_array}}%
\index{operator\%=@\tcode{operator\%=}!\tcode{mask_array}}%
\index{operator+=@\tcode{operator+=}!\tcode{mask_array}}%
\index{operator-=@\tcode{operator-=}!\tcode{mask_array}}%
\index{operator\^{}=@\tcode{operator\^{}=}!\tcode{mask_array}}%
\index{operator\&=@\tcode{operator\&=}!\tcode{mask_array}}%
\index{operator"|=@\tcode{operator"|=}!\tcode{mask_array}}%
\index{operator<<=@\tcode{operator\shl=}!\tcode{mask_array}}%
\index{operator>>=@\tcode{operator\shr=}!\tcode{mask_array}}%
\begin{itemdecl}
void operator*= (const valarray<T>&) const;
void operator/= (const valarray<T>&) const;
void operator%= (const valarray<T>&) const;
void operator+= (const valarray<T>&) const;
void operator-= (const valarray<T>&) const;
void operator^= (const valarray<T>&) const;
void operator&= (const valarray<T>&) const;
void operator|= (const valarray<T>&) const;
void operator<<=(const valarray<T>&) const;
void operator>>=(const valarray<T>&) const;
\end{itemdecl}

\begin{itemdescr}
\pnum
These computed assignments have reference semantics, applying the
indicated operation to the elements of the argument array and selected elements
of the
\tcode{valarray<T>}\
object to which the mask object refers.
\end{itemdescr}

\rSec3[mask.array.fill]{\tcode{mask_array}\ fill function}

\index{fill@\tcode{fill}!\tcode{mask_array}}%
\begin{itemdecl}
void operator=(const T&) const;
\end{itemdecl}

\begin{itemdescr}
\pnum
This function has reference semantics, assigning the value of its
argument to the elements of the
\tcode{valarray<T>}\
object to which the
\tcode{mask_array}\
object refers.
\end{itemdescr}

\rSec2[template.indirect.array]{Class template \tcode{indirect_array}}

\index{indirect_array@\tcode{indirect_array}}%
\begin{codeblock}
namespace std {
  template <class T> class indirect_array {
  public:
    typedef T value_type;

    void operator=  (const valarray<T>&) const;
    void operator*= (const valarray<T>&) const;
    void operator/= (const valarray<T>&) const;
    void operator%= (const valarray<T>&) const;
    void operator+= (const valarray<T>&) const;
    void operator-= (const valarray<T>&) const;
    void operator^= (const valarray<T>&) const;
    void operator&= (const valarray<T>&) const;
    void operator|= (const valarray<T>&) const;
    void operator<<=(const valarray<T>&) const;
    void operator>>=(const valarray<T>&) const;

    indirect_array(const indirect_array&);
   ~indirect_array();
    indirect_array& operator=(const indirect_array&);
    void operator=(const T&) const;
  private:
    indirect_array();
  };
}
\end{codeblock}

\pnum
This template is a helper template used by the indirect subscript operator

\begin{itemdecl}
indirect_array<T> valarray<T>::operator[](const valarray<size_t>&).
\end{itemdecl}

\begin{itemdescr}
\pnum
It has reference semantics to a subset of an array specified by an
\tcode{indirect_array}.
Thus the expression
\tcode{a[\brk{}indirect] = b;}\
has the effect of assigning the elements of
\tcode{b}\
to the elements in
\tcode{a}\
whose indices appear in
\tcode{indirect}.
\end{itemdescr}

\rSec3[indirect.array.cons]{\tcode{indirect_array}\ constructors}

\index{indirect_array@\tcode{indirect_array}!\tcode{indirect_array}}%
\begin{itemdecl}
indirect_array();
\end{itemdecl}

\begin{itemdescr}
\pnum
This constructor is declared to be private.
This constructor need not be defined.
\end{itemdescr}

\rSec3[indirect.array.assign]{\tcode{indirect_array}\ assignment}

\index{operator=@\tcode{operator=}!\tcode{indirect_array}}%
\begin{itemdecl}
void operator=(const valarray<T>&) const;
indirect_array& operator=(const indirect_array&);
\end{itemdecl}

\begin{itemdescr}
\pnum
These assignment operators have reference semantics, assigning the values
of the argument array elements to selected elements of the
\tcode{valarray<T>}\
object to which it refers.

\pnum
If the
\tcode{indirect_array}
specifies an element in the
\tcode{valarray<T>}\
object to which it refers more than once, the behavior is undefined.
\index{undefined}%

\pnum
\enterexample\ 
\begin{codeblock}
int addr[] = {2, 3, 1, 4, 4};
valarray<size_t> indirect(addr, 5);
valarray<double> a(0., 10), b(1., 5);
a[indirect] = b;
\end{codeblock}
results in undefined behavior since element 4 is specified twice in the
indirection.
\exitexample\ 
\end{itemdescr}

\rSec3[indirect.array.comp.assign]{\tcode{indirect_array}\ computed assignment}

\index{operator*=@\tcode{operator*=}!\tcode{indirect_array}}%
\index{operator*=@\tcode{operator*=}!\tcode{indirect_array}}%
\index{operator*=@\tcode{operator*=}!\tcode{indirect_array}}%
\index{operator/=@\tcode{operator/=}!\tcode{indirect_array}}%
\index{operator\%=@\tcode{operator\%=}!\tcode{indirect_array}}%
\index{operator+=@\tcode{operator+=}!\tcode{indirect_array}}%
\index{operator-=@\tcode{operator-=}!\tcode{indirect_array}}%
\index{operator\^{}=@\tcode{operator\^{}=}!\tcode{indirect_array}}%
\index{operator\&=@\tcode{operator\&=}!\tcode{indirect_array}}%
\index{operator"|=@\tcode{operator"|=}!\tcode{indirect_array}}%
\index{operator<<=@\tcode{operator\shl=}!\tcode{indirect_array}}%
\index{operator>>=@\tcode{operator\shr=}!\tcode{indirect_array}}%
\begin{itemdecl}
void operator*= (const valarray<T>&) const;
void operator/= (const valarray<T>&) const;
void operator%= (const valarray<T>&) const;
void operator+= (const valarray<T>&) const;
void operator-= (const valarray<T>&) const;
void operator^= (const valarray<T>&) const;
void operator&= (const valarray<T>&) const;
void operator|= (const valarray<T>&) const;
void operator<<=(const valarray<T>&) const;
void operator>>=(const valarray<T>&) const;
\end{itemdecl}

\begin{itemdescr}
\pnum
These computed assignments have reference semantics, applying the indicated
operation to the elements of the argument array and selected elements of the
\tcode{valarray<T>}\
object to which the
\tcode{indirect_array}\
object refers.

\pnum
If the
\tcode{indirect_array}\
specifies an element in the
\tcode{valarray<T>}\
object to which it refers more than once,
the behavior is undefined.
\index{undefined}
\end{itemdescr}

\rSec3[indirect.array.fill]{\tcode{indirect_array}\ fill function}

\index{fill@\tcode{fill}!\tcode{indirect_array}}%
\begin{itemdecl}
void operator=(const T&) const;
\end{itemdecl}

\begin{itemdescr}
\pnum
This function has reference semantics, assigning the value of its argument
to the elements of the
\tcode{valarray<T>}\
object to which the
\tcode{indirect_array}\
object refers.
\end{itemdescr}
