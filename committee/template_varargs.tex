\documentclass{article}
\usepackage{hyperref}
\usepackage{fullpage}

\begin{document}
\title{Typesafe Variable-length Function and Template Argument Lists}
\author{Douglas Gregor \and Gary Powell \and Jaakko J\"arvi}
\maketitle

\section{Introduction}
This proposal addresses three problems under a unified framework:
\begin{itemize}
\item The inability to instantiate class and function templates with an arbitrarily-long list of template parameters.
\item The inability to pass an arbitrary number of arguments to a function in a type-safe manner.
\item The argument forwarding problem.
\end{itemize}

The proposed resolution is to introduce a syntax and semantics for
variable-length template argument lists (usable with function templates
via explicit template argument specification and with class templates)
along with a method of argument building using the same mechanism to
pass an arbitrary number of function call arguments to a function in a
typesafe manner.

\section{Motivation}
\subsection{Variable-length template parameter lists}
Variable-length template parameter lists (varargs) allow a class or
function template to accept some number (possibly zero) of template
arguments beyond the number of template parameters specified. This
behavior can be simulated in C++ via a long list of defaulted template
parameters, e.g., a typelist wrapper may appear as:

\begin{verbatim}
struct unused;
template<typename T1 = unused, typename T2 = unused, 
         typename T3 = unused, typename T4 = unused, 
         /* up to */ typename TN = unused> class list;
\end{verbatim}

This technique is used by the Boost Tuples library~\cite{Tuples01},
for the specification of class template {\tt std::tuple<>} in the
library TR~\cite{Jarvi02}, and in the Boost metaprogramming
library~\cite{Gurtovoy02}. Unfortunately, this method leads to very
long type names in error messages (compilers tend to print the
defaulted arguments) and very long mangled names. It is also not
scalable to additional arguments without resorting to preprocessor
magic~\cite{Preprocessor01}. In all of these libraries (and presumably
many more), an implementation based on template varargs would be
shorter and would not suffer the limitations of the aforementioned
implementation. The declaration of the {\tt list<>} template above may
be:
\begin{verbatim}
template<...> class list;
\end{verbatim}

This language feature is primarily designed for library developers,
and is drawn from experience in library development where these
techniques have been simulated in one way or
another~\cite{Gurtovoy02,Lambda02,Tuples01,Function01,Bind01}. Novice
users aren't likely to use this feature, as generically manipulating
heterogeneous type container (tuples, in this case) requires some
degree of metaprogramming.

\subsection{Typesafe Variable-length Function Parameter Lists}
Variable-length function parameter lists allow more arguments to be
passed to a function than are declared by the function. This feature
is rarely used in C++ code (except for compatibility with C
libraries), because passing non-POD types through {\tt ...} invokes
undefined behavior. However, a typesafe form of such a feature would
be useful in many contexts, e.g., for implementing a typesafe C++
{\tt printf} that works for non-POD types. The lack of such a facility
has resulted in odd syntax for formatting~\cite{Krempp02}:
\begin{verbatim}
format("writing %1%, x=%2%: %3%-th try") % "toto" % 40.23 % 50
\end{verbatim}

\subsection{Argument forwarding}
The forwarding problem, as described in~\cite{Dimov02}, is:

\begin{quote}
  For a given expression $E(a_1, a_2, ..., a_n)$ that depends on the
  (generic) parameters $a_1, a_2, ..., a_n$, it is not possible to
  write a function (object) $f$ such that $f(a_1, a_2, ..., a_n)$ is
  equivalent to $E(a_1, a_2, ..., a_n)$.
\end{quote}

The primary issue with the argument forwarding problem is to deduce
the types of $a_1, a_2, ..., a_n$ such that they can be forwarded to
the underlying function $E$ without changing the semantics of the
expression. The common implementation idiom is to accept a (non-const)
reference for each parameter, and to provide overloads for some number
of arguments to forward, e.g.:
\begin{verbatim}
template<typename E, typename T1>
void f(E e, T1& a1) { e(a1); }

template<typename E, typename T1, typename T2>
void f(E e, T1& a1, T2& a2) { e(a1, a2); }

template<typename E, typename T1, typename T2, typename T3>
void f(E e, T1& a1, T2& a2, T3& a3) { e(a1, a2, a3); }

// repeat up to N arguments
\end{verbatim}

However, this idiom fails to accept literal values. This proposal
solves both issues: arguments are passed via the ``Const + Non-const
reference'' option \#3 enumerated in~\cite{Dimov02}, which results in
``perfect'' forwarding of arguments, but does not require $2^N$
overloads to support $N$ arguments: instead, a single overload
suffices for any number of arguments. The solution matches all three
criteria (C1-C3) of the argument forwarding paper~\cite{Dimov02} by
bundling the arguments into a single argument and providing an
unbundling method to pass the arguments on to another function.

\section{Syntax and Semantics}
\subsection{Variable-length template parameter lists}
\par The template parameter list to a function or class template can
be declared to accept an arbitrary number of extra template arguments
by terminating it with ``...'' optionally followed by an identifier
that will hold a tuple of the extra arguments. Any number of template
parameters may precede the ``...''. Thus we can define, for instance,
a class {\tt Foo} accepting a template type parameter followed by an
arbitrary number of template arguments as:
\begin{verbatim}
template<typename T, ... Elements> class Foo;
\end{verbatim}

Here, {\tt T} is the name of the template type parameter and {\tt
  Args} is the name of a {\tt std::tuple<>}~\cite{Jarvi02} containing
the (possibly empty) set of template arguments given to {\tt Foo}.
Given the instantiation {\tt Foo<int, float, double, std::string>},
{\tt Elements} would be {\tt std::tuple<float, double, std::string>}
(note that {\tt T} is {\tt int}). 

Nontype and template template arguments cannot be placed directly in
the tuple (see~\cite{Jarvi02}), so they are wrapped in {\tt
  nontype\_template\_arg} or {\tt template\_template\_arg}, respectively. 

{\tt nontype\_template\_arg} is defined as:
\begin{verbatim}
template<typename T, T V> struct nontype_template_arg {
  typedef T type;
  static const T value = V; 
};
\end{verbatim}

{\tt template\_template\_arg} is a class template with the following
signature: 
\begin{verbatim}
template<typename Designator>
struct template_template_arg {
  template<... Args> struct apply {
    typedef /* see below */ type;
  }
};
\end{verbatim}

The {\tt Designator} type is an unspecified type that uniquely
identifies the class template (e.g., {\tt std::vector}) but that is
itself not a template\footnote{This definition may be expanded to
  allow typedef templates or template aliases as well, if they can be
  passed as template arguments.}. The {\tt apply} member class
template accepts some number of template arguments and instantiates
the template designated by the {\tt Designator} with those parameters.
Thus instantiating {\tt Foo<int, std::vector>} results in {\tt Args}
being {\tt std::tuple<template\_template\_arg<X> >} (where {\tt X} is
unspecied and unique to {\tt std::vector}) and where {\tt
  template\_template\_arg<X>::apply<int>::type} is equivalent to {\tt
  std::vector<int>}.

Function templates may also be declared to accept variable-length
template parameter lists in the same manner as class templates. In
this case (and barring reuse of the vararg type as the final function
parameter, as described in the next section) arguments may only by
specified explicitly when calling the function template; they are not
deduced, e.g.,:

\begin{verbatim}
template<... Args> void f();
// ... 
f<int, double>(); // Args will be std::tuple<int, double>
\end{verbatim}

\subsection{Typesafe variable-length function parameter lists}
When a function template header contains an ellipsis (followed by an
identifier), that identifier may be reused as the type of the final
parameter in the function's parameter list to accept the tuple of
arguments passed to the function. For instance, a C++ {\tt printf} may
be declared as:
\begin{verbatim}
template<... Args>
  void printf(const char* format, Args args);
\end{verbatim}

{\tt Args} is deduced as an {\tt std::tuple} containing the types of
the arguments, then {\tt args} is a value of that type containing the
function call arguments passed after the {\tt format}
argument.\footnote{Note that, unlike C varargs, we do not require but
  do allow an argument prior to the argument bundle.} Arguments are
then accessed via any operations on a tuple. 

\subsection{Forwarding function arguments}
A tuple of values can be forwarded to a function object via a new
function template {\tt std::apply} that unbundles a tuple into
separate elements to be passed to a function object. The
pseudo-declarations for {\tt std::apply} are:
\begin{verbatim}
template<typename F, typename T1, typename T2, ..., typename TN>
the-return-type apply(F& f, tuple<T1, T2, ..., TN>& t);

template<typename F, typename Class, typename T1, typename T2, ..., typename TN>
the-return-type apply(F Class::*pm cv, Class& obj, tuple<T1, T2, ..., TN>& t);

template<typename F, typename Class, typename T1, typename T2, ..., typename TN>
the-return-type apply(F Class::*pm cv, const Class& obj, tuple<T1, T2, ..., TN>& t);
\end{verbatim}
A call expression {\tt std::apply(f, t)} where {\tt f} is a function
object and {\tt t} is of type {\tt std::tuple<T1, T2, ..., TN>}
returns {\tt f(get<1>(t), get<2>(t), ..., get<N>(t))}. Similarly, a
call expression {\tt std::apply(pmf, obj, t)}, where {\tt pmf} is a
pointer-to-member function, {\tt obj} is an object whose type is equal
to or derived from the type of the member function pointer's class
type, and {\tt t} is of type {\tt std::tuple<T1, T2, ..., TN>} returns
{\tt (obj.*pmf)(get<1>(t), get<2>(t), ..., get<N>(t))}.

With these definitions, we can define a function {\tt g} that forwards
perfectly to its function object parameter {\tt f}:
\begin{verbatim}
template<typename F, ... Args>
  the-return-type g(F f, Args args) {return std::apply(f, args);}
\end{verbatim}

\subsection{Argument type deduction}
The type of each element in the argument tuple is deduced to achieve
perfect argument forwarding. For an argument {\tt xi}, the type {\tt
  Ti} of the corresponding tuple element is the parameter type of the
{\tt forward} function determined by overload resolution using the
following two candidates:
\begin{verbatim}
template<typename T> void forward(T&);
template<typename T> void forward(const T &);
\end{verbatim}

\subsection{Partial ordering of vararg class template partial specializations}
The class template partial specialization partial ordering rules will
need to be augmented to include partial ordering with vararg
templates. Intuitively, a binding to a parameter that falls into a
template's variable-length argument list is weaker than a binding to a
specified template parameter. For instance, given:

\begin{verbatim}
template<...> struct foo;
template<typename T, ...> struct foo<T>; // #1
template<typename T, typename U> struct foo<T, U>; // #2
\end{verbatim}

Partial specialization \#2 is more specialized than partial
specialization \#1 because \#2 requires that the second template
argument by a type (and not a template or acceptable
literal). 

Formalizing this notion, we introduce a new type of template
parameter we call a template {\em variant} parameter. Template variant
parameters cannot be declared explicitly, but occur implicitly as
parameters for vararg templates. However, any template argument
(type, nontype, or template) can be passed to a template {\em variant}
parameter. 

Paragraph 3 of 14.5.5.2 [temp.func.order] describes the rules for
transforming a template for the purpose of partial ordering. To
support partial ordering with template varargs, introduce two
additional bullets:
\begin{itemize}
\item If after removing the variable-length template parameter list
  designator from both templates the template parameter lists of the
  templates are of different length, append unique template variant
  parameters to the shorter template parameter list until the template
  parameter lists are of equal length.
\item A binding of an argument to a variant parameter is weaker than a
  binding of an argument to a parameter of the same kind.
\end{itemize}

\subsection{Overloading}
\label{overloading}
The overloading rules need two minor changes to accomodate template
varargs:
\begin{itemize}
\item Template vararg functions follow the same overloading rules as
  non-template vararg functions.
\item If two overloads differ only in that one is a template vararg
  function and the other is a non-template vararg function, the
  template vararg function is more specialized.
\end{itemize}

Thus we prefer the typesafe template varargs to unsafe template
varargs, even though these rules conflict with 13.3.3p1,
which prefers non-template functions to template functions when the
conversion sequences are otherwise equal. 

\subsection{Function traits}
The Type Traits library proposal~\cite{Maddock03} defines a set of
class templates that report information about types. We propose to add
one additional class template {\tt function\_traits} that reports
information about function types. The pseudo-definition follows:
\begin{verbatim}
template<typename FunctionType> struct function_traits;

// For each 0 <= N <= implementation-defined maximum
template<typename R, typename T1, typename T2, ..., typename TN>
struct function_traits<R(T1, T2, ..., TN)>
{
  typedef R result_type;
  typedef tuple<T1, T2, ..., TN> argument_types;
};
\end{verbatim}

This class template can be implemented for a fixed number of
arguments in a library, but requires compiler support to handle an
arbitrary number of arguments. It is useful for many libraries that
deal with function and member function pointers as function objects,
where one may need to forward a set of function arguments whose types
are fixed (see Section~\ref{functionimpl}).

\section{Examples}
To demonstrate the use of varargs templates, this section gives
skeletal implementations of four library components using vararg
templates, including Tuples. All of these proposals were accepted by
the library working group into the first library TR, and are
implementable to some degree in C++03. However, all of them would
benefit from template varargs in that the implementations can be more
complete, more useful, and more readable than their C++03
counterparts. The implementations are not complete, but omit only
features that are irrelevant to the discussion of template varargs.

\subsection{Member pointer adaptor implementation}
The following code implements the enhanced member pointer adaptor
proposal~\cite{Dimov03a}. The {\tt enable\_if} and {\tt
  is\_function\_type} class templates are used but not defined, as
they are implementable directly in C++2003 and their definitions would
offer little additional insight.

\small
\begin{verbatim}
template<typename Class, typename FunctionType>
struct mem_fn_adaptor
{
  mem_fn_adaptor(FunctionType Class::*pmf) : pmf(pmf) {}
  
  template<... Args>
  the-return-type operator()(Class& object, Args args) const
    { return std::apply(pmf, object, args); }

  template<... Args>
  the-return-type operator()(const Class& object, Args args) const
    { return std::apply(pmf, object, args); }

  template<typename T, ... Args>
  the-return-type operator()(T& ptr, Args args) const
    { return std::apply(pmf, *ptr, args); }

  template<typename T, ... Args>
  the-return-type operator()(const T& ptr, Args args) const
    { return std::apply(pmf, *ptr, args); }

private:
  FunctionType Class::*pmf;
};

template<typename Class, typename T>
struct mem_ptr_adaptor
{
  mem_ptr_adaptor(T Class::*pm) : pm(pm) {}

  T& operator()(Class& object) const { return object.*pm; }
  const T& operator()(const Class& object) const { return object.*pm; }

  template<typename Ptr> const T& operator()(Ptr& ptr) const 
    { return (*ptr).*pm; }
  template<typename Ptr> const T& operator()(const Ptr& ptr) const
    { return (*ptr).*pm; }

private:
  T Class::*pm;
};

template<typename Class, typename FunctionType>
typename enable_if<(is_function_type<FunctionType>::value), 
                   mem_fn_adaptor<Class, FunctionType> >::type
mem_fn(FunctionType Class::*pmf)
  { return mem_fn_adaptor<Class, FunctionType>(pmf); }

template<typename Class, typename T>
typename enable_if<(!is_function_type<T>::value), 
                   mem_fn_adaptor<Class, T> >::type
mem_fn(T Class::*pm)
  { return mem_fn_adaptor<Class, T>(pm); }
\end{verbatim}
\normalsize 

\subsection{Tuple implementation}
\label{tupleimpl}
The following code implements the interesting portions of the Tuple
proposal~\cite{Jarvi02} using template varargs as proposed here. Note
that there is a mutual dependency: template varargs depend on tuples
and tuples are specified as a class template accepting a
variable-length template argument list. However,
this mutual dependency is not problematic, because no tuple types need
to be named until after the {\tt tuple} template has been
defined. The definition of tuples is recursive, such that a tuple of
length $N$ is implemented as a ``head'' element and a tuple of length
$N-1$, with a tuple of length $0$ specified as an explicit
specialization.

\small
\begin{verbatim}
// Tuple is tuple tuple<T1, T2, ..., TN>
template<typename Tuple> struct tuple_tail;

template<... Elements>
class tuple
{
  typedef typename tuple_element<Elements, 1>::type head_type;
  typedef typename tuple_tail<Elements>::type tail_type;
  typedef tuple self_type;

public:
  tuple() {}

  template<typename T1, ... Args> explicit tuple(const T1& t1, Args args)
    : head(t1), tail(args) {}

  template<typename Tuple>
  tuple(const Tuple& other,
        typename enable_if<(is_base_and_derived<tuple<>, self_type>::value),
                           void>::type * = 0)
    : head(other.head), tail(other.tail) {}

  template<typename Tuple>
  typename enable_if<(is_base_and_derived<tuple<>, self_type>::value),
                     tuple&>::type
  operator=(const Tuple& tuple)
  {
    head = other.head;
    tail = other.tail;
    return *this;
  }

private:
  head_type head;
  tail_type tail;
};

// Base case: a nullary tuple
template<> class tuple<> {};

// 2-element case, for pair specializations
template<typename T1, typename T2>
class tuple
{
  typedef T1 head_type;
  typedef tuple<T2> tail_type;

public:
  tuple() {}

  tuple(const T1& t1, const T2& t2) : head(t1), tail(t2) {}

  template<typename U1, typename U2>
  tuple(const pair<U1, U2>& other) : head(other.first), tail(other.second) {}

  template<typename U1, typename U2>
  tuple(const tuple<U1, U2>& other) : head(other.head), tail(other.tail) {}

  template<typename U1, typename U2>
  tuple& operator=(const pair<U1, U2>& other)
  {
    head = other.first;
    tail = tail_type(other.second);
    return *this;
  }

  template<typename U1, typename U2>
  tuple& operator=(const tuple<U1, U2>& other)
  {
    head = other.head;
    tail = other.tail;
    return *this;
  }

private:
  head_type head;
  tail_type tail;
};

// Tuple is tuple tuple<T1, T2, ..., TN>
template<typename Tuple>
struct tuple_tail
{
  // tuple<T2, T3, ..., TN>
  typedef /* ... */ type;

  // Returns tuple<T2, T3, ..., TN>(get<2>(t), get<3>(t), ..., get<N>(t))
  type get(const Tuple& t);
};

// Tuple is tuple<T1, T2, ..., TN>
template<typename Tuple>
struct make_tuple_result
{
  // tuple<U1, U2, ..., UN>, where:
  //  V1 = T1 with top-level lvalue(/rvalue) removed
  //  U1 = T& if V1 is reference_wrapper<T>, V1 otherwise
  typedef /* meta mumbo jumbo */ type;
};

template<... Args>
typename make_tuple_result<Args>::type
make_tuple(Args args)
{
  typedef typename make_tuple_result<Args>::type result_type;
  return result_type(args);
}

template<... Args> Args tie(Args args) { return args; }
\end{verbatim}
\normalsize

\subsection{Function implementation}
\label{functionimpl}
This section implements the core interesting portions of class
template {\tt function} from the function object wrapper
proposal~\cite{Gregor02}.  
\small
\begin{verbatim}
struct bad_function_call : public std::exception {};

template<typename R, typename Args>
struct function_invoker_base
{
  virtual ~function_invoker_base() {}
  virtual R invoke(Args& args) = 0;
  virtual function_invoker_base* clone() = 0;
};

template<typename R, typename Args, typename F>
struct function_invoker : public function_invoker_base<R, Args>
{
  function_invoker(F f) : f(f) {}
  R invoke(Args& args) { return std::apply(f, args); }
  function_invoker_base<R, Args>* clone() { return new function_invoker(f); }

private:
   F f;
};

template<typename Args, typename F>
struct function_invoker<void, Args, F> 
  : public function_invoker_base<void, Args>
{
  function_invoker(F f) : f(f) {}
  void invoke(Args & args) { std::apply(f, args); }
  function_invoker_base<void, Args>* clone() { return new function_invoker(f); }

private:
  F f;
};

template<typename Function> // Function type R(T1, T2, ..., TN)
struct function
{
  typedef typename function_traits<Function>::result_type result_type;
  typedef typename function_traits<Function>::argument_types argument_types;
  
private:
  typedef function_invoker_base<result_type, argument_types> invoker_type;
  
public:
  function() : invoker(0) {}
  
  function(const function& other) : invoker(0)
    { if (other.invoker) invoker = other.invoker->clone(); }

  template<typename F> function(F f) : invoker(0)
    { invoker = new function_invoker<result_type, argument_types, F>(f); }

  template<typename F> function(reference_wrapper<F> f) : invoker(0)
  { 
    invoker = new function_invoker<result_type, argument_types, F&>(f.get()); 
  }
  
  template<typename T, typename C> function(T C::*pm) : invoker(0)
    { *this = std::mem_fn(pm); }

  function(int) : invoker(0) {}
  
  ~function() { if (invoker) { delete invoker; invoker = 0; } }

  function& operator=(const function& other)
  {
    function tmp(other);
    tmp.swap(*this);
    return *this;
  }
  
  template<typename F> function& operator=(F f)
  {
    function tmp(f);
    tmp.swap(*this);
    return *this;
  }
  
  operator bool() const { return invoker; }

  template<... Args>
  result_type operator()(Args args) const
  {
    if (!invoker) throw bad_function_call;
    return invoker->invoke(args);
  }
  
  void swap(function& other)
  {
    using std::swap;
    swap(invoker, other.invoker);
  }
  
private:
  invoker_type* invoker;
};
\end{verbatim}
\normalsize

\bibliographystyle{abbrv}
\bibliography{template_varargs}
\end{document}
