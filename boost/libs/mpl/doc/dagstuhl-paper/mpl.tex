\documentclass{netobjectdays}

\newcommand{\Cpp}{C\kern-0.05em\texttt{+\kern-0.03em+}}
\newcommand{\mpl}{\code{boost::mpl}}


\usepackage{times}

\newif\ifpdf
\ifx\pdfoutput\undefined
   \pdffalse
\else
   \pdfoutput=1
   \pdftrue
\fi

\ifpdf
  \usepackage[
              pdftex,
              colorlinks=true,
              linkcolor=blue,filecolor=blue,pagecolor=blue,urlcolor=blue
              ]{hyperref}
\fi

\ifpdf
  \newcommand{\concept}[1]{\hyperref[concept:#1]{\textsf{#1}}}
  \newcommand{\stlconcept}[1]{\href{http://www.sgi.com/Technology/STL/#1.html}{\textsf{#1}}}
  \newcommand{\link}[2]{\hyperref[#1]{#2}}
\else
  \newcommand{\concept}[1]{\textsf{#1}}
  \newcommand{\stlconcept}[1]{\textsf{#1}}
  \newcommand{\href}[2]{#2}
  \newcommand{\link}[2]{#2}
\fi

\newcommand{\code}[1]{{\small \texttt{#1}}}

\newcommand{\Note}[1]{\marginpar{\begin{flushleft}%
  {%%\tiny %%\footnotesize
  {\bf Note:}  #1}%
\end{flushleft}}}


\begin{document}

\title{The Boost \Cpp\ Template Metaprogramming Library}

\author{Aleksey Gurtovoy$^\dag$ and David Abrahams$^\ddag$ \\
\\
$^\dag$ Meta Communications \\
\texttt{agurtovoy@meta-comm.com}\\
\\
$^\ddag$ Boost Consulting \\
\texttt{david.abrahams@rcn.com}
}

\maketitle

\begin{abstract} $\!$This paper describes the Boost \Cpp template
metaprogramming library (\mpl), an extensible compile-time framework
of algorithms, sequences and function classes. The library brings
together important abstractions from the generic and functional
programming worlds to build a powerful and easy-to-use
metaprogramming toolset which makes template metaprogramming practical
enough for the real-world environments. The MPL is heavily influenced
by its run-time equivalent - the Standard Template Library (STL), a
part of the C++ standard library. Like the STL, it defines an open
conceptual and implementation framework which can serve as a
foundation for future contributions in the domain. The library's
fundamental concepts and idioms enable the user to focus on solutions 
without navigating the universe of possible ad-hoc approaches to a 
given metaprogramming problem, even if no actual MPL code is used. 
{\mpl} also provides a compile-time lambda expression facility enabling
arbitrary currying and composition of class templates, a feature whose
runtime counterpart is often cited as missing from the STL. This paper
explains motivation, usage, design, and implementation of \mpl 
library, gives some advanced examples of its real-life applications, 
and offers some lessons learned about C++ template metaprogramming.
\end{abstract}


\section{Introduction}
\subsection{What is template metaprogramming? }
\subsection{Is it for real-world? }
\subsection{Motivation (Why a library is needed?)}
\subsection{What about portability? }
\subsection{Relation to other work.}

\section{Basic usage}
\subsection{Sequences, algorithms, and iterators}
\subsection{Function classes, simple composition}
\subsection{Lambda facility}

\section{Design of boost::mpl}
\subsection{Design goals}
\subsection{Design decisions}

\section{Advanced examples (What can I do with boost::mpl?)}
\subsection{Using \mpl\ to implement compile-time FSM generator}

\section{About implementation}

\section{Lessons learned}

\section{Conclusions}
\section{Acknowledgements}
\section{References}

\bibliographystyle{abbrv}
\bibliography{refs}

\end{document}
% LocalWords:  Aleksey David Gurtovoy Abrahams MPL STL Boost boost
