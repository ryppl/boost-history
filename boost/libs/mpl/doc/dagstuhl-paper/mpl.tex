\documentclass{netobjectdays}

\newcommand{\Cpp}{C\kern-0.05em\texttt{+\kern-0.03em+}}
\newcommand{\mpl}{\code{boost::mpl}}


\usepackage{times}

\newif\ifpdf
\ifx\pdfoutput\undefined
   \pdffalse
\else
   \pdfoutput=1
   \pdftrue
\fi

\ifpdf
  \usepackage[
              pdftex,
              colorlinks=true,
              linkcolor=blue,filecolor=blue,pagecolor=blue,urlcolor=blue
              ]{hyperref}
\fi

\ifpdf
  \newcommand{\concept}[1]{\hyperref[concept:#1]{\textsf{#1}}}
  \newcommand{\stlconcept}[1]{\href{http://www.sgi.com/Technology/STL/#1.html}{\textsf{#1}}}
  \newcommand{\link}[2]{\hyperref[#1]{#2}}
\else
  \newcommand{\concept}[1]{\textsf{#1}}
  \newcommand{\stlconcept}[1]{\textsf{#1}}
  \newcommand{\href}[2]{#2}
  \newcommand{\link}[2]{#2}
\fi

\newcommand{\code}[1]{{\small \texttt{#1}}}

\newcommand{\Note}[1]{\marginpar{\begin{flushleft}%
  {%%\tiny %%\footnotesize
  {\bf Note:}  #1}%
\end{flushleft}}}


\begin{document}

\title{The Boost \Cpp\ Template Metaprogramming Library}

\author{Aleksey Gurtovoy$^\dag$ and David Abrahams$^\ddag$ \\
\\
$^\dag$ Meta Communications \\
\texttt{agurtovoy@meta-comm.com}\\
\\
$^\ddag$ Boost Consulting \\
\texttt{david.abrahams@rcn.com}
}

\maketitle

\begin{abstract} $\!$This paper describes the Boost \Cpp template
metaprogramming library (\mpl), an extensible compile-time framework
of algorithms, sequences and function classes. The library brings
together important abstractions from the generic and functional
programming worlds to build a powerful and easy-to-use
metaprogramming toolset which makes template metaprogramming practical
enough for the real-world environments. The MPL is heavily influenced
by its run-time equivalent - the Standard Template Library (STL), a
part of the C++ standard library. Like the STL, it defines an open
conceptual and implementation framework which can serve as a
foundation for future contributions in the domain. The library's
fundamental concepts and idioms enable the user to focus on solutions 
without navigating the universe of possible ad-hoc approaches to a 
given metaprogramming problem, even if no actual MPL code is used. 
{\mpl} also provides a compile-time lambda expression facility enabling
arbitrary currying and composition of class templates, a feature whose
runtime counterpart is often cited as missing from the STL. This paper
explains motivation, usage, design, and implementation of \mpl 
library, gives some advanced examples of its real-life applications, 
and offers some lessons learned about C++ template metaprogramming.
\end{abstract}


\section{Introduction}

\subsection{What is Metaprogramming?}

Metaprogramming is usually defined as the creation of programs which
generate other programs. Parser generators such as YACC are examples
of one kind of program-generating program. The input language to YACC
is a context-free grammar in EBNF, and its output is a program which
parses that grammar. Note that in this case the metaprogram (YACC) is
written in a language (`C') which does not directly support the
description of generated programs. These specifications, which we'll
call \emph{meta-data}, are not written in `C', but in a
\emph{meta-language}. Because the the rest of the user's program
typically requires a general-purpose programming system and must
interact with the generated parser, the meta-data is translated into
`C', which is then compiled and linked together with the rest of the
system. The meta-data thus undergoes two translation steps, and the
user is always very conscious of the boundary between his meta-data
and the rest of his program.
% need bibliography reference for YACC

A more interesting form of metaprogramming is available in languages
such as Scheme, where the generated program specification is given in
the same language as the metaprogram itself. The metaprogrammer
defines his meta-language as a subset of the expressible forms of the
underlying language, and program generation can take place in the same
translation step used to process the rest of the user's program. This
allows users to switch transparently between ordinary programming,
generated program specification, and metaprogramming, often without
being aware of the transition.
% bib reference for Scheme metaprogramming

\subsection{Metaprogramming in \Cpp }

In \Cpp, it was discovered almost by accident that the template
mechanism provides a rich facility for computation at
compile-time. For example, the following tiny meta-function computes the
factorial of its argument:

{\footnotesize
\begin{verbatim}
template <unsigned n>
struct factorial
{
   static const unsigned value = n * factorial<n-1>::value;
};

template <>
struct factorial<0>
{
   static const unsigned value = 1;
};
\end{verbatim}
}

We might use this function, for example, to enumerate the values of a
given function when used to reduce all permutations of an array:

{\footnotesize
\begin{verbatim}
// Holder for an array of N doubles
template <unsigned N>
struct double_array
{
   double values[N];
};

template <unsigned N, class F>
double_array<factorial<N>::value>
accumulate_all_permutations(double (&input)[N], F f)
{
   double_array<factorial<N>::value> result;
   double* p = result.values;
   for (unsigned i = 0; i < N; ++i)
   {
       double copied[N];
       std::copy(input, input + N, copied);
       *p++ = std::accumulate(copied, copied+N, 0, f);
       std::next_permutation(input, input + N);
   }
   return result;
}
\end{verbatim}
}

Because of the hard line between the expression of compile-time and
runtime computation in \Cpp, metaprograms look different from their
runtime counterparts. Take for example a straightforward runtime
definition of the factorial function:

{\footnotesize
\begin{verbatim}
unsigned factorial(unsigned N)
{
   return N == 0 ? 1 : N * factorial(N - 1);
}
\end{verbatim}
}

This definition would be useless in \code{accumulate\_\-all\_\-permutations}
above, since in \Cpp array sizes are required to be computed at
compile-time. While it is easy to see the analogy between the two
recursive defintions, this feature is in fact more important to \Cpp
metaprograms than it is to runtime \Cpp. In contrast to languages such
as Lisp where recursion is idiomatic, \Cpp programmers will typically
avoid recursion when possible. This is often for efficiency reasons,
but also because of ``cultural momentum'': recursive programs are just
harder (for \Cpp programmers) to think about. Like pure Lisp, the \Cpp
template mechanism is a \emph{functional} programming language which
rules out the use of data mutation required to maintain loop
variables.

Another key difference between the runtime and compile-time factorial
functions is the expression of the termination condition: our
meta-factorial uses template specialization as a kind of
\emph{pattern-matching} mechanism to describe the behavior when
\code{N} is zero. The syntactic analogue in the runtime world would
require two separate definitions of the same function. In this case
the impact of the second definition is minimal, but in large
metaprograms the cost of maintaining and understanding the terminating
definitions can become significant.

\subsection{Why Metaprogramming?}

It's worth asking why anyone would want to do this. To see the answer,
let's examine the alternatives:

\begin{enumerate}

\item We could write programs to interpret the meta-data directly. For
  example, YACC might have been written as a function which accepts a
  string containing the grammar description and a pointer-to-function
  returning tokens from the stream to be parsed. This approach,
  however, would impose unacceptable runtime costs for most
  applications: either the parser would have to treat the grammar
  nondeterministically, exploring the space of possible parses, or it
  would have to begin by replicating at runtime the substantial
  table-generation work of the existing YACC for each input
  grammar. In our factorial example, the array size would become a
  runtime quantity with the associated cost of dynamic allocation.

\item We could replace the compile-time computation with our own
  analysis. After all, the size of arrays passed to
  \code{accumulate\_\-all\_\-permutations} are always known at
  compile-time, and thus can be known to its user. We could ask the
  user to supply the result size:
  {\footnotesize
  \begin{verbatim}
template <unsigned ResultSize, unsigned N, class F>
double_array<ResultSize>
accumulate_all_permutations(double (&input)[N], F f)
  \end{verbatim}
  }
  The costs to this approach are obvious: we give up expressivity (by
  requiring the user to explicitly specify implementation details),
  and correctness (by allowing the user to specify them
  incorrectly). Anyone who has had to write parser tables by hand will
  tell you that the impracticality of this approach is the very reason
  YACC's existence.
\end{enumerate}

So, the motivation for Metaprogramming comes down to the combination
of three factors: efficiency, expressivity, and correctness.

\subsection{Is it for real-world? }
\subsection{Motivation (Why a library is needed?)}
\subsection{What about portability? }
\subsection{Relation to other work.}

\section{Basic usage}
\subsection{Sequences, algorithms, and iterators}
\subsection{Function classes, simple composition}
\subsection{Lambda facility}

\section{Design of boost::mpl}
\subsection{Design goals}
\subsection{Design decisions}
\subsubsection{Use of iterators}
\subsubsection{Abstraction of sequences}

\section{Advanced examples (What can I do with boost::mpl?)}
\subsection{Using \mpl\ to implement compile-time FSM generator}

\section{About implementation}

\section{Lessons learned}

\section{Conclusions}
\section{Acknowledgements}
\section{References}

\bibliographystyle{abbrv} \bibliography{refs}

\end{document}
% LocalWords:  Aleksey David Gurtovoy Abrahams MPL STL Boost boost
